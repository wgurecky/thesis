The goal of this work is to capture small scale flow phenomena downstream of spacer grids
in COBRA-TF (CTF) \cite{salko12}, a subchannel-resolution reactor thermal hydraulic (TH) package, by utilizing higher fidelity results provided by CFD simulations.  Specifically, the information retained from the high fidelity CFD calculations will target
 improving CRUD-growth predictions in coupled CTF/MAMBA
simulations due to higher order flow phenomena that is not explicitly captured in CTF.  MAMBA
is a CRUD simulation code produced by a LANL and ORNL collaboration \cite{collins16}.  Improved treatment of spacer grid effects on CRUD growth will improve predictions of CRUD induced
power shift within a quick executing computational framework.

We begin by decomposting the surface temperature and shear stress fields destributed in space on the rods' surface into deterministic and stochastic componts (henceforth fields of interest FOI).
These FOI govern the growth rate of CRUD on the surface of the rod and the precipitation of boron in the porous crud layer.

By accepting the subchannel result as the deterministic portion of the
spatial FOI, the role of the CFD solution is to provide higher order
moments of the fields about the mean value prediced by CTF.
This is in the same vein as kriging techniques since one utilizes
the CFD result to inform a model of the random component of
the TH fields.  The deterministic portion of the solution is augmented by the
random component of the model to provide a descrption of the target random field(s).
This procedure is analogous to applying a shape function informed by CFD
to the subchannel result.
However, one cannot blindly apply a kriging model in this way since the
conditional quantiles (conditional on the location in the reacotor) of the FOI do not follow a gaussian distribution.
This property of the residual distributions of the FOI gives rise to a major challenge
in this work.

In this work we consider a physics-directed, statistically
based, surrogate model of the relevant small scale flow features.
In this case spatial distributions of the Reynolds-averaged temperature,
TKE, and boundary heat flux on the surface of the fuel pin(s) are treated
as random fields tallied over coarse patches on a rod surface.
Sklar's theorem is leveraged
to decompose multivariate probability densities into into independent copula
and marginal models. The free parameters within the copula model are predicted using a combination of supervised regression and classification machine learning techniques with training
data sets supplied by a suite of pre-computed CFD results spanning the typical TH envelop encounted in an operating reactor. A
gradient boosted regression tree (GBRT) methodology is proposed to provide the
required relationships between the copula model parameters and the
exogenous variables.


\section{Overview of Topics}

A literature review of Hi2Low modeling approaches to CFD-informed subchannel
problems is given in \autoref{chap:lit}.  Several advances in CRUD modeling are
also discused in \autoref{chap:lit} including a CASL developed high fidelity
CFD based CRUD tool currently under development.
Techniques for surrogate model construction are also reviewed.  Examples of
surrogate models developed using regression kriging are provided.  These
techniques are highly relevent to the current Hi2Low problem as these models
seek to describe the behavior of random fields given a sparse set of inputs.
Additionally a review
of the gradient boosted algorithm and applications to surrogate model construction
are provided.

In \autoref{chap:theory}, the overarching Hi2Low strategy is introduced.  A brief
intruduction to Sklar's theorem and gradient boosting is given with additional details
on these topics provided in \autoref{chap:app}.

Work that has been performed in support of the Hi2Low strategy is reviewed in
\autoref{chap:work}.  The development of a CFD data extraction tool with
additional post processing capabilities is provided.  This tool enables the
comparison of CFD result with subchannel results.  Comparisons between CFD and subchannel
subchannel results are presented to identify key flow regimes in which the
coarse subchannel based approach produces erroneous CRUD growth rates.
A demonstration of the proposed method's ability to predict CRUD grow rates on a
single rod at a single TH state point is also given in
\autoref{chap:work}.

Finally, future work is discussed in chapter \autoref{chap:fw}.  This section focuses on
time dependent CRUD simulation strategies and uncertainty quantification and propogation.
The application of a regression kriging technique is also discussed as a potential
avenue for future Hi2Low efforts involving the construction of surrogate models for
 spatial fields that exhibit significant small scale features distributed about a mean
 field pediction.

\subsection{CASL Challenge Problems}

CASL selected several problems identified by industry partners as critical
inadequately understood engineering scale phenomena which would provide financial and
safety benefits to the nuclear power industry if resolved.  An list of the CASL
challenge problems is provided in [ref].  The problem of interest in this
work is the prediction of Chalk River Unidentified Deposit (CRUD) growth rates.
The growth of CRUD comes with neutronic and thermal hydraulic repercussions.  A
phenomena known as CRUD Induced Power Shift (CIPS) as been identified as a high
impact problem by the industry council.  Additionally, the presence of CRUD on the rod
surface has been shown to exacerbate local oxide penetration rates of some zirconium alloys.
This is known as CRUD induced local corrosion.
The proposed work aims to improve predictions of CRUD build up in the core to ultimately improve the
accuracy and provide uncertainty estimates for key physical indicators of the CIPS and CILC phenomena.

The Virtual Environment for Reactor Applications (VERA) is a key component of
CASL's technical portfolio.  This meta-package integrates a variety of physics
packages and multiphysics coupling options to form a robust reactor simulation
capability.  For multi-cycle depletion computations, VERA relies upon MPACT, a
2D-1D method of characteristics neutronics package, coupled with a subchannel
thermal hydraulics code COBRA-TF (CTF).  An integrated CRUD modeling capability
is provided by MAMBA to address the CRUD induced power shift challenge problem.

To reduce computation times, the subchannel TH code discretized the reactor
domain into large, centimeter scale finite volumes.  A second order upwinding
scheme is employed along with the SIMPLE solution algorithm to resolve pressure
velocity coupling.  As a consequences of this discretization scheme, sub-centimeter scale
thermal hydraulic effects of the spacer grids on CRUD are averaged over
large regions on the fuel rods' surfaces.  Though small scale phenomena are not
explicitly modeled, they are approximately accounted for in a variety of empirically derived
closure relations.  In effect, a single constant estimate for the mean thermal
hydraulic conditions are obtained in each finite volume.

Previous Hi2Low focused work in CASL focused on utilizing experimental or CFD
datasets to improve closure models in CTF.  These studies leverage a multitude of
dimensionality reduction and regression techniques
to fit a parametric model to the accepted gold-standard empirical data.  This approach
is adequate for correcting biases in the bulk-average behavior of the flow (due to
the previously neglected physics).  Examples of such Hi2Low models are given in
\autoref{chap:lit}.

The traditional approach must be slightly modified to accommodate the CILC and
CIPS challenge problems.  Here arises the need to retain not only the effect of
fine-scale physics on the bulk, but also to predict if certain
temperature or TKE thresholds are exceeded in a given subchannel
volume.  Furthermore, for a complete treatment of thermal hydraulic impacts on
CRUD growth, the scale-bridging model must describe the frequency
distribution of extreme TH events above a given threshold.

The application of copula and gradient boosted regression and classification
techniques to the CIPS and CILC Hi2Low challenge problems is a novel within the CASL community.
At the core of this work is a semi-parametric method to reconstruct multivariate probability density functions via the use of boosted quantile surface regressions to build marginal distributions coupled with a boosted copula model.  To the author's knowledge this technique is not currently present in the literature.
Of course, the currently proposed work bears some similarities to work by [ref][ref] discussed in section \autoref{chap:lit}.

\section{Benefit and Novelty}

Accurately predicting CRUD induced power shift (CIPS) requires accurate CRUD
thickness and boron deposition estimates in all regions of the core.  Core wide
phenomena are out of scope for high fidelity CFD-Neutronic-CRUD coupled
simulations.  Instead, we rely on CASL's CTF-MAMBA1D coupling capability to
predict CIPS throughout a cycle.  CTF is not capable of simulating the detailed
flow patterns downstream the spacer grids.  Details in the flow field
downstream of spacer grids have important consequences for CRUD growth and
erosion, namely the local depression in surface temperatures and locally
increased shear stresses.  CRUD growth is extremely sensitive to the surface
temperature around the saturation point.  Therefore, growing crud at the bulk
average predicted TH conditions given by CTF in a coarse axial segment may not
yield the correct amount of CRUD growth in that region.  To overcome this issue
developing a surrogate model which is informed by CFD or experimental data that
captures small scale flow effects downsteam of spacer grids is a possible strategy.
Since experimental data and/or CFD data cannot be generated for every concivable
combination of thermal hydraulic and gemetric cases present in a reactor, a surrogate
serves to obtain .  The goal in the construction of a surrogate is to minimize the
number of CFD or experimental data sets required to drive down uncertainties in
the prediced quantities to an acceptable level.

To this end a novel method based on the marriage of copula and gradient boosting techniques
is proposed.  The spatial distributions are not resolved on each CTF patch, rather
multivariate probability density of the surface temperature, TKE, and bundary heat flux
are predicted by the model.
The resultant model produces CRUD estimates that account for the presence of
hot and cold spots on the rod surface induced by turbulent flow downstream of spacer grids.


\section{CRUD Overview}

CRUD growth is dominated by threshold physics: negligible growth or
erosion until a parameter is above some threshold value.  Therefore, it is
important to characterize the tails of the frequency-binned-temperature and
TKE distributions about the mean.  In other words, hot and cold spots
present downstream of spacer grids must be accurately resolved by the Hi2Low model
in order to predict the correct boron precipitation and maximum CRUD
thickness.  It is challenging to faithfully capture the peaks and valleys in
rod surface temperature and TKE distribution by traditional interpolation
techniques since such a model must guard against smearing out the sharp peaks
present in the spatial distributions.  This becomes increasingly difficult to
achieve when the number of parameters required to describe the spatial
distribution is very large.  The proposed work forgos a spatial interpolation
approach for a statistically driven approach which predicts the fractional
area of a rod's surface in excess of some critical temperature - but not
precisely where such maxima occur.  The aim is to reduce the total number of degrees of
freedom in the Hi2Low model so that fewer CFD training data sets are required
to make adequate predictions of key surface TH quantities that impact CRUD growth.
