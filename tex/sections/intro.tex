The goal of this work is to capture small scale flow phenomena downstream of
spacer grids in COBRA-TF (CTF) \cite{salko12}, a subchannel-resolution reactor
thermal hydraulic (TH) package, by utilizing higher fidelity results provided
by Computational Fluid Dynamics (CFD) simulations.  The information retained from the high
fidelity CFD calculations will target improving Chalk River Unidentified Deposite (CRUD) growth predictions in
multi physics core simulations due to higher order flow phenomena that is not
explicitly captured in CTF.  
Improved treatment of spacer grid
effects on CRUD growth will improve predictions of CRUD induced power shift
within a quick executing computational framework.

This proposal describes the development of a physics-directed, statistically
based, surrogate model of the small scale flow features that impact CRUD
growth.  The Reynolds-averaged temperature, turbulent kinetic energy, and
boundary heat flux on the surface of the fuel pin are treated as random
fields distributed over the rod's surface.  The surface temperature, heat flux and
turbulent kinetic energy - henceforth referred to as the fields of interest
(FOI) - distributed on the rods' surface are split into deterministic and
stochastic components.  These FOI govern the growth rate of CRUD on the surface
of the rod and the precipitation of boron in the porous crud layer.
The objective of the surrogate is to provide boundary conditions to the CRUD
model, and thus, the surrogate predicts the behavior of the FOI as a function
of position in the core, local thermal-hydraulic conditions, and perhaps 
more exogenous variables.

The subchannel result is interpret as the deterministic portion of the spatial
FOI distribution in this work; the role of the CFD solution is to provide
higher order moments of the about the mean value predicted by CTF.  This initial
decomposition suggest the application kriging techniques since a model of the
residual distribution of the spatial fields is sought.  Indeed, in the same
vein as kriging, the deterministic portion of the solution is augmented by the
predicted stochastic component of the model to provide a complete description of
the target FOIs.  However, it is shown in this proposal that we cannot blindly
apply a kriging model in this way since the conditional quantiles of the FOI
residual distributions do not follow a Gaussian distribution.  This property of
the residual distributions of the FOI gives rise to a major challenge in this
work.

Instead, Sklar's theorem is leveraged to decompose multivariate probability
densities into into independent copula and marginal models. The free parameters
within the copula model are predicted using a combination of supervised
regression and classification machine learning techniques with training data
sets supplied by a suite of pre-computed CFD results spanning the typical TH
envelop. A gradient boosted regression
(GBR) technique is proposed to develop relationships between the copula
model parameters and several covariates.


\section{Overview of Topics}

A literature review of Hi2Low modeling approaches to CFD-informed subchannel
problems is given in \autoref{chap:lit}.  Several advances in CRUD modeling are
also discussed in \autoref{chap:lit} including a CASL developed high fidelity
CFD based CRUD tool currently under development.  Techniques for surrogate
model construction are also reviewed.  Examples of surrogate models developed
using regression kriging are provided.  These techniques are highly relevant to
the current Hi2Low problem as these models seek to describe the behavior of
random fields given a sparse set of inputs.  Additionally a review of the
gradient boosted algorithm and applications to surrogate model construction are
provided.

In \autoref{chap:theory}, the overarching Hi2Low strategy is introduced.  A
brief introduction to Sklar's theorem and gradient boosting is given with
additional details on these topics provided in \autoref{chap:app}.

Work that has been performed in support of the Hi2Low strategy is reviewed in
\autoref{chap:work}.  The development of a CFD data extraction tool with
additional post processing capabilities is provided.  This tool enables the
comparison of CFD result with subchannel results.  Comparisons between CFD and
subchannel subchannel results are presented to identify key flow regimes in
which the coarse subchannel based approach produces erroneous CRUD growth
rates.  A demonstration of the proposed method's ability to predict CRUD grow
rates on a single rod at a single TH state point is also given in
\autoref{chap:work}.

Finally, future work is discussed in chapter \autoref{chap:fw}.  This section
focuses on time dependent CRUD simulation strategies and uncertainty
quantification and propagation.  The application of a regression kriging
technique is also discussed as a potential avenue for future Hi2Low efforts
involving the construction of surrogate models for spatial fields that exhibit
significant small scale features distributed about a mean prediction.

\subsection{CASL CRUD Challenge Problems}

CASL selected several problems identified by industry partners as critical
inadequately understood engineering scale phenomena which would provide
financial and safety benefits to the nuclear power industry if resolved.  A
list of the CASL challenge problems is provided in [ref].  The problem of
interest in this work is the prediction of Chalk River Unidentified Deposit
(CRUD) growth rates.  The growth of CRUD comes with neutronic and thermal
hydraulic repercussions that are of interest to CASL's industry partners.
A CRUD simulation code produced by a LANL
and ORNL collaboration \cite{collins16} under the name MAMBA is leveraged in this work.

A phenomena known as CRUD Induced Power Shift (CIPS) is caused by the presence
of elevated 10B concentrations in the CRUD layer.  Since CRUD is preferentially
deposited in the hot regions of the core, this leads to a slight shift in
power production towards to bottom of the core.  Predicting the severity of the
CRUD induced power shift is of particular concern when evaluating the thermal
margins of the fuel under a operating and accident conditions.  The prediction
of CIPS is especially important for older facilities seeking to uprate power
output.  Additionally, the presence of CRUD on the rod surface has been shown
to exacerbate local oxide penetration rates of some zirconium alloys [ref].
This is known as CRUD induced local corrosion (CILC).  Improvements in CRUD
simulation techniques ultimately improve the ability to predict the CIPS and
CILC phenomena.

The Virtual Environment for Reactor Applications (VERA) is a key component of
CASL's technical portfolio.  This meta-package integrates a variety of physics
packages and multiphysics coupling options to form a robust reactor simulation
capability.  For multi-cycle depletion computations, VERA relies upon MPACT, a
2D-1D method of characteristics neutronics package, coupled with a subchannel
thermal hydraulics code COBRA-TF (CTF).  An integrated CRUD modeling capability
is provided by MAMBA to address the CRUD induced power shift challenge problem.

To reduce computation times, the subchannel TH code discretized the reactor
domain into large, centimeter scale finite volumes. As a consequences of this
discretization scheme, sub-centimeter scale thermal hydraulic effects of the
spacer grids on CRUD are averaged over large regions on the fuel rods'
surfaces.  Though small scale phenomena are not explicitly modeled, they are
approximately accounted for in a variety of empirically derived closure
relations.  In effect, a single constant estimate for the mean thermal
hydraulic conditions are obtained in each finite volume.

Previous Hi2Low focused work in CASL focused on utilizing experimental or CFD
datasets to improve closure models in CTF.  These studies leverage a multitude
of dimensionality reduction and regression techniques to fit a parametric model
to the accepted gold-standard empirical data.  This approach is adequate for
correcting biases in the bulk-average behavior of the flow (due to the
previously neglected physics).  Examples of such Hi2Low models are given in
\autoref{chap:lit}.

The traditional approach must be modified to accommodate the CILC and CIPS
challenge problems.  Here arises the need to retain not only the effect of
fine-scale physics on the bulk, but also to predict if certain temperature or
TKE thresholds are exceeded in a given subchannel volume.  Furthermore, for a
complete treatment of thermal hydraulic impacts on CRUD growth, the
scale-bridging model must describe completely the frequency distribution of
extreme TH events above a given threshold.


\section{Benefit and Novelty}

Accurately predicting CRUD induced power shift (CIPS) requires accurate CRUD
thickness and boron deposition estimates in all regions of the core.  Core wide
phenomena are out of scope for high fidelity CFD-Neutronic-CRUD coupled
simulations.  Instead, a lower fidelity subchannel-CRUD coupling capability is used to
predict CIPS throughout a cycle.  CTF is not capable of simulating the detailed
flow patterns downstream spacer grids.  Details in the flow field
downstream of spacer grids have important consequences for CRUD growth and
erosion, namely the local depression in surface temperatures and locally
increased shear stresses.  CRUD growth is extremely sensitive to the surface
temperature around the saturation point.  Therefore, growing crud at the bulk
average predicted TH conditions given by CTF in a coarse axial segment may not
yield the correct amount of CRUD growth in that region.  To overcome this issue
a surrogate model which is informed by CFD or experimental data that captures
small scale flow effects downstream of spacer grids can provide the requisite
boundary conditions to the CRUD model.  Since experimental data and/or CFD data
cannot be generated for every conceivable combination of thermal hydraulic and
geometric cases present in a reactor, a surrogate serves to produce predicted
quantities in the entire feature space of interest rather than just at known
sample points.  The goal in the construction of a surrogate is to minimize the
number of CFD or experimental data sets required to drive down uncertainties in
the predicted quantities to an acceptable level.

To this end a novel method based on the marriage of copula and gradient
boosting techniques is proposed.  The proposed work forgos a spatial
interpolation approach for a statistically driven approach which predicts the
fractional area of a rod's surface in excess of some critical temperature - but
not precisely where such maxima occur.  The aim is to simplify the Hi2Low model
so that fewer CFD training data sets are required to make adequate predictions
of key surface TH quantities that impact CRUD growth.  The resultant model
retains the ability to account for the presence of hot and cold spots on the rod surface induced by
turbulent flow downstream of spacer grids when producing CRUD estimates.

Since CRUD growth is dominated by threshold physics
it is important to characterize the tails of the
frequency-binned-temperature and TKE distributions about the mean.  In other
words, hot and cold spots present downstream of spacer grids must be accurately
resolved by the Hi2Low model in order to predict the correct boron
precipitation and maximum CRUD thickness.
It is challenging to faithfully
preserve the peaks and valleys in rod surface temperature and TKE distribution
by traditional interpolation techniques since such a model must guard against
smearing out the sharp peaks present in the spatial distributions.  Developing
a predictive CRUD model entails overcoming this issue. The proposed method
addresses this problem by predicting the behavior of the tails of the frequency
binned FOI rather than the shape of the spatial distributions of the FOI on
the rod surface.  Over a coarse subchannel-scale patch on the rod surface, the
problem is reduced from predicting a 2D spatial field at CFD-like resolutions to
predicting the shape of multiple, independent 1D marginal density functions.
