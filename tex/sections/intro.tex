CRUD 

A literature review of Hi2Low modeling approaches to CFD-informed subchannel
problems is given in \autoref{chap:lit}.  Several advances in CRUD modeling are
also discused in \autoref{chap:lit} including a CASL developed high fidelity
CFD based CRUD tool currently under development.
Techniques for surrogate model construction are also reviewed.  Examples of
surrogate models developed using regression kriging are provided.  These
techniques are highly relevent to the current Hi2Low problem as these models
seek to describe the behavior of random fields given a sparse set of inputs.
Additionally a review
of the gradient boosted algorithm and applications to surrogate model construction
are provided.

In \autoref{chap:theory}, the overarching Hi2Low pipeline is introduced.  

Work that has been performed in support of this Hi2Low strategy is reviewed in
\autoref{chap:work}.  The development of a CFD data extraction tool with
additional post processing capabilities is provided.  This tool enables the
comparison of CFD result with CTF results.  Comparisons between CFD and CTF
subchannel results are presented to identify key flow regimes in which the
coarse subchannel based approach produces erroneous CRUD growth rates.

A demonstration of the copula based methodology to predict CRUD grow rates on a
single rod provided a single TH state point is also given in
\autoref{chap:work}.

Finally, future work is discussed in \autoref{chap:fw} with a detailed plan of
action laid out to clearly define project goals.

\section{Benefit and Novelty}

Accurately predicting CRUD induced power shift (CIPS) requires accurate CRUD
thickness and boron deposition estimates in all regions of the core.  Core wide
phenomena are out of scope for high fidelity CFD-Neutronic-CRUD coupled
simulations.  Instead, we rely on CASL's CTF-MAMBA1D coupling capability to
predict CIPS throughout a cycle.  CTF is not capable of simulating the detailed
flow patterns downstream the spacer grids.  Details in the flow field
downstream of spacer grids have important consequences for CRUD growth and
erosion, namely the local depression in surface temperatures and locally
increased shear stresses.  CRUD growth is extremely sensitive to the surface
temperature around the saturation point.  Therefore, growing crud at the bulk
average predicted TH conditions given by CTF in a coarse axial segment may not
yield the correct amount of CRUD growth in that region.

To this end a novel method based on the marriage of copula and gradient boosting techniques
is proposed.  The spatial distributions are not resolved on each CTF patch, rather
multivariate probability density of the surface temperature, TKE, and bundary heat flux
are predicted by the model.
The resultant model produces CRUD estimates that account for the presence of
hot and cold spots on the rod surface induced by turbulent flow downstream of spacer grids.

Copula are used to resolve the dependence structure underpinning correlated distributions
of rod surface temperature and TKE.  In this work, Sklar's theom is leveraged
to decompose multivariate probability densities into into indipendent copula
and marginal models. This technique is inspired from previous work conducted by
() in generating spatially dependent covarience models using vine-copula (Delft).

The free parameters within the copula model are prediced using a combination of
supervised regression and classification machine learning techniques. A
gradient boosted regression tree (GBRT) methodology is proposed to provide the
required relationships between the copula model parameters and the
explanatory variables.

\section{CASL Challenge Problems}

CASL selected several problems identified by industry partners as critical
inadequately understood engineering scale phenomena which would provide financial and
safety benefits to the nuclear power industry if resolved.  An list of the CASL
challenge problems is provided in [ref].

The Virtual Environment for Reactor Applications (VERA) is a key component of
CASL's technical portfolio.  This meta-package integrates a variety of physics
packages and multiphysics coupling options to form a robust reactor simulation
capability.  For multi-cycle depletion computations, VERA relies upon MPACT, a
2D-1D method of characteristics neutronics package, coupled with a subchannel
thermal hydraulics code COBRA-TF (CTF).  An integrated CRUD modeling capability
is provided by MAMBA to address the CRUD induced power shift challenge problem.

To reduce computation times, the subchannel TH code discretizes the reactor
domain into large, centimeter scale finite volumes.  A second order upwinding
scheme is employed along with the SIMPLE solution algorithm to resolve pressure
velocity coupling.  This coarse discretization scheme means that sub-centimeter scale
thermal hydraulic effects of the spacer grids on CRUD are averaged over
large regions on the fuel rods' surfaces.  Though small scale phenomena are not
explicitly modeled, they are approximately accounted for in a variety of empirically derived
closure relations.  In effect, a single constant estimate for the mean thermal
hydraulic conditions (Temperature, boundary heat flux, and wall shear stress)
are obtained in each finite volume.

Previous Hi2Low TH focused work in CASL focused on utilizing experimental or CFD
datasets to improve closure models in CTF.  These studies leverage a multitude of
dimensionality reduction and regression techniques
to fit a parametric model to the accepted gold-standard empirical data.  This approach
is adequate for correcting biases in the bulk-average behavior of the flow (due to
the previously neglected physics).  Examples of such Hi2Low models are given in
\autoref{chap:lit}.

The traditional approach must be slightly modified to accommodate the CILC and
CIPS challenge problems.  Here arises the need to retain not only the effect of
fine-scale physics on the bulk, but also to predict if certain
temperature or TKE \emph{thresholds} are exceeded in a given (CTF coarse)
volume.  Furthermore, for a complete treatment of thermal hydraulic impacts on
CRUD growth, the scale-bridging model must describe the \emph{frequency
distribution} of extreem TH events above a given threshold.

\subsection{CIPS \& CILC Challenge Problems}

Outline of CIPS and CILC challenge problem.  Role of CFD informed subchannel.

The CRUD induced power shift phenomina and CRUD induced local corrosion
challenge problems.  Currently avalible tools are unable to properly account
for spacer grid effects on the errosion, in subchannel-scale models.

To redemy this issue, a scale-bridging model is proposed in this work.  A
Hi2Low model, in the current context 



