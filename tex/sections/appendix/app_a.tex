%-----------------------------------------------------------------------------%
\section{Copula}

Intro to copula theory.

\begin{equation}
P(x, y) = P(x) \cdot P(y | x)
\label{joint_voi}
\end{equation}

This is the product rule of probability.  The comma denotes the conjunction
``and'' and the bar, $|$, is read ``given''.

Marginal distributions are defined below. Dependence information between $x$
and $y$ is not captured in the margins.

\begin{align*}
    P(x) &= \int P(y) P(x|y) dy \\ P(y) &= \int P(x) P(y|x) dx
\end{align*}

The process of decomposing a multivariate distribution into univariate marginal
distributions and an object which describes their conditional dependence was
formalized by Sklar (1959).  Shown in equation \ref{sklar1},  Sklar's Theorem
defines some \emph{copula}, $C$, that describes the dependence structure
underpinning multiple correlated variables.

Let $F_1\ \&\ F_2$ represent unbounded cumulative density functions (CDFs):

\begin{align*} 
F_1 &= P[X < x] = \int_{-\infty}^x P(x)dx \\
F_2 &= P[Y < y] = \int_{-\infty}^y P(y)dy
\end{align*}

For brevity, two variables $u, v$ that represent samples from their CDFs can be
introduced as follows:

\begin{align*} u &= F_1(x) \\ v &= F_2(y) \\ u,v &\in
[0, 1]
\end{align*}

\begin{equation}
H(x,y) = C(F_1(x), F_2(y))
\label{sklar1}
\end{equation}

The joint cumulative probability distribution, $H(x,y)$, can be computed from
constituent marginalized univariate distributions and the \emph{copula}, $C$.
The copula is given by equation \ref{sklar2}:

\begin{equation}
C(u, v) = F(F_1^{-1}(u), F_2^{-1}(v))
\label{sklar2}
\end{equation}

The power of Sklar's theorem resides in the ability to construct
models for the margins separately from a model of the dependence structure.
When combined according to Sklar's theorem, the margins and the copula completely
specify any multivariate probability density function.

Compared to rudimentary approaches based on an assumed linear dependence structure,
a copula based approach can treat skewed dependence structures in which the
strength of dependence (as measured via a local correlation coefficient) is allowed
to vary depending on location in the parameter space.
It can be shown that by restricting the dependence model to only consider elliptic Gaussian
copula, one achieves a multiple linear dependence model [ref] in which a covariance matrix
completely defines the dependence structure.

\section*{Copula Families}

\subsection*{Elliptic}

Provided an Elliptic copula lines of constant probability density form ellipses.

This family of copula are unsuitable for capturing skewed dependence structures.

\subsection*{Archimedian}

Archimedian copula are defined by the relationship.

Where $\psi$ is a monotonically decreasing and convex function.

Archimedian copula are not necessarily symmetric, and therefore allow one to capture
skewed dependence structures.

%-----------------------------------------------------------------------------%
\section{Vine Copula} 

Intro to vine copula theory.

A vine copula is a nested tree-like structure consisting of bivariate copula
edges and univariate marginal nodes which represents a decomposition of a high
($d>=3$) dimensional cumulative probability density function.  Traditionally,
three classes of vine copula are discussed:

\begin{enumerate}
    \item \textbf{Canonical vines} (C-vines):  In each tree in the vine, all edges share a common root node.
    \item \textbf{D-vines}:  All nodes have exactly two connecting edges.
    \item \textbf{Regular vines} (R-vines):  Super set of all possible vine structures.
\end{enumerate}

We will only consider canonical vines in this work.
