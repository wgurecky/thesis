\section{Hi2Low Literature Overview}

The practive of utilizing CFD and experimental data to inform parameterized scale bridging models is a common theme in previous Hi2Low studies.  Though the current Hi2Low strategy is targeted specifically to capture spacer grid effects on CRUD growth, it is in a similar vein to these past Hi2Low efforts.  The following section supplies examples of past Hi2Low work conducted by CASL.

Previous Hi2Low work focused on accounting for spacer grid effects on the heat transfer coefficient and on inter channel mixing.  This work targeted departure from nucleate boiling (DNB) predictions.  The strategy for these works involved adding additional terms in the momentum balance implemented in CTF that approximately capture fine scale flow effects introduced by the presence of the spacer grid arrays. In much the same way grid loss coefficients are determined, non mechanistic empirical models were fit to experimental data supplied by [].  The empirical mixing relationships inform the coefficients in the new momentum source terms.

A similar strategy was employed to develop empirical models which capture grid-enchanced heat transfer in CTF.  A function parameterized by vane angle is used as a multiplier map to adjust the heat transfer coefficent to reflect the presence of the spacer grids.  Due to CTF's coarse spatial discritization, the code cannot explicitly resolve the small scale voticies responsible for the increased heat transfer on the rod surface without these empirical enchancement functions. 

A Hi2Low stratagy proposed by Salko et. al [] specifically targets capturing spacer grid effects on CRUD growth  relies on mapping CFD born spatial distributions of the heat transfer coefficent and surface shear stress onto the coarse CTF grid.    To test the pocedure, a suite of 15 unique cases where devised spanning typical inlet flow rates and power levels.  Supplying consistent boundary conditions to each code preoduced matching pairs of CFD and CTF results which could be used for the mapping procedure.

In order to obey energy conservation on the target grid,  an iterative stratagey was used to converge on the rescaled temperature field on target CRUD grid since the boundary heat flux is non-trivially related to the surface temperature.  Additional details are provided in Appendix B \ref{chap:app_b}.

Initial results are indicate the method performs best at lower power levels when subcooled boiling does not occure.
Cases that have substantial amount of subcooled boiling tend to produce large difference between the CFD and CTF predictions.  

Though mapping spatial distributions is straight forward in principal it is difficult to account for all pin configurations in the core since a matching CFD solution must be avalible in order to perform a map.  If corresponding CFD data is unavailabe for some portions of the thermal hydraulic envelope present in the core at a given state point an interpolation procedure is required.  The effectiveness the interpolation procedure depends on the sensitivity of the spatial heat transfer and shear stress distributions on the rod surface to changes in other auxillery variables such as local power output.  If the spatial distributions are ill behaved as a function of rod orientaion and locally-averaged thermal hydraulic conditions in the core it can become prohibtively expensive to pre-compute the necissary CFD data required to build an interpolant with an acceptible level of uncertainty in the predicted quantites.  
A key metric of success can be assertained from this study:  The Hi2Low technique which utilizes the fewest number of CFD runs to drive down uncertainties in the prediced quantites of interest (those which significantly impact CRUD growth) is a suppirior model.


\section{CFD vs CTF Comparisons}


\section{Kriging Literature Review}

Kriging is popular technique used to construct surrogate models.  As discused in section \ref{chap:fw}, a regression kriging model is currently under consideration as a means interpolate high fidelity CFD-informed spatial data. The following is an overview of kriging applied to engineering problems.

The technique of Kriging was pioneered by Danie G. Krige in his master's thesis in an effort to model the spatial distribution of gold ore concentrations given sparse, uncertain ore concentration estimates [ref]. Following kriging's humble beginings, the technique has propogated through the geostatistics community. Typically, Kriging is applied when estimates of a random field are desired in between sparse training data samples [ref].  The technique has since been further explored by the machine learning community under the guise of gaussian process regression.
Kriging is related to Gaussian process regression since the underlying goal of both approaches is to model the spatial autocorrelation of a random field.  
