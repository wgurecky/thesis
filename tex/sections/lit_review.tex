\section{Hi2Low Literature Overview}

Utilizing CFD and experimental data to improve subchannel flow models is not a new practice.  Though the current Hi2Low strategy is targeted specifically at captureing spacer grid effects on CRUD growth, it is in a similar vein to these past CFD informed subchannel efforts.  The following examples of previously conducted Hi2Low work focused on accounting for spacer grid effects on the heat transfer coefficient and on inter channel mixing. 

Over time the CTF code has accumulated many incremental improvements.  A recent improvement targeted the treatment spacer grid induced cross-flow was devised by M. Avramova (2007) \cite{avramova2007}. 
The approach put forth in this work involved tuning coefficients of auxiliary terms included in the momentum and energy balances implemented in CTF.  These additional terms approximately capture fine scale turbulent mixing effects introduced by the presence of the spacer grid arrays on the bulk flow.  Semi-mechanistic models for the (mixing) coefficients were fit to data supplied by CFD computations.  Substantial improvement in bundle outlet temperature distributions from the base model was demonstrated with no additional computational overhead.  The nature of this work required large changes to the CTF source code as the momentum and energy balance equations were modified to accommodate the new spacer grid effects.   This work is representative of the state-of-the-art in utilizing CFD predictions to improve subchannel models, however the methods employed here were not focused on retaining fine scale detail from the CFD predictions as is required in the current work.

A similar strategy has been employed to capture grid-enhanced heat transfer in subchannel codes.   A function parametrized by vane angle is used to adjust the heat transfer coefficent to reflect the presence of the spacer grids.  An empirical relationship between the Nusselt number ratio and the vane angle, $\phi$, blockage ratio $\epsilon$, dimensionless distance from the grid, $x/D$, and fraction of flow area impeded by the vanes, $A$, was developed by S. C. Yao et. al (1982) [ref].
\begin{equation}
\frac{Nu}{Nu_0}  = \left[ 1 + 5.55 \epsilon^2 e^{-0.13(x/D)}\right] + \left[ 1 + A^2\mathrm{tan}^2\phi e^{-0.034(x/D)} \right]
\end{equation}
Where the first term is accounts for the effect of grid flow restriction and the second term represents the contribution of vane induced swirl on the heat transfer.

\begin{figure}[!htbp]
\centering
\includegraphics[width=8cm]{images/grid_nu_eff.png}
\caption{Localized heat transfer improvement due the the presence of a spacer grid \cite{salko17}.}
\label{fig:yao_nu}
\end{figure}

The relationship between the Nusselt number and heat transfer coefficient is given in Appendix B.
The Yao model is used in CTF to improve overall rod temperature and departure from nucleate boiling (DNB) predictions.  Due to CTF's coarse spatial discritization, the code cannot explicitly resolve the small scale vorticies responsible for the increased heat transfer on the rod surface without these empirical enhancement functions.   

A Hi2Low strategy proposed by Salko et. al (2017) specifically targets capturing spacer grid effects on CRUD growth \cite{salko17}.  This work is of particular importance because it represents a direct alternative to the proposed statistically based Hi2Low model for the purposes of enhancing the accuracy of CRUD predictions at the core-wide scale.  The model relies on mapping CFD born spatial distributions of the heat transfer coefficient (HTC) and surface TKE onto an intermediate grid upon which CRUD is grown.  The intermediate grid sits in between the CFD and CTF meshes and it's resolution is tunable.  An illustration of the intermediate grid setup is given in figure \ref{fig:cfd2ctf_map}.  In the figure, the turbulent kinetic energy is mapped on to the intermediate grid and is normalized by the CTF TKE prediction before passing the reconstructed TKE distribution to the CRUD simulator as a boundary condition. 

\begin{figure}[!htbp]
\centering
\includegraphics[width=8cm]{images/cfd_ctf_multi_grid.png}
\caption{Mapping from the CFD grid to the reconstruction (CRUD) grid.  Reproduced from \cite{salko17}.}
\label{fig:cfd2ctf_map}
\end{figure}

In order to obey energy conservation on the target grid,  an iterative method was used to converge the temperature field on target CRUD grid since the boundary heat flux is non-trivially related to the surface temperature.  Additional details on the relationship between surface temperature and the boundary heat flux are provided in Appendix B \ref{chap:app_b}.

After mapping, rescaling and computation of the new temperature field are complete, the FOI are passed to the CRUD simulation for all patches on the reconstruction grid.  A comparison of the CRUD results generated before and after the application of the spatial mapping procedure is provided in figure \ref{fig:crud_pre_map} and \ref{fig:crud_post_map}.

\begin{figure}[!htbp]
\centering
\begin{minipage}{.45\textwidth}
  %
  \includegraphics[width=7cm]{images/ctf_crud_orig.png}
\caption{CRUD growth on the rod \\ surface prior to HTC \\ and TKE field remapping.}
\label{fig:crud_pre_map}
\end{minipage}%
\begin{minipage}{.45\textwidth}
  %
  \includegraphics[width=7cm]{images/ctf_crud_reconstructed.png}
\caption{CRUD growth on the rod \\ surface post HTC and \\ TKE field remapping.}
\label{fig:crud_post_map}
\end{minipage}
\end{figure}

Initial results indicate that the method performs best at lower power levels when subcooled boiling does not occur.  Cases that have substantial amount of subcooled boiling tend to produce large differences between the CFD and CTF predictions which in turn reduces the predictive power gained by applying the rescaled HTC maps - essentially a garbage-in-garbage-out scinario that is of no fault to the remapping procedure.  An investigation into cause of these differences could be a subject of future work conducted in support of this Hi2Low approach.  

Though mapping and normalizing spatial distributions is straight forward, it is difficult to account for all pin configurations in the core since matching CFD and CTF solutions must be available in order to perform a simple map.  If corresponding CFD data is unavailable for some portions of the thermal hydraulic conditions present in the core an interpolation procedure is required.  The effectiveness the interpolation procedure depends on the sensitivity of the spatial heat transfer and TKE distributions on the rod surface to changes in rod bundle geometry and other auxiliary variables such as local power output.  If the spatial distributions are ill behaved as a function of rod orientation and locally-averaged thermal hydraulic conditions in the core it could become prohibitively expensive to pre-compute the necessary CFD data required to build an interpolate with an acceptable level of uncertainty in the predicted quantities.  In order to quantify this uncertainty, future development of this technique will focus on multi-rod, multi-state point cases.
A key metric of success can be ascertained from this study:  The Hi2Low technique which utilizes the fewest number of CFD runs to drive down uncertainties in the predicted quantities of interest - those which significantly impact CRUD growth - is a superior model.

\section{CRUD Modeling}

Previous studies have explored CRUD simulation at both the CFD scale and at the macroscopic core scale.  A CFD-CRUD 
Figure () depecits results from a coupled MAMBA1D and STARCCM+ calculation.    

An initial CIPS prediction capability has been demonstrated by CASL [ref B. Collins et. al PHYSOR paper].  This study revealed weaknesses in the predictive capability of coupled CTF-MAMBA simulations [refs] (cite MAMBA tuning and boiling issues).

TODO: Cicada description.

\begin{figure}[!htbp]
\centering
\includegraphics[width=16cm]{images/combo_180x.png}
\caption{CRUD thickness [m], Temperature [K], and TKE [J/kg] output from a coupled CFD-CRUD simulation.}
\label{fig:cfd2ctf_map}
\end{figure}

\section{Copula}

Formally introduced by Sklar in 1959 \cite{Sklar1959}, a copula is a function which relates marginal probability distributions to a multidimensional joint distribution.  Copula provide a flexible alternative to multidimensional Gaussian based models.  Copula are utilized in this work because of their ability to capture non-Gaussian dependence structure between two or more correlated random variables, for instance temperature and the TKE at a given point on a rod's surface.  Furthermore, Sklar's theorem is used in this work in order to decompose joint distributions into a product of uni-variate marginal distributions and a copula function.  

Copula have been leveraged in the finance industry to
predict correlated extreme value risks in credit portfolios
\cite{Geidosch2016}.  Copula have received additional attention in financial and mathematics communities since 
simpler Gaussian based dependence modeling techniques were revealed to make erroneous expected CDO portfolio loss predictions under the market conditions present in the financial crisis of
2008-2009 \cite{MacKenzie2013}, \cite{Li2000}.  Despite the widespread adoption of copula models in financial risk assesment community, only recently have copula been applied to flood risk
models \cite{Dupuis2007}, \cite{Ganguli2012}, and reliability analysis in nuclear plants
\cite{Kelly2007}.  The delayed adoption of the copula in the
engineering realm is speculated to be due to a substantial increase in computational
complexity required to construct and evaluate high dimensional copula over
incumbent Bayesian network and multidimensional Gaussian based methods.  
Though higher dimensional copula do pose significant challenges in fitting and sampling, it is straightforward to fit low dimensional copula models to empirical data
using a maximum likelihood or Markov Chain Monte Carlo approach \cite{Jouini1996}.
A method for drawing correlated samples from a copula is provided in Appendix A.

\section{Gradient Boosting}

Gradient boosting is a supervised machine learning method for classification and regression problems.
The first generalized gradient boosting algorithms were introduced by J. H. Friedman (1999) \cite{friedman2001}.  In his work, Friedman connected the technique to optimization of a cost function by gradient descent.  Prior to Fiedman's work, boosting algorithms such as AdaBoost were developed but were not easily modified to accommodate arbitrary loss functions.  Boosting is loosely defined to be the process of constructing a highly accurate predictor (regression model) from an ensemble of weak predictors in a stage wise manner.
Nowadays, gradient boosting is ubiquitous in the machine learning community.  Gradient boosting has found use in applications ranging from ecological modeling \cite{death2007} to webpage ranking problems \cite{Tyree2011}, \cite{chapelle2011}.  Particularly in the regression setting, gradient boosted trees are highly competitive with random forests and support vector machines \cite{moisen2006}.


\section{Kriging}

The term kriging was coined by Danie G. Krige in his master's thesis in an effort to model the spatial distribution of gold ore concentrations given sparse, uncertain estimates (1951) \cite{krige51}, \cite{Krige51a}. Following kriging's introduction, the technique has propagated through the geostatistics community.  Kriging is applicable when estimates of the mean and variance of a random field are desired in between sparse training data samples.  The technique has been further explored by the machine learning community under the guise of Gaussian process regression \cite{Williams96}.  Kriging is related to Gaussian process regression since the underlying goal of both approaches is to model the spatial autocorrelation of a random field.  An example of application of kriging to a soil quality data set is left to Hengl et. al \cite{Hengl07}.

Unfortunately the Hi2Low problem under consideration does not perfectly align with the typical applications of the technique since the spatial field data is extremely well resolved by CFD on the rod surface but is very sparse in other auxiliary dimensions, such as local core TH conditions.  Usually, kriging is applied to the opposite case where the data is known at a sparse spatial resolution but auxiliary predictors are known at a higher resolution.  Despite this, as discussed in section \ref{chap:fw}, a regression kriging model is currently being investigated as a means interpolate between known samples of CFD fields.

