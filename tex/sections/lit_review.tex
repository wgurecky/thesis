\section{Hi2Low Literature Overview}

The practice of utilizing CFD and experimental data to improve subchannel flow models is not a new practice.  Though the current Hi2Low strategy is targeted specifically to capture spacer grid effects on CRUD growth, it is in a similar vein to these past CFD informed subchannel efforts.  The following examples of previously conducted Hi2Low work focused on accounting for spacer grid effects on the heat transfer coefficient and on inter channel mixing. 

The CTF subchannel code has a long history of incremental improvements.  An improvement targeted the treatment spacer grid induced cross-flow was devised as part of a PhD dissertation by M. Avramova.
The approach put forth in this work involved tuning coefficients of auxillerary terms included in the momentum and energy balances implemented in CTF.  These additional terms approximately capture fine scale turbulent mixing effects introduced by the presence of the spacer grid arrays on the bulk flow.  Semi-mechanistic models for the (mixing) coefficients were fit to data supplied by CFD computations.  Substantial improvment from the base model was demonstrated with no additional computational overhead.  The nature of this work required large changes to the CTF source code as the momentum and energy balance equations were modified to accomidate the new spacer grid effects.  

A similar strategy was employed to develop models which capture grid-enchanced heat transfer in CTF.   A function parameterized by vane angle is used as a multiplier map to adjust the heat transfer coefficent to reflect the presence of the spacer grids.  This YHL model is used in CTF to improve departure from nucleate boiling (DNB) predictions.  Due to CTF's coarse spatial discritization, the code cannot explicitly resolve the small scale voticies responsible for the increased heat transfer on the rod surface without these empirical enchancement functions.   

A recently developed Hi2Low stratagy proposed by Salko et. al [] specifically targets capturing spacer grid effects on CRUD growth.  This work is of particular importance because it represents a direct alternative to the proposed statistically based Hi2Low model for the purposes of enhancing the accuracy of CRUD predictions at the core-wide scale.  The model relies on mapping CFD born spatial distributions of the heat transfer coefficent (HTC) and surface TKE onto an intermediate grid upon which CRUD is grown.  The intermediate grid sits in between the CFD and CTF meshes and it's resolution is tunable.  An illustration of the intermideate grid setup is given in figure ().  In the figure, the tubulent kinetic energy is mapped on to the intermediate grid and is normalized by the CTF TKE prediction before passing the reconstructed TKE distribution to the CRUD simulator as a boundary condition. 

In order to obey energy conservation on the target grid,  an iterative stratagey was used to converge on the rescaled temperature field on target CRUD grid since the boundary heat flux is non-trivially related to the surface temperature.  The iterative strategy is summarized in figure (). 
Additional details on the relationship between surface temperature and the boundary heat flux are provided in Appendix B \ref{chap:app_b}.

To test the pocedure, a suite of 15 unique cases where devised spanning typical inlet flow rates and power distributions.  Matching pairs of CFD and CTF results were produced for each case.  The matched results were then used used for the mapping procedure.

Initial results indicate that the method performs best at lower power levels when subcooled boiling does not occure.  Cases that have substantial amount of subcooled boiling tend to produce large differences between the CFD and CTF predictions.  An investigation into cause of these differences could be a subject of future work conducted in support of this Hi2Low approach.  

Though mapping spatial distributions is straight forward in principal it is difficult to account for all pin configurations in the core since matching CFD and CTF solutions must be avalible in order to perform a simple map.  If corresponding CFD data is unavailabe for some portions of the thermal hydraulic conditions present in the core an interpolation procedure is required.  The effectiveness the interpolation procedure depends on the sensitivity of the spatial heat transfer and shear stress distributions on the rod surface to changes in rod bundle geometry and other auxillery variables such as local power output.  If the spatial distributions are ill behaved as a function of rod orientaion and locally-averaged thermal hydraulic conditions in the core it could become prohibtively expensive to pre-compute the necissary CFD data required to build an interpolant with an acceptible level of uncertainty in the predicted quantites.  In order to quantify this uncertainty, future development of this technique will focus on multi-rod, multi-state point cases.
A key metric of success can be assertained from this study:  The Hi2Low technique which utilizes the fewest number of CFD runs to drive down uncertainties in the prediced quantites of interest (those which significantly impact CRUD growth) is a suppirior model.


\section{CFD vs CTF Comparisons}


\section{Kriging Literature Review}

Kriging is popular technique used to construct surrogate models.  As discused in section \ref{chap:fw}, a regression kriging model is currently under consideration as a means interpolate high fidelity CFD-informed spatial data. The following is an overview of kriging applied to engineering problems.

The technique of Kriging was pioneered by Danie G. Krige in his master's thesis in an effort to model the spatial distribution of gold ore concentrations given sparse, uncertain ore concentration estimates \cite{krige51}. Following kriging's humble beginings, the technique has propogated through the geostatistics community. Typically, Kriging is applied when estimates of a random field are desired in between sparse training data samples [ref].  The technique has since been further explored by the machine learning community under the guise of gaussian process regression.
Kriging is related to Gaussian process regression since the underlying goal of both approaches is to model the spatial autocorrelation of a random field.  
