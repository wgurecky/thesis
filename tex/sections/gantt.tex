%-----------------------------------------------------------------------------%
\pagebreak
\section*{Project Schedule}

\begin{enumerate}
\item \textbf{Literature review.}
\item \textbf{Develop coupled CFD-CRUD tool for high fidelity CRUD (CIPS/CILC) prediction.}  
          This is the platform used to generate datasets that are
          required to inform a scale-bridging model.
    \begin{enumerate}
        \item Create CFD data extraction and post processing utilities.
        \item Develop STAR-CCM+ plugin to extract cladding surface and volumetric TH quantities from a CFD computation.
        \item Develop tool to compute the required volume and surface integrals to
              distill finely resolved CFD datasets into subchannel-like results.
    \end{enumerate}
\item \textbf{Demonstrate differences between CFD and Subchannel predictions.} (NO CRUD)
    \begin{enumerate}
        \item Compute differences between volume/surface averaged CFD quantities and subchannel predictions.
        \item Identify deficiencies in subchannel predictions wrt. CRUD growth and CILC.
    \end{enumerate}
\item \textbf{Investigate CRUD model sensitives.}
    \begin{enumerate}
        \item Identify CRUD (MAMBA) sensitivities to TH boundary conditions
        \item Produce correlation coefficients and scatter plot depictions of the relationship(s) between input
              and output of the CRUD model.
        \item Identify TH conditions under which CFD scale CRUD predictions greatly diverge from subchannel-CRUD results.
    \end{enumerate}
\item \textbf{Preliminary methods implementation and demonstration} (Dec-Feb)
    \begin{enumerate}
        \item De-trend pointwise CFD datasets (Moving average based) \& compute residual distributions.
        \item Construct description of the spatially dependent co-variance between (residual) wall temperatures, surface shear stress, and
              boundary heat flux downstream of a grid span.
        \begin{enumerate}
            \item Develop flexible dependence modeling package capable of capturing skewed covariance behavior (Vine copula).
            \item Develop Kriging model to capture spatial dependence of covariance (may not be necessary).
        \end{enumerate}
        \item Construct a non-parametric regression model (Gradient boosted tree model [GBM]):
        \begin{enumerate}
            \item Regress covariance model parameters, $\bm{\hat\theta}$, on local-average CTF provided inputs.
            \item Evaluate regression: $[CTF\ Data] \rightarrow {P(T>t,...|\bm{\hat\theta})}$
        \end{enumerate}
        \item Compute areas of input space with highest regression model sensitivity (develop capability to super sample these regions).
        \item Demonstrate ability to propagate uncertainty through regression model.
        \item Demonstrate ability to rank inputs by \% output variance explained via PCA.  Also demonstrate ability to
              perform dimensionality reduction via PCA.
          \item Perform CFD informed subchannel based CRUD prediction for a \emph{single pin}.
    \end{enumerate}
\item \textbf{Write proposal document} (Dec-Feb)
\item \textbf{Methods analysis and improvement} (June)
    \begin{enumerate}
        \item Perform Large scale CFD runs to inform finalized scale-bridging model
        \item Large scale application of PCA: Determine best set of inputs to capture maximum output variance.
        \item Tune regression algorithm's (GBM) hyperparameters
        \item Validate CFD informed subchannel CRUD results against plant data (IF AVAILABLE).
    \end{enumerate}
\item \textbf{Final model construction and demonstration} (July-Aug)
    \begin{enumerate}
        \item Complete initial implementation of CFD informed subchannel model in VERA-CS.
        \item Demonstrate CFD informed subchannel model on a full-height, assembly scale problem.
        \item Assess CILC/CIPS prediction improvements vs. base model (no CFD informed TH).
        \item Validate CIPS predictions against experimental plant data.
    \end{enumerate}
\item \textbf{Write Dissertation} (July-Aug)
\end{enumerate}

\pagebreak

\begin{sidewaysfigure}
%-----------------------------------------------------------------------------%
\begin{ganttchart}[
        inline,
        x unit=1.5cm,
        y unit title =0.8cm,
        y unit chart=0.75cm,
        hgrid,
        vgrid,
        time slot format=isodate,
        compress calendar
    ]{2016-06-01}{2017-08-01}
    \gantttitlecalendar*[compress calendar, time slot format=isodate]{2016-06-01}{2017-08-01}{year, month=shortname} \\
    \gantttitlelist{1,...,15}{1} \\
%%%%%%% Phase 1
\ganttgroup{Proposal Epic}{2016-06-01}{2017-02-28} \\
\ganttbar{Lit. Review}{2016-06-01}{2016-11-01} \\
\ganttbar{CFD Post Proc. Dev.}{2016-06-01}{2016-08-20} \\
\ganttlinkedbar{CTF vs CFD}{2016-06-20}{2016-07-30} \ganttnewline
\ganttlinkedbar{MAMBA Sensitivity}{2016-07-01}{2016-08-30} \ganttnewline
\ganttbar{Covariance Model Construction}{2016-10-01}{2017-01-10}  \\
\ganttlinkedbar{Regression Model Dev.}{2016-12-01}{2017-02-28}  \\
\ganttbar{Write Proposal}{2016-12-01}{2017-02-28}
\ganttmilestone{Proposal}{2017-02-28} \ganttnewline
%%%%%%% Phase 2
\ganttgroup{Dissertation Epic}{2017-03-1}{2017-08-30} \\
%\ganttlinkedbar{Task 2}{3}{7} \ganttnewline
\ganttbar{Methods Refinement}{2017-03-01}{2017-05-01} \ganttnewline
\ganttlinkedbar{Final CFD }{2017-06-01}{2017-07-01} \ganttnewline
\ganttlinkedbar{Merge in VERA-CS}{2017-06-01}{2017-07-01} \ganttnewline
\ganttbar{Write Dissertation}{2017-06-01}{2017-08-01}
\ganttmilestone{Dissertation}{2017-08-01} \ganttnewline
%\ganttlink{elem2}{elem3}
%%%%%%%% Extra Task Linkages
%\ganttlink{elem6}{elem7}
\end{ganttchart}
%-----------------------------------------------------------------------------%
\end{sidewaysfigure}
\pagebreak
