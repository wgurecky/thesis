\section{Regression Kriging}

An alternative procedure based on regression kriging (RK) is under consideration for producing esimates and uncertainties of the desired FOI.
The available literature on the topic is substantial.  Regression kriging is typically introduced in the context of building
a surrogate model of a spatial field given a limited number of known sample points and a some predictive feature set.
The steps when constructing a RK model mirrors the currently proposed approach. \\

First the target FOI is decomposed into a deterministic and stocastic component, as in equation ().

Let the spatial coordiantes,$\mathbf{s} = {z, \theta}$,
where $z$ is the axial position and $\theta$ is the azimuthal coordinate on the rod surface. Let the FOI be denoted as $Z$.
\begin{equation}
Z(s) = \mu(s) + \varepsilon(s)
\end{equation}

The regression kriging method does not directly rely on the
CTF solutin to provide the deterministic portion of the solution. Instead
a seperate OLS procedure performed on the sampled CFD data to estimate the deterministic portion of the FOI.
In this case, the CTF results enter the kriging model as additional covariates in the construction of the regression model.

let $\mathbf{q}$ represent the predictive feature array. In the case of regression kriging, the predictied temperature,
TKE, and bounday heat flux from CTF are included as covariates.
In this example we examine case of kriging the surface temperature field.
\begin{equation}
$\mathbf{q}$ = {TKE, q'', T}_{CTF}
\end{equation}

In matrix form, the RK model is given by:

\begin{equation}
\hat z_\mathtt{RK}(\mathbf{s}_0 ) = \mathbf{q}_\mathbf{0}^\mathbf{T} \cdot \mathbf{\hat \beta}_\mathtt{GLS} + \mathbf{\lambda }_\mathbf{0}^\mathbf{T} \cdot (\mathbf{z}
- \mathbf{q} \cdot \mathbf{\hat \beta }_\mathtt{GLS} )
\end{equation}

The OLS regression coefficients are given by:


Shown in equation (), the kriging weights at the sample location, $\lambda_0$ are esimated by simple kriging:



\begin{figure}[hbtp]
\centering
\includegraphics[scale=.55]{images/rk_example.png}
\caption{Regression kriging example.}
\label{fit:rk}
\end{figure}

