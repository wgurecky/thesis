\section{Regression Kriging}

An alternative procedure based on regression kriging (RK) is under consideration for producing esimates and uncertainties of the desired FOI.
The available literature on the topic is substantial however much of the available resources focuses on kriging geostatistical data.  Regression kriging is typically introduced in the context of building
a surrogate model of a spatial field given a limited number of known sample points and a some set of predictive features that are known at a resolution greater than or equal to the endogenous variable sample resolution.
The initial steps taken to construct a RK model are similar to the currently proposed approach.  The primary focus of an RK model is to capture spatial auto-correlations in the field data.   If the ability to predict the full 2D distributions of the FOI are desired, an RK model could yeild the requisite fine scale surface distributions with uncertainty estimates. \\

First the target FOI is decomposed into a deterministic and stochastic component, as in equation \ref{eq:decomp}.
Let the spatial coordiantes, $\mathbf{s} = \{z, \theta\}$,
where $z$ is the axial position and $\theta$ is the azimuthal coordinate on the rod surface. Let the FOI be denoted as $T$.
\begin{equation}
T(\mathbf s) = \mu(\mathbf s) + \varepsilon(\mathbf s)
\label{eq:decomp}
\end{equation}

First, the mean field behavior must be estimated by either a least squares fit to the sampled CFD data, or by using the CTF result directly as the mean predictor, as is done in the proposed method. 
Additionally, the CTF results enter the kriging model as auxillery predictive variables.  These variables serve as predictors of spatial auto-correlation.

The following example examines case of kriging the surface temperature field, $\hat T_\mathtt{RK}$.
Let $\mathbf{q}$ represent a predictive feature array comprised of the temperature,
TKE, and bounday heat flux from CTF.  Also included in this ficticious example, is a parameter $\Delta z$ which represents the axial distance to the nearest upstream spacer grid.

\begin{equation}
\mathbf{q} = \{TKE, q'', T, \Delta z \}_{CTF}
\end{equation}

In matrix form, the RK model evaluated at a point $\mathbf{s}_0 $ on the rod surface is given by equation \ref{eq:rk_eval} \cite{Hengl07}:

\begin{equation}
\hat T_\mathtt{RK}(\mathbf{s}_0 ) = \mathbf{q}_\mathbf{0}^\mathbf{T} \cdot \mathbf{\hat \beta}_\mathtt{OLS} + \mathbf{\lambda }_\mathbf{0}^\mathbf{T} \cdot (\mathbf T_{CFD}
- \mathbf{q} \cdot \mathbf{\hat \beta }_\mathtt{OLS} )
\label{eq:rk_eval}
\end{equation}
Where $\mathbf{q}_\mathbf{0}= \mathbf{q}(\mathbf s_0)$. The kriging weights at the sample location, $\mathbf{\lambda_0}$, are estimated by \ref{eq:simple_krige_weights} with the covariance function: $c(q_i, q_0)= \mathrm{Cov}(\mathbf e(q_i), \mathbf e(q_0))$ where $\mathbf e(\mathbf q) = (\mathbf T_{CFD} - \mathbf{q} \cdot \mathbf{\hat \beta }_\mathtt{OLS}$). 
\begin{equation}
\begin{pmatrix}\lambda_{0_1} \\ \vdots \\ \lambda_{0_n} \end{pmatrix}=
\begin{pmatrix}c(q_1,q_1) & \cdots & c(q_1,q_n) \\
\vdots & \ddots & \vdots  \\
c(q_n,q_1) & \cdots & c(q_n,q_n) 
\end{pmatrix}^{-1}
\begin{pmatrix}c(q_1,q_0) \\ \vdots \\ c(q_n,q_0) \end{pmatrix}
\label{eq:simple_krige_weights}
\end{equation}

The covariences must be estimated from the available CFD data.  

The residual vector, $\mathbf e$, naively includes all available CFD data points, but in practice a cut-off distance from the sample location, $\mathbf s_0$, can be specified to drasitcally reduce the amount of information required to construct and evaluate a covariance model.  This effectively reduces the length of the vectors $\mathbf{\lambda_0}$ and $\mathbf e$.

\begin{figure}[hbtp]
\centering
\includegraphics[scale=.3]{images/rk_example.png}
\caption{Regression kriging example [ref].}
\label{fit:rk}
\end{figure}

RK is not straight forward to apply to the current Hi2Low problem because the available field estimates are very densely populated in space - but are sparse in the feature space of local core conditions. This is a non-typical situation for the application of RK. Therefore, building a RK model of the surface fields is left as an avenue for future investigation.
The application of RK is further complicated because the proposed set of auxillery variables are not smoothly distributed in space.  Without the application of a smoothing pre-processor, these
fields jump in value when passing between CTF control volumes.

