%! TEX root = ../dissertation_gurecky.tex
\label{sec:ml_cfd}

For deployment as an in-line statistically based downscaling tool which sits between a subchannel code and a crud simulation code in a core simulator such as VERA the model is required to perform the hi2lo mapping for all pins in the core at any operating condition.  In other words, the model must be evaluable at any local core conditions typical of an operating PWR.  Since the training data set cannot contain all possible pin geometries, loading configurations and operating conditions due to computational expense, the model produces a prediction for the copula and marginal distribution parameters between known states.

One might envision a table-lookup approach where high fidelity CFD flow field maps are pre-computed and stored for a wide array of flow and power conditions.  A nearest neighbor interpolation scheme could then be applied to extract the best-matching CFD map provided some local core state by VERA.  This is not tractable since the number of CFD computations to build the data base would be prohibitively large.  Instead of storing spatial CFD hi2lo maps, CFD data is distilled into a set of statistics and tabulated as a function of local core state.

% It is appealing to transform the problem of hi2lo construction from performing spatial flow map prediction into one of summary statistics prediction where much less information is required adequately fill out the operating envelope.

In this chapter the hi2lo methodology introduced in this work is exercised against a small CFD data set derived from a 5x5 fuel assembly operating at realistic PWR conditions.  A leave-one-out cross validation strategy is used to assess the predictive performance of the model.


\section{CFD Data Source}
\label{sec:cfd_data_source}

For the generation of high fidelity CFD data sets the Westinghouse 5x5 test stand shown in figures \ref{fig:5x5topdown} and \ref{fig:5x5side} was used to prepare the CAD geometry.  The CFD mesh consisted of approximately 25 million cells and 1e5 surface elements per pin.  A matching CTF input deck for the 5x5 assembly was also constructed with 100 axial zones.  The CTF and CFD codes were then executed for a variety of flow conditions and power levels.

\begin{figure}[H]
    \centering
    \includegraphics[width=0.5\linewidth]{figs/5x5/5x5_top_down}
    \caption[Top down view of 5x5 pin Westinghouse facility.]{Top down view of 5x5 pin Westinghouse facility.  Assembly dimensions and pin powers redacted.}
    \label{fig:5x5topdown}
\end{figure}

\begin{figure}[H]
    \centering
    \includegraphics[width=0.5\linewidth]{figs/5x5/5x5_side}
    \caption[Side view of 5x5 pin Westinghouse facility.]{Side view of 5x5 pin Westinghouse facility.  Pin dimensions redacted.}
    \label{fig:5x5side}
\end{figure}


\subsection{Preprocessing}
\label{sec:preprocessing}

Pre-processing requires paired CFD and CTF results for a given pin generated with consistent boundary conditions between the codes.
First each of the CFD $\{T,k,q''\}$ fields are aggregated onto the CTF face centers.  In the aggregation procedure spatial information is discarded within each CTF patch.  The aggregated CFD data are associated with a unique CTF face and corresponding local core conditions which comprise the predictive variable set given in table \ref{tab:features}.  The aggregated CFD field distributions are subtracted from the mean CTF predictions in each CTF face and the resultant (CFD-CTF) residual distributions are sored in an HDF5 table along with predictive variables.

Copula fitting by the maximum likelihood method with AIC model selection is carried out on each CTF face.  Additionally, the empirical Kendall's $\tau$ rank correlation coefficient is computed from the raw CFD data on each CTF face. Figure \ref{fig:copula_predicted} shows the copula parameters estimated from the raw CFD data on each pin as a function of axial position for the first 4 pins in the 5x5 CFD model.  There is a marked change in behavior of the copula between the pins.  This was an unexpected find since the flow patterns were speculated to be reasonably consistent from pin to pin.  Also, the influence of spacer grids on the correlation coefficient between the temperature and TKE fields is visible.  Across spacer grids the correlation coefficient first sharply falls indicating a tighter coupling between the TKE and temperature surface fields as the flow necks down when entering a grid.  This is followed by a sharp change in Kendall's $\tau$ towards unity indicating the temperature and TKE surface fields become less correlated immediately following the mixing vanes.  This change in correlation behavior is posited to be due to turbulent mixing effects.  The computed copula parameters are also stored alongside the raw temperature, TKE, and boundary heat flux aggregated residual distribution data in the HDF5 table.

\begin{figure}[H]%
    \centering
    \subfloat[Pin 1]{{\includegraphics[width=0.45\linewidth]{figs/preproc/copula_params_pin_1} }}%
    \qquad
    \subfloat[Pin 2]{{\includegraphics[width=0.45\linewidth]{figs/preproc/copula_params_pin_2} }}%
    \qquad
        \subfloat[Pin 3]{{\includegraphics[width=0.45\linewidth]{figs/preproc/copula_params_pin_3} }}%
    \qquad
        \subfloat[Pin 4]{{\includegraphics[width=0.45\linewidth]{figs/preproc/copula_params_pin_4} }}%
    \qquad
    \caption[Best fitting copula to CFD data.]{Best fitting copula determined by AIC model selection as a function of axial rod position.}%
    \label{fig:copula_predicted}%
\end{figure}

After pre-processing, the HDF5 table includes a list of predictive scalar values, $\mathbf p$, which are shown in table \ref{tab:features}, and a list of associated response variables; the copula parameters and residual sample distribution for $\{T,k,q''\}$ on each CTF face.

\subsection{Feature Engineering}

The objective of feature engineering is to select a predictive variable set that can describe the behavior of the conditional quantiles and copula everywhere in the assembly.

\begin{table}[h]
    \begin{center}
    \caption[Included exogenous training features.]{Features included in the gradient boosted models as exogenous variables.}
\begin{tabular}[h]{|l | l | l | l |}
    \hline
    Sym & Label & Feature & Unit \\
    \hline
    \hline
    $T$ & ctf\_twall\_avg & CTF Face surface temperature & $[K]$ \\
    $R_T$ & ctf\_twall\_range & Surface temperature range in 4 adjacent faces & $[K]$ \\
    $q''$ & ctf\_bhf\_avg & Local CTF face heat flux & $[W/m^2]$ \\
    $R_{q''}$ & ctf\_bhf\_range & Heat flux range in 4 adjacent faces & $[W/m^2]$ \\
    $u_z$ & w\_bulk & CTF subchannel bulk Z Velocity &  $[m/s]$ \\
    $k$ & ctf\_tke\_avg & Local CTF face near wall TKE &  $[J/kg]$ \\
    $R_k$ & ctf\_tke\_range & CTF TKE range in 4 adjacent faces & $[J/kg]$ \\
    $z$ & z & Global axial position & $[m]$ \\
    $\delta z_g$ & dz\_grid & Position relative to nearest spacer grid & $[m]$ \\
    $N_g$ & n\_upsteam\_grid  & Nearest upstream spacer grid ID & $[]$ \\
    $T_\infty$ & t\_bulk & Subchannel bulk temperature  &  $[K]$ \\
    \hline
\end{tabular}
\label{tab:features}
\end{center}
\end{table}

The predictive variables given in table \ref{tab:features} were selected based on two criteria:  Availability and orthogonality.  In order to evaluate the trained machine learning model at some TH state point the model, each conditioning variable should be made available by VERA or must be computable from CTF results alone.   The predictive exogenous variable set given in table \ref{tab:features} comprise the local core conditions.  Since the model uses the local core conditions as the conditioning set, these must be supplied to the hi2lo model in order to evaluate the model.

% This has to be true since when evaluating the hi2lo model the inputs to the hi2lo model are required to be derived from either previously stored information or information made available by CTF at runtime.

At this juncture, this criteria precludes using information such as geometric orientation of a given spacer grid since it is not possible to extract or infer this information from the CTF output.  Including additional geometric information into the exogenos variable set is postulated to increase the ability of the machine learning models to distinguish unique CTF faces in the core though testing of this hypothesis is left to a potential future investigation.

It is not useful to include features which are co-linear into the explanatory feature set. The bulk fluid density was not included in the predictive variable set as it strongly depends on the local temperature. Likewise the local static pressure was not used as a predictive variable since this would be one-to-one with the axial position.  The exclusion of this TH information is primarily done for computational saving when training the boosted models since, as opposed to other machine learning algorithms and statistical inference techniques, gradient boosting is robust to collinearity of features in the input space.

In the case of gradient boosting the inclusion of nuisance or collinear exogenous variables in the model will not necessarily reduce the model's ability to generalize to unseen data, only hamper computational efficiency.  The resulting feature importance plot shown in figure \ref{fig:ktauregfeatureimp} suggests that the relative axial position within a span does not provide predictive power since this information is redundant provided the absolute axial position and the nearest upstream spacer grid.

\begin{figure}[H]
    \centering
    \includegraphics[width=0.6\linewidth]{figs/ktau_reg_feature_imp}
    \caption[Relative feature importance.]{Relative feature importances on Kendall's $\tau$.}
    \label{fig:ktauregfeatureimp}
\end{figure}

Since the boosted regression (and classification) models are insensitive to multi-collinearity in the feature space, the application of principal component analysis to the training data set was not pursued.


\subsection{Cross Validation}

An estimate of the interpolation error incurred when evaluating the trained models at unknown CFD states can be made by performing a leave one out (LOO) cross validation study.  Cross validation is used to estimate how well the machine learning models employed in this work generalized to previously unseen local core conditions; i.e. core conditions that are not included in the training data set.

\begin{figure}[h]
    \centering
    \includegraphics[width=0.3\linewidth]{figs/drawings/5x5_loo}
    \caption[Example pin layout for leave-one-out cross validation procedure.]{Example pin layout for leave-one-out cross validation procedure.  The gradient boosted models are trained on CFD and CTF data extracted from the blue pins.  Crud predictions are made on the missing pin.}
    \label{fig:5x5loo}
\end{figure}

The leave-one-out cross validation procedure is depicted in figure \ref{fig:5x5loo}.  In this procedure a single CFD-CTF pin pair is removed from the database and then the model is trained on remaining data.  Following data culling and training, the machine learning model is evaluated and crud predictions are made
at the missing pin's TH conditions.  The predicted crud results are compared against crud results generated using the original CFD data for the missing pin.  This process is repeated by sequentially for each pin in the 5x5 pin set.  The differences are summarized and averaged to obtain a measure of model's predictive performance when applied to TH conditions that reside in the TH envelope included in the training set.

This cross validation technique only ascertains interpolation errors within the TH envelope enclosed by the original full 25 pin training set.  The resulting interpolation error estimates cannot be extrapolated to core conditions that lay outside of the training set.  For a robust interpolation error analysis, a much larger training data set is required which would essentially span all possible TH conditions encountered in an operational PWR.  This will require large scale CFD runs and is left as a avenue for future uncertainty quantification work.  A larger training set could also enable other multi-fold cross validation techniques to become viable.

\subsection{Quantile Regressors}

A principal goal of the machine learning model is to predict the shape of the conditional teperature and TKE distributions as a function of local core conditions.

Quantile-quantile (Q-Q) plots of the  temperature and TKE residual distributions are used to elucidate bias introduced by the machine learning model in the conditional quantiles at a variety of axial positions and local core conditions.  Estimated quantiles are obtained for the left-out pin by evaluating the trained reduced LOO model and are compared to the expected CFD result.
A subset of the TKE residual quantile regression results are given in figure \ref{fig:tkepin1} to \ref{fig:tkepin3}.  A complete set of quantile regression results are provide in chapter \ref{chap:app_ml}, Appendix A. The Q-Q plots summarize the biases in the conditional quantile distributions when compared to the target golden standard CFD data.  The maximum and average Kolmogorov–Smirnov (KS) statistic is provided in the Q-Q figures for each pin which can be computed by equation \ref{eq:ks_stat}.
\begin{equation}
KS = \mathrm{sup}(\{\hat F(\tau_i) - F(\tau_i)\})
\label{eq:ks_stat}
\end{equation}
Where $\mathrm{sup}(\cdot)$ is the supremum of the set of distances between the predicted and empirical cumulative densities $\{\hat F(\tau_i) - F(\tau_i)\}$.
This statistic was computed at each axial level on the CTF grid.

\begin{figure}[H]%
    \centering
    \subfloat[TKE quantile regression results. CFD in solid line.  Predicted values as dashed.]{{\includegraphics[width=0.45\linewidth]{figs/ml_fit/q_tke_regression_1} }}%
    \qquad
    \subfloat[Q-Q plot of TKE quantile regression predictions from LOO cross validation study]{{\includegraphics[width=0.45\linewidth]{figs/ml_fit/qq_tke_pin_1} }}%
    \caption[Q-Q LOO TKE pin 1 results.]{Pin 1 TKE quantile regression predictions from LOO cross validation study.}%
    \label{fig:tkepin1}%
\end{figure}

\begin{figure}[H]%
    \centering
    \subfloat[TKE quantile regression results. CFD in solid line.  Predicted values as dashed.]{{\includegraphics[width=0.45\linewidth]{figs/ml_fit/q_tke_regression_2} }}%
    \qquad
    \subfloat[Q-Q plot of TKE quantile regression predictions from LOO cross validation study]{{\includegraphics[width=0.45\linewidth]{figs/ml_fit/qq_tke_pin_2} }}%
    \caption[Q-Q LOO TKE pin 2 results.]{Pin 1 TKE quantile regression predictions from LOO cross validation study.}%
    \label{fig:tkepin2}%
\end{figure}

\begin{figure}[H]%
    \centering
    \subfloat[TKE quantile regression results. CFD in solid line.  Predicted values as dashed.]{{\includegraphics[width=0.45\linewidth]{figs/ml_fit/q_tke_regression_3} }}%
    \qquad
    \subfloat[Q-Q plot of TKE quantile regression predictions from LOO cross validation study]{{\includegraphics[width=0.45\linewidth]{figs/ml_fit/qq_tke_pin_3} }}%
    \caption[Q-Q LOO TKE pin 3 results.]{Pin 1 TKE quantile regression predictions from LOO cross validation study.}%
    \label{fig:tkepin3}%
\end{figure}


A subset of the Temperature residual quantile regression results are given in figure \ref{fig:temppin1} to \ref{fig:temppin3}.  Similar to the TKE conditional quantiles, the conditional temperature distribution exhibits sharp changes in behavior across the spacer grids.  Discontinuities enforced the choice of a machine learning algorithm which is resilient to extremely sharp gradients in the response surface.

\begin{figure}[H]%
    \centering
    \subfloat[Temperature quantile regression results. CFD in solid line.  Predicted values as dashed.]{{\includegraphics[width=0.45\linewidth]{figs/ml_fit/q_twall_regression_1} }}%
    \qquad
    \subfloat[Q-Q plot of Temperature quantile regression predictions from LOO cross validation study]{{\includegraphics[width=0.45\linewidth]{figs/ml_fit/qq_twall_pin_1} }}%
    \caption[Q-Q LOO Temperature pin 1 results.]{Pin 1 Temperature quantile regression predictions from LOO cross validation study.}%
    \label{fig:temppin1}%
\end{figure}

\begin{figure}[H]%
    \centering
    \subfloat[Temperature quantile regression results. CFD in solid line.  Predicted values as dashed.]{{\includegraphics[width=0.45\linewidth]{figs/ml_fit/q_twall_regression_2} }}%
    \qquad
    \subfloat[Q-Q plot of Temperature quantile regression predictions from LOO cross validation study]{{\includegraphics[width=0.45\linewidth]{figs/ml_fit/qq_twall_pin_2} }}%
    \caption[Q-Q LOO Temperature pin 2 results.]{Pin 2 Temperature quantile regression predictions from LOO cross validation study.}%
    \label{fig:temppin2}%
\end{figure}

\begin{figure}[H]%
    \centering
    \subfloat[Temperature quantile regression results. CFD in solid line.  Predicted values as dashed.]{{\includegraphics[width=0.45\linewidth]{figs/ml_fit/q_twall_regression_3} }}%
    \qquad
    \subfloat[Q-Q plot of Temperature quantile regression predictions from LOO cross validation study]{{\includegraphics[width=0.45\linewidth]{figs/ml_fit/qq_twall_pin_3} }}%
    \caption[Q-Q LOO Temperature pin 3 results.]{Pin 3 Temperature quantile regression predictions from LOO cross validation study.}%
    \label{fig:temppin3}%
\end{figure}

The ability of the gradient boosting method to predict conditional quantiles is

\subsection{Kendall's $\tau$ Regression}

The rank correlation coefficient, Kendall's $\tau$ ($\rho_\tau$), is used to quantify the strength of dependence between the temperature and TKE on the rod surface in each CTF face.  A separate gradient boosted regression model was tasked with predicting this statistic as a function of local core conditions.   The growth rate of crud was shown to be sensitive to   $\rho_\tau$ in section \ref{sec:crud_copula_sensi}, figure \ref{fig:patchcrudfit80}.  It is therefore important to understand the error and uncertainty carried by the predicted $\hat \rho_\tau$ values in each CTF face.

A subset of the 5x5 assembly's Kendall's $\tau$ regression results are given in figure \ref{fig:ktauregression} and the complete 5x5 $\rho_\tau$ LOO cross validation results are given in figure \ref{fig:ktauregressionmontage}.  There is a marked change in behavior of the rank correlation coefficient as a function of axial position in the core from pin to pin.  The influence of Kendall's $\tau$ on the CTF face-integrated crud results was discussed in section \ref{sec:crud_copula_sensi}, and it was shown to be an important parameter to accurately predict via the machine learning model.  Pins with large relative errors for Kendall's $\tau$ are expected to produce anomalously poor crud predictions.

The worst performing pin with respect to $\hat \rho_\tau$ prediction was pin 8, as indicated in table () and figure \ref{fig:ktauregressionmontage}.  Interestingly, this pin exhibited relatively good agreement between the predicted crud distribution and the expected CFD crud distribution as indicated in table \ref{tab:loo_crud_bmass} and figure \ref{fig:montageaxialbmasssm}.  This pin, however, was relatively cold in comparison to the others in the fuel bundle, growing only $5.9$e-2 $[g]$ of crud in 300 days when the hotest rods grew $1.4$e0 $[g]$ in the same time.  In the case of pin 8, since the majority of the rod surface exists below the saturation point the crud is not sensitive to the shape of the joint temperature and TKE distributions.

\begin{figure}[H]%
    \centering
    \subfloat[Pin 1]{{\includegraphics[width=0.45\linewidth]{figs/ml_fit/ktau_regression_1} }}%
    \qquad
    \subfloat[Pin 2]{{\includegraphics[width=0.45\linewidth]{figs/ml_fit/ktau_regression_2} }}%
    \qquad
    \subfloat[Pin 3]{{\includegraphics[width=0.45\linewidth]{figs/ml_fit/ktau_regression_3} }}%
    \qquad
    \subfloat[Pin 4]{{\includegraphics[width=0.45\linewidth]{figs/ml_fit/ktau_regression_4} }}%
    \qquad
    \caption[Kendall's $\tau$ regression LOO results.]{Kendall's $\tau$ regression results from LOO cross validation study.}%
    \label{fig:ktauregression}%
\end{figure}
To improve the performance of the Kendall's $\tau$ regressors, a larger training set could be generated in future work.  For this limited 25 pin data set, it is hypothesized that each pin has a substantially unique flow field when compared to the other 24 pins.  Expelling a pin from the training data set for cross validation causes the predictive performance of the model to suffer since the remaining pins in the training set do not provide the requisite information about the local core conditions vs. Kendall's $\tau$ relationship for the missing pin.

\subsection{Copula Classifier}

In addition to the rank correlation coefficient, Kendall's $\tau$, the copula family is also required to recover the copula density function on each CTF face.  To this end a gradient boosted classifier was trained on the available CFD data.  Copula information extracted from the raw CFD results is shown in figure \ref{fig:copula_predicted}.

Figure \ref{fig:confusionmatrixavg} summarizes the LOO cross validation results of the copula classifier as a confusion matrix.  The diagonal entries of the confusion matrix represent the correctly labeled copula predictions made by the reduced LOO trained classifier for each copula family average over the entire 5x5 assembly.  It is shown that on average the classifier predicts an incorrect result more often than not.

\begin{figure}[H]
    \centering
    \includegraphics[width=0.5\linewidth]{figs/confusion_matrix_avg}
    \caption[Copula classifier confusion matrix.]{Copula classifier confusion matrix.}
    \label{fig:confusionmatrixavg}
\end{figure}


The copula classifiers performance averaged over all pins is shown in figure \ref{fig:confusionmatrixavg}. The copula classifier struggles to predict the correct copula class given the local TH state and axial core position.  As indicated in figure \ref{fig:copula_predicted}, the behavior of the copula as a function of axial rod position are erratic and inconsistent from pin to pin.  Introducing other TH exogenous variables in addition to the axial position did not increase the classification score significantly.  We can conclude that the copula are not well described by local core condition and axial position.  It remains as future work to investigate if including additional geometric pin and grid attributes could improve the classification results.  Additional software infrastructure would be required to both write geometric pin and grid features from CTF and to utilize the geometric features in the current model.

Additional future work could include performing a transformation of the input space so that the copula labels are more easily separable in the transformed space.  A Potential candidate for building this transformation is the UMAP [ref] manifold learning algorithm.

A complementary strategy to transforming the original data set as a preprocessing step is a ensemble machine learning technique known as stacking [ref].  Stacking combines the predictions of multiple classifiers using a meta-classifier.  Since model tuning and ML performance is not a focus of this work, the application of this technique to improve copula classification results is left as future work.

Though improvements are possible, it should also be noted that section \ref{sec:crud_copula_sensi} and table \ref{tab:crud_totals_copula} show that the copula family does not significantly influence overall pin integrated crud results.  Gains in the copula classifier accuracy will not translate to a large improvement in crud prediction accuracy.

\section{Single Pin Comparisons}
\label{sec:single_pin_result}

The presented benchmark problem considers a single thermal hydrualic state (power profile and flow conditions) was held for 300 days.  The hi2lo model was stepped forward in time using a resampling step size of 50 days.  400 samples were used to estimate the crud distribution in each CTF face.  The importance sampling distribution parameters were set to values given in table ().  The default remapping
weights were used.

A single pin is selected from the 25 pin array for detailed comparison between the CFD, CTF, and hi2lo crud models.  Axial crud distrubtion comparisons were made at 300 $[days]$ of simulation time.  A summary of all pins included in this study is provided in section \ref{sec:multi_pin_result} and in appendices A \& B.

The comparisons presented are the result of the LOO cross validation study.  The hi2lo crud predictions were generated for the missing pin.


\begin{figure}[H]
    \centering
    \includegraphics[width=0.7\linewidth]{figs/5x5/imp/1_5_axial_bmass}
    \caption{Pin 1 CTF vs CFD vs Hi2lo axial crud boron mass distribution at 300 days.}
    \label{fig:15axialbmass}
\end{figure}
\begin{figure}[H]
    \centering
    \includegraphics[width=0.7\linewidth]{figs/5x5/imp/1_5_axial_cmass}
    \caption{Pin 1 CTF vs CFD vs Hi2lo axial crud mass distribution at 300 days.}
    \label{fig:15axialcmass}
\end{figure}
\begin{figure}[H]
    \centering
    \includegraphics[width=0.7\linewidth]{figs/5x5/imp/1_5_pin_bmass_time}
    \caption{Pin 1 CTF vs CFD vs Hi2lo integrated crud boron mass distribution as a function of time.}
    \label{fig:15pinbmasstime}
\end{figure}
\begin{figure}[H]
    \centering
    \includegraphics[width=0.7\linewidth]{figs/5x5/imp/1_5_pin_cmass_time}
    \caption{Pin 1 CTF vs CFD vs Hi2lo integrated crud mass distribution as a function of time.}
    \label{fig:15pincmasstime}
\end{figure}

Figures \ref{fig:2d_hi2loimppinbmass} and \ref{fig:2d_hi2loimppincmass} show the hi2lo predicted crud surface distributions at 300 days.  The result of re-ordering samples onto each CTF face to preserve hot spot stationarity in time is visible.  The stripped patterns are non-physical and are an artifact of the remapping procedure.  Recall that the overarching goal is not to reproduce the detailed inter-CTF face spatial crud distributions rather the model specifically attempts to reproduce the correct average crud behavior on each CTF face, even in regions near spacer grids, and estimate the frequency of extreem crud events so to be relevant for CILC risk estimates.


\begin{figure}[H]
    \centering
    \includegraphics[width=0.7\linewidth]{figs/5x5/imp/tstep_5/pin_1/hi2lo_imp_pin_bmass}
    \caption{Pin 1 hi2lo 2D surface map of crud boron mass density.}
    \label{fig:2d_hi2loimppinbmass}
\end{figure}
\begin{figure}[H]
    \centering
    \includegraphics[width=0.7\linewidth]{figs/5x5/imp/tstep_5/pin_1/hi2lo_imp_pin_cmass}
    \caption{Pin 1 hi2lo 2D surface map of crud mass density.}
    \label{fig:2d_hi2loimppincmass}
\end{figure}

When the crud surface fields are spatially aggregated to the coarse CTF grid, the crud distribution over each CTF face are approximately consistent with the gold-standard CFD result as shown in figures \ref{fig:dist_hi2loimppinzcthick}.  The average crud behavior as a function of axial position along the rod is given in figures \ref{fig:15axialbmass} and \ref{fig:15axialcmass}.  The axial crud root-mean-squared error is given in table \ref{tab:loo_crud_bmass} alongside other pins in the assembly.  Pin 1 exhibits good agreement between the hi2lo model's crud preditions and the CFD results.

The ability of the hi2lo model to accurately predict extreme crud thickness events is significant hampered by limitations of the quantile regression and the relative sparsity of the available training data.  According to equation \ref{eq:theory_qdist_1}, estimates for extreme crud distribution quantiles (i.e. estimates of how much of the rod surface experiences crud with a thickness exceeding some critical CILC crud threshold) will have high varience.  To the authors knowlege, using standard quantile regression techniques, estimating the extreme quantiles will be fraught with high relative uncertainty as dictated by equation \ref{eq:theory_qdist_1}.  Circumventing this issue will require an investigation of alternative representations of the marginal temperature and TKE distributions.

\begin{figure}[H]%
    \centering
    \subfloat[Hi2lo pin boron mass.]{{\includegraphics[width=0.46\linewidth]{figs/5x5/imp/tstep_5/pin_1/hi2lo_imp_pin_z_bmass} }}%
    \qquad
    \subfloat[CFD pin boron mass.]{{\includegraphics[width=0.46\linewidth]{figs/5x5/cfd/tstep_5/pin_1/CFD_pin_z_bmass} }}%
    \caption{Pin 1 crud boron mass results at 300 days.}%
    \label{dist_fig:hi2loimppinzbmass}
\end{figure}


\begin{figure}[H]%
    \centering
    \subfloat[Hi2lo pin crud thickness.]{{\includegraphics[width=0.46\linewidth]{figs/5x5/imp/tstep_5/pin_1/hi2lo_imp_pin_z_cthick} }}%
    \qquad
    \subfloat[CFD pin crud thickness.]{{\includegraphics[width=0.46\linewidth]{figs/5x5/cfd/tstep_5/pin_1/CFD_pin_z_cthick} }}%
    \caption{Pin 1 crud thickness results at 300 days.}%
    \label{fig:dist_hi2loimppinzcthick}
\end{figure}


% Redundant since we already have tempearture and TKE results
% \begin{figure}[H]
%    \centering
%    \includegraphics[width=0.7\linewidth]{figs/5x5/imp/tstep_5/pin_1/hi2lo_imp_pin_z_tke}
%    \caption{}
%    \label{fig:hi2loimppinztke}
% \end{figure}
% \begin{figure}[H]
%    \centering
%    \includegraphics[width=0.7\linewidth]{figs/5x5/imp/tstep_5/pin_1/hi2lo_imp_pin_z_twall}
%    \caption{}
%    \label{fig:hi2loimppinztwall}
% \end{figure}

\section{Multi Pin Comparisons}
\label{sec:multi_pin_result}

Results for each pin in the LOO cross validation study are presented here.  There was random variation in the prediction accuracy of the model accross the 5x5 assembly with no apparent spatial bias in the model prediction errors towards the edge of the assembbly, as one may expect.  This would indicate that some pins in the 5x5 assembly are, in a sense, more unique with respect to thermal hydraulic flow conditions than others.  Pins where a small difference between the hi2lo model predictions and the gold-standard CFD result show that the pin's thermal hydraulic conditions are well represented in the training set.

In table \ref{tab:loo_crud_bmass} and \ref{tab:loo_crud_cmass} rod integrated crud results for each pin are given at 300 days of simulation time.  The worst performing pin with respect to boron deposition prediction was pin 4 by relative percent difference between the hi2lo result and the CFD driven crud result.  The Kendall's $\tau$ boosted regressor produced large prediction errors as measured by the LOO cross validation for this pin, as shown in table \ref{tab:loo_rms}.  Incorrect predictions made for the correlation coefficient act in concert with a net under-prediction of the temperature quantiles, as shown in figure (), giving rise to a significant ($\approx -48\%$) net under prediction of the total crud mass on pin 4.

The error estimates provided by the present LOO cross validation study can be viewed as slighly conservative.  The present case of expunging an entire pin from an already limited trining pool of only 25 pins is not representative a production-ready training data set.  A training set to be used in production will have all possible pin geomotries included - that is all possible pin configurations within a bundle.  Not all combination of inlet and power conditions will be available, however.  If the shape of the flow fields are primarily driven, to $1^{st}$ order, by geometric factors, missing thermal hydraulic conditions is a less impactful a hole in the parameter space than a pin-geometry hole.

% Hi2lo Bmass results
\begin{table}[h]
    \begin{center}
    \caption[Hi2lo crud boron mass results]{Crud boron mass hi2lo LOO result summary at 300 days.}
    \begin{tabular}[h]{|c|c|c|c|c|c|}
        \hline
        Pin & CTF Bmass $[g]$ & CFD Bmass $[g]$ & Hi2lo Bmass $[g]$ & Hi2lo-CFD $[g]$ & Relative Diff \% \\
\hline
1  & 1.2940e-03 & 7.5489e-04 & 6.3163e-04 & -1.2326e-04 &  -16.3 \\
2  & 1.1458e-03 & 5.1953e-04 & 3.1487e-04 & -2.0466e-04 &  -39.4 \\
3  & 1.0265e-03 & 3.2678e-04 & 3.4192e-04 & 1.5140e-05 &  4.6 \\ 
4  & 1.0111e-03 & 5.6847e-04 & 2.9848e-04 & -2.6999e-04 &  $\bf{-47.5}$ \\
5  & 9.5319e-04 & 1.8974e-04 & 2.2723e-04 & 3.7490e-05 &  19.8 \\ 
6  & 4.0505e-04 & 9.1686e-05 & 1.0065e-04 & 8.9640e-06 &  9.8 \\ 
7  & 8.0907e-05 & 4.0210e-05 & 3.7515e-05 & -2.6950e-06 &  -6.7 \\
8  & 6.5705e-05 & 2.9861e-05 & 3.3475e-05 & 3.6140e-06 &  12.1 \\ 
9  & 6.7204e-05 & 3.3324e-05 & 3.3063e-05 & -2.6100e-07 &  -0.8 \\
10  &6.6850e-05 & 3.5892e-05 & 2.8171e-05 & -7.7210e-06 &  -21.5 \\
11  &8.7449e-05 & 4.0316e-05 & 3.7932e-05 & -2.3840e-06 &  -5.9 \\
12  &4.6693e-04 & 1.1722e-04 & 8.2669e-05 & -3.4551e-05 &  -29.5 \\
13  &1.0616e-03 & 2.2909e-04 & 2.6878e-04 & 3.9690e-05 &  17.3 \\ 
14  &1.0617e-03 & 2.7471e-04 & 3.9548e-04 & 1.2077e-04 &  44.0 \\ 
15  &1.0366e-03 & 4.5067e-04 & 3.3163e-04 & -1.1904e-04 &  -26.4 \\
16  &1.1594e-03 & 2.9707e-04 & 4.3385e-04 & 1.3678e-04 &  46.0 \\ 
17  &9.1090e-04 & 2.6739e-04 & 2.6569e-04 & -1.7000e-06 &  -0.6 \\
18  &7.1616e-04 & 2.3468e-04 & 2.0092e-04 & -3.3760e-05 &  -14.4 \\
19  &6.1329e-04 & 1.1289e-04 & 1.4341e-04 & 3.0520e-05 &  27.0 \\ 
20  &1.6370e-04 & 7.2277e-05 & 4.8991e-05 & -2.3286e-05 &  -32.2 \\
21  &1.2524e-04 & 4.5454e-05 & 4.4092e-05 & -1.3620e-06 &  -3.0 \\
22  &1.7007e-04 & 4.4576e-05 & 6.1729e-05 & 1.7153e-05 &  38.5 \\ 
23  &6.2594e-04 & 1.5092e-04 & 1.1521e-04 & -3.5710e-05 &  -23.7 \\
24  &7.1144e-04 & 1.8479e-04 & 2.1561e-04 & 3.0820e-05 &  16.7 \\ 
25  &4.1668e-04 & 1.3290e-04 & 8.6106e-05 & -4.6794e-05 &  -35.2 \\
\hline \hline
Totals & 1.5443e-02 & 5.2453e-03 & 4.7791e-03 & -4.6623e-04   & -8.88 \\
\hline
\end{tabular}
\label{tab:loo_crud_bmass}
\end{center}
\end{table}


% hi2lo cmass results
\begin{table}[h]
    \begin{center}
        \caption[Hi2lo crud mass results]{Crud mass hi2lo LOO result summary at 300 days.}
    \begin{tabular}[h]{|c|c|c|c|}
        \hline
  Pin & CTF Cmass Tot & CFD Cmass Tot & Hi2lo Cmass Tot  \\
\hline
1  & 2.4316e-04 & 1.4232e-04 & 1.1899e-04 \\
2  & 2.1537e-04 & 9.8041e-05 & 5.9479e-05 \\
3  & 1.9302e-04 & 6.1962e-05 & 6.4702e-05 \\
4  & 1.9040e-04 & 1.0737e-04 & 5.6557e-05 \\
5  & 1.7982e-04 & 3.6318e-05 & 4.3346e-05 \\
6  & 7.6893e-05 & 1.7713e-05 & 1.9431e-05 \\
7  & 1.5980e-05 & 7.9597e-06 & 7.4350e-06 \\
8  & 1.3032e-05 & 5.9122e-06 & 6.6394e-06 \\
9  & 1.3329e-05 & 6.6094e-06 & 6.5574e-06 \\
10  & 1.3259e-05 & 7.1162e-06 & 5.5866e-06 \\
11  & 1.7213e-05 & 7.9487e-06 & 7.5059e-06 \\
12  & 8.8404e-05 & 2.2533e-05 & 1.5989e-05 \\
13  & 2.0002e-04 & 4.3636e-05 & 5.1048e-05 \\
14  & 1.9970e-04 & 5.2130e-05 & 7.4719e-05 \\
15  & 1.9474e-04 & 8.4999e-05 & 6.2756e-05 \\
16  & 2.1787e-04 & 5.6215e-05 & 8.1922e-05 \\
17  & 1.7124e-04 & 5.0727e-05 & 5.0361e-05 \\
18  & 1.3477e-04 & 4.4664e-05 & 3.8244e-05 \\
19  & 1.1573e-04 & 2.1674e-05 & 2.7412e-05 \\
20  & 3.1488e-05 & 1.4063e-05 & 9.6251e-06 \\
21  & 2.4285e-05 & 8.9271e-06 & 8.6669e-06 \\
22  & 3.2634e-05 & 8.7701e-06 & 1.2055e-05 \\
23  & 1.1797e-04 & 2.8810e-05 & 2.2075e-05 \\
24  & 1.3379e-04 & 3.5204e-05 & 4.0989e-05 \\
25  & 7.8681e-05 & 2.5506e-05 & 1.6677e-05 \\
\hline
\end{tabular}
\label{tab:loo_crud_cmass}
\end{center}
\end{table}

% RMS error table
\begin{table}[h]
    \begin{center}
        \caption[Hi2lo crud RMS summary.]{Hi2lo crud RMS summary.}
    \begin{tabular}[h]{|c|c|c|c|c|c|c|c|}
        \hline
Pin  & RMS & RMS & RMS & RMS & RMS & RMS & RMS \\
\hline
1  & 6.2482e-04 & -2.3332e-05 & 3.3009e-07 & -1.2326e-08 & 7.5489e-08 & 1.4232e-04 & 0.0000e+00 \\
2  & 8.1392e-04 & -3.8562e-05 & 4.3183e-07 & -2.0466e-08 & 5.1953e-08 & 9.8041e-05 & 0.0000e+00 \\
3  & 2.0189e-04 & 2.7399e-06 & 1.0731e-07 & 1.5140e-09 & 3.2678e-08 & 6.1962e-05 & 0.0000e+00 \\
4  & 1.0737e-03 & -5.0812e-05 & 5.7008e-07 & -2.6998e-08 & 5.6847e-08 & 1.0737e-04 & 0.0000e+00 \\
5  & 2.2335e-04 & 7.0283e-06 & 1.1847e-07 & 3.7498e-09 & 1.8974e-08 & 3.6318e-05 & 0.0000e+00 \\
6  & 6.5623e-05 & 1.7178e-06 & 3.4608e-08 & 8.9595e-10 & 9.1686e-09 & 1.7713e-05 & 0.0000e+00 \\
7  & 2.1295e-05 & -5.2469e-07 & 1.0898e-08 & -2.6945e-10 & 4.0210e-09 & 7.9597e-06 & 0.0000e+00 \\
8  & 1.8572e-05 & 7.2717e-07 & 9.4019e-09 & 3.6144e-10 & 2.9861e-09 & 5.9122e-06 & 0.0000e+00 \\
9  & 1.0839e-05 & -5.1974e-08 & 5.4659e-09 & -2.6075e-11 & 3.3324e-09 & 6.6094e-06 & 0.0000e+00 \\
10  & 2.4348e-05 & -1.5296e-06 & 1.2298e-08 & -7.7213e-10 & 3.5892e-09 & 7.1162e-06 & 0.0000e+00 \\
11  & 2.9659e-05 & -4.4282e-07 & 1.5478e-08 & -2.3832e-10 & 4.0316e-09 & 7.9487e-06 & 0.0000e+00 \\
12  & 1.9073e-04 & -6.5447e-06 & 1.0111e-07 & -3.4547e-09 & 1.1722e-08 & 2.2533e-05 & 0.0000e+00 \\
13  & 3.9831e-04 & 7.4119e-06 & 2.1135e-07 & 3.9690e-09 & 2.2909e-08 & 4.3636e-05 & 0.0000e+00 \\
14  & 5.0327e-04 & 2.2589e-05 & 2.6950e-07 & 1.2076e-08 & 2.7471e-08 & 5.2130e-05 & 0.0000e+00 \\
15  & 5.5978e-04 & -2.2242e-05 & 2.9813e-07 & -1.1904e-08 & 4.5067e-08 & 8.4999e-05 & 0.0000e+00 \\
16  & 5.6026e-04 & 2.5707e-05 & 2.9830e-07 & 1.3678e-08 & 2.9707e-08 & 5.6215e-05 & 0.0000e+00 \\
17  & 1.2664e-04 & -3.6509e-07 & 6.6854e-08 & -1.7016e-10 & 2.6739e-08 & 5.0727e-05 & 0.0000e+00 \\
18  & 1.8205e-04 & -6.4199e-06 & 9.5976e-08 & -3.3756e-09 & 2.3468e-08 & 4.4664e-05 & 0.0000e+00 \\
19  & 1.2838e-04 & 5.7381e-06 & 6.8271e-08 & 3.0523e-09 & 1.1289e-08 & 2.1674e-05 & 0.0000e+00 \\
20  & 9.7915e-05 & -4.4377e-06 & 5.1674e-08 & -2.3286e-09 & 7.2277e-09 & 1.4063e-05 & 0.0000e+00 \\
21  & 1.7636e-05 & -2.6016e-07 & 9.2180e-09 & -1.3620e-10 & 4.5454e-09 & 8.9271e-06 & 0.0000e+00 \\
22  & 6.7042e-05 & 3.2845e-06 & 3.5199e-08 & 1.7153e-09 & 4.4576e-09 & 8.7701e-06 & 0.0000e+00 \\
23  & 1.7096e-04 & -6.7350e-06 & 9.0595e-08 & -3.5703e-09 & 1.5092e-08 & 2.8810e-05 & 0.0000e+00 \\
24  & 1.5885e-04 & 5.7841e-06 & 8.4133e-08 & 3.0827e-09 & 1.8479e-08 & 3.5204e-05 & 0.0000e+00 \\
25  & 1.9995e-04 & -8.8282e-06 & 1.0617e-07 & -4.6799e-09 & 1.3290e-08 & 2.5506e-05 & 0.0000e+00 \\
\hline
\end{tabular}
\label{tab:loo_rms}
\end{center}
\end{table}

At each resampling step the crud mass was summed over all pins in the assembly and presented in figure \ref{fig:asmcmasstime}.  The total crud predicted by CFD and the hi2lo model are in in close agreement.

\begin{figure}[H]
    \centering
    \includegraphics[width=0.7\linewidth]{figs/5x5/imp/asm_cmass_time}
    \caption{Assembly integrated CTF vs CFD vs Hi2lo crud mass as a function of time.}
    \label{fig:asmcmasstime}
\end{figure}

\begin{figure}[H]
    \centering
    \includegraphics[width=0.7\linewidth]{figs/5x5/imp/l2_boron_asm_errors_hmap}
    \caption{5x5 Axial RMS crud boron mass relative error distribution.}
    \label{fig:l2boronasmerrorshmap}
\end{figure}
\begin{figure}[H]
    \centering
    \includegraphics[width=0.7\linewidth]{figs/5x5/imp/l2_cmass_asm_errors_hmap}
    \caption{5x5 Axial RMS crud mass relative error distribution.}
    \label{fig:l2cmassasmerrorshmap}
\end{figure}


\begin{figure}[H]
    \centering
    \includegraphics[width=0.7\linewidth]{figs/5x5/imp/tot_bmass_rel_asm_errors_hmap}
    \caption{5x5 Integrated crud boron mass relative error distribution.}
    \label{fig:totbmassrelasmerrorshmap}
\end{figure}
\begin{figure}[H]
    \centering
    \includegraphics[width=0.7\linewidth]{figs/5x5/imp/tot_boron_asm_errors_hmap}
    \caption{5x5 Integrated crud boron mass absolute error distribution.}
    \label{fig:totboronasmerrorshmap}
\end{figure}
\begin{figure}[H]
    \centering
    \includegraphics[width=0.7\linewidth]{figs/5x5/imp/tot_cmass_asm_errors_hmap}
    \caption{5x5 Integrated crud mass absolute error distribution.}
    \label{fig:totcmassasmerrorshmap}
\end{figure}
\begin{figure}[H]
    \centering
    \includegraphics[width=0.7\linewidth]{figs/5x5/imp/tot_cmass_rel_asm_errors_hmap}
    \caption{5x5 Integrated crud mass relative error distribution.}
    \label{fig:totcmassrelasmerrorshmap}
\end{figure}

The correlation between errors committed by the multiple quantile and $\rho_\tau$ regressors on the errors seen in the crud results was investigated in an attempt to establish performance metrics.  If a statistically significant trend can be drawn between the ML model performance and predicted crud accuracy results then the trend could be used as an indicative tool for testing the a.  To do this, one could first compute the ML regression errors via a cross validation strategy.  Provided these error estimates and known sensitivities of the crud results to the ML errors one can estimate the expected accuracy of the crud predictions obtained via the hi2lo model.  This allows a user of the hi2lo tool to detect problems with the ML models before using the model to make crud predictions.

As shown in the upper-triangle of figure \ref{fig:asmerrorcorr} a Student-T test was conducted on the slope of each fitted linear trend model.  The null hypothesis was taken to be a slope of 0.  One can conclude that a  sample size greater than 25 pins should be used in future work to investigate the relative influence of machine learning errors on the predicted crud errors.  The standard deviation of the computed sensitivities is high when using a small sample size making it difficult to rigorously conclude that errors made by the machine learning models correspond to errors in the crud predictions.

Statistically significant trends were found between the RMS error committed by the TKE quantile regressors and the crud boron and mass distribution errors, as measured by root-mean-squared error.

\begin{figure}[H]
    \centering
    \includegraphics[width=0.95\linewidth]{figs/5x5/imp/asm_error_corr}
    \caption{Correlation of ML errors with crud prediction errors.}
    \label{fig:asmerrorcorr}
\end{figure}

\section{Section Takeaways}


\begin{itemize}
    \item Crud predictions made by the hi2lo model were compared against CFD/crud coupled results and stanalone CTF/crud results.  The Hi2lo model produced crud results closer to the CFD result than the CTF result, as anticipated.  The overall crud mass results for the 5x5 assembly favorably compared against the gold-standard CFD assembly integrated results.
    \item A leave one out cross validation strategy was utilized to quantify the predictive performance of the model.
    \item The prediction accuracy of the temperature and TKE quantile regression models was summarized though Q-Q plots for each pin in the LOO cross validation study.
    \item The prediction accuracy of the Kendall's $\tau$ regression model was assessed using the root-mean-square error for each pin in the LOO cross validation study.  \item Correlations between the errors committed by the machine learning models and the crud prediction errors were computed.  High uncertainty associated with these correlation measures did not permit a statistically significant link between poor Kendall's $\tau$ predictive performance and poor crud predictions to be drawn.
        \item The copula classifier performed poorly given the current set of considered explanatory variables and limited size of the training data set.  A Gaussian copula was assumed on each CTF face in place of the poorly predicted copula family from the classifier.  Recalling the results shown in \ref{sec:crud_copula_sensi}, this is not expected to reduce crud prediction accuracy.
\end{itemize}
