%! TEX root = ../dissertation_gurecky.tex

\utabstract
\index{Abstract}%
\indent
The development of a physics-directed, statistically based,
surrogate model of the small scale flow features that impact crud growth is presented in this work. 
The objective of the surrogate is to provide additional details of the rod surface temperature, heat
flux and near-wall turbulent kinetic energy fields which cannot be explicitly captured by the subchannel code, CTF. 

Operating as a mapping from the high fidelity CFD data to the low fidelity subchannel grid, the surrogate provides CFD-informed boundary conditions to the crud model executed on the subchannel pin surface mesh. The surface temperature, heat
flux and turbulent kinetic energy, henceforth referred to as the Fields of Interest (FOI),
govern the growth rate of crud on the surface of the rod and the
precipitation of boron in the porous crud layer. Therefore the surrogate predicts the behavior of the
fields of interest as a function of position in the core and local thermal-hydraulic (TH) conditions.

The subchannel code is substantially faster to execute than CFD
and produces an estimate for all the relevant TH quantities at a coarse spatial resolution everywhere in
the core.  The solution provided by CTF is augmented by a predicted stochastic
component of the FOI informed by CFD results to provide a more detailed description of the target
FOIs than CTF can provide alone.  To this end, a novel method based on the marriage of copula and
gradient boosting techniques is proposed. This approach forgoes a spatial interpolation approach
for a statistically driven approach which predicts the fractional area of a rod’s surface in excess of some
critical temperature - but not precisely where such maxima occur on the rod surface.
The resultant model retains the ability to account for the presence
of hot and cold spots on the rod surface induced by turbulent flow downstream of spacer grids when
producing crud estimates. Sklar’s theorem is leveraged to decompose multivariate probability densities
of the FOI into independent copula and marginal models. The free parameters within the copula model
are predicted using a combination of supervised regression and classification machine learning techniques
with training data sets supplied by a suite of pre-computed CFD results spanning a typical PWR TH
envelop.