%-----------------------------------------------------------------------------%
\pagebreak
\section*{Project Schedule}

Legend:
\begin{itemize}
    \item {\color{blue} Blue text: New task (added post-proposal).}
    \item (\checkmark)  Complete
    \item (\checkmark-)  Underway but incomplete
    \item ($\cdot$)  Not started
    \item (\xmark)  \sout{Not started.  Will not do.}
\end{itemize}


\begin{enumerate}
\item \textbf{(\checkmark-) Literature review.}
\item \textbf{(\checkmark) Develop CFD-CRUD tools for training data generation and extraction.}
          Platform used to generate datasets that are
          required to inform a scale-bridging model.
    \begin{enumerate}
        \item (\checkmark) Create CFD data extraction and post processing utilities.
        \item (\checkmark) Develop STAR-CCM+ plugin to extract cladding surface and volumetric TH quantities from a CFD computation.
        \item (\checkmark) Compute the required volume and surface integrals to
              distill finely resolved CFD datasets into subchannel-like results.
    \end{enumerate}
\item \textbf{(\checkmark-) Demonstrate differences between CFD and Subchannel predictions.}
    \begin{enumerate}
        \item (\checkmark-) Compute differences between volume/surface averaged CFD quantities and subchannel predictions.
        \item ($\cdot$) Identify deficiencies in subchannel predictions wrt. CRUD growth and CILC.
    \end{enumerate}
\item \textbf{(\checkmark-) Investigate CRUD model sensitives.}
    \begin{enumerate}
        \item (\checkmark) Identify CRUD (MAMBA) sensitivities to TH boundary conditions
        \item (\checkmark) Produce correlation coefficients and scatter plot depictions of the relationship(s) between input
              and output of the CRUD model.
          \item (\checkmark-) Identify TH conditions under which CFD scale CRUD predictions diverge from subchannel-CRUD results.
    \end{enumerate}
\item \textbf{(\checkmark-) Preliminary methods implementation and demonstration}
    \begin{enumerate}
        \item (\checkmark-) De-trend pointwise CFD datasets \& compute residual distributions.
        \begin{enumerate}
            \item (\checkmark) Moving averaged approach (assumes CTF and CFD will agree on the mean)
            \item (\checkmark-) CTF mean approach (requires CTF runs at identical CFD sample points)
        \end{enumerate}
        %
        \item (\checkmark) Create a tool to generate synthetic CFD-like training data sets to facilitate
            testing of the regression tools.
        %
        \item (\checkmark) Construct model of the spatially dependent co-variance between
              (residual) wall temperatures, surface shear stress, and
              boundary heat flux downstream of a grid span.
        \begin{enumerate}
            \item (\checkmark) Develop flexible dependence modeling package capable of capturing skewed covariance behavior (Copula).
            \item ($\cdot$) Develop Kriging model to capture spatial auto-correlation (may not be necessary?).
        \end{enumerate}
        %
        \item (\checkmark) Construct a regression model $G(\cdot)$ (Gradient boosted tree model [GBM]).
        \begin{enumerate}
            \item (\checkmark) Regress copula model parameters, $\bm{\hat\theta}$, on local-average CTF provided inputs.
            \item (\checkmark) Evaluate regression to recover joint TH distributions on any CTF patch:
                $G(CTF\ Patch\ Data) \rightarrow {P(T>t,TKE>tke,...|\bm{\hat\theta})}$
        \end{enumerate}
        %
        \item ($\cdot$) Uncertainty Propagation
        \begin{enumerate}
        	\item ($\cdot$) How many quantiles are required for temperature and TKE margin reconstruction
                                - do we have the data required to drive down uncertainty in CRUD predictions to an acceptable level?
            \item ($\cdot$) Estimate uncertainty introduced by the GBRT/Copula model
            \item (\xmark) \sout{Propagate uncertainty to ensemble CRUD calculations}
        \end{enumerate}
        %----------------- POST PROPOSAL ------------------%
        \item (\checkmark-) Multi-state point execution
        \begin{enumerate}
            \item (\checkmark) Implement time stepping scheme.  Support ability to change the power profile/flow conditions at each time step.
            \item (\checkmark-) Implement ability to propagate uncertainty through time steps.
            \begin{itemize}
                \item {\checkmark} Account for sampling-induced uncertainty and demonstrate how these errors accumulate over time.
                \item {\checkmark-} Account for uncertainty in CFD sample quantiles. The sample quantiles are known
                                    to be distributed according to a gaussian distribution.  It is possible to propagate
                                    this uncertainty through the model.  This type of uncertainty manifests itself as a fuzzyness
                                    of CFDs of temperature, heat flux, and tke.  So each time you sample from the CFD you: must
                                    first ``sample'' a CFD from.  Basically the parameters which govern the CDF become random variables.
            \end{itemize}
        \item{ ($\cdot$} {\color{blue} Detail how this hi2lo strategy fits into a multi-physics framework with TH/Neutronic/CRUD feedbacks.}
        \end{enumerate}
        % -------------------------------------------------%
        \item (\xmark) \sout{Compute areas of the prediction surface with highest
                             sensitivity to changes in input parameters \& develop capability to super sample these regions.}
        %
        \item (\xmark) \sout{Demonstrate ability to perform dimensionality reduction of the input space.}
        \item (\checkmark-) Perform CFD informed subchannel based CRUD prediction for a \emph{single pin}.
    \end{enumerate}
\item \textbf{(\checkmark-) Write proposal document}
\item \textbf{($\cdot$) Methods analysis and improvement}
    \begin{enumerate}
        \item (\checkmark) Perform Large scale CFD runs to inform finalized scale-bridging model.  (this was done by Bob Salko et. al for a 5x5 case)
        \item ($\cdot$) Tune regression algorithm's (GBM) hyperparameters
        \item ($\cdot$) Validate CFD informed subchannel CRUD results against plant data (IF AVAILABLE).
    \end{enumerate}
\item \textbf{($\cdot$) Final model construction and demonstration}
    \begin{enumerate}
        \item (\xmark) \sout{Demonstrate CFD informed subchannel model on a full-height, single assembly problem.}
        \item ($\cdot$) Propose how to extend this model to multi-pin, assembly scale cases.  Identify why this problem is
                        particularly messy and difficult.  This is a feature engineering problem.  Each pin, each CTF face must be
                        uniquely identified in the core.  Essentially every CTF face in the assembly must have an associated unique
                        set of features that identify it.
        \item ($\cdot$) Assess CILC/CIPS prediction improvements vs. base model (no CFD informed TH).
    \end{enumerate}
\item \textbf{($\cdot$) Write Dissertation}
\end{enumerate}

\begin{sidewaysfigure}
%---------------------------     YEAR ONE        ------------------------------%
\begin{ganttchart}[
        inline,
        x unit=1.5cm,
        y unit title =0.8cm,
        y unit chart=0.8cm,
        hgrid,
        vgrid,
        time slot format=isodate,
        compress calendar
    ]{2016-06-01}{2017-07-30}
    \gantttitlecalendar*[compress calendar, time slot format=isodate]{2016-06-01}{2017-07-30}{year, month=shortname} \\
    \gantttitlelist{1,...,14}{1} \\
%%%%%%% Phase 1
\ganttgroup{Proposal Era}{2016-06-01}{2017-06-30} \\  %elem0
\ganttbar{Lit. Review}{2016-06-01}{2016-11-01} \\  %elem1
\ganttbar{CFD Post Proc. Dev.}{2016-06-01}{2016-08-20} \\  %elem2
\ganttlinkedbar{CTF vs CFD}{2016-08-20}{2016-10-30} \ganttnewline  %elem3
\ganttbar{MAMBA Sensitivity}{2016-08-01}{2016-09-30} \ganttnewline  %elem4
\ganttbar{Copula Model Dev.}{2016-10-01}{2017-01-10}  \\  %elem5
\ganttbar{Synthetic Data Gen}{2016-12-01}{2017-01-30} \\  %elem6
\ganttlinkedbar{GBRT Model Dev.}{2016-12-01}{2017-03-30}  \\  %elem7
\ganttlinkedbar{Link Models w/ CRUD Sim.}{2017-03-01}{2017-06-30}  \\  %elem8
\ganttbar{Write Proposal}{2017-04-01}{2017-06-30}  %elem9
\ganttmilestone{Proposal}{2017-06-30} \ganttnewline  %elem10
%%%%%%%% Extra Task Linkages
\ganttlink[link mid=0.2]{elem2}{elem4}
\end{ganttchart}
%-----------------------------------------------------------------------------%
\end{sidewaysfigure}

\begin{sidewaysfigure}
%---------------------------     YEAR TWO       ------------------------------%
\begin{ganttchart}[
        inline,
        x unit=2.5cm,
        y unit title =0.8cm,
        y unit chart=0.8cm,
        hgrid,
        vgrid,
        time slot format=isodate,
        compress calendar
    ]{2017-06-01}{2018-05-01}
    \gantttitlecalendar*[compress calendar, time slot format=isodate]{2017-06-01}{2018-05-01}{year, month=shortname} \\
    \gantttitlelist{13,...,25}{1} \\
%%%%%%% Phase 2
\ganttbar{ }{2017-06-01}{2017-06-30}  %elem0
\ganttmilestone{Proposal}{2017-06-30} \ganttnewline  %elem1
\ganttgroup{Dissertation Era}{2017-07-1}{2018-05-01} \\  %elem2
\ganttbar{Time Stepping}{2017-07-01}{2017-01-30} \ganttnewline  %elem3
\ganttlinkedbar{Uncert. Prop}{2017-08-01}{2017-02-28} \ganttnewline  %elem4
\ganttbar{Methods Refinement}{2017-08-01}{2018-02-28} \ganttnewline  %elem5
\ganttbar{Final CFD Runs}{2018-01-01}{2017-02-28} \ganttnewline  %elem6
\ganttbar{Write Dissertation}{2018-01-01}{2018-03-30}  %elem7
\ganttmilestone{Defense}{2018-4-15} \ganttnewline  %elem8
\ganttlinkedbar{Final Revisions}{2018-04-15}{2018-05-01}  \ganttnewline  %elem9
\ganttmilestone{Project Conclusion}{2018-05-01} \ganttnewline  %elem10
%%%%%%%% Extra Task Linkages
\end{ganttchart}
%-----------------------------------------------------------------------------%
\end{sidewaysfigure}


\pagebreak
