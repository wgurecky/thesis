%! TEX root = ../dissertation_gurecky.tex



\section{Crud Background}

\begin{itemize}
        \item (\checkmark) What is CRUD?  What is CIPS?
                \begin{itemize}
                \item CRUD is deposits of primary loop corrosion products on the outer cladding surface.  NiFe2O4 and many other constituents, Ni-B-O and Fe-B-O compounds, also some LiBOH. [ref mamba theory manual - is this possible since it might be export controlled?].  CRUD  typically forms where temperatures are high and where sub cooled boiling occurs on the rod surface.
                \item Shift in power to bottom of core due to Boron trapped in the CRUD layer.
                \end{itemize}
        \item (\checkmark) How is CRUD typically modeled?  At the core wide scale, typically TH boundary conditions are supplied by subchannel or coarser [refs].  Some CFD/CRUD work has been conducted that predicted striping patterns, or high variations in azimuthal crud growth, downstream of spacer grids.  The aparse experimental crud scrape data also shows high azimuthal variation downstream spacer grids.
        \item (\checkmark) Where do current models fall short?
        \begin{itemize}
                \item Incorrect boundary conditions: Handing incorrect boundary conditions to the crud model will never produce the correct crud unless a fudge factor is applied to counteract the effects of bad BCs.  This is most severe downstream of spacer grids in situations where a subchannel code cannot resolve fine scale flow features that influence crud growth.  This work tries to address this problem by improving the accuracy of the boundary conditions handed to the crud model by leveraging a suite of pre-computed CFD results.
                \item Known missing physics in the crud model: missing Ni oxide, incorrect pore fill kinetics, incorrect crud model parameters (chimney heat transfer, Arrhenius rate constants, species diffusion constants.) These must be fixed via experiment and model calibration which is beyond the scope of this work.
        \end{itemize}
\end{itemize}

\section{Significance and Novelty}

CRUD growth is dominated by threshold physics [ref].  Hot and cold spots
present downstream of spacer grids must be accurately resolved by the Hi2Low model
in order to predict the maximum CRUD
thickness and boron precipitation within the CRUD layer.  

It is challenging to faithfully capture the peaks and valleys in
rod surface temperature and TKE distribution by traditional interpolation
techniques since such a model must guard against smearing out the sharp peaks
present in the spatial distributions.  
In the present method we forgo a spatial shape function mapping strategy
for a statistically driven approach which predicts the fractional
area of a rod's surface in excess of some critical temperature - but not
precisely where such maxima occur.

In this approach the method does not predict the fine scale flow/temperature field on the pin surface, rather, this approach estimates the joint T, TKE, BHF probability density on each CTF face.  The goal is retain the minimum amount of information required to get CRUD correct on each CTF face.  The amount of CRUD deposition downstream of spacer grids is influenced by the presence of hot and cold spots present due to the turbulent flow induced by mixing vanes.  CRUD is highly sensitive to the rod surface temperature, particularly around the saturation point, and therefore it is important to account for these small scale flow features when providing boundary conditions to MAMBA.

By capturing the action of local hot and cold spots on the crud deposition rate the method accounts for more physics when making predictions of the total integrated boron mass in the CRUD layer.  This results in an improvement in CIPS predictions since the total quantity of boron in the crud layer is of principle importance.  Additionally the ability to estimate the likelihood of extreme crud buildup enables the hi2lo methods developed in this work as a crud induced local corrosion (CILC) scoping tool.  It is envisioned that such a tool will identify potential CILC "hot spots" where a significant amount of cladding is consumed by oxide ingress, resulting in potential fuel failure.  The effectiveness of the model in this role is governed by the magnitude of propagated uncertainties through the hi2lo model.


\subsection{CASL CRUD Challenge Problems}

The Consortium for the Advanced Simulation of LWRs (CASL) selected several problems identified by industry partners as critical, inadequately understood, engineering scale phenomena which would provide
financial and safety benefits to the nuclear power industry if resolved \cite{Turinsky15}.
The problem of
interest in this work is the prediction of Chalk River Unidentified Deposit
(CRUD) growth rates.  The growth of CRUD comes with neutronic and thermal
hydraulic repercussions that are of interest to CASL's industry partners.
In an effort to simulate the
effects of CRUD on the power and burnup distribution, a code produced by a LANL
and ORNL collaboration \cite{collins16} under the name MAMBA was developed.
This previous code development effort provides a starting point for the currently proposed work.

A phenomena known as CRUD Induced Power Shift (CIPS) is caused by the presence
of elevated \ce{^{10}B} concentrations in the CRUD layer.  Since CRUD is preferentially
deposited on the fuel rods' in hot regions of the core, this leads to a slight shift in
power production towards to bottom of the core under steady state operation.  
CRUD induced power shift impacts the burnup distribution over a cycle
and is important to account for when computing thermal
margins of the fuel.  The prediction
of CIPS is especially important for older facilities seeking to uprate power
output.  Additionally, the presence of CRUD on the rod surface has been shown
to exacerbate local oxide penetration rates of some zirconium alloys \cite{adamson07}.
This is known as CRUD induced local corrosion (CILC).  Improvements in CRUD
simulation techniques ultimately improve the ability to predict the CIPS and
CILC phenomena.

The Virtual Environment for Reactor Applications (VERA) is a key component of
CASL's technical portfolio.  This meta-package integrates a variety of physics
packages and multiphysics coupling options to form a robust reactor simulation
capability.  For multi-cycle depletion computations, VERA relies upon MPACT, a
2D-1D method of characteristics neutronics package, coupled with the subchannel
thermal hydraulics code, CTF.  An integrated CRUD modeling capability
is provided by MAMBA to address the CRUD induced power shift challenge problem.

To reduce computation times, the subchannel TH code discretizes the reactor
domain into large, centimeter scale finite volumes. As a consequence of this
discretization scheme, sub-centimeter scale thermal hydraulic effects of the
spacer grids on CRUD are averaged over large regions on the fuel rods'
surfaces.  Though small scale phenomena are not explicitly modeled, they are
approximately accounted for in a variety of empirically derived closure
relations.  In effect, a single constant estimate for the mean thermal
hydraulic conditions are obtained in each finite volume.

Previous Hi2Low focused work in CASL focused on utilizing experimental or CFD
data sets to improve heat transfer and turbulent mixing models in CTF.  These studies focused on
correcting biases in the bulk-average behavior of the flow (due to the
previously neglected physics).  Examples of such Hi2Low models are given in
\autoref{chap:lit}.

The traditional approach must be modified to accommodate the CILC and CIPS
challenge problems.  Here arises the need to retain not only the effect of
fine-scale physics on the bulk, but also to predict if certain temperature or
TKE thresholds are exceeded in a given subchannel volume.  Furthermore, for a
complete treatment of thermal hydraulic impacts on CRUD growth, the
scale-bridging model must describe completely the frequency distribution of
extreme TH events above a given threshold.


\section{Hi2lo Discussion}

Hi2lo, or High to Low modeling, implies that a source of high fidelity gold standard data produced by an expensive to evaluate physics model is used to downscale and augment a low fidelity model of the same physics.
Provided sparsely available high fidelity data, this mapping must be possible even in the case where matching high and low fidelity results do not exist.  The hi2lo strategy put forward in this work can be viewed as a particular implementation of a statistical downscaling (SD) based model, of which a large variety exist in the literature and some of which are described in chapter ().  One interesting challenge in this work which is atypical of SD models is the requirement of co-prediction of multiple correlated fields, sometimes this is refereed to as multiple target regression.

The hi2lo mapping, which includes downscaling and augmentation of the CTF-predicted low fidelity fuel rod surface temperature, TKE, and boundary heat flux surface fields, must possible given any CTF pin at any operating condition - even when only a relatively small number of high fidelity results are available.  Furthermore it is assumed that the flow of information is uni-directional from the high fidelity data to the low fidelity model.  Feedback between the disparate scale models is not included.

Generally, a surrogate model replaces expensive to evaluate physics with a quick to evaluate model that preserves some aspects of the physics. The hi2lo strategy seeks to capture the action of high fidelity CFD resolved flow phenomena on crud growth without having to run the CFD model outright.  However, a key difference between the hi2lo model pursued in this work and a canonical dynamical system surrogate is that the hi2lo model does not seek to behave as a stand-in for a differential equation and should not be confused as such.  

