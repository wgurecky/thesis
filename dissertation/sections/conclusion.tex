\section{Discussion}

Positive outcomes:

\begin{itemize}
	\item  Ability to produce a hi2lo mapping in the case of missing CFD data at the desired TH state point has been demonstrated.
	\item  Includes more physics into the computation of the expected value of crud on each CTF face vs. base CTF/MAMBA estimates.
	\item  Straight forward method to propogate hi2lo-model-induced uncertainties though time due to Monte Carlo nature.
	\item  Enables probabilistic extreme value event estimation, although, these estimates were shown to have massive uncertainty.
\end{itemize}

\noindent Model deficiencies:  

\begin{itemize}
	\item  Not trivial to incorporate CRUD/TH feedback into the hi2lo model.  The hi2lo mapping must be modified to depend on the current crud state.  This CRUD/TH feedback was presently misssed in the current implementation.
	\item  Extreme upper and lower sample quantiles (upper and lower tail integrals) are plagued by high uncertainty which diminishes the applicability of this method to CILC.  
\end{itemize}

\section{Future Work}

\begin{itemize}
	\item  Investigate multi-state, many assembly cases.
	\begin{itemize}
		\item Requires extensive feature engineering in order to remain predictive under a wide variety of pin orientations and local core conditions.
	\end{itemize}
		
	\item Use parametric marginal models in place of the predicted piecewise conditional quantile function.  This would include identifying some class of distribution which accurately models the temperature and TKE variation on a CTF patch.
	
	\item Explore generating a heat transfer coefficient (HTC) map rather than predicting the temperature residual distribution.  This is avenue for future work is motivated by energy balance consistency concerns.  Under the current strategy the CTF energy balance is disobeyed because the mean surface temperatures are shifted to the CFD value in cases where the $CFD-CTF$ bias is nonzero.  In this scinario the Hi2lo model acts as a replacement for the Dittus-Boelter empirical heat transfer coefficient relation in CTF.
	
	\item Investigate using a spectral decomposition of the temperature and turbulent kinetic energy on each patch.  The machine learning problem becomes one of predicting the spectra in each CTF face as a function of local core conditions.  One could take advantage of the periodicity in the temperature and TKE surface fields as one traverses the rod in a spiral pattern.  These temperature and TKE spectra can be regarded as an isomorphism of the  joint distribution of temperature and TKE on the rod surface.
	
	\item Extend Hi2Lo model to quantify CILC risk.  Derive a CILC risk metric from the Hi2Lo crud result.
	
		\item Investigate including additional fudge factors in MAMBA crud kinetics to correct crud growth rates near spacer grids.  Tune fudge factors such that MAMBA + CTF to achieve the same integrated crud results as MAMBA ``fine'' + CFD for all entire relevant TH envelope.  This type of problem can be attacked using a standard Bayesian calibration technique.   This strategy would introduce introduce.  If the MAMBA CRUD kinetics change, the tuned fudge factors would need to be regenerated.
	
\end{itemize}
