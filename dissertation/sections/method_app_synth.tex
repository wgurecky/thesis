%! TEX root = ../dissertation_gurecky.tex

In this chapter the copula fitting, marginal reconstruction from quantiles, and Monte Carlo with importance sampling strategies are applied to a synthetic data set.  Synthetic data offers advantages over CFD born data for the puposes of testing and evaluating the efficacy of the proposed models.  Since synthetic data conforms to a known functional form with specified noise and bias the fitting and sampling routines can be easily checked against the expected result.
In this chapter we do not introduce any machine learning components and therefore, the copula and marginal models presented are not predictive but are reconstructions of the original synthetic data source.

The availability of synthetic data alleviates the need to generate comparatively expensive CFD results to test the hi2lo strategy.  Some aspects of CFD fields are preserved, including expected biases between CFD and CTF results that arise due to discrepancies in wall heat transfer closure models, among other differences, additionally turbulent dispersion of the temperature and surface shear distributions around spacer grids are specified.  Accounting for spatial auto-correlation in the surface fields was not pursued since the hi2lo procedures do preserve fine scale spatial features.  Any spatial autocorrelation present in the surface fields within a CTF face is lost.  The synthetic data is not a direct replacement to CFD data but serves as data source for method interrogation and integration testing.

The synthetic data generation routines specify the copula family, Kendall's $\tau$ rank correlation coefficient and marginal distribution parameters as a function of axial location and local TH conditions supplied by CTF.

\section{Generating Synthetic Data}

Synthetic data generation begins by first running standalone CTF on a single quarter symmetric pin and then augmenting the CTF result with tailored noise.  The augmented surface temperature field can be constructed by equation \ref{eq:synth_aug}.
\begin{equation}
    T(\mathbf z) = \mu_{ctf}(\mathbf z) + \varepsilon (\mathbf z; \theta(\mathbf z)) +  b(\mathbf z)
\label{eq:synth_aug}
\end{equation}
Where $\varepsilon(\mathbf z, \theta)$ is a user controlled spatially dependent residual distribution, with a mean of 0, which is
shifted by a bias function,
$ b(\mathbf z)$, where $\mathbf z=(z, \varphi)$ is the axial and azimuthal location on_the rod surface.
$_\theta$ represents user specified distribution parameters.

Equation \ref{eq:synth_aug} represents continuous random scalar surface field, however in practice a large number of indipendent and identically distributed samples are drawn in each CTF from the underlaying random field.  Individual temperature samples can be specified by equation ():

\begin{equation}
    T_{i_j} = \mu_{j_{ctf}} + b_j + X_{i_j};  \ X_{i_j} \sim \varepsilon_j(\theta_j)
    \label{eq:synth_aug_discrete}
\end{equation}
Where the index $j$ represents the $j^{th}$ CTF face on the rod, and the index $i$ is the sample index within the $j^{th}$ CTF face.  The distribution parameters, $\theta$ are fixed over a given CTF face.

The copula and marginal parameters are specified as inputs in the synthetic data generation tool.  To allow a great deal of flexibility, the copula and margin parameters can be provied as a function of local thermal hydraulic conditions and as a function of axial position.

\subsection{Single Pin Synthetic Data Set}

The original baseline CTF results are shown in figures \ref{fig:ctf_twall_orig} and \ref{fig:ctf_tke_orig}.  The CTF pin parameters are provided in table \ref{tab:pin_settings}.  The CTF result was produced from a quarter symmetric case, and therefore no azimuthal variation is observed.

\begin{table}[h]
    \begin{center}
        \caption{Single pin reference thermal hydraulic boundary conditions.}
        \begin{tabular}{|l|l|l|}
            \hline
            Setting & Value & Unit \\
            \hline
            Inlet Flow Rate & 0.3 & $[kg/s]$ \\
            Inlet Temperature & 565 & $[K]$ \\
            Pressure & 2250 & $[psia]$ \\
            Rod Outer Radius & 0.425 & $[cm]$ \\
            Pin Pitch & 1.26 & $[cm]$ \\
            Power Shape & constant & $[]$ \\
            Heat Flux & 85.86  & $[W/m^2]$ \\
            Rod Height & 3.6275 & $[m]$ \\
            Number of Grids & 3  & $[]$ \\
            Grid Locations & 2.0, 2.4, 2.8 & $[m]$ \\
            \hline
        \end{tabular}
    \label{tab:pin_settings}
    \end{center}
\end{table}

\begin{figure}[H]%
    \centering
    \subfloat[Axial CTF cladding surface temperature result.]{{\includegraphics[width=0.45\linewidth]{figs/synth/ctf_asm0_z_twall} }}%
    \qquad
    \subfloat[2D rod map of CTF result.]{{\includegraphics[width=0.45\linewidth]{figs/synth/ctf_pin_h5_temperature} }}%
    \caption[Single pin CTF baseline temperature result.  160\% nominal power conditions.]{Single pin CTF baseline temperature result.  160\% nominal power conditions.}%
    \label{fig:ctf_twall_orig}%
\end{figure}

\begin{figure}[H]%
    \centering
    \subfloat[Axial CTF cladding surface TKE result.]{{\includegraphics[width=0.45\linewidth]{figs/synth/ctf_asm0_z_tke} }}%
    \qquad
    \subfloat[2D rod map of CTF result.]{{\includegraphics[width=0.45\linewidth]{figs/synth/ctf_pin_h5_tke} }}%
    \caption[Single pin CTF baseline TKE result.  160\% nominal power conditions.]{Single pin CTF baseline TKE result.  160\% nominal power conditions.}%
    \label{fig:ctf_tke_orig}%
\end{figure}


The boundary heat flux was uniform at $85.86 [W/cm^2]$ which corresponds to approximately 160\% nominal PWR power conditions.

Next synthetic noise was generated using copula and marginal disrubution settings provided in table \ref{tab:synth_settings}. The complete synthetic data generation input deck for this case along with references to the code are provided in Appendix ().

\begin{table}[h]
    \begin{center}
        \caption{Per-span synthetic data generation settings.}
        \begin{tabular}{|l|l|l|l|l|}
            \hline
            \bf Span 1 & Node & Node location & Copula Settings  & Margin Settings \\
            \hline
            $N$: 8000  & 1  & 0.0 & $\Theta_c:$ Gaussian, $\theta:-0.6$ &  T:$\beta(5.0, 5.0)$ , k: $\mathcal{N}(0, 0.001)$ \\
                   & 2  & 2.0 & $\Theta_c:$ Gaussian, $\theta:-0.6$ &  T:$\beta(5.0, 5.0)$ , k: $\mathcal{N}(0, 0.001)$   \\
            \hline \hline
            \bf Span 2 & Node & Node location & Copula Settings  & Margin Settings \\
            \hline
             $N$: 8000 & 1  & 2.0 & $\Theta_c:$ Clayton-90, $\theta: 2.0$ &  T:$\beta(5.0, 2.7)$ , k: $\beta(1.75, 5.0)$ \\
            & 2  & 2.4 & $\Theta_c:$ Frank-90, $\theta: 8.0$ &  T:$\beta(5.0, 1.5)$ , k: $\beta(1.75, 5.0)$   \\
            \hline \hline
            \bf Span 3 & Node & Node location & Copula Settings  & Margin Settings \\
            \hline
             $N$: 8000 & 1  & 2.4 & $\Theta_c:$ Clayton-90, $\theta: 2.0$ &  T:$\beta(5.0, 2.7)$ , k: $\beta(1.75, 5.0)$ \\
            & 2  & 2.8 & $\Theta_c:$ Frank-90, $\theta: 8.0$ &  T:$\beta(5.0, 1.5)$ , k: $\beta(1.75, 5.0)$   \\
            \hline \hline
            \bf Span 4 & Node & Node location & Copula Settings  & Margin Settings \\
            \hline
            $N$: 8000 & 1  & 2.8 & $\Theta_c:$ Clayton-90, $\theta: 2.0$ &  T:$\beta(5.0, 2.7)$ , k: $\beta(1.75, 5.0)$ \\
            & 2  & 3.6 & $\Theta_c:$ Frank-90, $\theta: 8.0$ &  T:$\beta(5.0, 1.5)$ , k: $\beta(1.75, 5.0)$   \\
            \hline
        \end{tabular}
        \label{tab:synth_settings}
    \end{center}
\end{table}

In each span a separate copula mixture model was defined.  Samples are drawn with probability proportional to the inverse distance to the nearest specified node.

Let $d_{u_j}$ and  $d_{d_j}$ denote the distance from the centroid of the CTF face to the nearest upsteam and downstream copula nodes respectively.

The joint density at model in any given CTF face can be specified by equation \ref{eq:dist_weighted_synth}.

\begin{equation}
    \varepsilon_j = h_j \frac{d_{u_j}}{|d_{d} - d_{u}|} \varepsilon_u + \frac{d_{d_j}}{|d_{d} - d_{u}|} \varepsilon_d
    \label{eq:dist_weighted_synth}
\end{equation}
For simplicity, two copula nodes were sepcified per span though more are possible for a finer grained control over the marginal and copula distributions.  The copula node properties are given in table \ref{tab:synth_settings}.  The copula nodes were located at the span extrema. This node specification pattern allows the synthetic data to mimic the expected sharp change in copula and marginal distributions when moving across spacer grids as seen in the raw CFD data presented in chapter () in figures () and ().

The copula models were sampled in each span, the original CTF result was augmented with the synthetically generated noise in accordance with \ref{eq:synth_aug_discrete}.

\begin{figure}[H]%
    \centering
    \subfloat[Spatial axial augmented CTF result.]{{\includegraphics[width=0.45\linewidth]{figs/synth/pinH5TempOut} }}%
    \qquad
    \subfloat[2D rod map of sythetically augmented CTF result.]{{\includegraphics[width=0.45\linewidth]{figs/synth/cfd_pin_temperature} }}%
    \caption[Augmented CFD result.]{Augmented CTF temperature result.}%
    \label{fig:ctf_twall_aug}%
\end{figure}

\begin{figure}[H]%
    \centering
    \subfloat[Spatial axial augmented CTF result.]{{\includegraphics[width=0.45\linewidth]{figs/synth/pinH5TkeOut} }}%
    \qquad
    \subfloat[2D rod map of sythetically augmented CTF result.]{{\includegraphics[width=0.45\linewidth]{figs/synth/cfd_pin_tke} }}%
    \caption[Augmented CFD TKE result.]{Augmented CTF TKE result.}%
    \label{fig:ctf_tke_aug}%
\end{figure}

The augmented surface temperature and turbulent kinetic energy fields shown if figures \ref{fig:ctf_twall_aug} and \ref{fig:ctf_tke_aug} can be compared against the original CTF results provided in figures \ref{fig:ctf_twall_orig} and \ref{fig:ctf_tke_orig} respectively.  No azimuthal variations are present in the augmented result which would be present if a physics based model, such as CFD, were used.  Additionally, no spatial auto-correlation in the temperature and TKE cladding surface fields are included in the synthetic data.  Spatial autocorrelation could be captured with a kriging model in the future, however, the 1d nature of the crud simulation code used in this work dictates that the fine scale spatial detail in the surface fields are irrelivant when computing surface-integrated crud quantities.

In the following sections the goal is to fit separate copula and maginal models to the synthetic data in each CTF face.  Samples are then drawn from the fitted statistical models on each CTF face.  The samples are passed to the crud simulation code.  The reconsctuted crud results can be compared against crud results generated using the original synthtic surface field data.

\subsection{Single Pin Reconstruction}

Copula models are fit to the synthetic rod data in each CTF face according the the method of maximum likelyhood shown in equation () and AIC model selection, given in equatin ().  Additionally, in each CTF patch the sample quantiles of the local temperature, TKE, and boundary heat flux populations are computed by equation ().


The rod surface is divided into CTF faces.  In each face the emperical quantile distribution of temperature, turbulent kinetic energy, and boundary heat flux distributions are computed.  The number and spacing of quantiles used in the emperical quantile distribution is user specified.  Copula are fit to the synthetic data based on maximum likelihood and the Akiki information criterion in each CTF face.

\begin{itemize}
        \item (\checkmark) Compare CTF vs Synthetic CFD vs Hi2Lo Reconstructed CRUD results for a full pin for fixed TH conditions.
        \item (\checkmark) Show sensitivity of integrated crud mass and boron to the following.  Also show impact on uncertainty in these integrated results to:
        \begin{itemize}
                \item (\checkmark) Increase the number of samples drawn per patch
                \item (\checkmark) Force Gaussian copula vs. best fit copula approach.
        \end{itemize}
\end{itemize}

For the synthetic single pin data the difference between hi2lo and CTF predicted fractional surface area above a saturation temperature threshold ($T_{sat}$) is shown in figure \ref{fig:frac_a}.   CTF
is unable to capture small scale effects of the mixing vanes on the surface temperature distribution.

\begin{figure}[H]%
    \centering
    \subfloat[CTF predicted fractional area of each CTF face above the saturation point.]{{\includegraphics[width=0.45\linewidth]{figs/synth/hi2lo/ctf_pin_t_threshold} }}%
    \qquad
    \subfloat[Hi2lo predicted fractional area of each CTF face above the saturation point.]{{\includegraphics[width=0.45\linewidth]{figs/synth/hi2lo/hi2lo_pin_t_threshold} }}%
    \caption[]{Fraction area above the saturation point prediction comparison for the synthetic data set.}%
    \label{fig:frac_a}%
\end{figure}

The more significant the temperature thresholding behavior of the crud, the more important figure \ref{fig:frac_a} becomes with respect to computing the correct crud deposition rate on any given CTF surface.  This importance of this figure is echoed in figure (), which shows that a CTF and crud coupling without a hi2lo treatment of spacer grid effects might miss thresholding behavior entirely.

\begin{figure}[H]%
    \centering
    \subfloat[Hi2lo patch cladding wall temperature spatial distrubtion]{{\includegraphics[width=0.45\linewidth]{figs/synth/hi2lo/hi2lo_patch_t} }}%
    \qquad
    \subfloat[Hi2lo patch TKE spatial distrubtion.]{{\includegraphics[width=0.45\linewidth]{figs/synth/hi2lo/hi2lo_patch_tke} }}%
    \caption[]{Single hi2lo patch sample reordering demonstration.}%
    \label{fig:reshuffle_a}%
\end{figure}


\begin{figure}[H]
    \centering
    \includegraphics[width=0.8\linewidth]{figs/synth/hi2lo/hi2lo_pin_cmass}
    \caption[Single pin crud mass distribution from synthetic TH data.]{Single pin crud mass distribution from synthetic TH data at 160\% nominal power conditions at 300 days simulation time.}
    \label{fig:hi2lopincmass}
\end{figure}
\begin{figure}[H]
    \centering
    \includegraphics[width=0.8\linewidth]{figs/synth/hi2lo/hi2lo_pin_t}
    \caption{Single pin temperature distribution from synthetic TH data at 160\% nominal power conditions at 300 days.}
    \label{fig:hi2lopint}
\end{figure}
\begin{figure}[H]
    \centering
    \includegraphics[width=0.8\linewidth]{figs/synth/hi2lo/hi2lo_pin_tke}
    \caption{Single pin surface TKE distribution from synthetic TH data at 160\% nominal power conditions at 300 days.}
    \label{fig:hi2lopintke}
\end{figure}

\section{Single CTF Face Comparisons}
\begin{itemize}
    \item (\checkmark) Compare CTF vs Synthetic CFD vs Hi2Lo Reconstructed CRUD results on a single CTF face.
    \item (\checkmark) Show sensitivity of integrated crud mass and boron to the following tunable parameters.  Also show impact on uncertainty in these integrated results to:
    \begin{itemize}
        \item (\checkmark) The number of samples drawn per patch
        \item (\checkmark) Gaussian copula vs. best fit copula approach.
        \item (\checkmark) Number of quantiles used in the reconstruction of margins
    \end{itemize}
\end{itemize}

\begin{figure}[H]
    \centering
    \includegraphics[width=0.99\linewidth]{figs/synth/patch_scatter_0_4_0_6}
    \caption[Single patch synthetic CFD data vs hi2lo sampled data.]{Single patch synthetic data vs hi2lo sampled data from patch centered on the rod at (3.14 $[rad]$, 2.85 $[m]$) at 300$[days]$.}
    \label{fig:patchscatter}
\end{figure}

\subsubsection{Crud Copula Parameter Sensitivity}

Here the sensitivity of the crud result to the copula parameters is investigated.  Both the impact of the rank correlation coefficient, Kendall's $\tau$, and the Archimidean copula family are investigated.  The sensitivity results generated for the patch centered at $(\theta=3.14[rad], z=2.95[m])$ are shown in figure \ref{fig:patchcrudfit80}.  There is noise present in the crud predictions due to the Monte Carlo integration of equation \ref{eq:expected_crud} over the patch.  In this instance 2500 samples were used.  The crud is relatively insensitive to the choice of copula family, but the rank correlation coefficient is shown to have a significant influence on crud growth with an average boron deposition sensitivity of $\frac{\partial C_b}{\partial \rho_\tau} =$ -1.086e-7 $[g/cm^2/\tau]$ for this particular patch.  Accurately predicting Kendall's $\tau$ provided local core conditions is important.

\begin{figure}[H]
    \centering
    \includegraphics[width=0.7\linewidth]{figs/synth/patch_crud_fit_80}
    \caption{Single CTF face crud sensitivity to copula parameters.}
    \label{fig:patchcrudfit80}
\end{figure}

Next, two full single pin scenarios were considered. In the first scenario, shown if figure \ref{fig:crud_copula_fam_sensi}a, the best-fit copula on each patch as determined by the AIC metric is applied on each CTF face.  The second pin scenario enforces that a Gaussian copula model is used on every CTF face.  The crud results from these scenarios were then compared.  The data shows the choice of copula (between Gaussian, Frank, and Clayton) has a small overall impact on the total integrated rod boron mass.   The total integrated crud mass and crud boron mass for these scenarios at 300 days simulation time are given in table \ref{tab:crud_totals_copula}.

\begin{figure}[H]%
    \centering
    \subfloat[Best fit copula via AIC metric used in each CTF face.]{{\includegraphics[width=0.45\linewidth]{figs/synth/copula_compare/struct_pin_z_cmass_300_bestfit} }}%
    \qquad
    \subfloat[Gaussian copula used in each CTF face.]{{\includegraphics[width=0.45\linewidth]{figs/synth/copula_compare/struct_pin_z_cmass_300_gauss_only} }}%
    \caption[]{Influence of the choice parameters on the axial crud distribution.}%
    \label{fig:crud_copula_fam_sensi}%
\end{figure}


\begin{table}[h]
    \begin{center}
        \caption[Crud totals with different copula assumptions.]{Single pin crud totals at 300 days with different copula assumptions.}
        \begin{tabular}[h]{|l | l | l |}
            \hline
            Copula, $\Theta_c$ & Crud Boron Total: $C_B$ & Crud Mass Total: $C_m$ \\
            \hline
            Best Fit &  2.789529e-08 $[g]$ & 5.340151e-05 $[g]$ \\
            Gaussian &  2.783009e-08 $[g]$ & 5.327686e-05 $[g]$ \\
            \hline
            Rel Diff &  0.01724 $[\%]$ & 0.23396 $[\%]$ \\
            \hline
        \end{tabular}
        \label{tab:crud_totals_copula}
    \end{center}
\end{table}

The choice of the copula family, $\Theta_c$, has a negligible impact on the integrated crud results over a pin.  This result can reduce the complexity of the hi2lo model by removing the need to predict the correct copula family on each CTF patch in the core.  In section (), it is shown that CFD data exhibits a complex relationship between the best fitting copula family and the axial position along the rod.  This relationship is difficult to model using standard classification techniques, though further testing with a larger quantity of training data is warranted to ascertain if the copula family describing the dependence between the temperature and TKE fields on the rod surface can be accurately predicted given local core conditions.

\subsubsection{Crud Sample Size Study}

The number of samples, $N$, used to estimate the integral given in equation \ref{eq:expected_crud} is a parameter set at runtime of the hi2lo method.
As expected for a single rod the integrated crud variance is reduced by increasing the number of samples used per CTF face to estimate the integrated crud quantities of interest.   Section \ref{sec:Importance Sampling} demonstrates that improvements in sampling efficiency are possible by way of importance sampling.

Figure \ref{fig:cmprpintotalsviolinnsample} shows that increasing the number of samples used to in crud expected value Monte Carlo approximation yields an improvement in the variance of the predicted integrated crud results.  A single 300 day time step was conducted without re-sampling the underlaying density functions during this period.

\begin{figure}[H]
    \centering
    \includegraphics[width=0.7\linewidth]{figs/synth/nsample_study/cmpr_pin_totals_violin_nsample}
    \caption{Effect of sample size on the integrated crud results.}
    \label{fig:cmprpintotalsviolinnsample}
\end{figure}



\subsubsection{Importance Sampling}
\label{sec:Importance Sampling}

To obtain estimates for the efficiency gain offered by importance sampling to compute \ref{eq:expected_crud}, a singe patch was studied under synthetic TH data.

The choice of importance distribution is informed by the crud response.  To compute the integral \ref{eq:expected_crud} efficiently it is favorable to sample the TH distribution in regions which result in relatively large amounts of crud growth.  The response surface of the crud simulation code is presented in figures \ref{fig:crud_sensi1} to \ref{fig:crud_sensi3}.  Larger surface temperatures result in a higher crud growth rate.  Larger local TKE results in smaller crud growth rates due to the effects of erosion.  Additionally of note is the relatively small influence of the boundary heat flux.

\begin{figure}[H]%
    \centering
    \subfloat[Crud boron deposition sensitivity to temperature with TKE held fixed at $0.05 J/kg$]{{\includegraphics[width=0.45\linewidth]{../proposal/slides/seminar_slides/figs/dboron_dt_t} }}%
    \qquad
    \subfloat[Crud boron deposition sensitivity to TKE with temperature held fixed at $620 K$]{{\includegraphics[width=0.45\linewidth]{../proposal/slides/seminar_slides/figs/dboron_dt_tke} }}%
    \caption[]{Crud marginal response to varying temperature and TKE.}%
    \label{fig:crud_sensi1}%
\end{figure}

\begin{figure}[H]%
    \centering
    \subfloat[Crud boron deposition sensitivity with $q''=80 W/cm^2$.]{{\includegraphics[width=0.45\linewidth]{figs/crud/crud_t_tke_boron_response_80} }}%
    \qquad
    \subfloat[Crud boron deposition sensitivity $q''=120 W/cm^2$.]{{\includegraphics[width=0.45\linewidth]{figs/crud/crud_t_tke_boron_response_120} }}%
    \caption[]{Crud boron response to varying temperature and TKE.}%
    \label{fig:crud_sensi2}%
\end{figure}

\begin{figure}[H]%
    \centering
    \subfloat[Crud mass deposition sensitivity$q''=80 W/cm^2$.]{{\includegraphics[width=0.45\linewidth]{figs/crud/crud_t_tke_mass_response_bhf_80} }}%
    \qquad
    \subfloat[Crud mass deposition sensitivity $q''=120 W/cm^2$.]{{\includegraphics[width=0.45\linewidth]{figs/crud/crud_t_tke_mass_response_bhf_120} }}%
    \caption[]{Crud mass response to varying temperature and TKE.}%
    \label{fig:crud_sensi3}%
\end{figure}


The optimal importance distribution depends on the crud response surface and the TH probability density function.  Though an optimal importance distribution can be found [ref], the minimization problem is not solved in this work and is left as an avenue for future investigation.
Although the theoretically optimal importance distribution is forgone, a locally adaptive importance function was pursued in this work based on a distribution mixing approach.  With a known crud TH response surface existing in only $R^3$ it is feasible to design a near-optimal importance distribution by hand.

\begin{equation}
Q(\tilde{p}) = \sum_i^m a_{qi} Q_i(\tilde{p})
\end{equation}
Where $Q_i$ are the quantile functions as $a_{qi}$ are the weights.

Beta distributions with proscribed parameters are used in mixture with the original temperature and TKE density functions to produce a proposal density distribution for each patch.  The mixture weights can be adjusted.  This approach yields flexibility in the design of proposal density and by suitably tuning the parameters of the beta distributions.  Shown in figure \ref{fig:imp_sample2}, the sampling distribution can be skewed towards higher temperatures and lower TKE.

\begin{figure}[H]%
    \centering
    \subfloat[Temperature distributions.]{{\includegraphics[width=0.45\linewidth]{figs/imp_patch/temperature_importance_marginal_compare} }}%
    \qquad
    \subfloat[TKE distributions.]{{\includegraphics[width=0.45\linewidth]{figs/imp_patch/tke_importance_marginal_compare} }}%
    \caption[]{Proposal vs. original marginal distributions.}%
    \label{fig:imp_sample2}%
\end{figure}


\begin{figure}[H]%
    \centering
    \subfloat[Crud boron deposition rate.]{{\includegraphics[width=0.45\linewidth]{figs/imp_patch/bmass_sample_violin} }}%
    \qquad
    \subfloat[Crud mass deposition rate.]{{\includegraphics[width=0.45\linewidth]{figs/imp_patch/cmass_sample_violin} }}%
    \caption[]{Importance sampling trial results on a single CTF face.}%
    \label{fig:imp_sample1}%
\end{figure}

In figures \ref{fig:importancettkebmassscatter} the relative importance weight is denoted by the size of each point in the scatter plot.  Samples which have a small ratio $(h_i/\tilde h_i)$ appear as small points.  The sample weight is analogous to the rod surface area occupied by the sample.  In comparison, figure \ref{fig:originalttkebmassscatter} shows the same patch using a standard Monte Carlo sampling where each sample has the same weight.  The number of samples drawn in the upper tail of the temperature distribution is greater when importance sampling is applied, though these samples carry expectedly small sample weights.

\begin{figure}[H]
    \centering
    \includegraphics[width=0.99\linewidth]{figs/imp_patch/importance_t_tke_bmass_scatter}
    \caption[Importance sampled single patch crud result.]{Importance sampled single patch crud result.}
    \label{fig:importancettkebmassscatter}
\end{figure}

\begin{figure}[H]
    \centering
    \includegraphics[width=0.99\linewidth]{figs/imp_patch/original_t_tke_bmass_scatter}
    \caption[Standard Monte Carlo sampled patch crud result.]{Standard Monte Carlo sampled patch crud result.}
    \label{fig:originalttkebmassscatter}
\end{figure}

The importance sampling efficiency can be estimated by computing the variance ratio:  $\frac{\sigma^2_{MC}}{\sigma^2_{I}}$.  The variance of the patch-integrated crud result for the Monte Carlo and importance sampling schemes are provided figure \ref{fig:imp_sample1}.  The variance estimates were performed by running 1000 independent trials in which crud was grown on the patch for 300 days.  A total of 100 samples per patch per trial were used.  For the case studied, the application of importance sampling reduced the crud mass and boron mass sample variance by a factor of 2.02 [(4.979e-6)$^2$ / (3.503e-6)$^2$].  The mean crud predictions did not significantly deviate between two sampling schemes indicating that importance samples does not introduce any bias in the evaluation of the integral \ref{eq:expected_crud}.

The improvement in performance can be attributed to expending a larger proportion of the total available samples in the upper tail of the temperature distribution as this is a region which strongly contributes to crud growth.


\subsection{Single Pin with Time Stepping}

\subsubsection{Spatial Remapping with Time Stepping}
\begin{itemize}
        \item (\checkmark) Show influence of hot spot stationarity assumptions on crud growth.
\end{itemize}

\begin{itemize}
        \item (\checkmark) Show total integrated crud mass and crud boron as a function of time.
        \item (\checkmark) Show sensitivity of integrated crud mass and boron to the following.  Also show impact on uncertainty in these integrated results to:
        \begin{itemize}
                \item (\checkmark) Increase the number of samples drawn per patch.
                \item (\checkmark) Force Gaussian copula vs. best fit copula approach.
                \item (\checkmark) Increasing number of re-sampling time steps.
                We can re-sample from the joint $h(T, k, q'')$ distribution many times per TH state point.
                This reduces variance in the total integrated boron and crud mass quantities.
        \end{itemize}
\end{itemize}

\begin{figure}[H]
    \centering
    \includegraphics[width=0.7\linewidth]{figs/synth/cmpr_pin_totals_0_4_0_6}
    \caption[Total integrated crud born mass vs. time using approximately optimal remapping weights.]{Total integrated crud born mass vs. time using approximately optimal remapping weights ($\alpha_T=0.4, \alpha_{k}=0.6, \alpha_{q''}=0.0$).}
    \label{fig:cmprpintotals0406}
\end{figure}

\begin{figure}[H]
    \centering
    \includegraphics[width=0.7\linewidth]{figs/synth/cmpr_pin_totals_no_remap}
    \caption{Total integrated crud born mass vs. time without remapping samples.}
    \label{fig:cmprpintotalsnoremap}
\end{figure}

\begin{figure}[H]
    \centering
    \includegraphics[width=0.7\linewidth]{figs/synth/patch_fields_0_4_0_6}
    \caption{}
    \label{fig:patchfields0406}
\end{figure}
\begin{figure}[H]
    \centering
    \includegraphics[width=0.7\linewidth]{figs/synth/patch_fields_no_remap}
    \caption{}
    \label{fig:patchfieldsnoremap}
\end{figure}

\subsubsection{Time Step Re-Sample Frequency}

The frequency at which the distribution functions are sampled from on each CTF face influences the variance in the predicted integrated crud results.  To investigate this behavior, a parameter sweep was conducted in which the same pin was.  50 independent trials were conducted for each step size shown if table ().  Shown in figure \ref{fig:cmprpintotalsviolin} a smaller re-sampling steps size, $\Delta t_s$, results in a reduction in the variance of the rod integrated crud estimates at 300 days of simulation time.  It is also important to note that the variance of the rod integrated crud results increases as a function of time.

\begin{figure}[H]
    \centering
    \includegraphics[width=0.7\linewidth]{figs/synth/tstep_study/cmpr_pin_totals_violin}
    \caption{Influence of the re-sample frequency on the predicted integrated crud variance.}
    \label{fig:cmprpintotalsviolin}
\end{figure}

There is little computational penalty for reducing the re-sampling step size, and therefore taking a larger number of samples per VERA state, since the computational cost of sampling the underlaying joint density functions on each patch is far less than the time required to perform the crud simulations.


Section Takeaways:
\begin{itemize}
        \item Increasing the number of samples per patch decreases variance in total crud and total precipitated boron estimates.
        \item Increasing the number of re-sampling steps per VERA state point reduces variance in the final integrated crud results.
        \item After drawing samples from the predicted TH distribution a reordering of the samples on the rod surface is necessary to preserve hot spot stationarity.  This is required for an optimal time marching sampling strategy.
        \item Importance sampling was shown to reduce the variance in the integrated crud results.
        \item A synthetic data generation tool allows absolute control over the properties of joint distribution of TH boundary conditions which are fed into the crud simulation code.  Since the synthetic data has known properties, this data serves as an important data source for benchmarking and validation operations.
\end{itemize}
