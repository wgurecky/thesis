%! TEX root = ../outline.tex

%-----------------------------------------------------------------------------%
\pagebreak
\section*{Project Schedule}

Legend:
\begin{itemize}
    \item {\color{blue} Blue text: New task (added post-proposal).}
    \item (\checkmark)  Complete
    \item (\checkmark-)  Underway but incomplete
    \item ($\cdot$)  Not started. Future Work.
    \item (\xmark)  \sout{Not started.  Will not do.}
\end{itemize}
\bigskip

\begin{enumerate}
\item \textbf{(\checkmark) Literature review.}
\item \textbf{(\checkmark) Develop CFD-CRUD tools for training data generation and extraction.}
          Platform used to generate datasets that are
          required to inform a scale-bridging model.
    \begin{enumerate}
        \item (\checkmark) Create CFD data extraction and post processing utilities.
        \item (\checkmark) Develop STAR-CCM+ plugin to extract cladding surface and volumetric TH quantities from a CFD computation.
        \item (\checkmark) Compute the required volume and surface integrals to
              distill finely resolved CFD datasets into subchannel-like results.
    \end{enumerate}
\item \textbf{(\checkmark) Demonstrate differences between CFD and Subchannel predictions.}
    \begin{enumerate}
        \item (\checkmark) Compute differences between volume/surface averaged CFD quantities and subchannel predictions.
        \item (\checkmark) Identify deficiencies in subchannel predictions wrt. CRUD growth and CILC.
    \end{enumerate}
\item \textbf{(\checkmark) Investigate CRUD model sensitives.}
    \begin{enumerate}
        \item (\checkmark) Identify CRUD (MAMBA) sensitivities to TH boundary conditions
        \item (\checkmark) Produce correlation coefficients and scatter plot depictions of the relationship(s) between input
              and output of the CRUD model.
          \item (\checkmark) Identify TH conditions under which CFD scale CRUD predictions diverge from subchannel-CRUD results.
    \end{enumerate}
\item \textbf{(\checkmark) Write proposal document}
%
\item \textbf{(\checkmark) Methods implementation and demonstration}
    \begin{enumerate}
        \item (\checkmark) De-trend pointwise CFD datasets \& compute residual distributions.
        \begin{enumerate}
            \item (\checkmark) Moving averaged approach (assumes CTF and CFD will agree on the mean)
            \item (\checkmark) CTF mean approach (requires CTF runs at identical CFD sample points)
        \end{enumerate}
        %
        \item (\checkmark) Create a tool to generate synthetic CFD-like training data sets to facilitate
            testing of the regression tools.
        \begin{enumerate}
            \item (\checkmark) Develop flexible dependence modeling package capable of capturing skewed covariance behavior (Copula).
            \item (\xmark) \sout{Develop Kriging model to capture spatial auto-correlation (may not be necessary?)}
        \end{enumerate}
        %
        \item (\checkmark) Construct a regression model $M(\cdot)$ (Gradient boosted tree model [GBM]).
        \begin{enumerate}
            \item (\checkmark) Regress copula and marginal model parameters on local-average CTF core conditions.
            \item (\checkmark) Evaluate machine learning regressions' ability to recover joint TH distributions on any CTF patch.
        \end{enumerate}
        %
        \item (\checkmark-) Error Estimation \& Propagation
        \begin{enumerate}
        	\item (\checkmark-) How many quantiles are required for temperature and TKE margin reconstruction
                                - How do the number of quantiles used in the reconstruction impact the uncertainty in CRUD predictions?
            \item ($\cdot$) {\color{blue} Account for uncertainty in CFD sample quantiles. The sample quantiles are known to be distributed according to a Gaussian distribution.  It is possible to propagate this uncertainty through the hi2lo model into the crud predictions.}
                \item ($\cdot$) {Estimate uncertainty introduced by the gradient boosted/Copula model}
            \item ($\cdot$) {Propagate uncertainty to ensemble CRUD calculations}
        \end{enumerate}
        %----------------- POST PROPOSAL ------------------%
        \item (\checkmark-) Multi-state point execution
        \begin{enumerate}
            \item (\checkmark) Implement time stepping scheme.  Support ability to change the power profile/flow conditions at each time step.
            \item (\checkmark) Implement ability to propagate uncertainty through time steps.
            \begin{itemize}
                \item (\checkmark) Account for sampling-induced uncertainty and demonstrate how these errors compound over time.
                \item (\checkmark) {\color{blue} Account for hot spot stationarity in time.}

            \end{itemize}
        \item ($\cdot$) {\color{blue} Detail how this hi2lo strategy fits into a multi-physics framework with TH/Neutronic/CRUD feedbacks.}
        %
        \end{enumerate}
        % -------------------------------------------------%
        \item (\xmark) \sout{Compute areas of the machine learning models' prediction surface with highest
                             sensitivity to changes core state conditions \& develop capability to super sample these regions.  This is a design of experiments problem: Where do we run our CFD computations?}
        %
        \item (\xmark) \sout{Demonstrate ability to perform dimensionality reduction of the input space via PCA.}

    \end{enumerate}

\item \textbf{(\checkmark-) Methods analysis and improvement}
    \begin{enumerate}
        \item (\checkmark) Perform CFD informed subchannel based CRUD prediction (hi2lo) for a \emph{single pin}.
        \item (\checkmark) Perform large (assembly) scale CFD runs to supply realistic data to hi2lo model.  (this was done by Bob Salko et. al for a 5x5 case)
        \item (\checkmark) Tune hi2lo algorithm's hyperparameters (re-sampling coefficients, machine learning hyper parameters)
        \item (\checkmark) {\color{blue} Implement and demonstrate variance reduction strategy (importance sampling). }
        \item ($\cdot$) Validate CFD informed subchannel CRUD results against plant data (IF AVAILABLE).
    \end{enumerate}
\item \textbf{(\checkmark) Final hi2lo model construction and demonstration}
    \begin{enumerate}
        \item (\checkmark) Extend to assembly scale cases.  Identify why this problem is difficult.  From a feature engineering standpoint: Each pin \& each CTF face must be uniquely identified in the core by some set of exogenous variables.
        \item (\checkmark) Demonstrate CFD informed subchannel model on a full-height, single assembly problem.
        \item (\checkmark) Assess Crud/Axial boron (CIPS) prediction improvements vs. base CTF/MAMBA model.
    \end{enumerate}
\item \textbf{(\checkmark) Write Dissertation}
\end{enumerate}

\begin{sidewaysfigure}
%---------------------------     YEAR ONE        ------------------------------%
\begin{ganttchart}[
        inline,
        x unit=1.5cm,
        y unit title =0.8cm,
        y unit chart=0.8cm,
        hgrid,
        vgrid,
        time slot format=isodate,
        time slot unit=month
    ]{2016-06-01}{2017-07-30}
    \gantttitlecalendar*[]{2016-06-01}{2017-07-30}{year, month=shortname} \\
    \gantttitlelist{1,...,14}{1} \\
%%%%%%% Phase 1
\ganttgroup{Proposal Era}{2016-06-01}{2017-06-30} \\  %elem0
\ganttbar{Lit. Review}{2016-06-01}{2016-11-01} \\  %elem1
\ganttbar{CFD Post Proc. Dev.}{2016-06-01}{2016-08-20} \\  %elem2
\ganttlinkedbar{CTF vs CFD}{2016-08-20}{2016-10-30} \ganttnewline  %elem3
\ganttbar{MAMBA Sensitivity}{2016-08-01}{2016-10-30} \ganttnewline  %elem4
\ganttbar{Copula Model Dev.}{2016-10-01}{2017-01-10}  \\  %elem5
\ganttlinkedbar{Synthetic CFD}{2016-12-01}{2017-01-30} \\  %elem6
\ganttlinkedbar{Gradient Boosting Dev.}{2016-12-01}{2017-03-30}  \\  %elem7
\ganttlinkedbar{Link Models w/ Crud Sim.}{2017-03-01}{2017-06-30}  \\  %elem8
\ganttbar{Write Proposal}{2017-04-01}{2017-06-30}  %elem9
\ganttmilestone{Proposal}{2017-06-30} \ganttnewline  %elem10
%%%%%%%% Extra Task Linkages
\ganttlink[link mid=0.2]{elem2}{elem4}
\end{ganttchart}
%-----------------------------------------------------------------------------%
\end{sidewaysfigure}

\begin{sidewaysfigure}
%---------------------------     YEAR TWO       ------------------------------%
\begin{ganttchart}[
        inline,
        x unit= 1.2cm,
        y unit title =0.8cm,
        y unit chart=0.8cm,
        hgrid,
        vgrid,
        time slot format=isodate,
        time slot unit=month
    ]{2017-06-01}{2018-12-01}
    \gantttitlecalendar*[]{2017-06-01}{2018-12-01}{year, month=shortname} \\
    \gantttitlelist{13,...,31}{1} \\
%%%%%%% Phase 2
\ganttbar{ }{2017-06-01}{2017-06-30}  %elem0
\ganttmilestone{Proposal}{2017-06-30} \ganttnewline  %elem1
\ganttgroup{Dissertation Era}{2017-07-1}{2018-10-31} \\  %elem2
\ganttbar{Methods Refinement (Hi2lo Refactor)}{2017-07-01}{2018-02-28} \ganttnewline  %elem3
\ganttbar{5x5 CFD Runs}{2018-01-01}{2018-03-30} \ganttnewline  %elem4
\ganttbar{Time Stepping}{2017-08-01}{2018-02-28} \ganttnewline  %elem5
\ganttbar{Uncert. Prop}{2017-12-01}{2018-02-28} \ganttnewline  %elem6
\ganttbar{Importance Sampling}{2018-03-01}{2018-07-31} \ganttnewline  %elem7
\ganttbar{Hi2lo Runs}{2018-03-28}{2018-07-01} \ganttnewline  %elem8
\ganttlinkedbar{Generate Figs}{2018-06-01}{2018-08-31} \ganttnewline  %elem9
\ganttbar{Write Dissertation}{2018-06-01}{2018-10-30}  %elem10
\ganttmilestone{Defense}{2018-10-30} \ganttnewline  %elem11
\ganttlinkedbar{Final Rev.}{2018-10-30}{2018-11-30}  \ganttnewline  %elem12
\ganttmilestone{End.}{2018-11-30} \ganttnewline  %elem13
%%%%%%%% Extra Task Linkages
\ganttlink[link mid=0.1]{elem4}{elem8}
\ganttlink[link mid=0.2]{elem5}{elem7}
\ganttlink[link mid=0.2]{elem5}{elem8}
\end{ganttchart}
%-----------------------------------------------------------------------------%
\end{sidewaysfigure}


\pagebreak
