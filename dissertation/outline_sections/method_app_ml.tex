%! TEX root = ../outline.tex

\section{Machine Learning Model Trained on CFD Data Source}

\begin{itemize}
    \item (\checkmark-) Harvesting CFD data for training. Explain the pre-processing pipeline.
    \item ($\cdot$) Why use gradient boosted regression trees vs some other ML technique?
\end{itemize}

\subsection{Introduction Gradient Boosted Regression Trees}
\begin{itemize}
    \item (\checkmark-) Explain a classification and regression tree (CART).  (Move to theory or appendix?)
    \item (\checkmark-) Explain gradient boosting.  (Move to theory or appendix?)
\end{itemize}

\subsection{Feature Engineering}
\begin{itemize}
    \item ($\cdot$) Select a predictive variable set that can describe the behavior of the conditional quantiles and copula everywhere in the assembly.
    \item ($\cdot$) Justify choice of predictive variables.
    \item ($\cdot$) Perform parameter culling via first-pass Gradient boosting. (Maybe move to future work). Critical for multi-pin cases where it is
        important to different between each pin in the core.  For instance: what makes an edge pin special?  Can we predict the impact of neighboring
        guide tubes on the crud growth rate?
    \item ($\cdot$) Perform principle component analysis on the remaining predictive variables. (Maybe move to future work.)

\end{itemize}

\subsection{Training the Quantile Regressors}

\begin{itemize}
    \item ($\cdot$) Show temperature and TKE Q-Q plots elucidating bias introduced by the machine learning model at a variety of axial positions and local core conditions.
    \item (\xmark) \sout{Compute quantile uncertainty estimates associated with predicted quantiles provided by the trained ML model} (Unsure if feasible. Future work)
    \item (\xmark) \sout{Propagate ML model-induced quantile uncertainties} (Future work)
\end{itemize}

\subsection{Training the Copula Classifier}

\begin{itemize}
    \item ($\cdot$)  Show ML predicted copula vs CFD truth.  Compute the classification error rate.
        Summarize with a confusion matrix.
\end{itemize}


\subsection{Cross Validation}

An estimate of the interpolation error incurred when evaluating the trained models at unknown CFD states can be made by performing a leave one out cross validation study.
In this procedure, a single CFD/CTF pin pair is removed from the database and then the model is retrained on remaining data.  Following this retraining, a crud prediction is made 
at the missing pin's TH conditions.  The crud results are compared against crud results generated using the full training data set.  This process is repeated by
removing .  The differences are summarized and averaged to obtain a measure of model's predictive performance when applied to TH conditions that reside in the TH training envelope.

Note that this cross validation technique only ascertains interpolation errors within the TH envelope enclosed by the original full training set.  In other words, interpolation error
estimates cannot be extrapolated to core conditions that lay outside of the training set.  For a robust interpolation error analysis, a much larger training data set is required
that spans essentially all possible TH conditions encountered in an operational PWR.

\subsection{Multi State Point 5x5 Results}

\begin{itemize}
    \item ($\cdot$) Step a 5x5 assembly forward in time using pre-computed ctf-mpact results as input to the hi2lo model with the trained ML model supplying quantiles and copula.
    \item ($\cdot$) Report hi2lo crud model results under:
    \begin{itemize}
        \item ($\cdot$) Changes in power shapes.
        \item ($\cdot$) Changes in total rod power.
        \item ($\cdot$) Changes in flow rate.
    \end{itemize}

\end{itemize}
