%! TEX root = ../outline.tex

\section{Model Approach}

\begin{itemize}
        \item (\checkmark) Model overview.
    \item CTF estimates mean TH conditions everywhere in the core at a low spatial resolution.  The surrogate provides higher order moments about the mean.
    \begin{equation}
    \mathbf S(\mathbf z) = \underbrace{ \bm \mu(\mathbf{z})}_\text{CTF} +
    \underbrace{\varepsilon({\theta (\bm p(\mathbf z))})}_\text{CFD Informed} + \bm b(\mathbf{z})
    \end{equation}
    \begin{itemize}
        \item $\mathbf S$ is a three component vector field representing the cladding surface temperature, turbulent kinetic energy and boundary heat flux.
        \item $\mathbf z$ are spatial coordinates. $\mathbf p$ are a set of auxiliary predictors.  Auxiliary predictors are covariates that describe local core conditions and may be geometric or thermal hydraulic in nature.
        \item $\varepsilon$ is a random three-component vector field with components: $\{\varepsilon_T, \varepsilon_k, \varepsilon_{q''}\}$.
        \item $\varepsilon({\theta (\bm p(\mathbf z))})$ is a CFD informed model with $\theta$ representing free model parameters which must be fit to CFD data.
        \item $\bm b$ is bias ($\bm \mu_{CTF} - \bm \mu_{CFD}$)
        \item Field averages, $\bm \mu$, are piecewise constant over each CTF patch
        \item Note this is a additive model where Salko constructed a multiplicative model.
    \end{itemize}
\end{itemize}

Consider the case where the CFD results are normally distributed about the CTF results such that $\varepsilon \sim \mathcal N(0, \mathbf \theta(\mathbf p))$, where $\mathbf \theta(\mathbf p) = \bm \Sigma(\mathbf p)$ is a covariance matrix that depends on local core conditions.

Shifting this distribution by a constant vector $\bm c=\bm b + \bm \mu_{ctf}$, results in a new distribution denoted:
\begin{align}
    \left. h \right|_{\bm p} & = \mathcal N(\bm c, \bm \Sigma(\mathbf p)) \nonumber \\
    & = \left.
        \mathcal N \left(
        \begin{pmatrix}
            c_T \\
            c_k \\
            c_{q''}
        \end{pmatrix}
    ,
        \begin{pmatrix}
            \sigma_{T} \sigma_{T} & \sigma_{T} \sigma_{k} & \sigma_{T} \sigma_{q''} \\
            \sigma_{k} \sigma_{T} & \sigma_{k} \sigma_{k} & \sigma_{k} \sigma_{q''} \\
            \sigma_{q''} \sigma_{T} & \sigma_{q''} \sigma_{k} & \sigma_{q''} \sigma_{q''}
        \end{pmatrix}
    \right)
    \right|_{\mathbf p}
\end{align}


The goal is to estimate the expected crud on each CTF patch given by equation \ref{eq:expected_crud}.
This work could be summarized as an algorithm to estimate the following integral on each CTF face:

\begin{eqnarray}
        total\ crud\ [grams] = A \mu_g\ = A \E[g(\mathbf X|g_o, \mathbf I, \delta t)] \nonumber \\
        = A \iiint g(\mathbf X|g_o, \mathbf I, \delta t) h(\mathbf X|\theta) d \mathbf X
        \label{eq:expected_crud}
\end{eqnarray}
let $\mathbf X= \{T, k, q''\}$ denote a random vector of temperature, TKE, and BHF. $\mathbf I$ represents known crud parameters, $g_o$ is the crud state at the start of the time step and $\theta$ are distribution parameters.  The goal is to predict $\theta$, thereby specifying $h(\mathbf X|\hat \theta)$, in every CTF face.  The CRUD model, $g(\cdot)$, is common to all CTF faces.

To compute the total crud/boron in each CTF face:
\\

\begin{algorithm}[H]
\KwData{(1) Pre-process training set $\theta(\mathbf p)$.  (2) Train model $M(\mathbf p ; \gamma)$:  $ argmin_\gamma ||M(\mathbf p; \gamma) - \theta(\mathbf p)|| $}

\For{CTF face, $i$}{
Evaluate ML model $\hat \theta_i \leftarrow M(\mathbf p_{i})$ \;
Reconstruct $\hat h_i(x|\hat \theta_i)$ \;
Draw samples $\mathbf X \sim \hat h_i$ \;
Evaluate \ref{eq:expected_crud} via Monte Carlo approximation \;
}
\end{algorithm}
Where $\gamma$ are free parameters in the ML model.
\bigskip

In addition to improving the expected value prediction of CRUD on each CTF patch vs the CTF standalone case, the model provides the capability to estimate the likelihood of extreme value events (i.e. $\mathcal P_f \propto Pr(g(x) > g^*)$, where $g^*$ is some critical crud thickness and $\mathcal P_f$ is a cladding failure probability) on the rod surface.

A significant challenge is computing an estimate for $Var(\mathcal P_f) = \E(\mathcal P_f - \E(\mathcal P_f))^2$.


\section{Construction of the Hi2lo Map}

\subsection{Capturing Dependence Between Random Variables}
\begin{itemize}
        \item (\checkmark) Sklar's theorem.
        \item (\checkmark) Assumptions:
        \begin{itemize}
                \item Capture Temperature and TKE dependence structure.
                \item Assume 1) Temperature is uncorrelated with BHF.  2) TKE is uncorrelated with BHF.
                \item Justify 1) showing actual CFD data. 2) Relative variations in BHF are very small over a CTF face $(+/- 5\%)$.  3) Sensitivity of CRUD to BHF is small relative to sensitivity of CRUD to surface temperature.
        \end{itemize}
        \item (\checkmark) Show fallout of these assumptions on hi2lo model.
        \begin{equation}
                h(T, k, q'') = f_T f_k, f_{q''} \prod_c c_i(u_i, v_i)
        \end{equation}
        Where each bivariate copula model in $\prod c_i $ is known as a pair copula construction.

        Applying the aforementioned assumptions results in: $c(u_T, u_{q''}) = 1$ and $c(u_{k}, u_{q''}) = 1$. The simplified joint density is given by:
        \begin{equation}
                h(T, k, q'') \approx  f_T f_k, f_{q''} c(u_{T}, v_{k})  \cdot 1 \cdot 1
    \end{equation}
\end{itemize}


\subsection{Sample Quantiles}

\subsubsection{Non Parametric Representations of Univariate Distributions}

\begin{itemize}
    \item (\checkmark-) Introduce non-parametric representation of univariate distributions through sample quantiles.
    \item ($\cdot$) Explain typical alternatives and why they are not suitable:
        \begin{itemize}
            \item Compute sample moments.  Use method of moments to fit a model to sample moments.
            \item Compute sample cumulants.  Use sample cumulants to build an Edgeworth series.
            \item Do cumulants or traditional moments behave in a predictable manner as a function of local core conditions?  Do the quantiles behave in a predictable manner?
        \end{itemize}
\end{itemize}

\begin{itemize}
        \item (\checkmark) Detail how to compute sample quantiles.
        \item (\checkmark) Show reconstruction of the CDF from sample quantiles.
        \begin{itemize}
           \item Peicewise linear CDF leads to histogram PDF.
           \item PCHIP spline CDF preserves monotonicity of CDF and results in a smooth PDF. \cite{Fritsch80}
        \end{itemize}
        \item (\checkmark) Sampling from a known CDF via inverse transform sampling.
        \item (\checkmark-) Show that the order statistics are distributed according to a Gaussian distribution.
        \item (\checkmark-) Show that sample quantiles also follow a Gaussian distribution.  This directly follows from the distribution of the order statistics.
        \item ($\cdot$) Show the impact of the uncertainty in where the sample quantiles fall given CFD data on crud growth.  The reconstructed CDFs which comprise the hi2lo mapping are essentially fuzzy which means the temperature, TKE, and BHF distributions are artificially smeared out (artificially high densities in the tails).
\end{itemize}

\subsection{Monte Carlo CRUD Estimation}

\begin{itemize}
        \item (\checkmark-) Show importance sampling scheme to estimate \ref{eq:expected_crud}.  (Move to Appendix?)
        \begin{equation}
        \E(g(x)) \approx \frac{1}{N} \sum_i^N g(x_i) \frac{h(x_i)}{\tilde h(x_i)}, \ x \sim \tilde{h}
        \end{equation}
        \item (\checkmark-) Show examples of $g(x)$, $h(x)$ and $\tilde h(x)$ on a single CTF face.
        \item (\xmark) \sout{Find optimal proposal density distribution, $\tilde{h^*}$.}
\end{itemize}


\subsection{Propagating CRUD Through Time}
\subsection{Hot Spot Stationarity}
\begin{itemize}
        \item (\checkmark) Elucidate why we need a spatial remapping scheme. As time progresses it is likely that hot spots on the rod surface will stay stationary since the hot spot locations are predominately governed by geometric factors.  Note this applies to the time-averaged fields.
        \item (\checkmark) Detail how this mapping is produced:  This is a mapping between from a sample space to euclidean space on the rod surface.  This process requires that we can compute some multivariate order statistics by first performing weighted sum of three component random variables to generate a new univariate random variable for which traditional order statistics are available.
\end{itemize}

\subsection{Smearing Over Azimuth}

\begin{itemize}
        \item ($\cdot$) Smear the CFD data and CTF results over all azimuthal angles.  This will increase the number of CFD samples used for the hi2lo mapping construction by a factor of four, and therefore decrease the uncertainty in the sample quantiles by a factor of 2.  See section on the theoretical distribution of sample quantiles.  The sample quantiles follow $1/\sqrt N$ behavior.
        \item ($\cdot$) Since the primary goal of this method is to predict CIPS, it is possible neglect azimuthal variation entirely and focus solely on axial variation.  In this case, for any given pin, each CTF face at a fixed axial level will all get the same crud thickness and crud boron from the hi2lo model.
\end{itemize}

\section{Probability of Exceeding a Threshold}

\begin{itemize}
    \item (\checkmark-) The sampled temperatures may be tallied over each CTF face to estimate the fractional area that exceeds some threshold temperature.
    The proability of exceeding a threshold temperature can be obtained by integrating the PDF:
    \begin{equation}
        p_e &= Pr(T > T^*) = 1 - \int_0^{T^*} f_T dT \\
    \end{equation}
    Let $q_p = F_T^{-1}(1 - p_e)$
    denote the quantile associated with the threshold probability, $p_e$.
    $F_T^{-1}$ is the inverse CDF function and $f_T$ is the probability density function of temperature on the patch.

    The sample quantile corresponding to $p_e$ is distributed according to:
    \begin{align}
        q_p &\sim \mathcal N \left( F_T^{-1}(p), \sigma^2_{q_p} \right) \\
        \sigma^2_{q_p} &= \frac{p(1 - p)}{n[f_T(F_T^{-1}(p))]^2}
    \end{align}
    The variance of the exceedance probability estimate can be found by standard propagation of uncertainty principles:
    \begin{align}
        \sigma_p^2 = \left(\frac{\partial p_e}{\partial q_p} \right)^2 \cdot \sigma_{q_p}^2
    \end{align}
    Where
    \begin{align}
    \frac{\partial p_e}{\partial q_p} &= \frac{\partial}{\partial q_p} \left( 1 - \int_0^{q_p} f_T dT \right) \nonumber \\
    &= \frac{\partial}{\partial q_p} \left( -F_T(q_p) + F_T(0) \right) \nonumber \\
    &= -f_T(q_p)
    \end{align}
    This dictates that estimates of extreme upper tail integrals carry large relative uncertainties.
    \item The distribution for $p_e$ can be obtained by a change of variables, since $p_e$ and $q_p$ are related by:
    \begin{equation}
     q_p = F_T^{-1}(1 - p_e)
    \end{equation}
    Where $F_T^{-1}$ is the inverse CDF.  $p_e$ is distributed according to:
    \begin{align}
        f_{p_e} = f_{q_p} \left( F_T^{-1}(1-p_e), \sigma_{q_p}(p_e) \right) \cdot \abs{ \frac{\partial}{\partial p_e} F^{-1}(1-p_e) }
    \end{align}
\end{itemize}
