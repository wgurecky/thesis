%\documentclass[12pt,a4paper]{report}
\documentclass[12pt]{report}
% \usepackage[a4paper, total={6.00in, 9in}]{geometry}
\usepackage{pdflscape}
\usepackage{setspace}
\usepackage[utf8]{inputenc}
\usepackage{amsmath}
\usepackage{amsfonts}
\usepackage{amssymb}
\usepackage{mathtools}
\usepackage{listings}
\usepackage{hyperref}
\usepackage{pgfgantt}
\usepackage{rotating}
\usepackage{bm}
\usepackage{bbm}
\usepackage{lipsum}
\usepackage{pifont}
\usepackage{float}
\usepackage{color}
\usepackage{graphicx}
\usepackage{subfig}
\usepackage[]{algorithm}
\usepackage{algorithmic}
\usepackage{array}
\newcolumntype{B}{>{\centering\arraybackslash}m{3cm}}
\newcolumntype{C}{>{\centering\arraybackslash}m{4cm}}
\newcolumntype{L}{>{\arraybackslash}m{8cm}}
\usepackage[version=3]{mhchem} % Package for chemical equation typesetting \ce{}
\usepackage{siunitx} % Provides the \SI{}{} and \si{}
\usepackage[normalem]{ulem}  % stike-through by \sout{}

% diss template
\usepackage{utdiss2}
\author{William Ladd Gurecky}  	% Required

\address{9905 Chukar Circle\\ Austin, Texas 78758}  % Required

\title{ A CFD-Informed Model for Improving Subchannel Resolution Crud Prediction}
% Required

%%%%%%%%%%%%%%%%%%%%%%%%%%%%%%%%%%%%%%%%%%%%%%%%%%%%%%%%%%%%%%%%%%%%%%
% NOTICE: The total number of supervisors and other members %%%%%%%%%%
%%%%%%%%%%%%%%% MUST be seven (7) or less! If you put in more, %%%%%%%
%%%%%%%%%%%%%%% they are put on the page after the Committee %%%%%%%%%
%%%%%%%%%%%%%%% Certification of Approved Version page. %%%%%%%%%%%%%%
%%%%%%%%%%%%%%%%%%%%%%%%%%%%%%%%%%%%%%%%%%%%%%%%%%%%%%%%%%%%%%%%%%%%%%

%%%%%%%%%%%%%%%%%%%%%%%%%%%%%%%%%%%%%%%%%%%%%%%%%%%%%%%%%%%%%%%%%%%%%%
%
% Enter names of the supervisor and co-supervisor(s), if any,
% of your dissertation committee. Put one name per line with
% the name in square brackets. The name on the last line, however,
% must be in curly braces.
%
% If you have only one supervisor, the entry below will read:
%
%	\supervisor
%		{Supervisor's Name}
%
% NOTE: Maximum three supervisors. Minimum one supervisor.
% NOTE: The Office of Graduate Studies will accept only two supervisors!
% 
%
\supervisor
{Derek Haas}

%%%%%%%%%%%%%%%%%%%%%%%%%%%%%%%%%%%%%%%%%%%%%%%%%%%%%%%%%%%%%%%%%%%%%%
%
% Enter names of the other (non-supervisor) members(s) of your
% dissertation committee. Put one name per line with the name
% in square brackets. The name on the last line, however, must
% be in curly braces.
%
% NOTE: Maximum six other members. Minimum zero other members.
% NOTE: The Office of Graduate Studies may restrict you to a total
%	of six committee members.
%
%
\committeemembers
[Ben Leibowicz]
[Sheldon Landsberger]
[Kevin Clarno]
{Stuart Slattery}

%%%%%%%%%%%%%%%%%%%%%%%%%%%%%%%%%%%%%%%%%%%%%%%%%%%%%%%%%%%%%%%%%%%%%%
\previousdegrees{}
% \previousdegrees{B.S., M.S.}
% The abbreviated form of your previous degree(s).
% E.g., \previousdegrees{B.S., MBA}.
%
% The default value is `B.S., M.S.'

%\graduationmonth{...}      
% Graduation month, either May, August, or December, in the form
% as `\graduationmonth{May}'. Do not abbreviate.
%
% The default value (either May, August, or December) is guessed
% according to the time of running LaTeX.

%\graduationyear{...}   
% Graduation year, in the form as `\graduationyear{2001}'.
% Use a 4 digit (not a 2 digit) number.
%
% The default value is guessed according to the time of 
% running LaTeX.

%\typist{...}       
% The name(s) of typist(s), put `the author' if you do it yourself.
% E.g., `\typist{Maryann Hersey and the author}'.
%
% The default value is `the author'.



% image paths
\graphicspath{{../proposal/}{./}{./figs/crud}}


%===========================CUSTOM CMDS=======================================%
\DeclareMathOperator*{\E}{\mathbb{E}}
\newcommand{\xmark}{\ding{55}}%

\author{William Ladd Gurecky}

\begin{document}
\copyrightpage          % Produces the copyright page.
\commcertpage           % Produces the Committee Certification
%   of Approved Version page (doctoral)
%   or Signature page (masters).
%		20 Mar 2002	cwm

\titlepage              % Produces the title page.
	
\renewcommand{\thepage}{\roman{page}}
%=================================TITLE=======================================%
%-----------------------------------------------------------------------------%
% Title.tex
% Author: William Gurecky
% Info:  Title page, table of contents
% Changlog:

%-----------------------------------------------------------------------------%
\begin{titlepage}
	\centering
	{\scshape\LARGE The University of Texas at Austin \par}
	\vspace{1cm}
	{\scshape\Large Nuclear \& Radiation Engineering \par}
	\vspace{1.5cm}
	{\huge\bfseries A Gradient Boosted Copula Model for CFD-Informed CRUD Predictions \par}
	\vspace{2cm}
	{\Large William L. Gurecky \par}
	\vfill

	\begin{flushright}
	Dissertation Committee \par
	\bigskip
	Dr.~Derek \textsc{Haas}, Supervisor \par
	Dr. Sheldon Landsberger \par
	Dr. Benjamin Leibowicz \par
	Dr. Kevin Clarno \par
	Dr. Stuart Slattery \par
	\end{flushright}
	\vfill
	{\large \today\par}
\end{titlepage}
%-----------------------------------------------------------------------------%
\pagebreak
\tableofcontents
\pagebreak

%-----------------------------------------------------------------------------%
\clearpage
\vspace*{\fill}
\thispagestyle{empty} % suppress showing of page number
\begin{quotation}
\em % optional -- to switch to emphasis (italics) mode
In general, when building statistical models, we must not forget that the aim is to understand something about the real world. Or predict, choose an action, make a decision, summarize evidence, and so on, but always about the real world, not an abstract mathematical world: our models are not the reality—a point well made by George Box in his oft-cited remark that “all models are wrong, but some are useful”.
\medskip

--  David Hand
\end{quotation}
\vspace*{\fill}

\pagebreak


%! TEX root = ../dissertation_gurecky.tex

%\section*{Acknowledgments}
%\addcontentsline{toc}{chapter}{Acknowledgments}
\acknowledgments

I would like to acknowledge Dr.~Kevin Clarno, Dr.~Stuart Slattery, Dr.~Derek Haas, Dr.~Sheldon Landsberger, Dr.~Ben Leibowicz and Dr.~Robert Salko for their technical guidance and general advice throughout my graduate studies.

Special thanks to Dr.~Erich Schneider who fostered many a students' interest in nuclear engineering \& software development, including mine.

This research was supported by and performed in conjunction with the Consortium for Advanced Simulation of Light Water Reactors (http://www.casl.gov), an Energy Innovation Hub (http://www.energy.gov/hubs) for Modeling and Simulation of Nuclear Reactors under U.S. Department of Energy Contract No. DE-AC05-00OR22725.
\newpage


%! TEX root = ../dissertation_gurecky.tex

\utabstract
\index{Abstract}%
\indent
The development of a physics-directed, statistically based,
surrogate model of the small scale flow features that impact crud growth is presented in this work. 
The objective of the surrogate is to provide additional details of the rod surface temperature, heat
flux and near-wall turbulent kinetic energy fields which cannot be explicitly captured by the subchannel code, CTF. 

Operating as a mapping from the high fidelity CFD data to the low fidelity subhcannel grid, the surrogate provides CFD-informed boundary conditions to the crud model executed on the subchannel pin surface mesh. The surface temperature, heat
flux and turbulent kinetic energy, henceforth referred to as the Fields of Interest (FOI),
govern the growth rate of crud on the surface of the rod and the
precipitation of boron in the porous crud layer. Therefore the surrogate predicts the behavior of the
fields of interest as a function of position in the core and local thermal-hydraulic (TH) conditions.

The subchannel code is substantially faster to execute than CFD
and produces an estimate for all the relevant TH quantities at a coarse spatial resolution everywhere in
the core.  The solution provided by CTF is augmented by a predicted stochastic
component of the FOI informed by CFD results to provide a more detailed description of the target
FOIs than CTF can provide alone.  To this end, a novel method based on the marriage of copula and
gradient boosting techniques is proposed. This approach forgoes a spatial interpolation approach
for a statistically driven approach which predicts the fractional area of a rod’s surface in excess of some
critical temperature - but not precisely where such maxima occur on the rod surface.
The resultant model retains the ability to account for the presence
of hot and cold spots on the rod surface induced by turbulent flow downstream of spacer grids when
producing crud estimates. Sklar’s theorem is leveraged to decompose multivariate probability densities
of the FOI into independent copula and marginal models. The free parameters within the copula model
are predicted using a combination of supervised regression and classification machine learning techniques
with training data sets supplied by a suite of pre-computed CFD results spanning a typical PWR TH
envelop.

\pagebreak
\tableofcontents
\pagebreak

%-----------------------------------------------------------------------------%
%! TEX root = ../dissertation_gurecky.tex

% acryn.tex
% Author: William Gurecky
% Info:  Acronyms
% Changlog:

%-----------------------------------------------------------------------------%
\section*{Acronyms}
\begin{tabular}{l l}
BHF & Boundary Heat Flux \\
CASL & Consortium for Advanced Simulation of LWRs \\
CDF  & Cumulative Density Function \\
CFD &  Computational Fluid Dynamics \\
CILC & Crud Induced Local Corrosion \\
CIPS & Crud Induced Power Shift \\
CRUD & Chalk River Unidentified Deposit \\
CTF &  Coolant boiling in rod arrays–Two Fluid (COBRA-TF) \\
FOI &  Field of Interest \\
GBRM & Gradient Boosted Regression Model \\
GBRT & Gradient Boosted Regression Tree \\
HTC  & Convective Heat Transfer Coefficient \\
LANL & Los Alamos National Laboratory \\
LOO & Leave-one-out \\
LOOCV & Leave-one-out cross validation \\
LS  &  Least Squares \\
LWR & Light Water Reactor \\
ML  &  Maximum Likelihood \\
ORNL & Oak Ridge National Laboratory \\
PDF  &  Probability Density Function \\
PWR  & Pressurized Water Reactor \\
ROM &  Reduced Order Model \\
RV  & Random Variable \\
TH  &  Thermal Hydraulic \\
TKE &  Turbulent Kinetic Energy \\
VERA & Virtual Environment for Reactor Applications \\
\end{tabular}

\pagebreak

%-----------------------------------------------------------------------------%
\section*{Nomenclature}
\begin{tabular}{l l}
$c(\cdot)$ & Copula density function \\
$C(\cdot)$ & Copula cumulative density function \\
$\varphi(\cdot)$ & Copula generator function \\
$f(\cdot)$ & Marginal density function \\
$F(\cdot)$ & Marginal cumulative density function \\
$\mathcal G(\cdot)$ & CRUD generator function \\
$h(\cdot)$ & Joint density function \\
$H(\cdot)$ & Joint cumulative density function \\
$\mathcal F(\cdot)_M$ & Gradient boosted model \\
$\mathcal R$ & Mapping from sample space to a location on the rod surface \\
$t$ & Time \\
$T$ & Temperature \\
$k$ & Turbulent Kinetic Energy \\
$q''$ & Boundary heat flux \\
$Q_{\tau}$ & Quantile function (inverse CDF) \\
$q_{\tau}$ & The $\tau^{th}$ Quantile \\
$\rho_{\tau}$ & Kendall's tau \\
$\theta$ & Marginal distribution parameter \\
$\theta_c$ & Copula shape parameter \\
$\Theta_c$ & Archimedean Copula family \\
$\mathbf C$ & CRUD state vector \\
$C_m$ & CRUD mass density \\
$C_b$ & CRUD boron density \\
$C_t$ & CRUD thickness \\
$\mathbf X$ & Random vector $\mathbf X = \{X_0, ... X_n\}$. \\
$X_i$ & Random variable \\
\end{tabular}


\listoftables
%\addcontentsline{toc}{chapter}{List of Tables}

\listoffigures
%\addcontentsline{toc}{chapter}{List of Figures}

\pagebreak
%-----------------------------------------------------------------------------%


%=================================BODY========================================%
\renewcommand{\thepage}{\arabic{page}} % Arabic numerals for page counter
\setcounter{page}{1}
\onehalfspacing
% \doublespacing  % UT formatting spec
%-----------------------------------------------------------------------------%
\chapter{Introduction}
\label{chap:intro}
CRUD 

A literature review of Hi2Low modeling approaches to CFD-informed subchannel
problems is given in \autoref{chap:lit}.  Several advances in CRUD modeling are
also discused in \autoref{chap:lit} including a CASL developed high fidelity
CFD based CRUD tool currently under development.  Additionally a review
covering the non-parametric relm of gradient boosted regression trees is
provided.  Finally, work pertaining to correlated tail risk analysis is also
provided with a focus on engineering rather than financial applications.

In \autoref{chap:theory}, the overarching Hi2Low pipeline is introduced.  

Work that has been performed in support of this Hi2Low strategy is reviewed in
\autoref{chap:work}.  The development of a CFD data extraction tool with
additional post processing capabilities is provided.  This tool enables the
comparison of CFD result with CTF results.  Comparisons between CFD and CTF
subchannel results are presented to identify key flow regimes in which the
coarse subchannel based approach produces erroneous CRUD growth rates.

A demonstration of the copula based methodology to predict CRUD grow rates on a
single rod provided a single TH state point is also given in
\autoref{chap:work}.

Finally, future work is discussed in \autoref{chap:fw} with a detailed plan of
action laid out to clearly define project goals.

\section{Benefit and Novelty}

Accurately predicting CRUD induced power shift (CIPS) requires accurate CRUD
thickness and boron deposition estimates in all regions of the core.  Core wide
phenomena are out of scope for high fidelity CFD-Neutronic-CRUD coupled
simulations.  Instead, we rely on CASL's CTF-MAMBA1D coupling capability to
predict CIPS throughout a cycle.  CTF is not capable of simulating the detailed
flow patterns downstream the spacer grids.  Details in the flow field
downstream of spacer grids have important consequences for CRUD growth and
erosion, namely the local depression in surface temperatures and locally
increased shear stresses.  CRUD growth is extremely sensitive to the surface
temperature around the saturation point.  Therefore, growing crud at the bulk
average predicted TH conditions given by CTF in a coarse axial segment may not
yield the correct amount of CRUD growth in that region.

\section{CASL Challenge Problems}

CASL selected several problems identified by industry partners as critical
inadequately understood engineering scale phenomena which would provide financial and
safety benefits to the nuclear power industry if resolved.  An list of the CASL
challenge problems is provided in [ref].

The Virtual Environment for Reactor Applications (VERA) is a key component of
CASL's technical portfolio.  This meta-package integrates a variety of physics
packages and multiphysics coupling options to form a robust reactor simulation
capability.  For multi-cycle depletion computations, VERA relies upon MPACT, a
2D-1D method of characteristics neutronics package, coupled with a subchannel
thermal hydraulics code COBRA-TF (CTF).  An integrated CRUD modeling capability
is provided by MAMBA to address the CRUD induced power shift challenge problem.

To reduce computation times, the subchannel TH code discretizes the reactor
domain into large, centimeter scale finite volumes.  A second order upwinding
scheme is employed along with the SIMPLE solution algorithm to resolve pressure
velocity coupling.  This coarse discretization scheme means that sub-centimeter scale
thermal hydraulic effects of the spacer grids on CRUD are averaged over
large regions on the fuel rods' surfaces.  Though small scale phenomena are not
explicitly modeled, they are approximately accounted for in a variety of empirically derived
closure relations.  In effect, a single constant estimate for the mean thermal
hydraulic conditions (Temperature, boundary heat flux, and wall shear stress)
are obtained in each finite volume.

Previous Hi2Low TH focused work in CASL focused on utilizing experimental or CFD
datasets to improve closure models in CTF.  These studies leverage a multitude of
dimensionality reduction and regression techniques
to fit a parametric model to the accepted gold-standard empirical data.  This approach
is adequate for correcting biases in the bulk-average behavior of the flow (due to
the previously neglected physics).  Examples of such Hi2Low models are given in
\autoref{chap:lit}.

The traditional approach must be slightly modified to accommodate the CILC and
CIPS challenge problems.  Here arises the need to retain not only the effect of
fine-scale physics on the bulk, but also to predict if certain
temperature or TKE \emph{thresholds} are exceeded in a given (CTF coarse)
volume.  Furthermore, for a complete treatment of thermal hydraulic impacts on
CRUD growth, the scale-bridging model must describe the \emph{frequency
distribution} of extreem TH events above a given threshold.

\subsection{CIPS \& CILC Challenge Problems}

Outline of CIPS and CILC challenge problem.  Role of CFD informed subchannel.

The CRUD induced power shift phenomina and CRUD induced local corrosion
challenge problems.  Currently avalible tools are unable to properly account
for spacer grid effects on the errosion, in subchannel-scale models.

To redemy this issue, a scale-bridging model is proposed in this work.  A
Hi2Low model, in the current context 





%-----------------------------------------------------------------------------%
\chapter{Literature Review}
\label{chap:lit}
%! TEX root = ../dissertation_gurecky.tex

The problem of correcting thermal hydraulic predictions provided by a subchannel code using higher fidelity CFD results can be viewed from the perspective of statistical downscaling.  There are abundant examples of statistical downscaling techniques in the weather forecasting and geological literature.  One commonality across all studied procedures is the presence of a high and low fidelity data source and a goal to make credible predictions of the target field between known coarsely resolved sample locations.  The problem is one of data amalgamation, where the resultant downscaling model preserves some average aspects of the low fidelity model with the added benefits of uncertainty and spatial fidelity afforded by the finer scale data.

\section{Statistical Downscaling}

Statistical downscaling (SD) methods seek a statistical link between a low and high fidelity features.
In particular in the climate and weather data communities it is common to perform local bias-correction on coarsely resolved weather models provided local weather station measurements [ref]. A common situation which lends itself to SD methods is to have disparate resolution sample data; one set is typically provided by a coarsely resolved global circulation model (GCM) and a secondary set of finely resolved local rain and wind field measurements is provided by local weather stations, satellite or radar sources.  In addition to the longitude and latitude of these measurements, the fine scale data may also be asscosiated with auxiliary features at such as the terrain height.
\index{Statistical Downscaling}

Results from a SD model should be carefully interpreted since at fine scale resolutions point estimates for the fields represent a single realization of a random variable governed by a fitted underlaying distribution.  Depending on the models used to capture statistical variation in the spatial and temporal trends it is sometimes necessary to draw many samples from the SD model to estimate the mean and higher moments.  These mean and variance estimates can then be compared against a historical validation data set.

Precipitation estimates provided by statistically downscaled climate models are used as a boundary condition to local hydrology models for runoff \cite{wood2002}, flood, and aquifer replenishment studies.  One may draw a parallel with the current crud simulation work where bias-corrected subchannel TH results are passed to a corrosion chemistry or crud simulation package.

A particular class of SD methods known as bias-corrected spatial disaggregation (BCSD) rely on computing estimates of time-and-space aggregated rainfall percentiles given previous historical local station data \cite{wood2002}.  In this method, the spatially and temporally fine data is aggregated to the coarse scale GCM grid before percentiles are computed.  These models have been used to forecast the probability of extreme precipitation events in a given local region which is important for flood risk assessment studies [ref].  The problem is similar to highly threshold-sensitive crud problem since flood risk models require accurate frequency and magnitudes of extreme rainfall events which are difficult to quantify along with coarse scale GCMs alone.
\index{Statistical Downscaling!Bias Corrected}

Several factors prohibit the application of BSCD techniques directly to the hi2lo problem at hand.  The majority of BSCD literature does not consider the simultaneous prediction of multiple correlated random fields.  Additionally, the BSCD models do not typically consider a large number of exogenous covariates in their construction - only the spatial and temporal coordinates are used which constrains the statistical downscaling maps to be nontransferable to other geographic locations.  Finally, resolving fine spatial detail of the temperature and TKE fields in a given CTF face isn't necessary for accurate crud prediction when using a single dimensional crud simulation code since no azimuthal or axial variation in these surface fields are utilized in the crud package.  Therefore, the problem of finding the fractional area of a CTF face which exists above a threshold is preferable to spatial disaggregation techniques in the current hi2lo crud application.

It is also possible to nest a high fidelity simulation within a coarse fidelity weather simulation. Boundary conditions and constraints are supplied by the coarse fidelity model to the nested regional high resolution weather model.  The practice of coupling regional weather models with coarse scale global models is sometimes referred to as dynamical downscaling, though, this weather modeling strategy can also be viewed as a particular tightly coupled multi-scale model.  There are examples of such coupled simulation in reactor physics (see literature on coupled coarse RELAP and CFD models.  CFD is used where the flow is 'complicated' but RELAP handles the primary loop piping and heat exchanger)  The construction of dynamical downscaling models are not the focus of the current hi2lo work and will not be discussed further.

\section{Kriging}

Kriging or Gaussian process regression techniques are well suited when the errors can be assumed to be normally distributed about some mean prediction.  Simple kriging assumes that the mean of the underlaying field does not drift as a function location in the input space and therefore only requires the fitting of a co-variogram which describes the statistical spatial autocorrelation between the known data.  More advanced kriging strategies such as regression kriging (RK) forgo this simplification decomposing the interpolation problem into mean predicting and bias-correcting residual models.  In RK the spatial-autocorrelation in the residuals is modeled by a simple kriging model.  The mean response may be predicted by any regression strategy, with a common choice being an ordinary least squares model though works which investigate the use of random-forests or more advanced machine learning strategies in this role are pervasive [refs].

\index{Kriging}
\index{Kriging!Regression Kriging}

\section{Subchannel Hi2lo}

The utilization of CFD data to improve subchannel thermal hydraulic models does not necessarily take on a statistical downscaling characteristic.  Oftentimes the strategy by which one uses CFD data to improve a subchannel model can be developed using standard Bayesian inference techniques in which model parameters are inferred through comparing the low fidelity model to high fidelity experimental or CFD data.  This flavor of inverse problem oftentimes involves  model calibration, model selection and experimental design aspects.  A wide array of literature exists on each of these topics and will not be interrogated here.  Instead, a pointed literature review of the latest CFD-informed subchannel work is considered.

 M. Avramova developed CFD informed grid mixing models in CTF.  This work leveraged CFD results to improve the momentum balance and grid mixing models in CTF \cite{avramova2007}.  The lateral momentum equations implemented in CTF are provided in equation \ref{eq:ctf_lat_mom}.

    	\begin{align}
    	& \frac{\partial }{\partial t}(\alpha_l \rho_l \mathbf U_l)
    	+ \nabla \cdot (\alpha_l \rho_l \mathbf U_l \mathbf U_l) \nonumber \\
    	&= \alpha_l \rho_l \mathbf{g} - \alpha_l \nabla P + 
    	\nabla \cdot (\alpha_l \bm{\tau}_l) \nonumber \\
    	&+ M^L_l + M^d_l + M^T_l + M_l^{GDXF}
        \label{eq:ctf_lat_mom}
    	\end{align}
        
Where $l$ denotes the liquid phase and $\alpha$ is the volume fraction liquid.  The terms $M^L, M^d, M^T, M_l^{GDXF}$ account for droplet or bubble entrainment, phase interfacial drag, turbulent mixing and grid directed cross flow respectively.  Avramova devised a method to use CFD computations to obtain an accurate prediction of $M_l^{GDXF}$ for a variety of grid designs.
    	
The grid directed cross flow momentum source term used in Avramova's model is defined by equation \ref{eq:grid_en_xflow_coeff}.
    	\begin{equation}
    	M_l^{GDXF} = f^2_{sg}(z) u_l^2 \rho_l A_g S_g
        \label{eq:grid_en_xflow_coeff}
    	\end{equation}
    	Where $u_l$ is the axial liquid velocity. 
    	\begin{equation}
    	f_{sg}(z) = \frac{V^{CFD}_l(z-z_{in})}{U^{CFD}_{in}}
    	\end{equation}
    	
 The effectiveness of the grid enhanced cross flow model was determined by comparing exit bulk temperature profiles across a variety of assembly designs against experimental and CFD results.  The results indicated a marked improvement in the rod-assembly outlet temperature distribution at little additional computational cost as compared to CTF without CFD informed grid enhanced cross flow corrections.  
 
 The next bodies of work are closer in alignment with traditional downscaling techniques.  These class of hi2lo procedures are not statistical in nature, but rather seek to correct spatial biases in the field predictions made by a low fidelity subchannel code.

    
S. Yao et al. developed an empirical model of the heat transfer coefficient downstream of spacer grids \cite{yao82}.
    An empirical relationship between the Nusselt number ratio and the vane angle, $\phi$, blockage ratio $\epsilon$, dimensionless distance from the grid, $x/D$, and fraction of flow area impeded by the vanes, $A$, was produced.  This relationship is provided in equation \ref{eq:yao_htc}.
    
\begin{equation}
\frac{Nu}{Nu_0}  = \left[ 1 + 5.55 \epsilon^2 e^{-0.13(x/D)}\right] + \left[ 1 + A^2\mathrm{tan}^2\phi e^{-0.034(x/D)} \right]
\label{eq:yao_htc}
\end{equation}
Where the first term is accounts for the effect of grid flow restriction and the second term represents the contribution of vane induced swirl on the heat transfer.

\begin{figure}[H]
    \centering
    \includegraphics[width=0.6\linewidth]{../proposal/images/grid_nu_eff}
    \caption{S. Yao empirical Nusselt number ratio vs. distance from upstream spacer grid.}
    \label{fig:gridnueff}
\end{figure}

    Similar to Yao's approach for capturing rod-enhanced heat transfer,  B. Salko et al. developed a CFD-Informed hi2lo spatial remapping procedure for CILC/CIPS screening \cite{salko17}.  Rather than establishing a general empirical relationship between grid geometric features and the flow field, Salko developed grid-specific maps.  The developed multiplier maps are applicable only to the grid designed on which they are based.  In contrast to Yao's approach, this enables a much higher resolution flow field effects on the surface temperature distribution to be retained in the multiplier maps.  In addition to generating heat transfer multiplier maps, this method developed a TKE mapping procedure since both fields are required for accurate crud predictions.  Both maps are applied in conjunction to a baseline CTF result to produce grid enhanced surface temperature and TKE distributions at runtime of the CTF model.
    
    First, an intermediate coupling mesh is constructed on the surface rods with a resolution between the CFD mesh and the CTF grid is defined.  Next, the raw CFD surface fields are then mapped to the coupling mesh.  In this approach crud is to be grown on the intermediate coupling grid.  In theory, this grid can be refined to be equivalent to the CFD mesh size and indeed this would reduce interpolation error in the hi2lo procedure \cite{salko17}.
       
    The multiplier maps capture the ratio of the CFD predicted HTC and TKE surface distributions to the same surface distributions on a bare rod with no spacer grids present.
    Convective HTC remap is described by equations \label{eq:htc_remap_ctf_1} and \label{eq:htc_remap_ctf_2}.
    
    \begin{equation}
        \mathbf m_h = \frac{(Nu)_{cfd}}{(Nu)_{0}} = \frac{h_{cfd} L_{cfd} k_{0} }{h_{0}k_{cfd} L_{0}}
         \label{eq:htc_remap_ctf_1}
    \end{equation}
    Where $Nu$ is the Nusselt number.  Assuming equal length scales and thermal conductivities:
    \begin{equation}
        \mathbf m_h = \frac{h_{cfd}}{h_{0}} = \frac{q_{cfd}(T-T_\infty)_{0}}{q_{0}(T-T_\infty)_{cfd}}
        \label{eq:htc_remap_ctf_2}
    \end{equation}
    It is important to note that a uniform heat flux is used in both the bare and full rod case so that $q_{cfd}/q_0 =1 $.
    Apply HTC remap to the original CTF HTC by equation \ref{eq:htc_remap_ctf_3}.
    \begin{equation}
        \hat h_{l} = \mathbf m_h h_{ctf}
        \label{eq:htc_remap_ctf_3}
    \end{equation}
    $\hat h_l$ is the hi2lo remapped convective heat transfer coefficient.  In CTF the wall heat transfer is split between phases:
    \begin{equation}
        q'' = q''_{conv} + q''_{boil} = (\hat h_l)(T-T_{\infty}) + q''_{boil}(T)
    \end{equation}
    In order to compute augmented hi2lo surface temperatures
    several iterations are required to converge upon the correct surface temperature, $\hat T_s$ due to the surface boiling term.

    \begin{algorithm}[H]
        \captionsetup{labelfont={sc,bf}, labelsep=newline}
        \caption{Heat transfer coefficient map based hi2lo method for crud prediction (Salko. et. al.).}
    \setstretch{0.8}  % reduce spacing in algo sec
    \begin{algorithmic}[1]
    \STATE \textbf{Initialization} 
    \STATE Guess $T^{i=0}_s=T_0$.  Maximum number, $N$ iterations.

        \FOR {i in range($N$):}
           \STATE Evaluate effective multiphase CTF HTC: $h_{eff} = h_{{ctf}}(T^i_{s}, \hat h_l, q'')$ \;
           \STATE Compute new hi2lo surface temperatures: $T_{s} = \frac{q''}{h_{eff}} + T_\infty$ \;
           \STATE  Under relax  $T^{i+1}_{s} = \omega T_{s} + (1 - \omega) T^{i}_{s} ;\ \omega < 1.$ \;
           \STATE  \textbf{break if}:  $|T^{i+1}_s - T^i_s| < tol$ \;
        \ENDFOR 
    \STATE \textbf{return}: $\hat T_s = T^{i+1}_s$
    \end{algorithmic}
    \end{algorithm}
    Where $h_{ctf}(\cdot)$ is a callable CTF function that returns and effective multiphase HTC, $h_{eff}$.

    The TKE remap is constructed by evaluation the ratio given in equation \ref{eq:tke_map} on all CTF faces.
    \begin{equation}
       \mathbf m_{k} = \frac{k_{cfd}}{k_{0}}
       \label{eq:tke_map}
    \end{equation}
    Where $k_0$ is the TKE distribution for a bare rod without spacer grids.
    The TKE multiplier map is applied in the same manner as the HTC map.
       \begin{equation}
       \hat k = \mathbf m_k k_{ctf}
       \end{equation}
    Crud is grown on the coupling mesh using augmented temperature and TKE surface fields. By this method the integrated crud mass over a CTF face is given by equation \ref{eq:ctf_hi2lo_crud_est}.
     \begin{equation}
     C_m = \frac{1}{A} \sum_i^N \mathcal G(\hat T_{s_i}, \hat k_i, q''_i) a_i
     \label{eq:ctf_hi2lo_crud_est}
     \end{equation}
    Where $A$ is the area of the CTF face and $a_i$ is the area of each cell face on the crud coupling mesh.

    
    A key assumption that the multiplier maps are insensitive flow rate was made in the first implementation of this downscaling technique.  However this assumption is not strictly true: The multiplier maps carry some dependence on the inlet flow conditions.  An increase in flow rate changes the shape and extent of the wake region downstream of spacer grids which impacts the rod surface temperature and TKE fields.

    An extension of the multiplier map hi2lo procedure could linearly interpolate between multiplier maps developed at high and low inlet flow rate conditions.
    \begin{align*}
        \mathbf m_k &= \alpha \mathbf m_k^h + (1 - \alpha) \mathbf m_k^l \\
                    &= \alpha \frac{k^h_{cfd}}{k^h_0} + (1 - \alpha) \frac{k^l_{cfd}}{k^l_0} \\
        \alpha & = \frac{\dot m_i - \dot m_i^l }{\dot m_i^h - \dot m_i^l}
    \end{align*}
    Where $\dot m_i$ is the inlet mass flow rate.  The superscript, $(\cdot)^l$, represents low flow conditions and $(\cdot)^h$ represent high flow conditions.


\begin{figure}[H]%
    \centering
    \subfloat[CTF/MAMBA crud predictions without hi2lo remapping on a quarter symmetric pin.]{{\includegraphics[width=0.45\linewidth]{../proposal/images/ctf_crud_orig} }}%
    \qquad
    \subfloat[CTF/MAMBA crud predictions using hi2lo remapping on a quarter symmetric pin.]{{\includegraphics[width=0.45\linewidth]{../proposal/images/ctf_crud_reconstructed} }}% 
    \caption[The impact of spatial HTC hi2lo remapping on CTF/MAMBA crud predictions.]{The impact of spatial HTC hi2lo remapping on CTF/MAMBA crud predictions.}%
    \label{fig:htc_remap_crud}%
\end{figure}


    Some simplifications are made in this mapping.  For a given assembly, the multiplier maps have been shown to have a high span to span repeatability.  Therefore, a representative map is derived from a single span in a fully developed flow field.  This representative map is then applied to all other spans in the model.

    The multiplier map may not be transferable to other assemblies in the core due to geometric effects including the orientation of neighboring assemblies and TH/neutronics feedbacks.  This represents a limitation to the spatial mapping procedure as unique maps must be generated for different assemblies in the core.
     
    Blyth produced CFD informed grid enhanced heat transfer models for the advanced subchannel code, CTF.  This work presented strategies for processing CFD data for use in generating enhanced heat transfer maps and for computing the form loss coefficient across spacer grids.  Blyth's work served as a precursor to Salko's CFD informed method for developing HTC and TKE maps.

    T. Hengal. Regression Kriging application to soil composition prediction as function of elevation, distance to river, ect. \cite{Hengl07}
    
    Boosted regression and classification applications. \cite{moisen2006}, \cite{friedman2002}





%-----------------------------------------------------------------------------%
\chapter{Theory}
\label{chap:theory}
%! TEX root = ../dissertation_gurecky.tex

\section{Model Approach}

\begin{itemize}
        \item (\checkmark) Model overview.
    \item CTF estimates mean TH conditions everywhere in the core at a low spatial resolution.  The surrogate provides higher order moments about the mean.
    \begin{equation}
    \mathbf S(\mathbf z) = \underbrace{ \bm \mu(\mathbf{z})}_\text{CTF} +
    \underbrace{\varepsilon({\theta (\bm p(\mathbf z))})}_\text{CFD Informed} + \bm b(\mathbf{z})
    \end{equation}
    \begin{itemize}
        \item $\mathbf S$ is a three component vector field representing the cladding surface temperature, turbulent kinetic energy and boundary heat flux.
        \item $\mathbf z$ are spatial coordinates. $\mathbf p$ are a set of auxiliary predictors.  Auxiliary predictors are covariates that describe local core conditions and may be geometric or thermal hydraulic in nature.
        \item $\varepsilon$ is a random three-component vector field with components: $\{\varepsilon_T, \varepsilon_k, \varepsilon_{q''}\}$.
        \item $\varepsilon({\theta (\bm p(\mathbf z))})$ is a CFD informed model with $\theta$ representing free model parameters which must be fit to CFD data.
        \item $\bm b$ is bias ($\bm \mu_{CTF} - \bm \mu_{CFD}$)
        \item Field averages, $\bm \mu$, are piecewise constant over each CTF patch
        \item Note this is a additive model where Salko constructed a multiplicative model.
    \end{itemize}
\end{itemize}

Consider the case where the CFD results are normally distributed about the CTF results such that $\varepsilon \sim \mathcal N(0, \mathbf \theta(\mathbf p))$, where $\mathbf \theta(\mathbf p) = \bm \Sigma(\mathbf p)$ is a covariance matrix that depends on local core conditions.

Shifting this distribution by a constant vector $\bm c=\bm b + \bm \mu_{ctf}$, results in a new distribution denoted:
\begin{align}
    \left. h \right|_{\bm p} & = \mathcal N(\bm c, \bm \Sigma(\mathbf p)) \nonumber \\
    & = \left.
        \mathcal N \left(
        \begin{pmatrix}
            c_T \\
            c_k \\
            c_{q''}
        \end{pmatrix}
    ,
        \begin{pmatrix}
            \sigma_{T} \sigma_{T} & \sigma_{T} \sigma_{k} & \sigma_{T} \sigma_{q''} \\
            \sigma_{k} \sigma_{T} & \sigma_{k} \sigma_{k} & \sigma_{k} \sigma_{q''} \\
            \sigma_{q''} \sigma_{T} & \sigma_{q''} \sigma_{k} & \sigma_{q''} \sigma_{q''}
        \end{pmatrix}
    \right)
    \right|_{\mathbf p}
\end{align}


The goal is to estimate the expected crud on each CTF patch given by equation \ref{eq:expected_crud}.

\begin{eqnarray}
        total\ crud\ [grams] = A \mu_g\ = A \E[g(\mathbf X|g_o, \mathbf I, \delta t)] \nonumber \\
        = A \iiint g(\mathbf X|g_o, \mathbf I, \delta t) h(\mathbf X|\theta) d \mathbf X
        \label{eq:expected_crud}
\end{eqnarray}
let $\mathbf X= \{T, k, q''\}$ denote a random vector of temperature, TKE, and BHF. $\mathbf I$ represents additional known crud parameters, $g_o$ is the crud state at the start of the time step and $\theta$ are distribution parameters.  The goal is to predict what $h(\cdot)$ is in every CTF face.  The CRUD model, $g(\cdot)$, is common to all ctf faces.

To compute the total crud/boron in each CTF face:
\\

\begin{algorithm}[H]
\KwData{(1) Pre-process training set $\theta(\mathbf p)$.  (2) Train model $M(\mathbf p ; \gamma)$:  $ argmin_\gamma ||M(\mathbf p; \gamma) - \theta(\mathbf p)|| $}

\For{CTF face, $i$}{
Evaluate ML model $\hat \theta_i \leftarrow M(\mathbf p_{i})$ \;
Reconstruct $\hat h_i(x|\hat \theta_i)$ \;
Draw samples $\mathbf X \sim \hat h_i$ \;
Evaluate \ref{eq:expected_crud} via Monte Carlo approximation \;
}
\end{algorithm}
Where $\gamma$ are free parameters in the ML model.
\bigskip

In addition to improving the expected value prediction of CRUD on each CTF patch vs the CTF standalone case, the model provides the capability to estimate the likelihood of extreme value events (i.e. $\mathcal P_f \propto Pr(g(x) > g^*)$, where $g^*$ is some critical crud thickness and $\mathcal P_f$ is a cladding failure probability) on the rod surface.  This would be impossible to quantify with CTF/MAMBA alone.

A significant challenge is computing an estimate for $Var(\mathcal P_f) = \E[(\mathcal P_f - \E(\mathcal P_f))^2]$.


\section{Construction of the Hi2lo Map}

\subsection{Capturing Dependence Between Random Variables}
\begin{itemize}
        \item (\checkmark) Sklar's theorem.
        \item (\checkmark) Assumptions:
        \begin{itemize}
                \item Capture Temperature and TKE dependence structure.
                \item Assume 1) Temperature is uncorrelated with BHF.  2) TKE is uncorrelated with BHF.
                \item Justify 1) showing actual CFD data. 2) Relative variations in BHF are very small over a CTF face $(+/- 5\%)$.  3) Sensitivity of CRUD to BHF is small relative to sensitivity of CRUD to surface temperature.
        \end{itemize}
        \item (\checkmark) Show fallout of these assumptions on hi2lo model.
        \begin{equation}
                h(T, k, q'') = f_T f_k, f_{q''} \prod_c c_i(u_i, v_i)
        \end{equation}
        Where each bivariate copula model in $\prod c_i $ is known as a pair copula construction.

        Applying the aforementioned assumptions results in: $c(u_T, u_{q''}) = 1$ and $c(u_{k}, u_{q''}) = 1$. The simplified joint density is given by:
        \begin{equation}
                h(T, k, q'') \approx  f_T f_k, f_{q''} c(u_{T}, v_{k})  \cdot 1 \cdot 1
    \end{equation}
\end{itemize}


\subsection{Sample Quantiles}

\subsubsection{Non Parametric Representations of Univariate Distributions}

\begin{itemize}
    \item (\checkmark-) Introduce non-parametric representation of univariate distributions through sample quantiles.
    \item ($\cdot$) Explain typical alternatives and why they are not suitable:
        \begin{itemize}
            \item Compute sample moments.  Use method of moments to fit a model to sample moments.
            \item Compute sample cumulants.  Use sample cumulants to build an Edgeworth series.
            \item Do cumulants or traditional moments behave in a predictable manner as a function of local core conditions?  Do the quantiles behave in a predictable manner?
        \end{itemize}
\end{itemize}

\begin{figure}[H]
    \centering
    \includegraphics[width=0.3\linewidth]{../proposal/slides/seminar_slides/figs/margins_cdf_2}
    \caption[CDF from quantiles.]{Piecewise linear CDF interpolated from a set of quantiles.}
    \label{fig:marginscdf2}
\end{figure}


The $\tau^{th}$ quantile is $q_\tau = F^{-1}(\tau); $ where $\ F(t)=P[T \leq t]$.
$\tau \in [0, 1]$

The quantile loss function is given by equation \ref{eq:qloss_fn}.
\begin{equation}
\rho_\tau( u) = \mathbf u \cdot (\tau - \mathbb{I}_{( u < 0)})
\label{eq:qloss_fn}
\end{equation}
Where $\mathbb{I}$ is the indicator function which returns 1 if the argument is true and 0 otherwise.
In order to estimate a sample quantile given the empirical CDF $F$, minimize: $\E[\rho_\tau(T - q_\tau)]$ where $T$ is a random variable distributed according to $F$.
\begin{equation}
            \left.\begin{aligned}
            \hat q_{\tau_i} &= argmin_{q} \E[\rho(u)];\ \  u = T - q  \\
            \approx & argmin_q  \frac{1}{N} \sum_i^N \rho(u_i); \ u_i = t_i - q \\
            \approx & argmin_q \left[ (1-\tau) \sum_{y \leq q}( t_i - q ) - \tau \sum_{y > q} (t_i - q) \right]
            \end{aligned}\right.
\end{equation}


\begin{figure}[H]
    \centering
    \includegraphics[width=0.4\linewidth]{../proposal/slides/seminar_slides/figs/q_loss}
    \caption[Quantile loss function.]{Quantile loss function.}
    \label{fig:qloss}
\end{figure}



\begin{itemize}
        \item (\checkmark) Compute sample quantiles.
        \item (\checkmark) Building a cumulative density function from sample quantiles.
        \begin{itemize}
           \item Piecewise linear CDF leads to histogram PDF.
           \item PCHIP spline CDF preserves monotonicity of CDF and results in a smooth PDF. \cite{Fritsch80}
        \end{itemize}
        \item (\checkmark) Sampling from a known CDF via inverse transform sampling.
        \item (\checkmark-) Show that the order statistics are distributed according to a Gaussian distribution.
        \item (\checkmark-) Show that sample quantiles also follow a Gaussian distribution.  This directly follows from the distribution of the order statistics.
            The sample quantiles are distributed normally according to equation \ref{eq:theory_qdist}.
        \begin{eqnarray}
        q_p &\sim \mathcal N \left( F_T^{-1}(p), \sigma^2_{q_p} \right) \\
        \sigma^2_{q_p} &= \frac{p(1 - p)}{n[f_T(F_T^{-1}(p))]^2} \nonumber
        \label{eq:theory_qdist}
        \end{eqnarray}
        
        
\begin{figure}[H]
    \centering
    \includegraphics[width=0.5\linewidth]{figs/quantile_theory/q_beta_residual}
    \caption{}
    \label{fig:qbetaresidual}
\end{figure}
\begin{figure}[H]
    \centering
    \includegraphics[width=0.5\linewidth]{figs/quantile_theory/q_beta_residual_conditional_0_1}
    \caption{}
    \label{fig:qbetaresidualconditional01}
\end{figure}
\begin{figure}[H]
    \centering
    \includegraphics[width=0.5\linewidth]{figs/quantile_theory/q_beta_residual_conditional_0_5}
    \caption{}
    \label{fig:qbetaresidualconditional05}
\end{figure}
\begin{figure}[H]
    \centering
    \includegraphics[width=0.5\linewidth]{figs/quantile_theory/q_beta_residual_conditional_0_9}
    \caption{}
    \label{fig:qbetaresidualconditional09}
\end{figure}

\begin{figure}[H]
    \centering
    \includegraphics[width=0.5\linewidth]{figs/quantile_theory/q_gauss_residual}
    \caption{}
    \label{fig:qgaussresidual}
\end{figure}
\begin{figure}[H]
    \centering
    \includegraphics[width=0.5\linewidth]{figs/quantile_theory/q_gauss_residual_conditional_0_1}
    \caption{}
    \label{fig:qgaussresidualconditional01}
\end{figure}
\begin{figure}[H]
    \centering
    \includegraphics[width=0.5\linewidth]{figs/quantile_theory/q_gauss_residual_conditional_0_5}
    \caption{}
    \label{fig:qgaussresidualconditional05}
\end{figure}
\begin{figure}[H]
    \centering
    \includegraphics[width=0.5\linewidth]{figs/quantile_theory/q_gauss_residual_conditional_0_9}
    \caption{}
    \label{fig:qgaussresidualconditional09}
\end{figure}

        \item ($\cdot$) Show the impact of the uncertainty in where the sample quantiles fall given CFD data on crud growth.  The reconstructed CDFs which comprise the hi2lo mapping are essentially fuzzy which means the temperature, TKE, and BHF distributions are artificially smeared out (artificially high densities in the tails).
\end{itemize}

\begin{itemize}
    \item (\checkmark-) The sampled temperatures may be tallied over each CTF face to estimate the fractional area that exceeds some threshold temperature.
    The probability of exceeding a threshold temperature is shown in equation \ref{eq:pr_thresh}.
    
    \begin{equation}
    p_e = Pr(T > T^*) = 1 - \int_0^{T^*} f_T dT
    \label{eq:pr_thresh}
    \end{equation}
    
    Let $q_p = F_T^{-1}(1 - p_e)$
    denote the quantile associated with the threshold probability, $p_e$.
    $F_T^{-1}$ is the inverse CDF function and $f_T$ is the probability density function of temperature on the patch.
    
    The sample quantile corresponding to $p_e$ is distributed according to:

   \begin{eqnarray}
    q_p &\sim \mathcal N \left( F_T^{-1}(p), \sigma^2_{q_p} \right) \\
    \sigma^2_{q_p} &= \frac{p(1 - p)}{n[f_T(F_T^{-1}(p))]^2}
    \end{eqnarray}
    
    The variance of upper tail probability mass estimate can be found by standard propagation of uncertainty principles:
   \begin{equation}
    \sigma_p^2 = \left(\frac{\partial p_e}{\partial q_p} \right)^2 \cdot \sigma_{q_p}^2
    \end{equation}
    
    Where
   \begin{eqnarray}
    \frac{\partial p_e}{\partial q_p} &= \frac{\partial}{\partial q_p} \left( 1 - \int_0^{q_p} f_T dT \right) \nonumber \\
    &= \frac{\partial}{\partial q_p} \left( -F_T(q_p) + F_T(0) \right) \nonumber \\
    &= -f_T(q_p)
    \end{eqnarray}
    
    This dictates that estimates of extreme upper tail integrals carry large relative uncertainties.
    \item The distribution for $p_e$ can be obtained by a change of variables, since $p_e$ and $q_p$ are related by:
    \begin{equation}
    q_p = F_T^{-1}(1 - p_e)
    \end{equation}
    
    Where $F_T^{-1}$ is the inverse CDF.  $p_e$ is distributed according to:
    
    \begin{equation}
    f_{p_e} = f_{q_p} \left( F_T^{-1}(1-p_e), \sigma_{q_p}(p_e) \right) \cdot \left|{ \frac{\partial}{\partial p_e} F^{-1}(1-p_e) } \right|
    \end{equation}
    
\end{itemize}


\subsection{Introduction Gradient Boosted Regression Trees}
\begin{itemize}
    \item (\checkmark-) Explain a classification and regression tree (CART).  (Move to appendix?)
    \item (\checkmark-) Explain gradient boosting.  (Move to appendix?)
\end{itemize}

\begin{figure}[H]
    \centering
    \includegraphics[width=0.5\linewidth]{../proposal/slides/seminar_slides/figs/cart}
    \caption[Regression tree stump.]{Regression tree stump comparing a fit of depth 1 and 2.}
    \label{fig:cart}
\end{figure}

The generalized gradient boosting algorithm was developed by Friedman (1999).

\begin{algorithm}[H]
    \KwData{ (1) Training set $\{(p_i, y_i)\}_{i=1}^n$. (2) Differentiable loss function $L(y, F(p))$. (3) Number of iterations ${{M}}$.
        Initialize model with a constant value:
        $F_0(p) = \underset{\gamma}{\arg\min} \sum_{i=1}^n L(y_i, \gamma).$}
    
    \For{${{m}} = 1$ to ${{M}}$}{
        Compute the pseudo-residuals:
        
        \For{$i=1,\ldots,n $}{
            $r_{im} = -\frac{\partial L(y_i, F_{m-1}(p_i))}{\partial F_{m-1}(p_i)}$
        }
        
        Fit a weak learner $h_m(p)$ to pseudo-residuals, $r_{m}$: Training data set is $\{(p_i, r_{im})\}_{i=1}^n$ \;
        
        Compute multiplier $\gamma_m$ :
        $\gamma_m = \underset{\gamma}{\operatorname{arg\,min}} \sum_{i=1}^n L\left(y_i, F_{m-1}(p_i) + \gamma h_m(p_i)\right)$\;
        Update the model:
        $F_m(p) = F_{m-1}(p) + \nu \gamma_m h_m(p).$
    }
    Output $F_M(p).$
\end{algorithm}
Where $\nu$ is a tunable constant in $[0, 1]$ called the learning rate.


\subsection{Monte Carlo CRUD Estimation}

\begin{itemize}
        \item (\checkmark-) Show importance sampling scheme to estimate \ref{eq:expected_crud}.  (Move to Appendix?)
        \begin{equation}
        \E(g(x)) \approx \frac{1}{N} \sum_i^N g(x_i) \frac{h(x_i)}{\tilde h(x_i)}, \ x \sim \tilde{h}
        \end{equation}
        \item (\checkmark-) Show examples of $g(x)$, $h(x)$ and $\tilde h(x)$ on a single CTF face.
        \item (\xmark) \sout{Find optimal proposal density distribution, $\tilde{h^*}$.}
\end{itemize}



\subsection{Propagating CRUD Through Time}
\subsubsection{Hot Spot Stationarity}
\begin{itemize}
        \item (\checkmark) Elucidate why we need a spatial remapping scheme.
        \item (\checkmark) Detail how this mapping is produced.
\end{itemize}

\subsection{Smearing Over Azimuth}

\begin{itemize}
        \item ($\cdot$) Smear the CFD data and CTF results over all azimuthal angles.  This will increase the number of CFD samples used for the hi2lo mapping construction by a factor of four, and therefore decrease the uncertainty in the sample quantiles by a factor of 2.  See section on the theoretical distribution of sample quantiles (the sample quantile distributions follow 1/sqrt(N) behavior).
        \item ($\cdot$) Since the primary goal of this method is to predict CIPS, it is possible neglect azimuthal variation entirely and focus solely on the axial variation.  In this case, for any given pin, each CTF face at a fixed axial level will all get the same crud thickness and crud boron from the hi2lo model.
\end{itemize}


%-----------------------------------------------------------------------------%
\chapter{Method Exploration Under a Synthetic Data Source}
\label{chap:work}
%! TEX root = ../dissertation_gurecky.tex

In this chapter the univariate distribution reconstruction from quantiles, copula parameter fitting, Monte Carlo and importance sampling strategies are exercised with a synthetic data set.  Synthetic data offers advantages over CFD born data for the purposes of testing and evaluating the efficacy of the proposed models.  Since synthetic data conforms to a known functional form with specified distribution and bias parameters the fitting and sampling routines can be checked to ensure that the fitted models retain key statistical properties from the synthetic data set. This chapter does not introduce machine learning components and does not explore forward model predictions.  See chapter \ref{sec:ml_cfd} for hi2lo model performance when used in a predictive capacity.

The availability of synthetic data alleviates the need to generate comparatively expensive CFD results to test the hi2lo strategy.  Some aspects of CFD fields are preserved in the synthetic data, including expected biases between CFD and CTF results that arise due to discrepancies in wall heat transfer closure models, among other differences.  Turbulent dispersion of the temperature and near wall TKE distributions around spacer grids are emulated by the synthetic data model.  Additionally, the dependence structure between the cladding surface temperature, boundary heat flux, and near wall TKE can be enforced by the synthetic data generation tool.  Accounting for spatial auto-correlation in the surface fields was not pursued.  Spatial autocorrelation present in the surface fields within a CTF face is not captured, however.  Consequently the synthetic data is not a direct replacement to CFD data but serves as data source for method interrogation and integration testing.

Semiparametric copula and marginal distributions are fit to the synthetic data in each CTF face independently.  Samples are drawn from the fitted joint temperature, TKE and BHF density models on each patch using standard Monte Carlo methods or importance sampling.  The surface samples are provided to a crud simulation package as cladding-surface boundary conditions.  Additional required bulk coolant properties such as the bulk fluid temperature and bulk concentration of soluble boron are supplied by CTF.

Time dependent crud simulation is also discussed.  The interaction between the sample surface remapping strategy and the integrated crud results are discussed.  Furthermore, the process by which importance sample weights are averaged over discrete time steps is discussed.

Speedups afforded by importance sampling are presented and contrasted against standard Monte Carlo sampling results.  The sampling distributions utilized in importance sampling are informed by the physics of crud growth.

\section{Generating Synthetic Data}
\index{Synthetic CFD Data}

Synthetic data generation begins by running standalone CTF on a single quarter symmetric pin.   The CTF result is then augmented with tailored noise.  The augmented synthetic surface fields can be constructed by equation \ref{eq:synth_aug}.
\begin{align}
    \bm X &= \bm \mu_{ctf} + \bm b + \bm \varepsilon \nonumber \\
          &=
    \begin{pmatrix}
        T \\
        k \\
        q''
    \end{pmatrix}
    =
    \begin{pmatrix}
        \mu_{T} \\
        \mu_k \\
        \mu_{q''}
    \end{pmatrix}_{ctf}
    + \begin{pmatrix}
        b_{T} \\
        b_k \\
        b_{q''}
    \end{pmatrix}
    + \bm{\varepsilon} (\mathbf z; \bm \theta),
\label{eq:synth_aug}
\end{align}
Where $\bm \varepsilon(\mathbf z, \bm \theta)$ is a user controlled spatially dependent residual random vector with a mean of 0.  This residual is
shifted by a bias vector
$\mathbf b$, where $\mathbf z=\{z, \varphi\}$ is the axial and azimuthal location on the rod surface.
$\bm \theta$ represents user specified distribution parameters.

Equation \ref{eq:synth_aug} represents three continuous random surface fields.  In practice a large number of independent and identically distributed samples are drawn in each CTF face from the underlying random field.  Individual surface samples can be specified by equation \ref{eq:synth_aug_discrete}.

\begin{equation}
    X_{ij} = \mu_{j,\mathrm{ctf}} + b_j + \varepsilon_{ij};\ \   \varepsilon_{j} \sim h_j
    \label{eq:synth_aug_discrete}
\end{equation}
Where the index $j$ represents the $j^{th}$ CTF face on the rod, and the index $i$ is the sample index within the $j^{th}$ CTF face.  The distribution parameters are constant over a given CTF face and give by $\bm \theta = \{\theta_c, \{\theta_x\}\}$ where $\theta_c$ is the copula parameter and  $\{\theta_x\}$ are marginal parameters, 


Shown in equation \ref{eq:synth_aug_face}, according to Sklar's theorem the surface residual temperature and TKE joint distribution may be decomposed into copula an marginal models on each CTF face:

\begin{equation}
    h_j = c_j(F_k(k), F_T(T); \theta_{c_j}) f_{T}(T; \theta_{T_j}) f_{k}(k; \theta_{k_j})
    \label{eq:synth_aug_face}
\end{equation}
Where the copula parameter $\theta_{c_j}$ and the marginal temperature and TKE distribution parameters $\theta_{T_j}$ and $\theta_{k_j}$ are set at runtime of the synthetic data generation tool.

To allow for a great deal of flexibility in the synthetic data the copula family, Kendall's $\tau$ rank correlation coefficient and marginal distribution parameters are specified as a function of axial location and local TH conditions supplied by CTF.  The copula's shape parameter, $\theta_c$ may be related to the rank correlation coefficient by equation \ref{eq:tauar} which is a one to one function for the Archimedean copula considered in this work.


\subsection{Single Pin Synthetic Data Set}

The original baseline CTF results are shown in figures \ref{fig:ctf_twall_orig} and \ref{fig:ctf_tke_orig}.  The CTF pin parameters are provided in table \ref{tab:pin_settings}.  The CTF result was produced from a quarter symmetric case, and therefore no azimuthal variation is observed.

\begin{table}[h]
    \begin{center}
        \caption{Single pin reference thermal hydraulic boundary conditions.}
        \begin{tabular}{|l|l|l|}
            \hline
            Setting & Value & Unit \\
            \hline
            Inlet Flow Rate & 0.3 & $[kg/s]$ \\
            Inlet Temperature & 565 & $[K]$ \\
            Pressure & 2250 & $[psia]$ \\
            Rod Outer Radius & 0.425 & $[cm]$ \\
            Pin Pitch & 1.26 & $[cm]$ \\
            Power Shape & constant & $[]$ \\
            Heat Flux & 85.86  & $[W/m^2]$ \\
            Rod Height & 3.6275 & $[m]$ \\
            Number of Grids & 3  & $[]$ \\
            Grid Locations & 2.0, 2.4, 2.8 & $[m]$ \\
            \hline
        \end{tabular}
    \label{tab:pin_settings}
    \end{center}
\end{table}

\begin{figure}[H]%
    \centering
    \subfloat[Axial CTF cladding surface temperature result.]{{\includegraphics[width=0.45\linewidth]{figs/synth/ctf_asm0_z_twall} }}%
    \qquad
    \subfloat[2D rod map of CTF result.]{{\includegraphics[width=0.45\linewidth]{figs/synth/ctf_pin_h5_temperature} }}%
    \caption[Single pin CTF baseline temperature result.  160\% nominal power conditions.]{Single pin CTF baseline temperature result.  160\% nominal power conditions.}%
    \label{fig:ctf_twall_orig}%
\end{figure}

\begin{figure}[H]%
    \centering
    \subfloat[Axial CTF cladding surface TKE result.]{{\includegraphics[width=0.45\linewidth]{figs/synth/ctf_asm0_z_tke} }}%
    \qquad
    \subfloat[2D rod map of CTF result.]{{\includegraphics[width=0.45\linewidth]{figs/synth/ctf_pin_h5_tke} }}%
    \caption[Single pin CTF baseline TKE result.  160\% nominal power conditions.]{Single pin CTF baseline TKE result.  160\% nominal power conditions.}%
    \label{fig:ctf_tke_orig}%
\end{figure}


The boundary heat flux was uniform at $85.86 [W/cm^2]$ which corresponds to approximately 160\% nominal PWR power conditions.

Next synthetic noise was generated using copula and marginal distribution settings provided in table \ref{tab:synth_settings}. The complete synthetic data generation input deck for this case along with references to the code are provided in chapter \ref{chap:app_c}, (Appendix C).

\begin{table}[h]
    \begin{center}
        \caption{Per-span synthetic data generation settings.}
        \begin{tabular}{|l|l|l|l|l|}
            \hline
            \bf Span 1 & Node & $z$ & Copula Settings  & Margin Settings \\
            \hline
            $N$: 8000  & 1  & 0.0 & $\Theta_c:$ Gaussian, $\theta_c:-0.6$ &  $T\sim\beta(5.0, 5.0)$,$k\sim\mathcal{N}(0, 0.001)$ \\
                   & 2  & 2.0 & $\Theta_c:$ Gaussian, $\theta_c:-0.6$ &  $T\sim\beta(5.0, 5.0)$, $k\sim\mathcal{N}(0, 0.001)$   \\
            \hline \hline
            \bf Span 2 & Node & $z$ & Copula Settings  & Margin Settings \\
            \hline
             $N$: 8000 & 1  & 2.0 & $\Theta_c:$ Clayton-90, $\theta_c: 2.0$ &  $T\sim\beta(5.0, 2.7)$, $k\sim\beta(1.75, 5.0)$ \\
            & 2  & 2.4 & $\Theta_c:$ Frank-90, $\theta_c: 8.0$ &  $T\sim\beta(5.0, 1.5)$, $k\sim\beta(1.75, 5.0)$   \\
            \hline \hline
            \bf Span 3 & Node & $z$ & Copula Settings  & Margin Settings \\
            \hline
             $N$: 8000 & 1  & 2.4 & $\Theta_c:$ Clayton-90, $\theta_c: 2.0$ &  $T\sim\beta(5.0, 2.7)$, $k\sim\beta(1.75, 5.0)$ \\
            & 2  & 2.8 & $\Theta_c:$ Frank-90, $\theta_c: 8.0$ &  $T\sim\beta(5.0, 1.5)$, $k\sim\beta(1.75, 5.0)$   \\
            \hline \hline
            \bf Span 4 & Node & $z$ & Copula Settings  & Margin Settings \\
            \hline
            $N$: 8000 & 1  & 2.8 & $\Theta_c:$ Clayton-90, $\theta_c: 2.0$ &  $T\sim\beta(5.0, 2.7)$, $k\sim\beta(1.75, 5.0)$ \\
            & 2  & 3.6 & $\Theta_c:$ Frank-90, $\theta_c: 8.0$ &  $T\sim \beta(5.0, 1.5)$, $k\sim\beta(1.75, 5.0)$   \\
            \hline
        \end{tabular}
        \label{tab:synth_settings}
    \end{center}
\end{table}

Samples are drawn with probability proportional to the inverse distance to the nearest specified node.
Let the subscript $(\cdot)_u$ denote the location of the upstream span and $(\cdot)_d$ denote the downstream grid. $d_{u_j}$ and  $d_{d_j}$ denote the distance from the centroid of the CTF face to the nearest upstream and downstream copula nodes respectively.  

The mixture joint density model in any given CTF face can be specified by equation \ref{eq:dist_weighted_synth}.
\begin{equation}
    h_j = \left( \frac{d_{u_j}}{|d_{d} - d_{u}|} \right) h_u +
    \left( \frac{d_{d_j}}{|d_{d} - d_{u}|} \right) h_d
    \label{eq:dist_weighted_synth}
\end{equation}
Where $h_u$ and $h_d$ are the upstream and downstream joint density models respectively with parameters specified in table \ref{tab:synth_settings}.
For simplicity, two copula nodes were specified per span though more are possible for a finer grained control over the marginal and copula distributions.    The copula nodes were located at the span extrema. This node specification pattern allows the synthetic data to mimic the expected sharp change in copula and marginal distributions when moving across spacer grids as seen in the raw CFD data presented in section \ref{sec:preprocessing} in figure \ref{fig:copula_predicted}.

The copula models were sampled in each span, the original CTF result was augmented with the synthetically generated noise in accordance with equation \ref{eq:synth_aug_discrete}.

\begin{figure}[H]%
    \centering
    \subfloat[Spatial axial augmented CTF result.]{{\includegraphics[width=0.45\linewidth]{figs/synth/pinH5TempOut} }}%
    \qquad
    \subfloat[2D rod map of sythetically augmented CTF result.]{{\includegraphics[width=0.45\linewidth]{figs/synth/cfd_pin_temperature} }}%
    \caption[Augmented CFD result.]{Augmented CTF temperature result.}%
    \label{fig:ctf_twall_aug}%
\end{figure}

\begin{figure}[H]%
    \centering
    \subfloat[Spatial axial augmented CTF result.]{{\includegraphics[width=0.45\linewidth]{figs/synth/pinH5TkeOut} }}%
    \qquad
    \subfloat[2D rod map of sythetically augmented CTF result.]{{\includegraphics[width=0.45\linewidth]{figs/synth/cfd_pin_tke} }}%
    \caption[Augmented CFD TKE result.]{Augmented CTF TKE result.}%
    \label{fig:ctf_tke_aug}%
\end{figure}

The augmented surface temperature and turbulent kinetic energy fields shown if figures \ref{fig:ctf_twall_aug} and \ref{fig:ctf_tke_aug} can be compared against the original CTF results provided in figures \ref{fig:ctf_twall_orig} and \ref{fig:ctf_tke_orig} respectively.  No azimuthal variations are present in the augmented fields which would be present if a physics based model, such as CFD, were used.  Additionally, no spatial auto-correlation in the temperature and TKE cladding surface fields are included in the synthetic data.  Spatial autocorrelation could be captured with a kriging model in the future, however, the one dimensional nature of the crud simulation code used in this work dictates that the fine scale spatial detail in the surface fields are irrelevant when computing surface-integrated crud quantities.


\subsection{Single Pin Reconstruction}

The rod surface is subdivided into CTF faces before fitting and reconstructing the synthetic data.  The location and extent of the CTF faces on the rod surface can be determined from a CTF output file.  

In each face the empirical quantile distributions of temperature, turbulent kinetic energy, and boundary heat flux distributions were computed.  The number quantiles used in the empirical quantile distribution was set at 20.  A uniform spacing of quantiles was used.  Copula were fit to the synthetic data based on maximum likelihood and the Akaike information criterion (AIC) in each CTF face.  Maximum likelihood estimation is described in section \ref{sec:fitting_copula} and the AIC may be computed from equation \ref{eq:cop_aic}.

For the synthetic single pin data the hi2lo predicted fractional surface area above a saturation temperature threshold ($T_{sat}$) is shown in figure \ref{fig:frac_a}.

\begin{figure}[H]%
    \centering
    \subfloat[CTF predicted fractional area of each CTF face above the saturation point.]{{\includegraphics[width=0.45\linewidth]{figs/synth/hi2lo/ctf_pin_t_threshold} }}%
    \qquad
    \subfloat[Hi2lo predicted fractional area of each CTF face above the saturation point.]{{\includegraphics[width=0.45\linewidth]{figs/synth/hi2lo/hi2lo_pin_t_threshold} }}%
    \caption[]{Fraction area above the saturation point prediction comparison for the synthetic data set.}%
    \label{fig:frac_a}%
\end{figure}

Provided that crud growth exhibits a temperature thresholding behavior about the saturation point it is important to predict the fractional area of the rod surface which exists above this critical temperature.  This can be performed by evaluating equation \ref{eq:pr_thresh} in each CTF face. Figure \ref{fig:frac_a} shows a substantial difference in the fractional area predicted above the saturation point in each CTF face when utilizing the hi2lo model rather than the predictions generated from a CTF computation alone.  The more significant the thresholding behavior of crud growth, the more important it becomes to accurately compute areas of the rod surface in excess of the saturation point.

Figure \ref{fig:patchscatter} examines the surface frequency distributions of temperature, crud boron mass and TKE for the CTF patch denoted by the red box in figure \ref{fig:hi2lo_tke_t}.  A marked change in behavior of the crud boron mass vs cladding surface temperature scatter plot is exhibited at $\approx 619[K]$.  For samples which fell at or below this temperature, little crud was grown and thus the deposited crud boron mass is small.  Past this temperature threshold, there is an approximately linear relationship between the surface temperature and the crud deposition rate.  

An additional feature of note in figure  \ref{fig:patchscatter} is the shape of the TKE vs. surface temperature scatter plot.  The dependence structure exhibits tighter coupling between these two fields at lower TKE values and less correlation at higher TKE.  This behavior was imposed by specifying a clayton copula in this location on the rod surface with the copula parameters given in table \ref{tab:synth_settings}.  The ability of the copula fitting routines to correctly preserve this skewed dependence structure is demonstrated in the figure and could not be achieved with Gaussian models.

\begin{figure}[H]
    \centering
    \includegraphics[width=0.99\linewidth]{figs/synth/patch_scatter_0_4_0_6}
    \caption[Single patch synthetic CFD data vs hi2lo sampled data.]{Single patch synthetic data vs hi2lo sampled data from patch centered on the rod at (3.14 $[rad]$, 2.85 $[m]$) at 300$[days]$.}
    \label{fig:patchscatter}
\end{figure}

\begin{figure}[H]%
    \centering
    \subfloat[Hi2lo temperature reconstruction.]{{\includegraphics[width=0.45\linewidth]{figs/synth/best_fit/struct_pin_twall} }}%
    \qquad
    \subfloat[Hi2lo TKE reconstruction.]{{\includegraphics[width=0.45\linewidth]{figs/synth/best_fit/struct_pin_tke_marked} }}%
    \caption{Hi2lo reconstruction of 2D surface temperature and TKE fields from synthetic CFD data source. }%
    \label{fig:hi2lo_tke_t}%
\end{figure}


After samples are independently drawn in each CTF face the temperature, TKE, and boundary heat flux samples were passed to a crud simulation packages as the cladding-side boundary conditions.  The CTF bulk fluid properties were used as the coolant-side boundary conditions.  The crud simulation was stepped forward for 300 days with a resample step size of $\Delta t_s =50$ days.  400 samples per CTF face were drawn at each resampling event. The resultant crud distribution at 300 days is given in figure \ref{fig:hi2lopincmass}.  Good agreement between the hi2lo predictions and the target synthetic data for the axial crud, temperature, and TKE distributions is exhibited.

\begin{figure}[H]%
    \centering
    \subfloat[Hi2lo axial crud thickness.]{{\includegraphics[width=0.45\linewidth]{figs/synth/best_fit/struct_pin_z_cthick300} }}%
    \qquad
    \subfloat[Hi2lo axial crud mass desnity.]{{\includegraphics[width=0.45\linewidth]{figs/synth/best_fit/struct_pin_z_cmass_300} }}%
    \caption{Hi2lo axial crud results compared to synthetic CFD/crud results at 300 days simulation time. }%
    \label{fig:hi2lopincmass}%
\end{figure}


Table \ref{tab:crud_totals_2} summarizes the hi2lo crud predictions for the synthetic single pin data set.  Good agreement is seen for the rod-integrated crud results.  Again, the hi2lo model was not used in a predictive manner and this represents a best-case scenario in which the copula and marginal distribution parameters can be directly estimated from the known (synthetic) CFD and CTF data sets.

\begin{table}[h]
    \begin{center}
        \caption[Crud totals for synthetic and hi2lo models.]{Single pin crud totals at 300 days.}
        \begin{tabular}[h]{|l | l | l |}
            \hline
            Copula, $\Theta_c$ & Crud Boron Total: $C_B$ & Crud Mass Total: $C_m$ \\
            \hline  \hline
            Synthetic CFD &  2.79049E-4 $[g]$ & 5.34151E-1 $[g]$ \\
            Hi2lo Reconstruction &  2.78012E-4  $[g]$ & 5.32146E-1 $[g]$ \\
            \hline
            Rel Diff &  0.374 $[\%]$ & 0.377 $[\%]$ \\
            \hline
        \end{tabular}
        \label{tab:crud_totals_2}
    \end{center}
\end{table}



\subsubsection{Crud Copula Parameter Sensitivity}
\label{sec:crud_copula_sensi}

Here the sensitivity of the crud result to the copula parameters is investigated.  Both the impact of the rank correlation coefficient, Kendall's $\tau$, and the Archimedean copula family are investigated.  The sensitivity results generated for the patch centered at $\{ \phi=3.14[rad], z=2.95[m]\}$ are shown in figure \ref{fig:patchcrudfit80}.  There is noise present in the crud predictions due to the Monte Carlo integration of equation \ref{eq:expected_crud} over the patch.  In this instance 2500 samples were used in the computation of the integral.  The crud is relatively insensitive to the choice of copula family, but the rank correlation coefficient is shown to have a significant influence on crud growth with an average boron deposition sensitivity of $\frac{\partial C_b}{\partial \rho_\tau} =$ -1.086e-7 $[g/cm^2/\tau]$ for this particular patch.  Accurately predicting Kendall's $\tau$ provided local core conditions is important.
\index{Copula!Crud Sensitivity}

\begin{figure}[H]
    \centering
    \includegraphics[width=0.7\linewidth]{figs/synth/patch_crud_fit_80}
    \caption{Single CTF face crud sensitivity to copula parameters.}
    \label{fig:patchcrudfit80}
\end{figure}

Next, two full single pin scenarios were considered. In the first scenario, shown if figure \ref{fig:crud_copula_fam_sensi}a, the best-fit copula on each patch as determined by the AIC metric is applied on each CTF face.  The second pin scenario enforces that a Gaussian copula model is used on every CTF face.  The crud results from these scenarios were then compared.  The data shows the choice of copula (between Gaussian, Frank, and Clayton) has a small overall impact on the total integrated rod boron mass.   The total integrated crud mass and crud boron mass for these scenarios at 300 days simulation time are given in table \ref{tab:crud_totals_copula}.

\begin{figure}[H]%
    \centering
    \subfloat[Best fit copula via AIC metric used in each CTF face.]{{\includegraphics[width=0.45\linewidth]{figs/synth/copula_compare/struct_pin_z_cmass_300_bestfit} }}%
    \qquad
    \subfloat[Gaussian copula used in each CTF face.]{{\includegraphics[width=0.45\linewidth]{figs/synth/copula_compare/struct_pin_z_cmass_300_gauss_only} }}%
    \caption[]{Influence of the choice parameters on the axial crud distribution.}%
    \label{fig:crud_copula_fam_sensi}%
\end{figure}


\begin{table}[h]
    \begin{center}
        \caption[Crud totals with different copula assumptions.]{Single pin crud totals at 300 days with different copula assumptions.}
        \begin{tabular}[h]{|l | l | l |}
            \hline
            Copula, $\Theta_c$ & Crud Boron Total: $C_B$ & Crud Mass Total: $C_m$ \\
            \hline  \hline
            Best Fit &  2.78953e-04 $[g]$ & 5.34015e-01 $[g]$ \\
            Gaussian &  2.78301e-04 $[g]$ & 5.32769e-01 $[g]$ \\
            \hline
            Rel Diff &  0.017 $[\%]$ & 0.234 $[\%]$ \\
            \hline
        \end{tabular}
        \label{tab:crud_totals_copula}
    \end{center}
\end{table}

The choice of the copula family, $\Theta_c$, has a negligible impact on the integrated crud results over a pin.  This result can reduce the complexity of the hi2lo model by removing the need to predict the correct copula family on each CTF patch in the core.  In section \ref{sec:preprocessing} it is shown that CFD data exhibits a complex relationship between the best fitting copula family and the axial position along the rod.  This relationship proved difficult to model using standard classification techniques, though further testing with a larger quantity of training data is warranted to ascertain if the copula family describing the dependence between the temperature and TKE fields on the rod surface can be accurately predicted given local core conditions.

\subsubsection{Crud Sample Size Study}

The number of samples, $N$, used to estimate the integral given in equation \ref{eq:mc_expected_crud} is a parameter set at runtime of the hi2lo method.  Here, it is shown that the integrated crud variance is reduced by increasing the number of samples used per CTF face to estimate the integrated crud quantities of interest.   Furthermore, section \ref{sec:Importance Sampling} demonstrates that improvements in sampling efficiency are possible by way of importance sampling.

Figure \ref{fig:cmprpintotalsviolinnsample} shows the variance of the crud expectation value at 300 days of simulation time computed by the Monte Carlo approximation when using a sample sizes of 100, 400, and $800  [\frac{N}{\mathrm{CTF_{face}}}]$.  80 independent trials were conducted for each sample size to estimate the variance of the pin integrated crud results at 300 days.

To isolate the impact of increasing the sample size on the crud variance, a single 300 day time step was conducted without re-sampling the underlying density functions during this period.  Importance sampling was not applied in this study.

\begin{figure}[H]
    \centering
    \includegraphics[width=0.7\linewidth]{figs/synth/nsample_study/cmpr_pin_totals_violin_nsample}
    \caption{Effect of sample size on the integrated crud results.}
    \label{fig:cmprpintotalsviolinnsample}
\end{figure}

The standard deviation of the rod-integrated crud results at 300 days simulation time are summarized in table \ref{tab:crud_sample_size}.

\begin{table}[h]
	\begin{center}
		\caption[Estimated sensitivity of the pin-integrated crud boron variance to the number of samples used per CTF face.]{Estimated sensitivity of the pin-integrated crud boron variance to the number of samples used per CTF face. Variance estimated using 80 independent trials.}
		\begin{tabular}[h]{| c | C | C |}
			\hline
			N  & Mean Pin Crud Boron Total $[g]$ & Pin Integrated Crud Boron Variance $[g]$ \\
			\hline  \hline
			100 &  2.78953e-04  & 3.80e-6  \\
			400 &  2.78301e-04  & 1.69e-6  \\
			800 & 2.78411e-04 & 8.93e-7 \\
			\hline
		\end{tabular}
		\label{tab:crud_sample_size}
	\end{center}
\end{table}


\subsubsection{Importance Sampling}
\label{sec:Importance Sampling}

To obtain estimates for the efficiency gain offered by importance sampling to compute the expected crud value by equation \ref{eq:mc_imp_expected_crud}, a singe patch was studied under synthetic TH data.

Here, the design of the sampling routines and the physics of crud growth is intertwined.  To compute the integral \ref{eq:expected_crud} efficiently it is favorable to sample the TH distribution in regions which result in relatively large amounts of crud growth.  To guide the design of the importance distributions the response surface of the crud simulation code is presented in figures \ref{fig:crud_sensi1} to \ref{fig:crud_sensi3}.  Larger surface temperatures result in a higher crud growth rate.  Larger local TKE results in smaller crud growth rates due to the effects of erosion.  Additionally of note is the relatively small influence of the boundary heat flux.  
% Crud growth rates are less sensitive to the boundary heat flux than to the cladding outer surface temperature.

\begin{figure}[H]%
    \centering
    \subfloat[Crud boron deposition sensitivity to temperature with TKE held fixed at $0.05$ {[J/kg]}]{{\includegraphics[width=0.45\linewidth]{../proposal/slides/seminar_slides/figs/dboron_dt_t} }}%
    \qquad
    \subfloat[Crud boron deposition sensitivity to TKE with temperature held fixed at $620$ {[K]}]{{\includegraphics[width=0.45\linewidth]{../proposal/slides/seminar_slides/figs/dboron_dt_tke} }}%
    \caption[]{Crud marginal response to varying temperature and TKE.}%
    \label{fig:crud_sensi1}%
\end{figure}

\begin{figure}[H]%
    \centering
    \subfloat[Crud boron deposition sensitivity with $q''=80 {[W/cm^2]}$.]{{\includegraphics[width=0.45\linewidth]{figs/crud/crud_t_tke_boron_response_80} }}%
    \qquad
    \subfloat[Crud boron deposition sensitivity $q''=120 {[W/cm^2]}$.]{{\includegraphics[width=0.45\linewidth]{figs/crud/crud_t_tke_boron_response_120} }}%
    \caption[]{Crud boron response surface to varying temperature and TKE.}%
    \label{fig:crud_sensi2}%
\end{figure}

\begin{figure}[H]%
    \centering
    \subfloat[Crud mass deposition sensitivity$q''=80 {[W/cm^2]}$.]{{\includegraphics[width=0.45\linewidth]{figs/crud/crud_t_tke_mass_response_bhf_80} }}%
    \qquad
    \subfloat[Crud mass deposition sensitivity $q''={[120 W/cm^2]}$.]{{\includegraphics[width=0.45\linewidth]{figs/crud/crud_t_tke_mass_response_bhf_120} }}%
    \caption[]{Crud mass response surface to varying temperature and TKE.}%
    \label{fig:crud_sensi3}%
\end{figure}


Though an optimal importance distribution, $\tilde{h}^* $, can be found by equation \ref{eq:optimal_imp},

\begin{equation}
\tilde{h}^* = \mathrm{argmin}_{\tilde{h}}\mathrm{Var} \left[ \frac{\mathcal{G}(x)h(x)}{\tilde{h}(x)} \right] 
\label{eq:optimal_imp}
\end{equation}

the requisite minimization problem is not solved in this work and is left as an avenue for future investigation. It can be shown that the optimal importance distribution follows the form: $\tilde{h}^* \propto |\mathcal{G}(x)|h(x)$, \cite{rubinstein2011}, \cite{mcbook}.

Although the theoretically optimal importance distribution is not achieved in this work, a locally adaptive importance function was adopted based on a distribution mixing approach.  Through the mixture formulation, the importance distributions can be made to depend on the temperature and TKE marginal distributions in a particular CTF face.  The temperature and TKE distributions in each CTF face are mixed with a beta distribution who's parameters are set at runtime of the hi2lo tool.  

The importance  mixture quantile functions are defined in equations \ref{eq:imp_mix_k} and \ref{eq:imp_mix_T}.

\begin{align}
\tilde Q_{j,k} &= \lambda_{0,k} \hat Q_{j,k}  + \lambda_{1,k} Q_{\beta_k}(k; \vartheta_k),  \label{eq:imp_mix_k} \\
\tilde Q_{j,T} &= \lambda_{0,T} \hat Q_{j,T}  + \lambda_{1,T} Q_{\beta_T}(T; \vartheta_T);  \label{eq:imp_mix_T} \\
\sum_i \lambda_i &= 1, \ \ \tilde F_{j,T} = \tilde Q^{-1}_{j,T},\ \  \tilde F_{j,k} = \tilde Q^{-1}_{j,k}
\end{align}
Where $\tilde Q_{j,T}$ is the quantile function for the proposal temperature distribution in the $j^{th}$ CTF face. $\lambda_i$ are user set mixture weights. Standard Monte Carlo sampling can be recovered by setting $\lambda_{0,k}=1,\ \lambda_{0,T}=1$ and  $\lambda_{1,k}=0,\ \lambda_{1,T}=0$.

Beta distributions, with quantile functions $Q_{\beta_k}$ and $Q_{\beta_T}$,  with proscribed parameters, $\{ \vartheta_k, \vartheta_T \}$, are used in mixture with the original target temperature and TKE density functions to produce a proposal density distribution for each patch.  To provide additional flexibility in the design of proposal density, the mixture weights can be adjusted. By suitably tuning the parameters of the beta distributions and mixture weights, one can target the hot locations of the rod which occur in coincidence with low TKE.  Mixture settings denoted in table \ref{tab:hi2lo_params} were adopted for this work.  Shown in figure \ref{fig:imp_sample2} with proper tuning of the mixture distribution parameters, the sampling distribution can be skewed towards higher temperatures and lower TKE.

\begin{figure}[H]%
    \centering
    \subfloat[Temperature distributions.]{{\includegraphics[width=0.45\linewidth]{figs/imp_patch/temperature_importance_marginal_compare} }}%
    \qquad
    \subfloat[TKE distributions.]{{\includegraphics[width=0.45\linewidth]{figs/imp_patch/tke_importance_marginal_compare} }}%
    \caption[]{Proposal (red) vs. original (blue) marginal distributions for the (a) surface temperature and (b) TKE.  Generated using importance distribution parameters: $\vartheta_T = \{1, 0.9\}$,  $\vartheta_k = \{1.1, 1.2\}$,  $\lambda_{0,T}=0.6,\ \lambda_{1,T}=0.4$ and $\lambda_{0,k}=0.7,\ \lambda_{1,k}=0.3$.}%
    \label{fig:imp_sample2}%
\end{figure}


\begin{figure}[H]%
    \centering
    \subfloat[Crud boron mass deposition.]{{\includegraphics[width=0.45\linewidth]{figs/imp_patch/bmass_sample_violin} }}%
    \qquad
    \subfloat[Crud mass deposition.]{{\includegraphics[width=0.45\linewidth]{figs/imp_patch/cmass_sample_violin} }}%
    \caption[]{Importance sampling trial results on a single CTF face for (a) crud boron mass deposition and (b) Crud mass deposition at 300 days.  Red denotes importance samples and blue denotes standard Monte Carlo samples.  The sample population variance is shown above the figure for each case and the full sample distributions are given in the margins.}%
    \label{fig:imp_sample1}%
\end{figure}

In figure \ref{fig:importancettkebmassscatter} the relative importance weight is denoted by the size of each point in the scatter plot.  Samples which have a small ratio $(h_i/\tilde h_i)$ appear as small points.  The sample weight is analogous to the rod surface area occupied by the sample.  In comparison, figure \ref{fig:originalttkebmassscatter} shows the same patch using a standard Monte Carlo sampling where each sample has the same weight.  The number of samples drawn in the upper tail of the temperature distribution is greater when importance sampling is applied, though these samples carry expectedly small sample weights.

\begin{figure}[H]
    \centering
    \includegraphics[width=0.99\linewidth]{figs/imp_patch/importance_t_tke_bmass_scatter}
    \caption[Importance sampled single patch crud result.]{Importance sampled single patch crud result. Key: \textbf{quant\_t\_imp}: Importance samples from quantile-reconstructed patch temperature $[K]$ distribution with $N_{Q_T}=20$,  \textbf{quant\_tke\_imp}: Importance samples from quantile-reconstructed patch TKE $[J/kg]$ distribution with $N_{Q_k}=20$,  \textbf{bmass}:  Resultant crud boron mass density samples in $[g/cm^2]$. Relative importance weights are denoted by the point size.}
    \label{fig:importancettkebmassscatter}
\end{figure}

\begin{figure}[H]
    \centering
    \includegraphics[width=0.99\linewidth]{figs/imp_patch/original_t_tke_bmass_scatter}
    \caption[Standard Monte Carlo sampled patch crud result.]{Standard Monte Carlo sampled patch crud result. Key: \textbf{quant\_t}: Samples from quantile-reconstructed patch temperature $[K]$ distribution with $N_{Q_T}=20$,  \textbf{quant\_tke}: Samples from quantile-reconstructed patch TKE $[J/kg]$ distribution with $N_{Q_k}=20$,  \textbf{bmass}: Resultant crud boron mass density samples in $[g/cm^2]$.  }
    \label{fig:originalttkebmassscatter}
\end{figure}
\index{Importance Sampling!Application}

The importance sampling efficiency can be estimated by computing the variance ratio:  $\sigma^2_{MC}/\sigma^2_{I}$.  The variance of the patch-integrated crud result for the Monte Carlo and importance sampling schemes are provided figure \ref{fig:imp_sample1}.  The variance estimates were computed by running 1000 independent trials in which crud was grown on the patch for 300 days.  A total of 100 samples per patch per trial were used.  For the case studied, the application of importance sampling reduced the crud mass and boron mass sample variance by a factor of 2.02. $\frac{\sigma^2_{MC}}{\sigma^2_{I}} \approx (\num{4.979e-6})^2 / (\num{3.503e-6})^2  \approx 2.02$.  The mean crud predictions did not significantly deviate between two sampling schemes indicating that importance sampling does not introduce any bias in the evaluation of the integral in equation \ref{eq:expected_crud}.

The improvement in performance afforded by importance sampling can be attributed to expending a larger proportion of the total available samples in the upper tail of the temperature distribution as this is a region which strongly contributes to crud growth.  Some samples are necessarily expended in cold regions of the rod surface but occur with less frequency when compared to a standard Monte Carlo sampling routine and the samples are appropriately weighted to avoid biasing the integrated result.


\subsection{Single Pin with Time Stepping}

Stepping the crud simulation forward under the application of hi2lo supplied boundary conditions demands careful treatment of hot spot stationarity assumptions.  The time evolution of the crud simulation on the rod surface is strongly influenced by choices made both in the number of re-sampling steps taken as well as tunable constants which govern the sample remapping procedure and surface temperature mixing.

\subsubsection{Spatial Remapping with Time Stepping}

The influence of hot-spot stationarity assumptions on the overall integrated crud mass on the rod as a function of time can be seen in figures \ref{fig:cmprpintotals0406} and \ref{fig:cmprpintotalsnoremap}.  When the surface temperature is allowed to randomly mix on each re-sampling event in each CTF face, the influence of the hot spots are smeared over the surface of the rod which leads to an overall under prediction in the total integrated crud mass.  Reordering the samples in each CTF face by their temperature improves the ability of the hi2lo model to preserve the impact of stationary hot and cold spots on the rod surface.  Good agreement with the original coupled synthetic CFD-Crud simulation data was achieved by tuning the constants introduced in equation \ref{eq:weighting} to values of $w_T = 0.4$, and of $w_k = 0.6$.  This weighting seeks to preserve a heuristic thermal-hydraulic metric on the rod surface  where the metric can be interpreted as some linear combination of cladding surface temperature and near-wall TKE. 


\begin{figure}[H]
    \centering
    \includegraphics[width=0.7\linewidth]{figs/synth/cmpr_pin_totals_0_4_0_6}
    \caption[Total integrated crud boron mass vs. time using approximately optimal remapping weights.]{Total integrated crud boron mass vs. time using approximately optimal remapping weights ($w_T=0.4, w_{k}=0.6, w_{q''}=0.0$).}
    \label{fig:cmprpintotals0406}
\end{figure}
\begin{figure}[H]
    \centering
    \includegraphics[width=0.7\linewidth]{figs/synth/cmpr_pin_totals_no_remap}
    \caption{Total integrated crud boron mass vs. time without remapping samples.}
    \label{fig:cmprpintotalsnoremap}
\end{figure}

A spatial representation of the samples pre and post-remapping are shown in figure \ref{fig:remmap_comp}.  A visual representation of the remapping strategy presented is presented in figure \ref{fig:samplemapping}.  While the spatial distribution of the temperature and TKE fields are distinctly different after the remapping procedure is applied, the joint density distributions formed by the sample population over the patch are identical. 

\begin{figure}[H]%
    \centering
    \subfloat[Remapped surface samples.]{{\includegraphics[width=0.45\linewidth]{figs/synth/patch_fields_0_4_0_6} }}%
    \qquad
    \subfloat[Non-remapped surface samples]{{\includegraphics[width=0.45\linewidth]{figs/synth/patch_fields_no_remap} }}%
    \caption[Re-mapped and non remaped temperature and TKE surface samples]{Re-mapped (a) and non remaped (b) temperature and TKE surface samples for a single CTF face.}%
    \label{fig:remmap_comp}%
\end{figure}

\subsection{Re-Sample Frequency}
\label{sec:resample_freq_study}

The frequency at which the distribution functions are sampled from on each CTF face influences the variance in the predicted integrated crud results.  To investigate this behavior, a parameter sweep was conducted in which the same pin was.  50 independent trials were conducted for each step size.  Shown in figure \ref{fig:cmprpintotalsviolin} a smaller re-sampling steps size, $\Delta t_s$, results in a reduction in the variance of the rod integrated crud estimates at 300 days of simulation time.   Additionally, the sampling-induced rod-integrated crud mass uncertainties were shown to be approximately distributed according to a normal distribution.  It is also important to note that the variance of the rod integrated crud results increases as a function of time.

\begin{figure}[H]
    \centering
    \includegraphics[width=0.7\linewidth]{figs/synth/tstep_study/cmpr_pin_totals_violin}
    \caption[Influence of the re-sample frequency on the predicted integrated crud variance.]{Influence of the re-sample frequency on the predicted rod-integrated crud variance.  Variance estimates are given in the right hand side of the figure from 50 independent trials.}
    \label{fig:cmprpintotalsviolin}
\end{figure}

Performing a larger number of re-sampling events per VERA state results in reduced variance at little additional computational effort.
This is in part due to the minimal computational requirements of sampling the joint temperature and TKE distribution on each patch.  Drawing samples from a bivariate copula density model is straight forward and can be done in parallel since each patch is treated as an independent sampling zone in this hi2lo approach.  The crud computation, by comparison, is more expensive.  Increasing the re-sampling frequency does not increase the total number of samples used per pin per time step, rather, this process only increases the number of (re-sample) steps per VERA state point.  The reduction in variance stems from an improved sample density throughout time of the underlying random field.  In time, the underlaying random field is fixed throughout a VERA state point.  Repeatedly drawing samples from this field at small time steps rather than sampling the random field only once at the beginning of the VERA state point vastly increases the number of samples used to perform the time integration of the crud result on each CTF face.


\section{Section Takeaways}
\begin{itemize}
        \item  Good rod-integrated crud agreement between the synthetic source data and the predicted results was observed.  The rod-integrated results were summarized in table \ref{tab:crud_totals_1}.  Fitting copula and marginal distributions to the sample data, re-sampling from these models, and then growing curd using these samples reproduces the correct total amount of crud at each time step and correctly reproduces the axial distribution of crud as shown in figure \ref{fig:hi2lopincmass}.
        \item Increasing the number of samples drawn per CTF patch decreases variance in total crud mass and total precipitated boron estimates.  The number of samples used is an adjustable runtime parameter and can be increased depending on the available computational resources and accuracy desired.
        \item Increasing the number of re-sampling steps per VERA state point reduces variance in the final integrated crud results. 
        \item After drawing samples from the joint density distribution a reordering of the samples on the rod surface is necessary to preserve hot spot stationarity.  Demonstrated in figure \ref{fig:cmprpintotals0406} and \ref{fig:cmprpintotalsnoremap}, sample remapping with weights of $w_T=0.4,\ w_k=0.6$ was performed to achieve an optimal time marching sampling strategy.
        \item Importance sampling was shown to reduce the variance in the integrated crud results.
        \item A synthetic data generation tool allows absolute control over the properties of joint distribution of TH boundary conditions which are fed into the crud simulation code.  Since the synthetic data has known statistical properties, this data serves as an important data source for benchmarking and validation operations.
\end{itemize}

%! TEX root = ../dissertation_gurecky.tex


%-----------------------------------------------------------------------------%
\chapter{Model Performance Under a CFD Data Source}
\label{chap:fw}
%! TEX root = ../dissertation_gurecky.tex
\label{sec:ml_cfd}

For deployment as an in-line statistically based downscaling tool which sits between a subchannel code and a crud simulation code in a core simulator such as VERA the model is required to perform the hi2lo mapping for all pins in the core at any operating condition.  In other words, the model must be evaluable at any local core conditions typical of an operating PWR.  Since the training data set cannot contain all possible pin geometries, loading configurations and operating conditions due to computational expense, the model produces a prediction for the copula and marginal distribution parameters between known states.

One might envision a table-lookup approach where high fidelity CFD flow field maps are precomputed and stored for a wide array of flow and power conditions.  A nearest neighbor interpolation scheme could then be applied to extract the best-matching CFD map provided some local core state by VERA.  This is not tractable since the number of CFD computations to build the data base would be prohibitively large.  Instead of storing spatial CFD hi2lo maps, CFD data is distilled into a set of statistics tabulated as a function of local core state.

% It is appealing to transform the problem of hi2lo construction from performing spatial flow map prediction into one of summary statistics prediction where much less information is required adequately fill out the operating envelope.

In this chapter the hi2lo methodology introduced in this work is exercised against a small CFD data set derived from a 5x5 fuel assembly operating at realistic PWR conditions.  A leave-one-out cross validation strategy is used to assess the predictive performance of the model.


\section{CFD Data Source}
\label{sec:cfd_data_source}

For the generation of high fidelity CFD data sets the Westinghouse 5x5 test stand shown in figures \ref{fig:5x5topdown} and \ref{fig:5x5side} was used to prepare the CAD geometry.  The CFD mesh consisted of approximately 25 million cells and 1e5 surface elements per pin.  A matching CTF input deck for the 5x5 assembly was also constructed with 100 axial zones.  The CTF and CFD codes were then executed for a variety of flow conditions and power levels.  StarCCM+ was utilized for the CFD simulations in this work.

\begin{figure}[H]
    \centering
    \includegraphics[width=0.5\linewidth]{figs/5x5/5x5_top_down}
    \caption[Top down view of 5x5 pin Westinghouse facility.]{Top down view of 5x5 pin Westinghouse facility.  Assembly dimensions and pin powers redacted.}
    \label{fig:5x5topdown}
\end{figure}

\begin{figure}[H]
    \centering
    \includegraphics[width=0.5\linewidth]{figs/5x5/5x5_side}
    \caption[Side view of 5x5 pin Westinghouse facility.]{Side view of 5x5 pin Westinghouse facility.  Pin dimensions redacted.}
    \label{fig:5x5side}
\end{figure}

For this rod configuration, the axial pin power, total power and CFD simulation rod surface temperature distributions are available in external references.  This information is purposely withheld from this document to protect the intellectual property of the Westinghouse electric company.


\subsection{Preprocessing}
\label{sec:preprocessing}

Preprocessing requires paired CFD and CTF results for a given pin generated with consistent boundary conditions between the codes.  This requires consistent geometry, inlet, outlet and power distributions between the codes.

The cladding surface temperature and near-wall TKE CFD fields are spatially aggregated onto the CTF face centers.  The aggregation requires that the location and extent of each CTF face is known.  These CTF face attributes are accessible from a CTF output file.  In the aggregation procedure spatial information is discarded within each CTF patch as the spatial fields are agglomerated into sample distributions.
In each CTF face each of the features given in table \ref{tab:features} are computed from the available CTF (or VERA) results.
The aggregated CFD data fields are then associated with their corresponding feature set.
In each face, the aggregated CFD field distributions are subtracted from the mean CTF predictions and the resultant (CFD-CTF) residual distributions are stored in a HDF5 table along with the associated predictive variables.

Next, correlation statistics are computed from the residual distributions.
Copula fitting by the maximum likelihood method with AIC model selection is carried out on each CTF face.  Additionally, the empirical Kendall's $\tau$ rank correlation coefficient is computed from the raw CFD data on each CTF face. Figure \ref{fig:copula_predicted} shows the copula parameters estimated from the raw CFD data on each pin as a function of axial position for the first 4 pins in the 5x5 CFD model.  There is a marked change in behavior of the copula between the pins.  This was an unexpected find since the flow patterns were speculated to be reasonably consistent from pin to pin.  Also, the influence of spacer grids on the correlation coefficient between the temperature and TKE fields is visible.  Across spacer grids the correlation coefficient sharply falls indicating a tighter coupling between the TKE and temperature surface fields as the flow necks down when entering a grid.  This is followed by a sharp change in Kendall's $\tau$ towards zero indicating the temperature and TKE surface fields become less correlated immediately following the mixing vanes.  This change in correlation behavior is posited to be due to turbulent mixing effects.  The computed copula parameters are also stored alongside the raw temperature, TKE, and boundary heat flux aggregated residual distribution data in the HDF5 table.

\begin{figure}[H]%
    \captionsetup[subfigure]{justification=centering}
    \centering
    \subfloat[Pin 1]{{\includegraphics[width=0.53\linewidth]{figs/preproc/copula_params_pin_1} }}\hspace*{-1.0em}%
    \subfloat[Pin 2]{{\includegraphics[width=0.53\linewidth]{figs/preproc/copula_params_pin_2} }}\hspace*{-1.0em}%
    \\
    \subfloat[Pin 3]{{\includegraphics[width=0.53\linewidth]{figs/preproc/copula_params_pin_3} }}\hspace*{-1.0em}%
    \subfloat[Pin 4]{{\includegraphics[width=0.53\linewidth]{figs/preproc/copula_params_pin_4} }}\hspace*{-1.0em}%
    \caption[Best fitting copula to CFD data.]{Best fitting copula determined by AIC model selection as a function of axial rod position.}%
    \label{fig:copula_predicted}%
\end{figure}

After pre-processing, the HDF5 table includes a list of predictive scalar values, which are shown in table \ref{tab:features}, and a list of associated response variables comprised of the copula parameters and residual sample distribution for $\{T,k,q''\}$ on each CTF face.

\subsection{Feature Engineering}
\label{sec:feature_eng}

The objective of feature engineering is to select a predictive variable set that describes the behavior of the conditional quantiles and copula everywhere in the assembly.

\begin{table}[h]
    \begin{center}
    \caption[Included exogenous training features.]{Features included in the gradient boosted models as exogenous variables.}
\begin{tabular}[h]{|l | l | l | l |}
    \hline
    Sym & Label & Feature & Unit \\
    \hline
    \hline
    $T$ & ctf\_twall\_avg & CTF Face surface temperature & $[K]$ \\
    $R_T$ & ctf\_twall\_range & Surface temperature range in 4 adjacent faces & $[K]$ \\
    $q''$ & ctf\_bhf\_avg & Local CTF face heat flux & $[W/m^2]$ \\
    $R_{q''}$ & ctf\_bhf\_range & Heat flux range in 4 adjacent faces & $[W/m^2]$ \\
    $u_z$ & w\_bulk & CTF subchannel bulk Z Velocity &  $[m/s]$ \\
    $k$ & ctf\_tke\_avg & Local CTF face near wall TKE &  $[J/kg]$ \\
    $R_k$ & ctf\_tke\_range & CTF TKE range in 4 adjacent faces & $[J/kg]$ \\
    $z$ & z & Global axial position & $[m]$ \\
    $\delta z_g$ & dz\_grid & Position relative to nearest spacer grid & $[m]$ \\
    $N_g$ & n\_upsteam\_grid  & Nearest upstream spacer grid ID & $[]$ \\
    $T_\infty$ & t\_bulk & Subchannel bulk temperature  &  $[K]$ \\
    \hline
\end{tabular}
\label{tab:features}
\end{center}
\end{table}

The predictive variables given in table \ref{tab:features} were selected based on two criteria:  Availability and orthogonality.  In order to evaluate the trained machine learning model at a TH state point each conditioning variable should be made available by VERA or must be computable from CTF results and supplied as input to the trained machine learning models.   The exogenous variable set given in table \ref{tab:features} comprise the local core conditions at any given CTF face.  The machine learning model uses the local core conditions as the exogenous feature set, thus these features must be supplied to the fitted gradient boosted regressors at runtime in order to evaluate the model.

% This has to be true since when evaluating the hi2lo model the inputs to the hi2lo model are required to be derived from either previously stored information or information made available by CTF at runtime.

At this juncture, the availability criteria precludes using some geometric information such as the orientation of a given spacer grid since it is not possible to extract or infer this information from the CTF output.  Including additional geometric information into the exogenous variable set could potentially increase the ability of the machine learning models to distinguish unique CTF faces in the core though testing of this hypothesis is left to a future investigation.  Additional software infrastructure is required to include and extract additional features from the CTF or VERA output files.

It is not useful to include features which are co-linear into the explanatory feature set. The bulk fluid density was not included in the predictive variable set as it strongly depends on the local temperature. Likewise the local static pressure was not used as a predictive variable since this would be approximately one-to-one with the axial position.  The exclusion of this TH information is primarily done for computational saving when training the boosted models since, as opposed to other machine learning algorithms and statistical inference techniques, gradient boosting is robust to collinearity of features in the input space.

In the case of gradient boosting the inclusion of nuisance or collinear exogenous variables in the model will not necessarily reduce the model's ability to generalize to unseen data, only hamper computational efficiency.  The resulting feature importance plot shown in figure \ref{fig:ktauregfeatureimp} suggests that the relative axial position within a span does not provide predictive power since this information is redundant provided the absolute axial position and the nearest upstream spacer grid are included in the feature set.

\begin{figure}[H]
    \centering
    \includegraphics[width=0.6\linewidth]{figs/ktau_reg_feature_imp}
    \caption[Relative feature importance.]{Relative feature importances on Kendall's $\tau$.}
    \label{fig:ktauregfeatureimp}
\end{figure}

Since the boosted regression (and classification) models are insensitive to multi-collinearity in the feature space, the application of principal component analysis to the training data set was not pursued.


\subsection{Cross Validation}

An estimate of the per pin crud prediction errors incurred when evaluating the trained models at unknown CFD states were made by performing a leave one out (LOO) cross validation study.  Cross validation is used to estimate how well the machine learning models employed in this work generalized to previously unseen local core conditions; i.e. core conditions that are not included in the training data set.

\begin{figure}[h]
    \centering
    \includegraphics[width=0.4\linewidth]{figs/drawings/5x5_loo}
    \caption[Example pin layout for leave-one-out cross validation procedure.]{Example pin layout for leave-one-out cross validation procedure.  The gradient boosted models are trained on CFD and CTF data extracted from the blue pins.  Crud predictions are made on the missing pin.}
    \label{fig:5x5loo}
\end{figure}

The LOO cross validation procedure is depicted in figure \ref{fig:5x5loo}.  In this procedure a single CFD-CTF pin pair is removed from the database and then the model is trained on remaining data.  Following data culling and training, the machine learning model is evaluated and crud predictions are made at the missing pin's TH conditions.
The predicted crud results are compared against crud results generated using the original CFD data for the missing pin.  This process is repeated for each pin in the 5x5 assembly.
% The differences are summarized and averaged to obtain a measure of model's predictive performance when applied to TH conditions that reside in the TH envelope included in the training set.

This cross validation technique ascertains crud prediction errors within the TH envelope enclosed by the original full 25 pin training set.  The resulting crud prediction error estimates cannot be extrapolated to core conditions that lay outside of the thermal hydraulic envelope formed by the training set.  For a robust crud prediction error analysis, a much larger training data set is required which would essentially span all possible TH conditions encountered in an operational PWR.  This will require large scale CFD runs and is left as an avenue for future uncertainty quantification work.  A larger training set would also increase the viability of other multi-fold cross validation techniques which require permutations of stratified chunks to be excised from the training pool in their application.  This would involve removing multiple pins from the training set.

\subsection{Quantile Regressors}

A principal goal of the machine learning model is to predict the conditional quantiles of the temperature and TKE distributions as a function of local core conditions.  In this light, the trained quantile regression models are compared against the left-out CFD data set on each pin.  The accuracy of both the TKE and temperature quantile regressors is assessed using both quantile-quantile plots and quantile vs axial rod position comparisons.

Quantile-quantile (Q-Q) plots of the temperature and TKE residual distributions are used to elucidate bias introduced by the machine learning model in the conditional quantiles at a variety of axial positions and local core conditions.  Estimated quantiles are obtained for the left-out pin by evaluating the trained reduced LOO model and are compared to the expected CFD result.
A subset of the TKE residual quantile regression results are given in figures \ref{fig:tkepin1} to \ref{fig:tkepin3}.  A complete set of quantile regression results are provide in appendix \ref{chap:app_ml}. The Q-Q plots summarize the biases in the conditional quantile distributions when compared to the target golden standard CFD data.  The maximum and average Kolmogorov–Smirnov (KS) statistic is provided in the Q-Q figures for each pin.  The KS statistic is given by equation \ref{eq:ks_stat}.
\begin{equation}
    KS = \mathrm{sup}(\{\hat F(q_{\tau}) - F(q_{\tau})\})
\label{eq:ks_stat}
\end{equation}
\index{Kolmogorov–Smirnov Testing}

Where $\mathrm{sup}(\cdot)$ is the supremum of the set of distances between the predicted and empirical cumulative densities.  The cumulative densities are supported at the specified quantile levels: $\{\mathbf \tau\} = \{0.000, 0.0526, 0.1052, ... 1.000 \}$ since the number of quantiles used in the reconstruction of the marginal temperature and TKE distributions was set to be 20 and were evenly spaced.
The KS statistic was computed at each axial level on the CTF grid.

A two sample KS test was performed on the temperature and TKE distribution reconstructions from predicted quantiles on each CTF axial edit for every pin in the assembly. The null hypothesis is that the predicted and empirical (CFD derived) distributions are the same on a given CTF axial zone.  To reject the null hypothesis the KS distance must satisfy the inequality  \ref{eq:ks_crit}.

\begin{equation}
    KS_D^* > c(\alpha) \sqrt{\frac{n+m}{nm}}
\label{eq:ks_crit}
\end{equation}

Where $n=20$ in this case since the predicted CDFs are supported at 20 locations. The number of CFD surface samples available to construct the empirical distribution, $m$, on each axial edit was approximately $800$, though this varied slightly from zone to zone and is dependent on the CFD mesh density on the rod surface.  In general $c(\alpha) = \sqrt{-\frac{1}{2} \mathrm{ln}\alpha}$, therefore at $\alpha=0.1$, $KS_D^* \approx 0.243$.  A summary of the KS distances and test results are provided in table \ref{tab:ks_temp} for the temperature quantiles and in table \ref{tab:ks_tke} for the predicted TKE quantiles.

\begin{table}[h]
    \begin{center}
    \caption[KS statistic TKE summary.]{TKE distribution KS statistic summary. Values in bold result in rejection of the null hypothesis at significance level $\alpha=0.1$.}
    \begin{tabular}[h]{|c|c|c|}
        \hline
        Pin & $KS_\mu$ & $KS_{max}$ \\
\hline
1 &  2.949e-02 &  \textbf{3.7027e-01} \\
2 &  8.487e-02 &  \textbf{3.7030e-01} \\
3 &  2.999e-02 &  1.9407e-01 \\
4 &  1.111e-01 &  \textbf{5.5811e-01} \\
5 &  3.175e-02 &  1.8512e-01 \\
6 &  2.891e-02 &  2.3469e-01 \\
7 &  4.466e-02 &  2.2317e-01 \\
8 &  5.642e-02 &  \textbf{3.2113e-01} \\
9 &  1.943e-02 &  1.3405e-01 \\
10 & 5.650e-02 & \textbf{2.8362e-01} \\
11 & 3.155e-02 & 1.5097e-01 \\
12 & 6.701e-02 & \textbf{4.3280e-01} \\
13 & 3.847e-02 & 2.3050e-01 \\
14 & 4.220e-02 & \textbf{3.5174e-01} \\
15 & 5.040e-02 & \textbf{2.7588e-01} \\
16 & 9.001e-02 & \textbf{4.1846e-01} \\
17 & 2.129e-02 & 2.0583e-01 \\
18 & 2.702e-02 & 1.7656e-01 \\
19 & 2.959e-02 & 1.5939e-01 \\
20 & 3.804e-02 & 1.8140e-01 \\
21 & 3.092e-02 & 2.3890e-01 \\
22 & 2.263e-02 & 1.6278e-01 \\
23 & 2.181e-02 & 1.4727e-01 \\
24 & 3.227e-02 & 1.3258e-01 \\
25 & 2.728e-02 & 1.5842e-01 \\
\hline
\end{tabular}
\label{tab:ks_tke}
\end{center}
\end{table}

By inspecting the KS test results presented in tables \ref{tab:ks_tke} and \ref{tab:ks_temp} it can be concluded that the prediction of the conditional temperature distribution on each CTF axial edit was far more difficult than predicting the conditional TKE distribution.  A large maximum KS temperature distribution distance was seen for the majority of the pins in the assembly. The worst performing pins in this respect were pin 4, pin 9, and pin 20. This is due to the aforementioned high span-to-span and pin-to-pin repeatability of the TKE distributions and conversely the low repeatability of the temperature distribution.  Since the maximum KS distance may occur in CTF axial edit which do not contain temperatures in excess of the saturation point, the maximum KS distance is not an indicator of poor crud predictive performance.  The presented KS tests only serve to quantify the ability of the gradient boosted quantile regressors to reproduce the expected distributions.

Pin-average KS distances indicate that the the null hypothesis was not rejected in the majority of CTF axial zones, however.  This indicates that the distributions predicted by the gradient boosted quantile regressors were, on average, statistically indistinguishable from the empirical CFD distributions. 

\begin{table}[h]
    \begin{center}
    \caption[KS statistic temperature summary.]{Temperature distribution KS statistic summary. Values in bold result in rejection of the null hypothesis at significance level $\alpha=0.1$.}
    \begin{tabular}[h]{|c|c|c|}
        \hline
        Pin & $KS_\mu$ & $KS_{max}$ \\
\hline
1 &  4.329e-02 &  \textbf{3.2094e-01} \\
2 &  8.477e-02 &  \textbf{3.9065e-01} \\
3 &  6.725e-02 &  \textbf{4.4959e-01} \\
4 &  1.509e-01 &  \textbf{6.0957e-01} \\
5 &  4.483e-02 &  \textbf{2.7359e-01} \\
6 &  8.718e-02 &  \textbf{4.4920e-01} \\
7 &  8.848e-02 &  \textbf{3.2938e-01} \\
8 &  5.668e-02 &  \textbf{4.2560e-01} \\
9 &  7.273e-02 &  \textbf{6.3407e-01} \\
10 & 1.292e-01 &  \textbf{4.6647e-01} \\
11 & 6.765e-02 &\textbf{ 3.5969e-01} \\
12 & 1.124e-01 & \textbf{4.9874e-01} \\
13 & 5.064e-02 & \textbf{3.2152e-01} \\
14 & 6.728e-02 & \textbf{4.7099e-01} \\
15 & 1.429e-01 &\textbf{ 5.4111e-01} \\
16 & 7.260e-02 & \textbf{3.9793e-01} \\
17 & 3.975e-02 & 2.3933e-01 \\
18 & 3.209e-02 & 2.5577e-01 \\
19 & 6.852e-02 &\textbf{ 3.7904e-01} \\
20 & 1.273e-01 &\textbf{ 8.3754e-01} \\
21 & 3.059e-02 &\textbf{ 3.8896e-01} \\
22 & 1.147e-01 &\textbf{ 6.6852e-01} \\
23 & 6.098e-02 &\textbf{ 4.6728e-01} \\
24 & 6.525e-02 &\textbf{ 4.7225e-01} \\
25 & 4.811e-02 & \textbf{ 3.3235e-01} \\
\hline
\end{tabular}
\label{tab:ks_temp}
\end{center}
\end{table}


Caution should be observed when drawing conclusions from this goodness-of-fit study.
The two sample KS test is generally regarded as a statistically weak, requiring a relatively large number of samples and high KS distance to reject the null hypothesis \cite{ks_power} \cite{KHAMIS1990317}.  The statistical power of a hypothesis test is defined as the probability of avoiding a type II error.  In the currently considered case there is high probability of committing type II errors, or in other words, failing to reject the null hypothesis.  For this reason and provided only 20 quantiles available for use in the KS test there is insufficient evidence to conclude the gradient boosted quantile regression models properly reproduced the expected temperature and TKE distributions on each face.  In future work, a larger number of CFD data points and larger number of quantile regressors should be used to improve the ability of the KS test to identify incongruence between the model predictions and the expected distributions.

Note that since a LOO CV technique was used for comparing the predicted distributions to the empirical CFD distributions complications in the KS test which arise when the parameters of the predicted distribution are estimated from the target empirical data set were avoided \cite{kstestInfo}.   Under these circumstances the KS test would no longer be valid though methods based on bootstrap resampling have been proposed to resolve this specific limitation of the traditional KS test \cite{kstestInfo}.

\begin{figure}[H]%
    \captionsetup[subfigure]{justification=centering}
    \centering
    \subfloat[][TKE quantile regression results. \\ CFD in dashed line.  Predicted values as solid. \\ Azimuthally integrated.]{{\includegraphics[width=0.53\linewidth]{figs/ml_fit/q_tke_regression_1} }}\hspace*{-1.0em}%
    \subfloat[][Q-Q plot of TKE quantile regression \\ predictions from LOO cross validation study]{{\includegraphics[width=0.56\linewidth]{figs/ml_fit/qq_tke_pin_1} }}%
    \caption[Q-Q LOO TKE pin 1 results.]{Pin 1 TKE quantile regression predictions from LOO cross validation study.}%
    \label{fig:tkepin1}%
\end{figure}

\begin{figure}[H]%
    \captionsetup[subfigure]{justification=centering}
    \centering
    \subfloat[][TKE quantile regression results. \\ CFD in dashed line.  Predicted values as solid. \\ Azimuthally integrated.]{{\includegraphics[width=0.53\linewidth]{figs/ml_fit/q_tke_regression_2} }}\hspace*{-1.0em}%
    \subfloat[][Q-Q plot of TKE quantile regression \\ predictions from LOO cross validation study]{{\includegraphics[width=0.56\linewidth]{figs/ml_fit/qq_tke_pin_2} }}%
    \caption[Q-Q LOO TKE pin 2 results.]{Pin 2 TKE quantile regression predictions from LOO cross validation study.}%
    \label{fig:tkepin2}%
\end{figure}

\begin{figure}[H]%
    \captionsetup[subfigure]{justification=centering}
    \centering
    \subfloat[][TKE quantile regression results. \\ CFD in dashed line.  Predicted values as solid. \\ Azimuthally integrated.]{{\includegraphics[width=0.53\linewidth]{figs/ml_fit/q_tke_regression_3} }}\hspace*{-1.0em}%
    \subfloat[][Q-Q plot of TKE quantile regression \\ predictions from LOO cross validation study]{{\includegraphics[width=0.56\linewidth]{figs/ml_fit/qq_tke_pin_3} }}%
    \caption[Q-Q LOO TKE pin 3 results.]{Pin 3 TKE quantile regression predictions from LOO cross validation study.}%
    \label{fig:tkepin3}%
\end{figure}

Shown in the axial plots in figures \ref{fig:tkepin1} to \ref{fig:tkepin3}, the TKE distribution is drastically influenced by spacer grids.  The maximum near-wall TKE sharply increases following a spacer grid followed by a decay towards a more orderly flow.  The location of minimum predicted near wall TKE also immediately follows the spacer grids.  In addition to increasing the net turbulent kinetic energy of the flow, mixing vanes also produce eddy regions of stagnant flow thus giving rise to regions of low near wall TKE.  The hi2lo model retains both of these properties of the flow field resolved by CFD.

Good overall performance of the TKE quantile regression models may be attributed to high pin-to-pin and span-to-span similarities of the surface TKE distributions. The observation of high span-to-span repeatability of the TKE distributions is consistent with those found in other hi2lo studies by Salko et. al \cite{salko17}.

Temperature residual quantile regression results are given in figures \ref{fig:temppin1} to \ref{fig:temppin3}.  Similar to the TKE conditional quantiles, the conditional temperature distribution exhibits sharp changes in behavior across the spacer grids. Unlike the TKE residual distribution the surface temperature distributions do not exhibit the same degree of similarity from span to span or from pin to pin.  The presence of discontinuities in the temperature distributions enforced the choice of the gradient boosted tree machine learning algorithm which is resilient to steep gradients in the response surface.

\begin{figure}[H]%
    \captionsetup[subfigure]{justification=centering}
    \centering
    \subfloat[][Temperature quantile regression results. \\ CFD in dashed line.  Predicted values as solid. \\ Azimuthally integrated.]{{\includegraphics[width=0.53\linewidth]{figs/ml_fit/q_twall_regression_1} }}\hspace*{-1.0em}%
    \subfloat[][Q-Q plot of Temperature quantile regression \\ predictions from LOO cross validation study]{{\includegraphics[width=0.56\linewidth]{figs/ml_fit/qq_twall_pin_1} }}%
    \caption[Q-Q LOO Temperature pin 1 results.]{Pin 1 Temperature quantile regression predictions from LOO cross validation study.}%
    \label{fig:temppin1}%
\end{figure}

\begin{figure}[H]%
    \captionsetup[subfigure]{justification=centering}
    \centering
    \subfloat[][Temperature quantile regression results. \\ CFD in dashed line.  Predicted values as solid. \\ Azimuthally integrated.]{{\includegraphics[width=0.53\linewidth]{figs/ml_fit/q_twall_regression_2} }}\hspace*{-1.0em}%
    \subfloat[][Q-Q plot of Temperature quantile regression \\ predictions from LOO cross validation study]{{\includegraphics[width=0.56\linewidth]{figs/ml_fit/qq_twall_pin_2} }}%
    \caption[Q-Q LOO Temperature pin 2 results.]{Pin 2 Temperature quantile regression predictions from LOO cross validation study.}%
    \label{fig:temppin2}%
\end{figure}

\begin{figure}[H]%
    \captionsetup[subfigure]{justification=centering}
    \centering
    \subfloat[][Temperature quantile regression results. \\ CFD in dashed line.  Predicted values as solid. \\ Azimuthally integrated.]{{\includegraphics[width=0.53\linewidth]{figs/ml_fit/q_twall_regression_3} }}\hspace*{-1.0em}%
    \subfloat[][Q-Q plot of Temperature quantile regression \\ predictions from LOO cross validation study]{{\includegraphics[width=0.56\linewidth]{figs/ml_fit/qq_twall_pin_3} }}%
    \caption[Q-Q LOO Temperature pin 3 results.]{Pin 3 Temperature quantile regression predictions from LOO cross validation study.}%
    \label{fig:temppin3}%
\end{figure}

Inspecting the axial quantile difference figures \ref{fig:temppin1}a to \ref{fig:temppin3}a, on average the extreme quantiles exhibit the largest axial RMS prediction errors.  The magnitude of the extreme quantile prediction errors may also gauged by inspecting the upper and lower regions in the Q-Q plots.

Estimates of the extreme quantiles from a sample population are naturally fraught with high variance as described by equation \ref{eq:theory_qdist_1}.  Recall that this fact was also experimentally demonstrated using a simple test quantile regression problem in section \ref{chap:GBRT}.  In both cases the distribution of the residuals between the gradient boosted quantile predictions and the empirical sample quantiles increased in variance when estimating the more extreme conditional quantiles.


\subsection{Kendall's $\tau$ Regression}

The rank correlation coefficient, Kendall's $\tau$ ($\rho_\tau$), is used to quantify the strength of correlation between the temperature and TKE on the rod surface in each CTF face.  A separate gradient boosted regression model was tasked with predicting this statistic as a function of local core conditions.   The growth rate of crud was shown to be sensitive to   $\rho_\tau$ in section \ref{sec:crud_copula_sensi}, figure \ref{fig:patchcrudfit80}.  It is therefore important to understand the error and uncertainty carried by the predicted $\hat \rho_\tau$ values in each CTF face.

A subset of the 5x5 assembly's Kendall's $\tau$ regression results are given in figure \ref{fig:ktauregression} and the complete 5x5 $\rho_\tau$ LOO cross validation results are given in figure \ref{fig:ktauregressionmontage}.  There is a marked change in behavior of the rank correlation coefficient as a function of axial position in the core from pin to pin.  The influence of Kendall's $\tau$ on the CTF face-integrated crud results was discussed in section \ref{sec:crud_copula_sensi}, and it was shown to be an important parameter to accurately predict via the machine learning model.  Pins with large relative errors for Kendall's $\tau$ are expected to produce anomalously poor crud predictions.

The worst performing pin with respect to $\hat \rho_\tau$ prediction was pin 8, as indicated in figure \ref{fig:ktauregressionmontage}.  Interestingly, this pin exhibited relatively good agreement between the predicted crud distribution and the expected CFD crud distribution as indicated in table \ref{tab:loo_crud_bmass} and figure \ref{fig:montageaxialbmasssm}.  This pin, was relatively cold in comparison to the others in the fuel bundle which resulted growing only $5.9$e-2 $[g]$ of crud in 300 days when the hottest rods grew $\approx 1.4$e0 $[g]$ in the same time.  In the case of pin 8, since the majority of the rod surface exists below the saturation point the crud result was not sensitive to the shape of the joint temperature and TKE distributions, and thus, even with poor $\rho_\tau$ predictions the axial and integrated crud results agree with the original CFD result.

\begin{figure}[H]%
    \centering
    \subfloat[Pin 1]{{\includegraphics[width=0.53\linewidth]{figs/ml_fit/ktau_regression_1} }}\hspace*{-1.0em}%
    \subfloat[Pin 2]{{\includegraphics[width=0.53\linewidth]{figs/ml_fit/ktau_regression_2} }}\hspace*{-1.0em}%
    \\
    \subfloat[Pin 3]{{\includegraphics[width=0.53\linewidth]{figs/ml_fit/ktau_regression_3} }}\hspace*{-1.0em}%
    \subfloat[Pin 4]{{\includegraphics[width=0.53\linewidth]{figs/ml_fit/ktau_regression_4} }}\hspace*{-1.0em}%
    \caption[Kendall's $\tau$ regression LOO results.]{Azimuthally integrated Kendall's $\tau$ regression results from LOO cross validation study.}%
    \label{fig:ktauregression}%
\end{figure}
To improve the performance of the Kendall's $\tau$ regressors, a larger training set could be generated in future work.  For this limited 25 pin data set, it is hypothesized that each pin has a substantially unique flow field when compared to the other 24 pins.  Expelling a pin from the training data set for cross validation causes the predictive performance of the model to suffer since the remaining pins in the training set do not provide the requisite information about the local core conditions vs. Kendall's $\tau$ relationship for the missing pin.

\subsection{Copula Classifier}

In addition to the rank correlation coefficient, Kendall's $\tau$, the copula family is also required to recover the copula density function on each CTF face.  To this end a gradient boosted classifier was trained on the available CFD data.  Copula information extracted from the raw CFD results is shown in figure \ref{fig:copula_predicted}.

Figure \ref{fig:confusionmatrixavg} summarizes the LOO cross validation results of the copula classifier as a confusion matrix.  The diagonal entries of the confusion matrix represent the correctly labeled copula predictions made by the reduced LOO trained classifier for each copula family average over the entire 5x5 assembly.  It is shown that on average the classifier predicts an incorrect result more often than not.

\begin{figure}[H]
    \centering
    \includegraphics[width=0.6\linewidth]{figs/confusion_matrix_avg}
    \caption[Copula classifier confusion matrix.]{Copula classifier confusion matrix.}
    \label{fig:confusionmatrixavg}
\end{figure}


It is clear that the copula classifier struggles to predict the correct copula class given the local TH conditions.  As previously indicated in figure \ref{fig:copula_predicted}, the behavior of the copula as a function of axial rod position is erratic and inconsistent from pin to pin.  This erratic behavior proved too difficult to capture provided the limited training data set.  It is not possible to conclude that the copula are well-described by the local thermal hydraulic conditions and axial position.  It remains as future work to investigate if including additional geometric pin and grid attributes could improve the classification results.  Additional software infrastructure would be required to both write geometric pin and grid features from the CTF code and to utilize these geometric features in the current model.

Future work could include performing a transformation of the input space so that the copula family labels are separable in the transformed space.  A potential candidate for building this transformation is the UMAP manifold learning algorithm \cite{UMAP18}.

Further improvements in prediction accuracy are possible by applying an ensemble
machine learning technique known as stacking.  Stacking combines the predictions of multiple classifiers using a meta-classifier.
Stacking increases model complexity since each classifier in the ensemble contains hyper-parameters which require tuning.
Since machine learning model tuning and performance is not a focus of this work, the application of this technique to improve copula classification results is left as future work.

Though improvements are possible, it should also be noted that section \ref{sec:crud_copula_sensi} and table \ref{tab:crud_totals_copula} show that the copula family does not substantially influence pin integrated crud results.  Due to this, gains in the copula classifier accuracy will not necessarily translate to a large improvement in crud prediction accuracy.

\section{Crud Results}

The presented case considered the 5x5 array operating at a single state with fixed power profile and flow conditions for 300 days.  The hi2lo model was marched forward in time using a resampling step size of 50 days.  A sample size of 400 was used to estimate the crud distribution in each CTF face.  The importance sampling distribution parameters were set to values given in table \ref{tab:hi2lo_params}.  The default remapping weights of $w_T=0.4, \ w_k=0.6$ were used in this case.

The comparisons presented are the result of the LOO cross validation study.
Ergo the hi2lo model was used in a properly predictive manner since distribution parameters had to be inferred from the machine learning model at local core conditions outside of the training set.

The error estimates provided by the present LOO cross validation study may be viewed as conservative.  The LOO strategy expunged an entire pin from an already limited training pool of only 25 pins . This is not representative a production-ready training data set.  A training set to be used in a production environment will have all possible pin geometries represented within it; that is all possible pin configurations within a bundle.  Not all combinations of inlet and power conditions will be simulated by CFD due to computational time limitations and so interpolation error may be expected even if a provided a geometrically rich training set.  

% If the flow fields are principally governed by geometric factors the 

% A question for future invesigation is decomposing the interpolation error into parts:  how much interpolation error is attributed to geometric factors vs thermal hydraulic ones.

%If the shape of the flow fields are primarily driven to first order by geometric factors, missing thermal hydraulic conditions is a less impactful a hole in the parameter space than a pin-geometry hole.

\subsection{Single Pin Comparisons}
\label{sec:single_pin_result}

A single pin was selected from the 25 pin array for detailed comparison of the CFD, CTF, and hi2lo models.  For this pin, axial crud distribution comparisons were made at 300 $[days]$ of simulation time are shown in figures \ref{fig:15axialbmass} and \ref{fig:15axialcmass}.  Axial crud distributions of all pins are provided in appendix B.  The CTF standalone case generally predicts a greater amount of crud at all axial positions.  Since the CTF model did not include any grid-enhanced heat transfer model it is to be expected that surface temperature downstream spacer grids would be over-predicted since the influence of the mixing vanes on the rod surface temperature distributions are partially neglected.  The Hi2lo model preserves the influence of the spacer grids on the crud distributions predicted by CFD computations.

\begin{figure}[H]
    \centering
    \includegraphics[width=0.7\linewidth]{figs/5x5/imp/1_5_axial_bmass}
    \caption{Pin 1 CTF vs CFD vs Hi2lo axial crud boron mass distribution at 300 days.}
    \label{fig:15axialbmass}
\end{figure}
\begin{figure}[H]
    \centering
    \includegraphics[width=0.7\linewidth]{figs/5x5/imp/1_5_axial_cmass}
    \caption{Pin 1 CTF vs CFD vs Hi2lo axial crud mass distribution at 300 days.}
    \label{fig:15axialcmass}
\end{figure}

The total crud mass and total boron hideout mass were computed at each resampling step and presented in figures \ref{fig:15pinbmasstime} and \ref{fig:15pincmasstime}.  The time evolution of the crud total mass for all pins is given in appendix B.  The hi2lo model under predicted the crud mass on pin 1 when compared to the CFD model.

\begin{figure}[H]
    \centering
    \includegraphics[width=0.7\linewidth]{figs/5x5/imp/1_5_pin_bmass_time}
    \caption{Pin 1 CTF vs CFD vs Hi2lo integrated crud boron mass distribution as a function of time.}
    \label{fig:15pinbmasstime}
\end{figure}
\begin{figure}[H]
    \centering
    \includegraphics[width=0.7\linewidth]{figs/5x5/imp/1_5_pin_cmass_time}
    \caption{Pin 1 CTF vs CFD vs Hi2lo integrated crud mass distribution as a function of time.}
    \label{fig:15pincmasstime}
\end{figure}

Figures \ref{fig:2d_hi2loimppinbmass} and \ref{fig:2d_hi2loimppincmass} show the hi2lo predicted crud surface distributions at 300 days.  The result of re-ordering samples onto each CTF face to preserve hot spot stationarity in time is visible.  The stripped patterns are non-physical and are an artifact of the remapping procedure.  Recall that the overarching goal is not to reproduce the detailed intra-CTF face spatial crud distributions rather the model specifically attempts to reproduce the correct average crud behavior on each CTF face, even in regions near spacer grids, and estimate the frequency of extreme crud events so to be relevant for CILC risk estimates.
Note that the crud surface field results are left in the area-normalized form with units of $[g/cm^2]$ which is the natural result from the 1D crud growth package.

\begin{figure}[H]
    \centering
    \includegraphics[width=0.7\linewidth]{figs/5x5/imp/tstep_5/pin_1/hi2lo_imp_pin_bmass}
    \caption{Pin 1 hi2lo 2D surface map of crud boron mass density.}
    \label{fig:2d_hi2loimppinbmass}
\end{figure}
\begin{figure}[H]
    \centering
    \includegraphics[width=0.7\linewidth]{figs/5x5/imp/tstep_5/pin_1/hi2lo_imp_pin_cmass}
    \caption{Pin 1 hi2lo 2D surface map of crud mass density.}
    \label{fig:2d_hi2loimppincmass}
\end{figure}

The average crud behavior as a function of axial position along the rod is given in figures \ref{fig:15axialbmass} and \ref{fig:15axialcmass}.  The axial crud root-mean-squared error is given in table \ref{tab:loo_crud_bmass} alongside other pins in the assembly.  Pin 1 exhibits good agreement between the hi2lo model's crud predictions and the CFD results for the axial crud distribution when compared to other pins in the assembly.  The rod integrated crud mass is also consistent between the two.

The crud density distributions predicted by the Hi2lo procedure are approximately consistent with the gold-standard CFD result as shown in figures \ref{fig:dist_hi2loimppinzbmass} and \ref{fig:dist_hi2loimppinzcthick}. Some difficulty in capturing the extreme quantiles of the crud distributions as a function of axial position along the pin is shown in the figures. 

The ability of the hi2lo model to accurately predict the fraction of the rod surface which experiences extreme crud thickness, a precursor quantity to CILC risk estimation, is hampered by limitations of the quantile regression and the relative sparsity of the available training data.  Recall that a given large-sample quantile follows a Gaussian distribution according to equation \ref{eq:theory_qdist_1}. By the propagation of uncertainty to upper tail integrals of the probability density detailed in equations \ref{eq:pr_thresh} and \ref{eq:pr_thresh_uncert}, estimates for extreme crud distribution quantiles (i.e. estimates of how much of the rod surface experiences crud with a thickness exceeding some critical CILC crud threshold) will have high variance.  Difficulty in predicting extreme quantiles by standard quantile regression reflects basic facts about the large sample limit of extreme quantiles.
Circumventing these difficulties is a non-trivial undertaking.  Without making assumptions for the functional form of the surface temperature distribution, thereby adopting a parametric model, it is difficult to estimate the likelihood of extreme crud events.

\begin{figure}[H]%
    \centering
    \subfloat[Hi2lo pin boron mass.]{{\includegraphics[width=0.53\linewidth]{figs/5x5/imp/tstep_5/pin_1/hi2lo_imp_pin_z_bmass} }}\hspace*{-1.0em}%
    \subfloat[CFD pin boron mass.]{{\includegraphics[width=0.53\linewidth]{figs/5x5/cfd/tstep_5/pin_1/CFD_pin_z_bmass} }}%
    \caption[Pin 1 crud boron mass density results at 300 days]{Pin 1 crud boron mass density results at 300 days.  Select crud quantiles are indicated via colored bands.  The agreement of the mean axial crud boron density distribution between the hi2lo vs CFD models is better than in the upper quantiles. }%
    \label{fig:dist_hi2loimppinzbmass}
\end{figure}


\begin{figure}[H]%
    \centering
    \subfloat[Hi2lo pin crud thickness.]{{\includegraphics[width=0.53\linewidth]{figs/5x5/imp/tstep_5/pin_1/hi2lo_imp_pin_z_cthick} }}\hspace*{-1.0em}%
    \subfloat[CFD pin crud thickness.]{{\includegraphics[width=0.53\linewidth]{figs/5x5/cfd/tstep_5/pin_1/CFD_pin_z_cthick} }}%
    \caption[Pin 1 crud thickness results at 300 days]{Pin 1 crud thickness results at 300 days.  Select crud quantiles are indicated via colored bands.  The maximum crud thickness predicted by the hi2lo model is approximately 70 microns at 300 days.  Likewise the maximum crud thickness predicted by coupled CFD/crud computations was approximately 72 microns.  Additionally note that the the mean crud thickness deviates from the median indicating asymmetry in the crud thickness density distribution.}%
    \label{fig:dist_hi2loimppinzcthick}
\end{figure}

% Redundant since we already have tempearture and TKE results
% \begin{figure}[H]
%    \centering
%    \includegraphics[width=0.7\linewidth]{figs/5x5/imp/tstep_5/pin_1/hi2lo_imp_pin_z_tke}
%    \caption{}
%    \label{fig:hi2loimppinztke}
% \end{figure}
% \begin{figure}[H]
%    \centering
%    \includegraphics[width=0.7\linewidth]{figs/5x5/imp/tstep_5/pin_1/hi2lo_imp_pin_z_twall}
%    \caption{}
%    \label{fig:hi2loimppinztwall}
% \end{figure}

\subsection{Multi Pin Comparisons}
\label{sec:multi_pin_result}

Results for each pin in the LOO cross validation study are presented here.  There was random variation in the prediction accuracy of the model across the 5x5 assembly with no apparent spatial bias in the model prediction errors towards the edge of the assembly, as one may expect.  This would indicate that some pins in the 5x5 assembly are, in a sense, more unique with respect to thermal hydraulic flow conditions than others.  Some pins, especially pin 9, show a small difference between the hi2lo model predictions and the gold-standard CFD result.  The thermal hydraulic conditions surrounding these high performing pins are well represented in the training set.

In table \ref{tab:loo_crud_bmass} and \ref{tab:loo_crud_cmass} rod integrated crud results for each pin are given at 300 days of simulation time.  The worst performing pin with respect to boron deposition prediction was pin 4 by relative percent difference between the hi2lo result and the CFD driven crud result.  The boosted regressor produced large Kendall's $\tau$ prediction errors for this pin as shown in table \ref{tab:loo_rms}.  Incorrect predictions made for Kendall's $\tau$ acted in concert with a net under-prediction of the temperature quantiles, as shown in figure \ref{fig:qqtwallmontagesm}, which gave rise to a significant ($\approx -48\%$) net under prediction of the total crud mass on pin 4.  This highlights the importance of correctly predicting the conditional quantiles of the temperature distribution on the rod surface as a function of local core conditions since crud growth is highly sensitive to the outer cladding temperature.


% Hi2lo Bmass results
\begin{table}[h]
    \begin{center}
    \caption[Hi2lo crud boron mass results]{Crud boron mass hi2lo LOO result summary at 300 days.}
    \begin{tabular}[h]{|c|c|c|c|c|c|}
        \hline
        Pin & CTF Bmass $[g]$ & CFD Bmass $[g]$ & Hi2lo Bmass $[g]$ & Hi2lo-CFD $[g]$ & Rel Diff \% \\
\hline
1  & 1.2940e-03 & 7.5489e-04 & 6.3163e-04 & -1.2326e-04 &  -16.3 \\
2  & 1.1458e-03 & 5.1953e-04 & 3.1487e-04 & -2.0466e-04 &  -39.4 \\
3  & 1.0265e-03 & 3.2678e-04 & 3.4192e-04 & 1.5140e-05 &  4.6 \\ 
4  & 1.0111e-03 & 5.6847e-04 & 2.9848e-04 & -2.6999e-04 &  $\bf{-47.5}$ \\
5  & 9.5319e-04 & 1.8974e-04 & 2.2723e-04 & 3.7490e-05 &  19.8 \\ 
6  & 4.0505e-04 & 9.1686e-05 & 1.0065e-04 & 8.9640e-06 &  9.8 \\ 
7  & 8.0907e-05 & 4.0210e-05 & 3.7515e-05 & -2.6950e-06 &  -6.7 \\
8  & 6.5705e-05 & 2.9861e-05 & 3.3475e-05 & 3.6140e-06 &  12.1 \\ 
9  & 6.7204e-05 & 3.3324e-05 & 3.3063e-05 & -2.6100e-07 &  -0.8 \\
10  &6.6850e-05 & 3.5892e-05 & 2.8171e-05 & -7.7210e-06 &  -21.5 \\
11  &8.7449e-05 & 4.0316e-05 & 3.7932e-05 & -2.3840e-06 &  -5.9 \\
12  &4.6693e-04 & 1.1722e-04 & 8.2669e-05 & -3.4551e-05 &  -29.5 \\
13  &1.0616e-03 & 2.2909e-04 & 2.6878e-04 & 3.9690e-05 &  17.3 \\ 
14  &1.0617e-03 & 2.7471e-04 & 3.9548e-04 & 1.2077e-04 &  44.0 \\ 
15  &1.0366e-03 & 4.5067e-04 & 3.3163e-04 & -1.1904e-04 &  -26.4 \\
16  &1.1594e-03 & 2.9707e-04 & 4.3385e-04 & 1.3678e-04 &  46.0 \\ 
17  &9.1090e-04 & 2.6739e-04 & 2.6569e-04 & -1.7000e-06 &  -0.6 \\
18  &7.1616e-04 & 2.3468e-04 & 2.0092e-04 & -3.3760e-05 &  -14.4 \\
19  &6.1329e-04 & 1.1289e-04 & 1.4341e-04 & 3.0520e-05 &  27.0 \\ 
20  &1.6370e-04 & 7.2277e-05 & 4.8991e-05 & -2.3286e-05 &  -32.2 \\
21  &1.2524e-04 & 4.5454e-05 & 4.4092e-05 & -1.3620e-06 &  -3.0 \\
22  &1.7007e-04 & 4.4576e-05 & 6.1729e-05 & 1.7153e-05 &  38.5 \\ 
23  &6.2594e-04 & 1.5092e-04 & 1.1521e-04 & -3.5710e-05 &  -23.7 \\
24  &7.1144e-04 & 1.8479e-04 & 2.1561e-04 & 3.0820e-05 &  16.7 \\ 
25  &4.1668e-04 & 1.3290e-04 & 8.6106e-05 & -4.6794e-05 &  -35.2 \\
\hline \hline
Totals & 1.5443e-02 & 5.2453e-03 & 4.7791e-03 & -4.6623e-04   & -8.88 \\
\hline
\end{tabular}
\label{tab:loo_crud_bmass}
\end{center}
\end{table}


% hi2lo cmass results
\begin{table}[h]
    \begin{center}
        \caption[Hi2lo crud mass results]{Crud mass hi2lo LOO result summary at 300 days.}
    \begin{tabular}[h]{|c|c|c|c|c|c|}
        \hline
        Pin & CTF Cmass $[g]$& CFD Cmass $[g]$& Hi2lo Cmass $[g]$& Hi2lo-CFD $[g]$ & Rel Diff \% \\
\hline
1	 & 2.4316e+00 & 1.4232e+00 & 1.1899e+00 & -2.3330e-01 & -16.4\\
2	 & 2.1537e+00 & 9.8041e-01 & 5.9479e-01 & -3.8562e-01 & -39.3\\
3	 & 1.9302e+00 & 6.1962e-01 & 6.4702e-01 & 2.7400e-02 &  4.4\\
4	 & 1.9040e+00 & 1.0737e+00 & 5.6557e-01 & -5.0813e-01 &  $\mathbf{-47.3}$\\
5	 & 1.7982e+00 & 3.6318e-01 & 4.3346e-01 & 7.0280e-02 &  19.4\\
6	 & 7.6893e-01 & 1.7713e-01 & 1.9431e-01 & 1.7180e-02 &  9.7\\
7	 & 1.5980e-01 & 7.9597e-02 & 7.4350e-02 & -5.2470e-03 & -6.6\\
8	 & 1.3032e-01 & 5.9122e-02 & 6.6394e-02 & 7.2720e-03 &  12.3\\
9	 & 1.3329e-01 & 6.6094e-02 & 6.5574e-02 & -5.2000e-04 & -0.8\\
10	 &1.3259e-01 & 7.1162e-02 & 5.5866e-02 & -1.5296e-02  & -21.5\\
11	 &1.7213e-01 & 7.9487e-02 & 7.5059e-02 & -4.4280e-03  & -5.6\\
12	 &8.8404e-01 & 2.2533e-01 & 1.5989e-01 & -6.5440e-02  & -29.0\\
13	 &2.0002e+00 & 4.3636e-01 & 5.1048e-01 & 7.4120e-02 & 17.0\\
14	 &1.9970e+00 & 5.2130e-01 & 7.4719e-01 & 2.2589e-01 & 43.3\\
15	 &1.9474e+00 & 8.4999e-01 & 6.2756e-01 & -2.2243e-01  & -26.2\\
16	 &2.1787e+00 & 5.6215e-01 & 8.1922e-01 & 2.5707e-01 & 45.7\\
17	 &1.7124e+00 & 5.0727e-01 & 5.0361e-01 & -3.6600e-03  & -0.7\\
18	 &1.3477e+00 & 4.4664e-01 & 3.8244e-01 & -6.4200e-02  & -14.4\\
19	 &1.1573e+00 & 2.1674e-01 & 2.7412e-01 & 5.7380e-02 & 26.5\\
20	 &3.1488e-01 & 1.4063e-01 & 9.6251e-02 & -4.4379e-02  & -31.6\\
21	 &2.4285e-01 & 8.9271e-02 & 8.6669e-02 & -2.6020e-03  & -2.9\\
22	 &3.2634e-01 & 8.7701e-02 & 1.2055e-01 & 3.2849e-02 & 37.5\\
23	 &1.1797e+00 & 2.8810e-01 & 2.2075e-01 & -6.7350e-02  & -23.4\\
24	 &1.3379e+00 & 3.5204e-01 & 4.0989e-01 & 5.7850e-02 & 16.4\\
25	 &7.8681e-01 & 2.5506e-01 & 1.6677e-01 & -8.8290e-02  & -34.6\\
\hline \hline
Totals	 & 2.9128e+01 & 9.9713e+00 & 9.0877e+00 & -8.8360e-01  & -8.9\\
\hline
\end{tabular}
\label{tab:loo_crud_cmass}
\end{center}
\end{table}


At each resample step the crud mass was summed over all pins in the assembly and presented in figure \ref{fig:asmcmasstime}.  The total assembly crud mass predicted by CFD and the hi2lo model are in in close agreement.  At 300 days simulation time the relative difference in the crud mass results between the hi2lo and the CFD model was -8.9\%, as shown in table \ref{tab:loo_crud_cmass}.  In general the expected error produced by the hi2lo model depends on the quantity and quality of the training data available.  A study of the relative crud error as a function of training data set size is left to future work as a study of this nature would require performing a significantly larger number of CFD computations than was performed here.

\begin{figure}[]
    \centering
    \includegraphics[width=0.7\linewidth]{figs/5x5/imp/asm_cmass_time}
    \caption{Assembly integrated CTF vs CFD vs Hi2lo crud mass as a function of time.}
    \label{fig:asmcmasstime}
\end{figure}

The RMS axial crud distribution errors are summarized in table \ref{tab:loo_rms}.  To facilitate examination of the hi2lo model predictions for geometric biases across the assembly top-down views of the RMS crud mass and boron mass error distributions were generated and presented in figure \ref{fig:l2boronasmerrorshmap} and \ref{fig:l2cmassasmerrorshmap} respectively.  The pins which reside on the edge of the assembly did not exhibit any increase in crud error on average.  The root mean squared error is given by: $RMS = \sqrt{\frac{1}{J}\sum^J_j(\mathrm{Hi2lo}_j - \mathrm{CFD}_j)^2}$ where $j$ is the CTF face index on a given pin.

% RMS error table
\begin{table}[h]
    \begin{center}
        \caption[Hi2lo crud RMS summary.]{Hi2lo vs CFD crud RMS summary.}
    \begin{tabular}[h]{|c|C|C|}
        \hline
        Pin  & Axial RMS Error Crud Mass $[g/cm^2]$ & Axial RMS Error Crud Boron Mass $[g/cm^2]$ \\
\hline \hline
1  & 6.2482e-04 &  3.3009e-07  \\
2  & 8.1392e-04 &  4.3183e-07  \\
3  & 2.0189e-04 &  1.0731e-07  \\
4  & $\mathbf{1.0737\mathrm{\bf e-}03}$ &  5.7008e-07  \\
5  & 2.2335e-04 &  1.1847e-07  \\
6  & 6.5623e-05 &  3.4608e-08  \\
7  & 2.1295e-05 &  1.0898e-08  \\
8  & 1.8572e-05 &  9.4019e-09  \\
9  & 1.0839e-05 &  5.4659e-09  \\
10 & 2.4348e-05 &  1.2298e-08  \\
11 & 2.9659e-05 &  1.5478e-08  \\
12 & 1.9073e-04 &  1.0111e-07  \\
13 & 3.9831e-04 &  2.1135e-07  \\
14 & 5.0327e-04 &  2.6950e-07  \\
15 & 5.5978e-04 &  2.9813e-07  \\
16 & 5.6026e-04 &  2.9830e-07  \\
17 & 1.2664e-04 &  6.6854e-08  \\
18 & 1.8205e-04 &  9.5976e-08  \\
19 & 1.2838e-04 &  6.8271e-08  \\
20 & 9.7915e-05 &  5.1674e-08  \\
21 & 1.7636e-05 &  9.2180e-09  \\
22 & 6.7042e-05 &  3.5199e-08  \\
23 & 1.7096e-04 &  9.0595e-08  \\
24 & 1.5885e-04 &  8.4133e-08  \\
25 & 1.9995e-04 &  1.0617e-07  \\
\hline
\end{tabular}
\label{tab:loo_rms}
\end{center}
\end{table}

\begin{figure}[H]
    \centering
    \includegraphics[width=0.7\linewidth]{figs/5x5/imp/l2_boron_asm_errors_hmap}
    \caption{5x5 average axial RMS crud boron mass error distribution.  Top down bundle view.}
    \label{fig:l2boronasmerrorshmap}
\end{figure}
\begin{figure}[H]
    \centering
    \includegraphics[width=0.7\linewidth]{figs/5x5/imp/l2_cmass_asm_errors_hmap}
    \caption{5x5 average axial RMS crud mass error distribution.  Top down bundle view.}
    \label{fig:l2cmassasmerrorshmap}
\end{figure}

Consistent under prediction of crud across the assembly was not observed. The top down assembly view of the crud prediction error distribution provided in figures \ref{fig:totbmassrelasmerrorshmap} and \ref{fig:totcmassrelasmerrorshmap} do not exhibit a tilt or other regular geometric pattern.  The hi2lo model over predicts the total amount of crud for some pins in the assembly, particularly pins 14 and 16 but strongly under predicts the total crud in pin 4.  This high variance in the prediction errors across the assembly can be partially attributed to small training sample size, however a more in depth cross validation should be conducted as part of a future study in which a larger CFD training pool is available.  

\begin{figure}[H]
    \centering
    \includegraphics[width=0.7\linewidth]{figs/5x5/imp/tot_bmass_rel_asm_errors_hmap}
    \caption{5x5 integrated crud boron mass relative error distribution. Top down bundle view.}
    \label{fig:totbmassrelasmerrorshmap}
\end{figure}
\begin{figure}[H]
    \centering
    \includegraphics[width=0.7\linewidth]{figs/5x5/imp/tot_boron_asm_errors_hmap}
    \caption{5x5 integrated crud boron mass absolute error distribution. Top down bundle view. }
    \label{fig:totboronasmerrorshmap}
\end{figure}
\begin{figure}[H]
    \centering
    \includegraphics[width=0.7\linewidth]{figs/5x5/imp/tot_cmass_asm_errors_hmap}
    \caption{5x5 integrated crud mass absolute error distribution. Top down bundle view.}
    \label{fig:totcmassasmerrorshmap}
\end{figure}
\begin{figure}[H]
    \centering
    \includegraphics[width=0.7\linewidth]{figs/5x5/imp/tot_cmass_rel_asm_errors_hmap}
    \caption{5x5 integrated crud mass relative error distribution. Top down bundle view.}
    \label{fig:totcmassrelasmerrorshmap}
\end{figure}

Correlations between crud prediction errors and errors committed by the quantile regressors were investigated in an attempt to establish performance metrics.  Understanding the correlation between the machine learning model prediction accuracy and the crud prediction errors is helpful if one wished to estimate the expect crud growth errors before employing the hi2lo model in a production setting.

Provided sensitivities of the crud results to the machine learning prediction errors one may estimate the expected accuracy of the crud predictions obtained via the hi2lo model by standard propagation of error procedure shown in equation \ref{eq:error_prop}.

\begin{equation} 
    E_{c_j} \approx \sqrt{ \sum_l \left( \frac{\partial E_{c,j}}{\partial E_{l,j}}\right)^2 E_{l,j}^2 }
    \label{eq:error_prop}
\end{equation}
Where $E_{c,j}$ is the $j^{th}$ pin crud mass error and $E_l$ is the error associated with the $l^{th}$ component of the ML model predictions.  $E_{l,j}$ can be computed at training time since this quantity does not depend on the crud growth rate.   Figure \ref{fig:asmerrorcorr} displays estimates of the partial derivatives, $\frac{\partial E_{c,j}}{\partial E_{l,j}}$.  This information is required in order for a user of the hi2lo model to detect problems with the trained quantile regression and copula models prior to employing the model to make crud predictions.

A Student-T test was conducted on the slope of each fitted linear trend lines. The results of the Student-T tests are shown in the upper-triangle of figure \ref{fig:asmerrorcorr}.  The null hypothesis was taken to be a slope of zero.  The standard deviation of the computed sensitivities is high when using a small sample size making it difficult to rigorously conclude that errors made by the machine learning models correspond to errors in the crud predictions.
% One can conclude that a  sample size greater than 25 pins should be used in future work to investigate the relative influence of machine learning errors on the predicted crud errors

\begin{figure}[H]
    \centering
    \includegraphics[width=0.95\linewidth]{figs/5x5/imp/asm_error_corr}
    \caption{Correlation of ML errors with crud prediction errors.}
    \label{fig:asmerrorcorr}
\end{figure}

Statistically significant trends were found between the RMS error committed by the TKE quantile regressors and the crud boron and mass distribution errors, as measured by root-mean-squared error.  This suggests there is a link between being able to accurately predict the conditional quantiles of the TKE distributions and obtaining accurate crud estimates.

A strong positive correlation was observed between the root-mean-squared crud errors and the total crud mass prediction errors.  This is a trivial result since one expects pins which exhibited large axial crud distribution RMS errors would also be likely to experience a large total integrated crud mass error unless, by happenstance, there was a cancellation of errors.


\section{Section Takeaways}


\begin{itemize}
	\item Pre-processing the data set requires generating paired CFD and subchannel results with congruent geometries and inlet boundary conditions.  The fine CFD data is first aggregated onto the subchannel grid.  Explanatory features are extracted from the available subchannel results and paired with statistical properties of the residual distribution of the CFD result about the subchannel result in each subchannel face. The paired explanatory feature set and distribution properties are written to an HDF5 file for use as a training data set.
    \item Crud predictions made by the hi2lo model were compared against CFD/crud coupled results and CTF/crud results.  The axial and integrated crud results produced by the hi2lo model compared favorably to the CFD results.  The impact of spacer grids on the crud distribution was captured by the hi2lo model.  Shown in figure \ref{fig:asmcmasstime}, the assembly integrated crud mass results for the 5x5 assembly differed from the gold-standard CFD assembly integrated results by -8.8360e-01 $[g]$ for a relative difference of -8.9\%.
    \item A leave-one-out cross validation strategy was utilized to estimate the predictive performance of the model.
    \item The prediction accuracy of the temperature and TKE quantile regression models was summarized through Q-Q plots for each pin in the LOO cross validation study.
    \item The prediction accuracy of the Kendall's $\tau$ regression model was assessed using the root-mean-square error for each pin in the LOO cross validation study.  \item Correlations between the errors committed by the machine learning models and the crud prediction errors were computed.  High uncertainty associated with these correlation measures did not permit a statistically significant link between poor Kendall's $\tau$ predictive performance and poor crud predictions to be drawn.
    \item The copula classifier performed poorly given the current set of considered explanatory variables and limited size of the training data set.  A Gaussian copula was assumed on each CTF face in place of the poorly predicted copula family from the classifier.  Recalling the results shown in section \ref{sec:crud_copula_sensi}, this is not expected to reduce crud prediction accuracy.
\end{itemize}


%-----------------------------------------------------------------------------%
\chapter{Conclusion}
\label{chap:conc}
%-----------------------------------------------------------------------------%

The proposed technique does not try to provide a detailed spatial distribution
of the temperature, TKE, and CRUD on the rods' surface, rather the
statistical approach seeks to provide a detailed frequency distribution
of these fields.  The end result is model which correctly captures hot and cold
spot TH conditions that give rise to the largest and smallest boron
precipitation concentrations without precisely knowing where on the rods'
surface gave rise to the sampled TH conditions.

A large body of future work will focus on time steeping and uncertainty propagation.
The changing TH conditions through a cycle must be accounted for in the simulation of CRUD build up over time.  Furthermore, the feedback between the CRUD layer and the rod surface temperature due to augmentation of the thermal resistance should be accounted for when stepping the simulation forward in time.
Uncertainties present in the gradient boosted models should be computed and
propagated into the CRUD results.   Uncertainties are expected to compound in time, thus there exists a strong incentive to minimize model induced uncertainties.

Additionally, an assessment of the regression kriging techniques should be made for this Hi2Low application.  Leveraging this technique for interpolating spatial fields is attractive because it naturally supplies uncertainty estimates for the value of the FOI at all interpolated locations.  However, the issue of sparsely available auxiliary variables must be addressed before RK becomes applicable to this work.

A key measure of success for this Hi2Low work with respect to  CRUD predictions
in the vicinity of spacer grids is the computation time needed to build the
training data sets upon which a regression model is developed.
It has yet to be proven that the proposed Copula and GBM based framework outperforms
either a table-lookup approach or a spatial interpolation approach in which
spatial CFD information is explicitly preserved in the Hi2Low model.  This assessment
of computation requirements is a key avenue for future work.

%-----------------------------------------------------------------------------%
% %-----------------------------------------------------------------------------%
\pagebreak
\section*{Project Schedule}

Legend:
\begin{itemize}
    \item (\checkmark)  Complete
    \item (\checkmark-)  Underway but incomplete
    \item ($\cdot$)  Not started
\end{itemize}


\begin{enumerate}
\item \textbf{(\checkmark-) Literature review.}
\item \textbf{(\checkmark) Develop CFD-CRUD tools for training data generation and extraction.}
          Platform used to generate datasets that are
          required to inform a scale-bridging model.
    \begin{enumerate}
        \item (\checkmark) Create CFD data extraction and post processing utilities.
        \item (\checkmark) Develop STAR-CCM+ plugin to extract cladding surface and volumetric TH quantities from a CFD computation.
        \item (\checkmark) Compute the required volume and surface integrals to
              distill finely resolved CFD datasets into subchannel-like results.
    \end{enumerate}
\item \textbf{(\checkmark-) Demonstrate differences between CFD and Subchannel predictions.}
    \begin{enumerate}
        \item (\checkmark-) Compute differences between volume/surface averaged CFD quantities and subchannel predictions.
        \item ($\cdot$) Identify deficiencies in subchannel predictions wrt. CRUD growth and CILC.
    \end{enumerate}
\item \textbf{(\checkmark-) Investigate CRUD model sensitives.}
    \begin{enumerate}
        \item (\checkmark) Identify CRUD (MAMBA) sensitivities to TH boundary conditions
        \item (\checkmark) Produce correlation coefficients and scatter plot depictions of the relationship(s) between input
              and output of the CRUD model.
          \item (\checkmark-) Identify TH conditions under which CFD scale CRUD predictions diverge from subchannel-CRUD results.
    \end{enumerate}
\item \textbf{(\checkmark-) Preliminary methods implementation and demonstration}
    \begin{enumerate}
        \item (\checkmark-) De-trend pointwise CFD datasets \& compute residual distributions.
        \begin{enumerate}
            \item (\checkmark) Moving averaged approach (assumes CTF and CFD will agree on the mean)
            \item (\checkmark-) CTF mean approach (requires CTF runs at identical CFD sample points)
        \end{enumerate}
        %
        \item (\checkmark) Create a tool to generate synthetic CFD-like training data sets to facilitate
            testing of the regression tools.
        %
        \item (\checkmark) Construct model of the spatially dependent co-variance between
              (residual) wall temperatures, surface shear stress, and
              boundary heat flux downstream of a grid span.
        \begin{enumerate}
            \item (\checkmark) Develop flexible dependence modeling package capable of capturing skewed covariance behavior (Copula).
            \item ($\cdot$) Develop Kriging model to capture spatial auto-correlation (may not be necessary?).
        \end{enumerate}
        %
        \item (\checkmark) Construct a regression model $G(\cdot)$ (Gradient boosted tree model [GBM]).
        \begin{enumerate}
            \item (\checkmark) Regress copula model parameters, $\bm{\hat\theta}$, on local-average CTF provided inputs.
            \item (\checkmark) Evaluate regression to recover joint TH distributions on any CDF patch:
                $G(CTF\ Patch\ Data) \rightarrow {P(T>t,TKE>tke,...|\bm{\hat\theta})}$
        \end{enumerate}
        %
        \item ($\cdot$) Uncertainty Propagation
        \begin{enumerate}
            \item ($\cdot$) Estimate uncertainty introduced by the GBRT/Copula model
            \item ($\cdot$) Propogate uncertainty to ensemble CRUD calculations
        \end{enumerate}
        %----------------- POST PROPOSAL ------------------%
        \item (\checkmark-) (\checkmark-) Multi-state point execution
        \begin{enumerate}
            \item (\checkmark-) Implement time stepping scheme
            \item ($\cdot$) Propagate uncertainty through time steps
        \end{enumerate}
        % -------------------------------------------------%
        \item ($\cdot$) Compute areas of the prediction surface with highest
            sensitivity to changes in input parameters \& develop capability to super sample these regions.
        %
        \item ($\cdot$) Demonstrate ability to perform dimensionality reduction of the input space via PCA.
        \item (\checkmark-) Perform CFD informed subchannel based CRUD prediction for a \emph{single pin}.
    \end{enumerate}
\item \textbf{(\checkmark-) Write proposal document}
\item \textbf{($\cdot$) Methods analysis and improvement}
    \begin{enumerate}
        \item ($\cdot$) Perform Large scale CFD runs to inform finalized scale-bridging model
        \item ($\cdot$) Tune regression algorithm's (GBM) hyperparameters
        \item ($\cdot$) Validate CFD informed subchannel CRUD results against plant data (IF AVAILABLE).
    \end{enumerate}
\item \textbf{($\cdot$) Final model construction and demonstration}
    \begin{enumerate}
        \item ($\cdot$) Demonstrate CFD informed subchannel model on a full-height, single assembly problem.
        \item ($\cdot$) Assess CILC/CIPS prediction improvements vs. base model (no CFD informed TH).
    \end{enumerate}
\item \textbf{($\cdot$) Write Dissertation}
\end{enumerate}

\begin{sidewaysfigure}
%---------------------------     YEAR ONE        ------------------------------%
\begin{ganttchart}[
        inline,
        x unit=1.5cm,
        y unit title =0.8cm,
        y unit chart=0.8cm,
        hgrid,
        vgrid,
        time slot format=isodate,
        compress calendar
    ]{2016-06-01}{2017-07-30}
    \gantttitlecalendar*[compress calendar, time slot format=isodate]{2016-06-01}{2017-07-30}{year, month=shortname} \\
    \gantttitlelist{1,...,14}{1} \\
%%%%%%% Phase 1
\ganttgroup{Proposal Era}{2016-06-01}{2017-06-30} \\  %elem0
\ganttbar{Lit. Review}{2016-06-01}{2016-11-01} \\  %elem1
\ganttbar{CFD Post Proc. Dev.}{2016-06-01}{2016-08-20} \\  %elem2
\ganttlinkedbar{CTF vs CFD}{2016-08-20}{2016-10-30} \ganttnewline  %elem3
\ganttbar{MAMBA Sensitivity}{2016-08-01}{2016-09-30} \ganttnewline  %elem4
\ganttbar{Copula Model Dev.}{2016-10-01}{2017-01-10}  \\  %elem5
\ganttbar{Synthetic Data Gen}{2016-12-01}{2017-01-30} \\  %elem6
\ganttlinkedbar{GBRT Model Dev.}{2016-12-01}{2017-03-30}  \\  %elem7
\ganttlinkedbar{Link Models w/ CRUD Sim.}{2017-03-01}{2017-06-30}  \\  %elem8
\ganttbar{Write Proposal}{2017-04-01}{2017-06-30}  %elem9
\ganttmilestone{Proposal}{2017-06-30} \ganttnewline  %elem10
%%%%%%%% Extra Task Linkages
\ganttlink[link mid=0.2]{elem2}{elem4}
\end{ganttchart}
%-----------------------------------------------------------------------------%
\end{sidewaysfigure}

\begin{sidewaysfigure}
%---------------------------     YEAR TWO       ------------------------------%
\begin{ganttchart}[
        inline,
        x unit=2.5cm,
        y unit title =0.8cm,
        y unit chart=0.8cm,
        hgrid,
        vgrid,
        time slot format=isodate,
        compress calendar
    ]{2017-06-01}{2018-01-30}
    \gantttitlecalendar*[compress calendar, time slot format=isodate]{2017-06-01}{2018-01-30}{year, month=shortname} \\
    \gantttitlelist{13,...,20}{1} \\
%%%%%%% Phase 2
\ganttbar{ }{2017-06-01}{2017-06-30}  %elem0
\ganttmilestone{Proposal}{2017-06-30} \ganttnewline  %elem1
\ganttgroup{Dissertation Era}{2017-07-1}{2018-01-01} \\  %elem2
\ganttbar{Time Stepping}{2017-07-01}{2017-08-01} \ganttnewline  %elem3
\ganttlinkedbar{Uncert. Prop}{2017-08-01}{2017-09-01} \ganttnewline  %elem4
\ganttbar{Methods Refinement}{2017-08-01}{2017-10-01} \ganttnewline  %elem5
\ganttbar{Final CFD Runs}{2017-10-01}{2017-11-01} \ganttnewline  %elem6
\ganttbar{Write Dissertation}{2017-10-01}{2017-12-15}  %elem7
\ganttmilestone{Defense}{2017-12-15} \ganttnewline  %elem8
\ganttlinkedbar{Final Revisions}{2018-01-01}{2018-01-30}  \ganttnewline  %elem9
\ganttmilestone{Project Conclusion}{2018-01-30} \ganttnewline  %elem10
%%%%%%%% Extra Task Linkages
\end{ganttchart}
%-----------------------------------------------------------------------------%
\end{sidewaysfigure}


\pagebreak


%-----------------------------------------------------------------------------%

%=================================APPX========================================%
%-----------------------------------------------------------------------------%
\chapter{Appendix A}
\label{chap:app_ml}


%! TEX root = ../dissertation_gurecky.tex

\section{5x5 Leave-One-Out Machine Learning Results}


Gradient boosted quantile regression model results are presented in figures \ref{fig:qtwallregressionmontagesm} and \ref{fig:qtkeregressionmontagesm}.  For each pin, the predictions are made for the left-out-pin and compared to the original CFD training data.  The gradient boosted results are shown as solid lines and the original CFD-CTF data is shown as broken lines. 

The presented quantile regression results are shown as a function of axial position along the rod for the residual surface temperature and TKE distributions, e.g $\hat q_{\tau}(z) = \mathbf b(z) + \varepsilon(z)$, where $\mathbf b(z) = \mu_{\mathrm{cfd}} - \mu_{\mathrm{ctf}}$.   The results were averaged over the 4 azimuthal CTF faces at each axial level in the CTF grid.   The root-mean-square (RMS) error of select quantiles prediction vs axial location are given in each figure.

Figures \ref{fig:qqtwallmontagesm} to \ref{fig:qqtkepinmontage} show quantile-quantile (Q-Q plots) for each pin in the 5x5 LOO results.  Each Q-Q plot summarizes the overall prediction quality afforded by the quantile regression averaged over the entire pin length.  At each CTF axial grid level, the Kolmogorov–Smirnov (KS) statistic was computed to quantify the goodness-of-fit of the quantile distribution reconstruction to the original, empirical CFD-CTF distribution. The average and maximum KS statistic encountered is recorded in each figure.

Additionally, the predicted rank correlation coefficient as a function of axial position made by the LOO-trained models are compared to the expected result in \ref{fig:ktauregressionmontage}.  The RMS error between the predicted $\hat \rho_\tau(z)$ and the CFD computed $\rho_\tau(z)$ is shown in each figure.

\newgeometry{left=1cm,right=1cm,top=1cm,bottom=1.5cm}
\begin{landscape}
\begin{figure}[H]
    \centering
    \includegraphics[width=0.96\linewidth]{figs/ml_fit/q_twall_regression_montage_sm}
    \caption{5x5 Axial surface temperature residual (CFD-CTF)  quantile predictions.}
    \label{fig:qtwallregressionmontagesm}
\end{figure}

\begin{figure}[H]
    \centering
    \includegraphics[width=0.96\linewidth]{figs/ml_fit/q_tke_regression_montage_sm}
    \caption{5x5 Axial TKE residual (CFD-CTF) quantile predictions.}
    \label{fig:qtkeregressionmontagesm}
\end{figure}

\begin{figure}[H]
    \centering
    \includegraphics[width=0.96\linewidth]{figs/ml_fit/qq_twall_montage_sm}
    \caption{5x5 surface temperature quantile predictions Q-Q goodness-of-fit summary.}
    \label{fig:qqtwallmontagesm}
\end{figure}

\begin{figure}[H]
    \centering
    \includegraphics[width=0.96\linewidth]{figs/ml_fit/qq_tke_pin_montage}
    \caption{5x5 TKE quantile predictions Q-Q goodness-of-fit summary.}
    \label{fig:qqtkepinmontage}
\end{figure}

\begin{figure}[H]
    \centering
    \includegraphics[width=0.96\linewidth]{figs/ml_fit/ktau_regression_montage}
    \caption{5x5 Kendall's $\tau$ vs axial position predictions.}
    \label{fig:ktauregressionmontage}
\end{figure}
\end{landscape}
\restoregeometry



\chapter{Appendix B}
\label{chap:app_b}
%! TEX root = ../dissertation_gurecky.tex

\section{5x5 Results}

All axial crud results for the 5x5 model are shown in figures \ref{fig:montageaxialbmasssm} and \ref{fig:montageaxialcmasssm}.  The axial distributions are shown at a simulated time of 300 days. Pin-integrate crud results plotted as a function of time are provided in figures \ref{fig:montagetimebmasssm} and \ref{fig:montagetimecmasssm}.

\begin{landscape}
\begin{figure}[H]
    \centering
    \includegraphics[width=.9\linewidth]{figs/5x5/imp/montage_axial_bmass_sm}
    \caption{5x5 axial crud boron mass results at 300 days.}
    \label{fig:montageaxialbmasssm}
\end{figure}
\begin{figure}[H]
    \centering
    \includegraphics[width=.9\linewidth]{figs/5x5/imp/montage_axial_cmass_sm}
    \caption{5x5 axial crud mass results at 300 days.}
    \label{fig:montageaxialcmasssm}
\end{figure}

\begin{figure}[H]
    \centering
    \includegraphics[width=0.9\linewidth]{figs/5x5/imp/montage_time_bmass_sm}
    \caption{5x5 rod integrated crud boron mass vs time.}
    \label{fig:montagetimebmasssm}
\end{figure}


\begin{figure}[H]
    \centering
    \includegraphics[width=0.9\linewidth]{figs/5x5/imp/montage_time_cmass_sm}
    \caption{5x5 rod integrated crud mass vs time.}
    \label{fig:montagetimecmasssm}
\end{figure}

\end{landscape}


\chapter{Appendix C}
\label{chap:app_c}
%! TEX root = ../dissertation_gurecky.tex

\section{Gradient Boosting Toolkit}
\index{Gradient Boosting!Software Implementation}

A gradient boosting library was developed in the python programming language to support the hi2lo work.  This package provides an easily extensible loss function class that a user can use to implement arbitrary loss functions in the gradient boosting framework.  As required by the hi2lo work, both quantile and least squares loss functions are included.  The package is applicable to both regression and classification problems.  CART tree construction controls are also provided allowing fine grained control over the weak learners.
The library interface was constructed to be similar to Scikit-learn's gradient boosting API so that the newly developed boosting algorithms can stand as drop in replacements for those available in Scikit-learn.

The gradient boosting package \emph{pCRTree} is available at \url{https://github.com/wgurecky/pCRTree.git}.

\section{Copula Toolkit}
\index{Copula!Software Implementation}

For copula simulation, the CDvine toolkit (GPLv3 licensed) is available for the R programming language. This packages does not implement all rotations of copula making it burdensome to handle negative dependence structures out-of-the-box.  Furthermore, the maximum likelihood fitting method included in CDVine does not allow the user to specify sample weights, a key feature for the CFD data under consideration since the CFD mesh cells vary in size.

To circumvent these deficiencies and potential license compatibility issues with VERA, a new copula toolkit was developed in python and is BSD3 licensed.
Careful attention was paid to develop a flexible abstract copula class which enables custom copula functions to be specified.  Importantly, all copula rotations are supported by default allowing one to model positive and negative dependence structures without duplication of code.
Canonical vine-copula construction and sampling algorithms are included in this package to handle the decomposition of arbitrary joint density functions of any dimension.
Copula parameters can be determined by a weighted maximum likelihood fit to empirically supplied data with included sample weights or by specifying a rank correlation coefficient in the case of Archimedean copula.  In the current hi2Low work, both capabilities are leveraged.

The \emph{StarVine} copula package and documentation is available at \url{https://github.com/wgurecky/StarVine.git}.

\section{Python Interfaces to Crud Codes}
\index{MAMBA!Software Implementation}

As part of this work, python interfaces were developed for both the legacy CASL crud tool known as MAMBA1D and the state-of-the art crud package, Mamba.  The python wrappers to these Fortran codes facilitate rapid prototyping of hi2lo procedures which provide boundary conditions to the crud codes.  Additionally, the high level interface simplifies the process of orchestrating large crud sensitivity studies.

The python wrappers are available in the Virtual Environment for Reactor Analysis (VERA) developed by CASL \url{https://www.casl.gov}.

\section{Hi2lo Code}
\index{Hi2lo!Software Implementation}

A package that leverages all the aforementioned tools to produce estimates of crud growth rates was developed.  This high level package is the primary user facing result of the current work.  It should be noted this package is heavily dependent on crud simulation, copula construction, and gradient boosting technologies.
This package orchestrates the construction and evaluation of gradient boosted regression trees which provide the copula and marginal distribution parameters as a function of local core conditions.
Currently, multi pin, multi state point simulation is implemented with future work focused on parallelization, training data acquisition, and improvements to the machine learning model implementation.

The hi2lo crud growth package and documentation is available at \url{https://github.com/wgurecky/crudBoost.git}.


\section{Synthetic Training Data Generation}
\label{chap:synth}
\index{Synthetic CFD Data!Software Implementation}

A toolkit to overlay custom noise atop a CTF solution was developed to provide a secondary source of training data sets aside from running a CFD code.  The synthetic data generation tool provides training data sets with lower computational cost than CFD calculations.  Some properties of a true CFD solution field are preserved by the tool, namely that the shape of the marginal and copula distributions change as a function of position and local thermal hydraulic conditions in the core.  The synthetic data is not to be viewed as complete substitute for CFD data since it lacks the ability to capture spatial auto-correlation in the predicted spatial fields that arise naturally from the governing PDEs.  Neighboring points on the rod surface do not exchange any TH information in this tool.  Despite the unphysical nature of the synthetic data, the tool provides a means to verify that known relationships between the explanatory variables and the copula parameters are recovered by the gradient boosted regression model.  This is possible because the user specifies these relationships up-front as inputs to the surface field sampling routines. \\

An excerpt of an input to generate a synthetic single pin data set is given below:
\tiny
\begin{lstlisting}[language=XML]
{
    "pinID": 1,
    "chanID": 1,
    "averageHeatFlux": 1.2e6,
    "spans": {
              "0.0": {"model": "lower", "samples": 1000},
              "2.01": {"model": "upper", "samples": 4000},
              "2.53": {"model": "upper", "samples": 4000},
              "2.98": {"model": "upper", "samples": 4000}
    },
    "upper": {
            "0.0": {"copula":  {"family": "gauss", "params": [-0.5], "rot": 0},
                "tke": {"type": "gauss", "params": [0.001, 0.02]},
                "temp": {"type": "beta", "params": [5.0, 2.7], "loc": -9.2, "scale": 12.0},
                "bhf": {"type": "gauss", "params": [0.001, 2.6e4]}
                },
            "0.3": {"copula":  {"family": "gauss", "params": [-0.6], "rot": 0},
                "tke": {"type": "gauss", "params": [0.01, 0.008]},
                "temp": {"type": "beta", "params": [5.0, 1.7], "loc": -7.0, "scale": 8.0},
                "bhf": {"type": "gauss", "params": [0.01, 1.1e4]}
                },
            "1.0": {"copula":  {"family": "frank", "params": [4.0], "rot": 1},
                "tke": {"type": "gauss", "params": [0.01, 0.005]},
                "temp": {"type": "beta", "params": [5.0, 1.5], "loc": -4.0, "scale": 5.0},
                "bhf": {"type": "gauss", "params": [0.01, 0.9e4]}
                }
            },
    "lower": {
            "0.0": {"copula":  {"family": "gauss", "params": [-0.6]},
                "tke": {"type": "gauss", "params": [0.001, 0.0001]},
                "temp": {"type": "beta", "params": ["5.0*(t)/600.0", 5.0], "loc": -2.0, "scale": 4.0},
                "bhf": {"type": "gauss", "params": [0.01, 1.0e3]}
                },
            "1.0": {"copula":  {"family": "gauss", "params": [-0.6]},
                "tke": {"type": "gauss", "params": [0.001, 0.0002]},
                "temp": {"type": "beta", "params": [5.0, 5.0], "loc": -2.0, "scale": 4.0},
                "bhf": {"type": "gauss", "params": [0.01, 1.0e3]}
                }
            }
}
\end{lstlisting}
\normalsize
The synthetic data generation tool is available for download at \url{https://github.com/wgurecky/ctfpurt.git}


\chapter{Appendix D}
\label{chap:app_d}
%! TEX root = ../dissertation_gurecky.tex
% ------------------------- TH definitions ----------------------- %
\section{Subcooled Boiling and DNB}

The relationship between the surface temperature of a internally heated object and the heat flux from the surface into the surrounding fluid is shown in figure \ref{fig:boiling_curve}.

\begin{figure}[H]
    \centering
    \includegraphics[width=0.7\linewidth]{../proposal/images/boiling_curve}
    \caption{Boiling curve.}
    \label{fig:boilingcurve}
\end{figure}


The curve can be approximated by equation \ref{eq:boil_h}.  Note that surface temperature $T_s$ is equivalent to the wall temperature $T_w$ in the equations which follow.
The critical heat flux (CHF) is the point at which film boiling begins to dominate and is accompanied by a precipitous drop in the heat transfer and a rise in the surface temperature.  This condition is known as departure from nucleate boiling (DNB) and must be avoided when operating a PWR.

\begin{equation}
q''(T_w) = 
\begin{cases}
      h(T_w-T_{\infty}), & \mbox{if } T_w < T_{sat} \\
      h(T_w-T_{\infty}) + q''_{nb} ,  & \mbox{if } T_{sat} \leq T_w < T_{CHF} 
\end{cases}
\label{eq:boil_h}
\end{equation}
Where $h$ is the single phase convective heat transfer coefficient which is in turn a function of the Nusselt number given in equations \ref{eq:htc} and \ref{eq:db}.  The contribution of nucleate boiling to the heat transfer can be approximated by the Rohsenow model given in \ref{eq:ros} \cite{rohsenow51}.

\begin{equation}
q''_{nb} = {{\mu }_{L}}{{h}_{fg}}{{\left[ \frac{g\left( {{\rho }_{L}}-{{\rho }_{v}} \right)}{\sigma } \right]}^{{}^{1}\!\!\diagup\!\!{}_{2}\;}}{{\left[ \frac{{{c}_{pL}}\left( {{T}_{w}}-{{T}_{sat}} \right)}{{{C}_{sf}}{{h}_{fg}}Pr_{L}^{n}} \right]}^{3}\;}
\label{eq:ros}
\end{equation}
Where $h_{fg}$ is the latent heat of vaporization, $\mu_L$ is the liquid viscosity, $\rho_v,\ \rho_L$ are the vapor and liquid phase densities, ${c}_{pL}$ is the specific heat of the liquid phase, and ${C}_{sf}$ is a tunable empirical constant.

\begin{equation}
h = \frac{k_l \mathrm{Nu}}{L} = \frac{q''}{T_w-T_{\infty}}
\label{eq:htc}
\end{equation}
Where $k_l$ is the thermal conductivity of the liquid, $L$ is the characteristic length scale, and $Nu$ is the Nusselt number.  For non-boiling flows over a flat vertical surface, the Nusselt number can be approximated by the
Dittus-Boelter equation:

\begin{equation}
\mathrm{Nu} = 0.023\, \mathrm{Re}^{4/5}\, \mathrm{Pr}^{n}
\label{eq:db}
\end{equation} 
Where $\mathrm{Re}$ is the Reynolds number and $\mathrm{Pr}$ is the Prandtl number.  $n$ is an empirically derived constant and is typically 0.4 for a heated flow.


%! TEX root = ../outline.tex

\bibliographystyle{unsrt}
\bibliography{sections/refs}



\end{document}
