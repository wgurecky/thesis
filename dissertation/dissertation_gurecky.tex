%\documentclass[12pt,a4paper]{report}
\documentclass[12pt]{report}
% \usepackage[a4paper, total={6.00in, 9in}]{geometry}
\usepackage{pdflscape}
\usepackage{setspace}
\usepackage[utf8]{inputenc}
\usepackage{amsmath}
\usepackage{amsfonts}
\usepackage{amssymb}
\usepackage{mathtools}
\usepackage{listings}
\usepackage{hyperref}
\usepackage{pgfgantt}
\usepackage{rotating}
\usepackage{bm}
\usepackage{bbm}
\usepackage{lipsum}
\usepackage{pifont}
\usepackage{float}
\usepackage{color}
\usepackage{graphicx}
\usepackage{subfig}
\usepackage[]{algorithm}
\usepackage{algorithmic}
\usepackage{array}
\newcolumntype{B}{>{\centering\arraybackslash}m{3cm}}
\newcolumntype{C}{>{\centering\arraybackslash}m{4cm}}
\newcolumntype{L}{>{\arraybackslash}m{8cm}}
\usepackage[version=3]{mhchem} % Package for chemical equation typesetting \ce{}
\usepackage{siunitx} % Provides the \SI{}{} and \si{}
\usepackage[normalem]{ulem}  % stike-through by \sout{}

% diss template
\usepackage{utdiss2}
\author{William Ladd Gurecky}  	% Required

\address{9905 Chukar Circle\\ Austin, Texas 78758}  % Required

\title{ A CFD-Informed Model for Improving Subchannel Resolution Crud Prediction}
% Required

%%%%%%%%%%%%%%%%%%%%%%%%%%%%%%%%%%%%%%%%%%%%%%%%%%%%%%%%%%%%%%%%%%%%%%
% NOTICE: The total number of supervisors and other members %%%%%%%%%%
%%%%%%%%%%%%%%% MUST be seven (7) or less! If you put in more, %%%%%%%
%%%%%%%%%%%%%%% they are put on the page after the Committee %%%%%%%%%
%%%%%%%%%%%%%%% Certification of Approved Version page. %%%%%%%%%%%%%%
%%%%%%%%%%%%%%%%%%%%%%%%%%%%%%%%%%%%%%%%%%%%%%%%%%%%%%%%%%%%%%%%%%%%%%

%%%%%%%%%%%%%%%%%%%%%%%%%%%%%%%%%%%%%%%%%%%%%%%%%%%%%%%%%%%%%%%%%%%%%%
%
% Enter names of the supervisor and co-supervisor(s), if any,
% of your dissertation committee. Put one name per line with
% the name in square brackets. The name on the last line, however,
% must be in curly braces.
%
% If you have only one supervisor, the entry below will read:
%
%	\supervisor
%		{Supervisor's Name}
%
% NOTE: Maximum three supervisors. Minimum one supervisor.
% NOTE: The Office of Graduate Studies will accept only two supervisors!
% 
%
\supervisor
{Derek Haas}

%%%%%%%%%%%%%%%%%%%%%%%%%%%%%%%%%%%%%%%%%%%%%%%%%%%%%%%%%%%%%%%%%%%%%%
%
% Enter names of the other (non-supervisor) members(s) of your
% dissertation committee. Put one name per line with the name
% in square brackets. The name on the last line, however, must
% be in curly braces.
%
% NOTE: Maximum six other members. Minimum zero other members.
% NOTE: The Office of Graduate Studies may restrict you to a total
%	of six committee members.
%
%
\committeemembers
[Ben Leibowicz]
[Sheldon Landsberger]
[Kevin Clarno]
{Stuart Slattery}

%%%%%%%%%%%%%%%%%%%%%%%%%%%%%%%%%%%%%%%%%%%%%%%%%%%%%%%%%%%%%%%%%%%%%%
\previousdegrees{}
% \previousdegrees{B.S., M.S.}
% The abbreviated form of your previous degree(s).
% E.g., \previousdegrees{B.S., MBA}.
%
% The default value is `B.S., M.S.'

%\graduationmonth{...}      
% Graduation month, either May, August, or December, in the form
% as `\graduationmonth{May}'. Do not abbreviate.
%
% The default value (either May, August, or December) is guessed
% according to the time of running LaTeX.

%\graduationyear{...}   
% Graduation year, in the form as `\graduationyear{2001}'.
% Use a 4 digit (not a 2 digit) number.
%
% The default value is guessed according to the time of 
% running LaTeX.

%\typist{...}       
% The name(s) of typist(s), put `the author' if you do it yourself.
% E.g., `\typist{Maryann Hersey and the author}'.
%
% The default value is `the author'.



% image paths
\graphicspath{{../proposal/}{./}{./figs/crud}}


%===========================CUSTOM CMDS=======================================%
\DeclareMathOperator*{\E}{\mathbb{E}}
\newcommand{\xmark}{\ding{55}}%

\author{William Ladd Gurecky}

\begin{document}
\copyrightpage          % Produces the copyright page.
\commcertpage           % Produces the Committee Certification
%   of Approved Version page (doctoral)
%   or Signature page (masters).
%		20 Mar 2002	cwm

\titlepage              % Produces the title page.
	
\renewcommand{\thepage}{\roman{page}}
%=================================TITLE=======================================%
%-----------------------------------------------------------------------------%
%! TEX root = ../dissertation_gurecky.tex

% Title.tex
% Author: William Gurecky
% Info:  Title page, table of contents
% Changlog:

%-----------------------------------------------------------------------------%
\begin{titlepage}
	\centering
	{\scshape\LARGE The University of Texas at Austin \par}
	\vspace{1cm}
	{\scshape\Large Nuclear \& Radiation Engineering \par}
	\vspace{1.5cm}
	
	        {\huge\bfseries Draft \par}
	{\large\bfseries A CFD-Informed Model for Improving Subchannel Resolution CRUD Predictions\par}

	\vspace{2cm}
	{\Large William L. Gurecky \par}
	\vfill

	\begin{flushright}
	Dissertation Committee \par
	\bigskip
	Dr.~Derek \textsc{Haas}, Supervisor \par
	Dr. Sheldon Landsberger \par
	Dr. Benjamin Leibowicz \par
	Dr. Kevin Clarno \par
	Dr. Stuart Slattery \par
	\end{flushright}
	\vfill
	{\large \today\par}
\end{titlepage}
%-----------------------------------------------------------------------------%
\pagebreak
\tableofcontents
\pagebreak

%-----------------------------------------------------------------------------%
\clearpage
\vspace*{\fill}
\thispagestyle{empty} % suppress showing of page number
\begin{quotation}
\em % optional -- to switch to emphasis (italics) mode
TODO: Dedication
\end{quotation}
\vspace*{\fill}

\pagebreak


%! TEX root = ../dissertation_gurecky.tex

\section*{Acknowledgments}
\addcontentsline{toc}{chapter}{Acknowledgments}

I would like to acknowledge Dr. Kevin Clarno, Dr. Stuart Slattery, Dr. Derek Haas, Dr. Sheldon Landsberger, and Dr. Ben Leibowicz for their technical guidance and general advice throughout my graduate studies.

Special thanks to Dr. Erich Schneider who fostered many a student's interest in nuclear engineering \& software development, including mine.

This research is supported by and performed in conjunction with the Consortium for Advanced Simulation of Light Water Reactors (http://www.casl.gov), an Energy Innovation Hub (http://www.energy.gov/hubs) for Modeling and Simulation of Nuclear Reactors under U.S. Department of Energy Contract No. DE-AC05-00OR22725.
\newpage


%! TEX root = ../dissertation_gurecky.tex

\utabstract
\index{Abstract}%
\indent
A physics-directed, statistically based,
surrogate model of the small scale flow features that impact Chalk River unidentified deposit (crud) growth is presented in this work. 
The objective of the surrogate is to provide additional details of the rod surface temperature, heat
flux, and near-wall turbulent kinetic energy fields which cannot be explicitly captured by a subchannel code. 

Operating as a mapping from the high fidelity computational fluid dynamics (CFD) data to the low fidelity subchannel grid (hi2lo), the model provides CFD-informed boundary conditions to the crud model executed on the subchannel pin surface mesh. The surface temperature, heat
flux, and turbulent kinetic energy, henceforth referred to as the fields of interest (FOI),
govern the growth rate of crud on the surface of the rod and the
precipitation of boron in the porous crud layer. Therefore the model predicts the behavior of the
FOIs as a function of position in the core and local thermal-hydraulic (TH) conditions.

The subchannel code produces an estimate for all crud-relevant TH quantities at a coarse spatial resolution everywhere in
the core and executes substantially faster than CFD.  In the hi2lo approach, the solution provided by the subchannel code is augmented by a predicted stochastic
component of the FOI informed by CFD results to provide a more detailed description of the target
FOIs than subchannel can provide alone.  To this end, a novel method based on the marriage of copula and
gradient boosting techniques is proposed. This methodology forgoes a spatial interpolation procedure
for a statistically driven approach, which predicts the fractional area of a rod’s surface in excess of some
critical temperature but not precisely where such maxima occur on the rod surface.
The resultant model retains the ability to account for the presence
of hot and cold spots on the rod surface induced by turbulent flow downstream of spacer grids when
producing crud estimates. Sklar’s theorem is leveraged to decompose multivariate probability densities
of the FOI into independent copula and marginal models. The free parameters within the copula model
are predicted using a combination of supervised regression and classification machine learning techniques
with training data sets supplied by a suite of precomputed CFD results spanning a typical pressurized water reactor TH
envelope.

Results show that compared to the subchannel standalone case, the hi2lo method more accurately preserves the influence of spacer grids on the crud growth rate.  Or more precisely, the hi2lo method recovers key statistical properties of the FOI which impact crud growth.  Compared to gold standard high fidelity CFD/crud coupled results in a single assembly test case, the hi2lo model produced a relative total crud mass difference of -8.9\% compared to the standalone subchannel relative crud mass difference of 192.1\%.


\pagebreak
\tableofcontents
\pagebreak

%-----------------------------------------------------------------------------%
% acryn.tex
% Author: William Gurecky
% Info:  Acronyms
% Changlog:

%-----------------------------------------------------------------------------%
\section*{Acronyms}
\begin{tabular}{l l}
CASL & Consortium for the Advanced Simulation of LWRs \\
CDF  & Cumulative Probability Density Function \\
CFD &  Computational Fluid Dynamics \\
CILC & CRUD Induced Local Corrosion \\
CIPS & CRUD Induced Power Shift \\
CRUD & Chalk River Unidentified Deposit \\
CTF &  Cobra-TF \\
FOI &  Field of Interest \\
GBRM & Gradient Boosted Regression Model \\
GBRT & Gradient Boosted Regression Tree \\
LANL & Los Alamos National Laboratory \\
LS  &  Least Squares \\
ML  &  Maximum Likelihood \\
MOM &  Method of Moments \\
ORNL & Oak Ridge National Laboratory \\
PC  &  Pair Copula \\
PDF  &  Probability Density Function \\
ROM &  Reduced Order Model \\
TH  &  Thermal Hydraulic \\
VERA & Virtual Environment for Reactor Applications \\
\end{tabular}

\section*{Symbols}
\begin{tabular}{l l}
$c(\cdot)$ & Copula density function \\
$C(\cdot)$ & Copula cumulative density function \\
$f(\cdot)$ & Marginal density function \\
$F(\cdot)$ & Marginal cumulative density function \\
$t$ & Time \\
$T$ & Temperature \\
$K$ & Turbulent Kinetic Energy \\
$\tau^{th}$ & The $^{th}$ Quantile \\
$q''$ & Boundary heat flux \\
$\rho_{\tau}$ & Kendall's tau \\
$\theta$ & Copula shape parameter \\
$\Theta_c$ & Archimedean Copula family \\
$\phi$ & Scalar flux \\
\end{tabular}

\pagebreak
%-----------------------------------------------------------------------------%


%=================================BODY========================================%
\renewcommand{\thepage}{\arabic{page}} % Arabic numerals for page counter
\setcounter{page}{1}
\onehalfspacing
% \doublespacing  % UT formatting spec
%-----------------------------------------------------------------------------%
\chapter{Introduction}
\label{chap:intro}


\section{Overview of Topics}

A literature review of Hi2Low modeling approaches to CFD-informed subchannel
problems is given in \autoref{chap:lit}.  Several advances in CRUD modeling are
also discused in \autoref{chap:lit} including a CASL developed high fidelity
CFD based CRUD tool currently under development.
Techniques for surrogate model construction are also reviewed.  Examples of
surrogate models developed using regression kriging are provided.  These
techniques are highly relevent to the current Hi2Low problem as these models
seek to describe the behavior of random fields given a sparse set of inputs.
Additionally a review
of the gradient boosted algorithm and applications to surrogate model construction
are provided.

In \autoref{chap:theory}, the overarching Hi2Low strategy is introduced.  A brief
intruduction to Sklar's theorem and gradient boosting is given with additional details
on these topics provided in \autoref{chap:app}.

Work that has been performed in support of this Hi2Low strategy is reviewed in
\autoref{chap:work}.  The development of a CFD data extraction tool with
additional post processing capabilities is provided.  This tool enables the
comparison of CFD result with subchannel results.  Comparisons between CFD and subchannel
subchannel results are presented to identify key flow regimes in which the
coarse subchannel based approach produces erroneous CRUD growth rates.
A demonstration of the copula based methodology to predict CRUD grow rates on a
single rod at a single TH state point is also given in
\autoref{chap:work}.

Finally, future work is discussed in chapter \autoref{chap:fw}.  This section focuses on
time dependent CRUD simulation strategies and uncertainty quantification and propogation.

\section{CASL Benefit and Novelty}

Accurately predicting CRUD induced power shift (CIPS) requires accurate CRUD
thickness and boron deposition estimates in all regions of the core.  Core wide
phenomena are out of scope for high fidelity CFD-Neutronic-CRUD coupled
simulations.  Instead, we rely on CASL's CTF-MAMBA1D coupling capability to
predict CIPS throughout a cycle.  CTF is not capable of simulating the detailed
flow patterns downstream the spacer grids.  Details in the flow field
downstream of spacer grids have important consequences for CRUD growth and
erosion, namely the local depression in surface temperatures and locally
increased shear stresses.  CRUD growth is extremely sensitive to the surface
temperature around the saturation point.  Therefore, growing crud at the bulk
average predicted TH conditions given by CTF in a coarse axial segment may not
yield the correct amount of CRUD growth in that region.  To overcome this issue
developing a surrogate model which is informed by CFD or experimental data that
captures small scale flow effects downsteam of spacer grids is a possible strategy.
Since experimental data and/or CFD data cannot be generated for every concivable
combination of thermal hydraulic and gemetric cases present in a reactor, a surrogate
serves to obtain .  The goal in the construction of a surrogate is to minimize the
number of CFD or experimental data sets required to drive down uncertainties in
the prediced quantities to an acceptable level.

To this end a novel method based on the marriage of copula and gradient boosting techniques
is proposed.  The spatial distributions are not resolved on each CTF patch, rather
multivariate probability density of the surface temperature, TKE, and bundary heat flux
are predicted by the model.
The resultant model produces CRUD estimates that account for the presence of
hot and cold spots on the rod surface induced by turbulent flow downstream of spacer grids.

Copula are used to resolve the dependence structure underpinning correlated distributions
of rod surface temperature and TKE.  In this work, Sklar's theorem is leveraged
to decompose multivariate probability densities into into independent copula
and marginal models. The free parameters within the copula model are predicted using a combination of
supervised regression and classification machine learning techniques. A
gradient boosted regression tree (GBRT) methodology is proposed to provide the
required relationships between the copula model parameters and the
explanatory variables.

\subsection{CASL Challenge Problems}

CASL selected several problems identified by industry partners as critical
inadequately understood engineering scale phenomena which would provide financial and
safety benefits to the nuclear power industry if resolved.  An list of the CASL
challenge problems is provided in [ref].  The problem of interest in this
work is the prediction of Chalk River Unidentified Deposit (CRUD) growth rates.
The growth of CRUD comes with neutronic and thermal hydraulic repercussions.  A
phenomena known as CRUD Induced Power Shift (CIPS) as been identified as a high
impact problem by the industry council.  Additionally, the presence of CRUD on the rod
surface has been shown to exacerbate local oxide penetration rates of some zirconium alloys.
This is known as CRUD induced local corrosion.
The proposed work aims to improve predictions of CRUD build up in the core to ultimately improve the
accuracy and provide uncertainty estimates for key physical indicators of the CIPS and CILC phenomena.

The Virtual Environment for Reactor Applications (VERA) is a key component of
CASL's technical portfolio.  This meta-package integrates a variety of physics
packages and multiphysics coupling options to form a robust reactor simulation
capability.  For multi-cycle depletion computations, VERA relies upon MPACT, a
2D-1D method of characteristics neutronics package, coupled with a subchannel
thermal hydraulics code COBRA-TF (CTF).  An integrated CRUD modeling capability
is provided by MAMBA to address the CRUD induced power shift challenge problem.

To reduce computation times, the subchannel TH code discretized the reactor
domain into large, centimeter scale finite volumes.  A second order upwinding
scheme is employed along with the SIMPLE solution algorithm to resolve pressure
velocity coupling.  As a consequences of this discretization scheme, sub-centimeter scale
thermal hydraulic effects of the spacer grids on CRUD are averaged over
large regions on the fuel rods' surfaces.  Though small scale phenomena are not
explicitly modeled, they are approximately accounted for in a variety of empirically derived
closure relations.  In effect, a single constant estimate for the mean thermal
hydraulic conditions are obtained in each finite volume.

Previous Hi2Low focused work in CASL focused on utilizing experimental or CFD
datasets to improve closure models in CTF.  These studies leverage a multitude of
dimensionality reduction and regression techniques
to fit a parametric model to the accepted gold-standard empirical data.  This approach
is adequate for correcting biases in the bulk-average behavior of the flow (due to
the previously neglected physics).  Examples of such Hi2Low models are given in
\autoref{chap:lit}.

The traditional approach must be slightly modified to accommodate the CILC and
CIPS challenge problems.  Here arises the need to retain not only the effect of
fine-scale physics on the bulk, but also to predict if certain
temperature or TKE thresholds are exceeded in a given subchannel
volume.  Furthermore, for a complete treatment of thermal hydraulic impacts on
CRUD growth, the scale-bridging model must describe the frequency
distribution of extreme TH events above a given threshold.

The application of copula and gradient boosted regression and classification
techniques to the CIPS and CILC Hi2Low challenge problems is a novel within the CASL community.
At the core of this work is a semi-parametric method to reconstruct multivariate probability density functions via the use of boosted quantile surface regressions to build marginal distributions coupled with a boosted copula model.  To the author's knowledge this technique is not currently present in the literature.
Of course, the currently proposed work bears some similarities to work by [ref][ref] discussed in section \autoref{chap:lit}.



%-----------------------------------------------------------------------------%
\chapter{Literature Review}
\label{chap:lit}
%! TEX root = ../dissertation_gurecky.tex

The problem of correcting thermal hydraulic predictions provided by a subchannel code using higher fidelity CFD results can be viewed from the perspective of statistical downscaling.  There are abundant examples of statistical downscaling techniques in the weather forecasting and geological interpolation and mapping literature.  One commonality across all studied procedures is the presence of a high and low fidelity data source and a goal to make credible predictions of the target field between known coarsely resolved sample locations.  The problem is one of data amalgamation, where the resultant downscaling model preserves some average aspects of the low fidelity model with the added benefits of uncertainty and spatial fidelity afforded by the finer scale data.

\section{Statistical Downscaling}

Statistical downscaling methods seek a statistical link between a low and high fidelity features. 
In particular, in the climate and weather data communities, it is common to perform local bias-correction on coarsely resolved weather models provided local weather station measurements [ref].    A common situation which lends itself to SD methods is to have disparate resolution sample data, one set typically provided by a coarsely resolved global climate model and a secondary set of finely resolved local rain and wind field measurements, and to have a wide array of auxiliary predictive features at ones disposal such as the terrain height.

Results from an SD model should be carefully interpreted since at fine scale resolutions point estimates for the fields represent a single realization of a random variable governed by a fitted underlaying distribution.  Depending on the models used to to capture statistical variation in the spatial and temporal trends it is sometimes necessary to draw many samples from the SD model to estimate the mean and higher moments.  These mean and variance estimates can then be compared against a historical validation data set.  In the case of a fully parametric SD model, the standard deviation and mean field estimate can analytically computed.

A particular class of SD methods known as bias-corrected spatial disaggregation (BCSD) focus on establishing local area estimates of higher order moments about the mean from a coarse model given previous historical local station data.  These models have been used to forecast the probability of extreme precipitation events in a given local region which is important for flood risk assessment studies [ref].

Several factors prohibit the application of BSCD techniques directly to the hi2lo problem at hand.  The majority of BSCD literature does not consider the simultaneous prediction of multiple correlated random fields.  Additionally, the BSCD models do not typically consider a large number of exogenous covariates in their construction - only the spatial and temporal coordinates are used which constrains the statistical downscaling maps which are produced to be nontransferable to other geographic locations.  Finally, resolving fine spatial detail of the temperature and TKE fields in a given CTF face isn't necessary for accurate crud prediction when using a single dimensional crud simulation code since no azimuthal or axial variation in these surface fields are utilized in the crud package.  Therefore, the problem of finding the fractional area of a CTF face which exists above a threshold is preferable to spatial disaggregation techniques in the current hi2lo crud application.


It is also possible to nest a high fidelity simulation within a coarse fidelity weather simulation. Boundary conditions and constraints are supplied by the coarse fidelity model to the nested regional high resolution weather model.  The practice of coupling regional weather models with coarse scale global models is sometimes referred to as dynamical downscaling, though, this weather modeling strategy can also be viewed as a particular tightly coupled multi-scale model.  There are examples of such coupled simulation in reactor physics (see literature on coupled coarse RELAP and CFD models.  CFD is used where the flow is 'complicated' but RELAP handles the primary loop piping and heat exchanger)  The construction of dynamical downscaling models are not the focus of the current hi2lo work and will not be discussed further.

\section{Kriging}

Besides the BCSD based tools, other bias-correction and statistical augmentation techniques are well suited when the errors can be assumed to be normally distributed about some mean prediction.  Simple kriging assumes that the mean of the underlaying field does not drift as a function location in the input space and therefore only requires the fitting of a co-variogram which describes the statistical spatial autocorrelation between the known data.  More advanced kriging strategies such as regression kriging (RK) forgo this simplification decomposing the interpolation problem into mean predicting and bias-correcting residual models.  In RK the spatial-autocorrelation in the residuals is modeled by a simple kriging model.  The mean response may be predicted by any regression strategy, with a common choice being an ordinary least squares model though works which investigate the use of random-forests or more advanced machine learning strategies in this role are pervasive [refs].

\section{Subchannel Hi2lo}

\begin{itemize}
    \item  M. Avramova.  CFD informed grid mixing coefficients in CTF.  This work leveraged CFD results to improve the momentum balance (grid mixing models) in CTF. \cite{avramova2007}
    
    \begin{itemize}
    	
    	\item The lateral momentum equations implemented in CTF are as follows:
    	\begin{align}
    	& \frac{\partial }{\partial t}(\alpha_l \rho_l \mathbf U_l)
    	+ \nabla \cdot (\alpha_l \rho_l \mathbf U_l \mathbf U_l) \nonumber \\
    	&= \alpha_l \rho_l \mathbf{g} - \alpha_l \nabla P + 
    	\nabla \cdot (\alpha_l \bm{\tau}_l) \nonumber \\
    	&+ M^L_l + M^d_l + M^T_l + M_l^{GDXF}
    	\end{align}
    	Where $l$ denotes the liquid phase and $\alpha$ is the volume fraction liquid.  The coefficients $M^L, M^d, M^T, M_l^{GDXF}$ account for droplet or bubble entrainment, phase interfacial drag, turbulent mixing and grid directed cross flow.  Avramova devised a method to use CFD computations to obtain an accurate prediction for $M_l^{GDXF}$ for a variety of grid designs.
    	
    	\item The grid directed cross flow source term is defined as:
    	\begin{equation}
    	M_l^{GDXF} = f^2_{sg}(z) u_l^2 \rho_l A_g S_g
    	\end{equation}
    	Where $w_l$ is the axial liquid velocity.
    	With
    	\begin{equation}
    	f_{sg}(z) = \frac{V^{CFD}_l(z-z_{in})}{U^{CFD}_{in}}
    	\end{equation}
    	
    	\item The effectiveness of the grid enhanced cross flow model was determined by comparing exit bulk temperature profiles across a variety of assembly designs against experimental and CFD results.
    	
    \end{itemize}
    
    \item  Blyth produced CFD informed grid enhanced heat transfer models for the advanced subchannel code, CTF.  This work presented strategies for processing CFD data for use in generating enhanced heat transfer maps and for computing the form loss coefficient across spacer grids.  Blyth's work served as a precursor to Salko's CFD informed method for developing HTC and TKE maps.
    
    
    \item S. Yao developed an empirical model of the heat transfer coefficient downstream of spacer grids. \cite{yao82}
    An empirical relationship between the Nusselt number ratio and the vane angle, $\phi$, blockage ratio $\epsilon$, dimensionless distance from the grid, $x/D$, and fraction of flow area impeded by the vanes, $A$, was developed by S. C. Yao et al. (1982) [ref].
\begin{equation}
\frac{Nu}{Nu_0}  = \left[ 1 + 5.55 \epsilon^2 e^{-0.13(x/D)}\right] + \left[ 1 + A^2\mathrm{tan}^2\phi e^{-0.034(x/D)} \right]
\end{equation}
Where the first term is accounts for the effect of grid flow restriction and the second term represents the contribution of vane induced swirl on the heat transfer.

\begin{figure}[H]
    \centering
    \includegraphics[width=0.6\linewidth]{../proposal/images/grid_nu_eff}
    \caption{S. Yao empirical Nusselt number ratio vs. distance from upstream spacer grid.}
    \label{fig:gridnueff}
\end{figure}

    \item  B. Salko et al. developed a CFD-Informed hi2lo spatial remapping procedure for CILC/CIPS screening. \cite{salko17}
    Similar to Yao approach.  Additionally developed a TKE remap.
    The multiplier maps capture the ratio of the CFD predicted HTC and TKE surface distributions to the same surface distributions on a bare rod with no spacer grids present.
    Convective HTC remap:
    \begin{equation}
        \mathbf m_h = \frac{(Nu)_{cfd}}{(Nu)_{0}} = \frac{h_{cfd} L_{cfd} k_{0} }{h_{0}k_{cfd} L_{0}}
    \end{equation}
    Where $Nu$ is the Nusselt number.  Assuming equal length scales and thermal conductivities:
    \begin{equation}
        \mathbf m_h = \frac{h_{cfd}}{h_{0}} = \frac{q_{cfd}(T-T_\infty)_{0}}{q_{0}(T-T_\infty)_{cfd}}
    \end{equation}
    It is important to note that a uniform heat flux is used in both the bare and full rod case so that $q_{cfd}/q_0 =1 $.
    Apply HTC remap:
    \begin{equation}
        \hat h_{l} = \mathbf m_h h_{ctf}
    \end{equation}
    $\hat h_l$ is the hi2lo remapped convective heat transfer coefficient.  In CTF the wall heat transfer is split between phases:
    \begin{equation}
        q'' = q''_{conv} + q''_{boil} = (\hat h_l)(T-T_{\infty}) + q''_{boil}(T)
    \end{equation}
    In order to compute augmented hi2lo surface temperatures
    several iterations are required to converge upon the correct surface temperature, $\hat T_s$ due to the surface boiling term.

    \begin{algorithm}[H]
        \captionsetup{labelfont={sc,bf}, labelsep=newline}
        \caption{Heat transfer coefficient map based hi2lo method for crud prediction (Salko. et. al.).}
    \setstretch{0.8}  % reduce spacing in algo sec
    \begin{algorithmic}[1]
    \STATE \textbf{Initialization} 
    \STATE Guess $T^{i=0}_s=T_0$.  Maximum number, $N$ iterations.

        \FOR {i in range($N$):}
           \STATE Evaluate effective multiphase CTF HTC: $h_{eff} = h_{{ctf}}(T^i_{s}, \hat h_l, q'')$ \;
           \STATE Compute new hi2lo surface temperatures: $T_{s} = \frac{q''}{h_{eff}} + T_\infty$ \;
           \STATE  Under relax  $T^{i+1}_{s} = \omega T_{s} + (1 - \omega) T^{i}_{s} ;\ \omega < 1.$ \;
           \STATE  \textbf{break if}:  $|T^{i+1}_s - T^i_s| < tol$ \;
        \ENDFOR 
    \STATE \textbf{return}: $\hat T_s = T^{i+1}_s$
    \end{algorithmic}
    \end{algorithm}
    Where $h_{ctf}(\cdot)$ is a callable CTF function that returns and effective multiphase HTC, $h_{eff}$.

    TKE remap:
    \begin{equation}
       \mathbf m_{k} = \frac{k_{cfd}}{k_{0}}
    \end{equation}
    Where $k_0$ is the TKE distribution for a bare rod without spacer grids.
    Apply TKE remap:
       \begin{equation}
       \hat k = \mathbf m_k k_{ctf}
       \end{equation}
     Grow crud using augmented temperature and tke surface fields and compute average crud results over the CTF face:
     \begin{equation}
     \mu_g = \frac{1}{A} \sum_i^N g(\hat T_{s_i}, \hat k_i, q''_i) a_i
     \end{equation}
    Where $A$ is the area of the CTF face and $a_i$ is the area of each cell face on the CRUD coupling mesh.

    \begin{itemize}
    \item \textbf{Claim:} The multiplier maps are insensitive flow rate.  An additive model is sensitive to the inlet flow conditions.

     The claim is not strictly true: The multiplier maps carry some dependence on the inlet flow conditions.  An increase in flow rate changes the shape and extent of the wake region downstream of spacer grids which impacts the rod surface temperature and TKE fields.

    An extension of the multiplier map hi2lo procedure could linearly interpolate between multiplier maps developed at high and low inlet flow rate conditions.
    \begin{align*}
        \mathbf m_k &= \alpha \mathbf m_k^h + (1 - \alpha) \mathbf m_k^l \\
                    &= \alpha \frac{k^h_{cfd}}{k^h_0} + (1 - \alpha) \frac{k^l_{cfd}}{k^l_0} \\
        \alpha & = \frac{\dot m_i - \dot m_i^l }{\dot m_i^h - \dot m_i^l}
    \end{align*}
    Where $\dot m_i$ is the inlet mass flow rate.  The superscript, $(\cdot)^l$, represents low flow conditions and $(\cdot)^h$ represent high flow conditions.


\begin{figure}[H]%
    \centering
    \subfloat[CTF/MAMBA crud predictions without hi2lo remapping on a quarter symmetric pin.]{{\includegraphics[width=0.45\linewidth]{../proposal/images/ctf_crud_orig} }}%
    \qquad
    \subfloat[CTF/MAMBA crud predictions using hi2lo remapping on a quarter symmetric pin.]{{\includegraphics[width=0.45\linewidth]{../proposal/images/ctf_crud_reconstructed} }}% 
    \caption[The impact of spatial HTC hi2lo remapping on CTF/MAMBA crud predictions.]{The impact of spatial HTC hi2lo remapping on CTF/MAMBA crud predictions.}%
    \label{fig:htc_remap_crud}%
\end{figure}


     Some simplifications are made in this mapping.  For a given assembly, the multiplier maps have been shown to have a high span to span repeatability.  Therefore, a representative map is derived from a single span in a fully developed flow field.  This representative map is then applied to all other spans in the model.

     \item The multiplier map may not be transferable to other assemblies in the core due to geometric effects including the orientation of neighboring assemblies and TH/neutronics feedbacks.  This represents a limitation to the spatial mapping procedure as unique maps must be generated for different assemblies in the core.
    \end{itemize}
    \item  T. Hengal. Regression Kriging application to soil composition prediction as function of elevation, distance to river, ect. \cite{Hengl07}
    \item  Boosted regression and classification applications. \cite{moisen2006}, \cite{friedman2002}
\end{itemize}




%-----------------------------------------------------------------------------%
\chapter{Theory}
\label{chap:theory}

\section{Hi2Low Approach}

In this approach the the deterministic and random components of the spatial field are modeled seperately.  The deterministic component is supplied by the coarse CTF solution and an model that describes small purtubations from the mean is constructed upon available CFD data sets.

The availability of the deterministic portion of the fields of interest via CTF is a boon the proposed Hi2Low methodology.  Typically, additional modeling descisions must be made to construct an estimator for the average behavior of the output fields.


%-----------------------------------------------------------------------------%
\chapter{Method Exploration Under a Synthetic Data Source}
\label{chap:work}
%! TEX root = ../dissertation_gurecky.tex

In this chapter marginal reconstruction from quantiles, copula fitting, Monte Carlo and importance sampling strategies are exercised with a synthetic data set.  Synthetic data offers advantages over CFD born data for the purposes of testing and evaluating the efficacy of the proposed models.  Since synthetic data conforms to a known form with specified distribution and bias parameters the fitting and sampling routines can be checked against the pre-specified expected result.
This chapter does not introduce machine learning components and does not explore forward model predictions.  See chapter \ref{sec:ml_cfd} for hi2lo model performance when used in a predictive capacity.

The availability of synthetic data alleviates the need to generate comparatively expensive CFD results to test the hi2lo strategy.  Some aspects of CFD fields are preserved in the synthetic ata, including expected biases between CFD and CTF results that arise due to discrepancies in wall heat transfer closure models, among other differences. Additionally turbulent dispersion of the temperature and surface shear distributions around spacer grids are emulated by the synthetic data model.  Accounting for spatial auto-correlation in the surface fields was not pursued.  Spatial autocorrelation present in the surface fields within a CTF face is not captured.  Consequently the synthetic data is not a direct replacement to CFD data but serves as data source for method interrogation and integration testing.

Semi-parametric copula and marginal distributions are fit to the synthetic data in each CTF face independently.  Samples are drawn from the fitted joint temperature, TKE and BHF density models on each patch using standard Monte Carlo methods or importance sampling.  The surface samples are provided to a crud simulation package as cladding-surface boundary conditions.  Additional required bulk coolant properties are supplied by CTF.
In this chapter some consequences of adopting a statistically based hi2lo map  on a multistate point, or time dependent crud simulation are discussed.

Finally, speedups afforded by importance sampling are presented.  The sampling distributions utilized in importance sampling are informed by the physics of crud growth.

\section{Generating Synthetic Data}

Synthetic data generation begins by first running standalone CTF on a single quarter symmetric pin and then augmenting the CTF result with tailored noise.  The augmented surface fields can be constructed by equation \ref{eq:synth_aug}.
\begin{align}
    \bm X &= \bm \mu_{ctf} + \bm b + \bm \varepsilon \nonumber \\
          &=
    \begin{pmatrix}
        T \\
        k \\
        q''
    \end{pmatrix}
    =
    \begin{pmatrix}
        \mu_{T} \\
        \mu_k \\
        \mu_{q''}
    \end{pmatrix}_{ctf}
    + \begin{pmatrix}
        b_{T} \\
        b_k \\
        b_{q''}
    \end{pmatrix}
    + \bm{\varepsilon} (\mathbf z; \bm \theta),
\label{eq:synth_aug}
\end{align}
Where $\bm \varepsilon(\mathbf z, \bm \theta)$ is a user controlled spatially dependent residual random vector with a mean of 0.  This residual is
shifted by a bias vector
$\mathbf b$, where $\mathbf z=\{z, \varphi\}$ is the axial and azimuthal location on the rod surface.
$\bm \theta$ represents user specified distribution parameters.

Equation \ref{eq:synth_aug} represents three continuous random surface fields.  In practice a large number of independent and identically distributed samples are drawn in each CTF face from the underlaying random field.  Individual surface samples can be specified by equation \ref{eq:synth_aug_discrete}.

\begin{equation}
    X_{ij} = \mu_{j,\ ctf} + b_j + \varepsilon_{ij};\ \   \varepsilon_{ij} \sim h_j(\bm \theta)
    \label{eq:synth_aug_discrete}
\end{equation}
Where the index $j$ represents the $j^{th}$ CTF face on the rod, and the index $i$ is the sample index within the $j^{th}$ CTF face.  The distribution parameters are constant over a given CTF face and give by $\bm \theta = \{\theta_c, \{\theta_x\}\}$ where $\theta_c$ is the copula parameter and  $\{\theta_x\}$ are marginal parameters, 


Shown in equation \ref{eq:synth_aug_face}, according to Sklar's theorem the surface residual temperature and TKE joint distribution may be decomposed into copula an marginal models on each CTF face:

\begin{equation}
    h_j = c_j(F_k(k), F_T(T); \theta_{c_j}) f_{T_i}(T; \theta_{T_j}) f_{k_i}(k; \theta_{k_j})
    \label{eq:synth_aug_face}
\end{equation}
Where the copula parameter $\theta_{c_j}$ and the marginal temperature and TKE distribution parameters $\theta_{T_j}$ and $\theta_{k_j}$ are set at runtime of the synthetic data generation tool.

To allow for a great deal of flexibility in the synthetic data the copula family, Kendall's $\tau$ rank correlation coefficient and marginal distribution parameters are specified as a function of axial location and local TH conditions supplied by CTF.  The copula's shape parameter, $\theta_c$ may be related to the rank correlation coefficient by equation \ref{eq:tauar} which is a one to one function for the Archimidean copula considered in this work.


\subsection{Single Pin Synthetic Data Set}

The original baseline CTF results are shown in figures \ref{fig:ctf_twall_orig} and \ref{fig:ctf_tke_orig}.  The CTF pin parameters are provided in table \ref{tab:pin_settings}.  The CTF result was produced from a quarter symmetric case, and therefore no azimuthal variation is observed.

\begin{table}[h]
    \begin{center}
        \caption{Single pin reference thermal hydraulic boundary conditions.}
        \begin{tabular}{|l|l|l|}
            \hline
            Setting & Value & Unit \\
            \hline
            Inlet Flow Rate & 0.3 & $[kg/s]$ \\
            Inlet Temperature & 565 & $[K]$ \\
            Pressure & 2250 & $[psia]$ \\
            Rod Outer Radius & 0.425 & $[cm]$ \\
            Pin Pitch & 1.26 & $[cm]$ \\
            Power Shape & constant & $[]$ \\
            Heat Flux & 85.86  & $[W/m^2]$ \\
            Rod Height & 3.6275 & $[m]$ \\
            Number of Grids & 3  & $[]$ \\
            Grid Locations & 2.0, 2.4, 2.8 & $[m]$ \\
            \hline
        \end{tabular}
    \label{tab:pin_settings}
    \end{center}
\end{table}

\begin{figure}[H]%
    \centering
    \subfloat[Axial CTF cladding surface temperature result.]{{\includegraphics[width=0.45\linewidth]{figs/synth/ctf_asm0_z_twall} }}%
    \qquad
    \subfloat[2D rod map of CTF result.]{{\includegraphics[width=0.45\linewidth]{figs/synth/ctf_pin_h5_temperature} }}%
    \caption[Single pin CTF baseline temperature result.  160\% nominal power conditions.]{Single pin CTF baseline temperature result.  160\% nominal power conditions.}%
    \label{fig:ctf_twall_orig}%
\end{figure}

\begin{figure}[H]%
    \centering
    \subfloat[Axial CTF cladding surface TKE result.]{{\includegraphics[width=0.45\linewidth]{figs/synth/ctf_asm0_z_tke} }}%
    \qquad
    \subfloat[2D rod map of CTF result.]{{\includegraphics[width=0.45\linewidth]{figs/synth/ctf_pin_h5_tke} }}%
    \caption[Single pin CTF baseline TKE result.  160\% nominal power conditions.]{Single pin CTF baseline TKE result.  160\% nominal power conditions.}%
    \label{fig:ctf_tke_orig}%
\end{figure}


The boundary heat flux was uniform at $85.86 [W/cm^2]$ which corresponds to approximately 160\% nominal PWR power conditions.

Next synthetic noise was generated using copula and marginal disrubution settings provided in table \ref{tab:synth_settings}. The complete synthetic data generation input deck for this case along with references to the code are provided in Appendix ().

\begin{table}[h]
    \begin{center}
        \caption{Per-span synthetic data generation settings.}
        \begin{tabular}{|l|l|l|l|l|}
            \hline
            \bf Span 1 & Node & Node location & Copula Settings  & Margin Settings \\
            \hline
            $N$: 8000  & 1  & 0.0 & $\Theta_c:$ Gaussian, $\theta:-0.6$ &  T:$\beta(5.0, 5.0)$ , k: $\mathcal{N}(0, 0.001)$ \\
                   & 2  & 2.0 & $\Theta_c:$ Gaussian, $\theta:-0.6$ &  T:$\beta(5.0, 5.0)$ , k: $\mathcal{N}(0, 0.001)$   \\
            \hline \hline
            \bf Span 2 & Node & Node location & Copula Settings  & Margin Settings \\
            \hline
             $N$: 8000 & 1  & 2.0 & $\Theta_c:$ Clayton-90, $\theta: 2.0$ &  T:$\beta(5.0, 2.7)$ , k: $\beta(1.75, 5.0)$ \\
            & 2  & 2.4 & $\Theta_c:$ Frank-90, $\theta: 8.0$ &  T:$\beta(5.0, 1.5)$ , k: $\beta(1.75, 5.0)$   \\
            \hline \hline
            \bf Span 3 & Node & Node location & Copula Settings  & Margin Settings \\
            \hline
             $N$: 8000 & 1  & 2.4 & $\Theta_c:$ Clayton-90, $\theta: 2.0$ &  T:$\beta(5.0, 2.7)$ , k: $\beta(1.75, 5.0)$ \\
            & 2  & 2.8 & $\Theta_c:$ Frank-90, $\theta: 8.0$ &  T:$\beta(5.0, 1.5)$ , k: $\beta(1.75, 5.0)$   \\
            \hline \hline
            \bf Span 4 & Node & Node location & Copula Settings  & Margin Settings \\
            \hline
            $N$: 8000 & 1  & 2.8 & $\Theta_c:$ Clayton-90, $\theta: 2.0$ &  T:$\beta(5.0, 2.7)$ , k: $\beta(1.75, 5.0)$ \\
            & 2  & 3.6 & $\Theta_c:$ Frank-90, $\theta: 8.0$ &  T:$\beta(5.0, 1.5)$ , k: $\beta(1.75, 5.0)$   \\
            \hline
        \end{tabular}
        \label{tab:synth_settings}
    \end{center}
\end{table}

Samples are drawn with probability proportional to the inverse distance to the nearest specified node.
Let the subscript $_u$ denote the location of the upstream span and $_d$ denote the downstream grid. $d_{u_j}$ and  $d_{d_j}$ denote the distance from the centroid of the CTF face to the nearest upsteam and downstream copula nodes respectively.

The mixture joint density model in any given CTF face can be specified by equation \ref{eq:dist_weighted_synth}.
\begin{equation}
    h_j = \left( \frac{d_{u_j}}{|d_{d} - d_{u}|} \right) h_u +
    \left( \frac{d_{d_j}}{|d_{d} - d_{u}|} \right) h_d
    \label{eq:dist_weighted_synth}
\end{equation}
For simplicity, two copula nodes were specified per span though more are possible for a finer grained control over the marginal and copula distributions.  The copula node properties are given in table \ref{tab:synth_settings}.  The copula nodes were located at the span extrema. This node specification pattern allows the synthetic data to mimic the expected sharp change in copula and marginal distributions when moving across spacer grids as seen in the raw CFD data presented in chapter () in figures () and ().

The copula models were sampled in each span, the original CTF result was augmented with the synthetically generated noise in accordance with equation \ref{eq:synth_aug_discrete}.

\begin{figure}[H]%
    \centering
    \subfloat[Spatial axial augmented CTF result.]{{\includegraphics[width=0.45\linewidth]{figs/synth/pinH5TempOut} }}%
    \qquad
    \subfloat[2D rod map of sythetically augmented CTF result.]{{\includegraphics[width=0.45\linewidth]{figs/synth/cfd_pin_temperature} }}%
    \caption[Augmented CFD result.]{Augmented CTF temperature result.}%
    \label{fig:ctf_twall_aug}%
\end{figure}

\begin{figure}[H]%
    \centering
    \subfloat[Spatial axial augmented CTF result.]{{\includegraphics[width=0.45\linewidth]{figs/synth/pinH5TkeOut} }}%
    \qquad
    \subfloat[2D rod map of sythetically augmented CTF result.]{{\includegraphics[width=0.45\linewidth]{figs/synth/cfd_pin_tke} }}%
    \caption[Augmented CFD TKE result.]{Augmented CTF TKE result.}%
    \label{fig:ctf_tke_aug}%
\end{figure}

The augmented surface temperature and turbulent kinetic energy fields shown if figures \ref{fig:ctf_twall_aug} and \ref{fig:ctf_tke_aug} can be compared against the original CTF results provided in figures \ref{fig:ctf_twall_orig} and \ref{fig:ctf_tke_orig} respectively.  No azimuthal variations are present in the augmented fields which would be present if a physics based model, such as CFD, were used.  Additionally, no spatial auto-correlation in the temperature and TKE cladding surface fields are included in the synthetic data.  Spatial autocorrelation could be captured with a kriging model in the future, however, the one dimensional nature of the crud simulation code used in this work dictates that the fine scale spatial detail in the surface fields are irrelevant when computing surface-integrated crud quantities.


\subsection{Single Pin Reconstruction}

The rod surface is subdivided into CTF faces before fitting and reconstructing the synthetic data.  The location and extent of the CTF faces on the rod surface can be determined from a CTF output file.  

In each face the empirical quantile distribution of temperature, turbulent kinetic energy, and boundary heat flux distributions are computed.  The number and spacing of quantiles used in the empirical quantile distribution is user specified.  Copula are fit to the synthetic data based on maximum likelihood and the Akaike information criterion (AIC) in each CTF face.  Maximum likelihood estimation is described in section \ref{sec:fitting_copula} and the AIC may be computed from equation \ref{eq:cop_aic}.

For the synthetic single pin data the hi2lo predicted fractional surface area above a saturation temperature threshold ($T_{sat}$) is shown in figure \ref{fig:frac_a}.

\begin{figure}[H]%
    \centering
    \subfloat[CTF predicted fractional area of each CTF face above the saturation point.]{{\includegraphics[width=0.45\linewidth]{figs/synth/hi2lo/ctf_pin_t_threshold} }}%
    \qquad
    \subfloat[Hi2lo predicted fractional area of each CTF face above the saturation point.]{{\includegraphics[width=0.45\linewidth]{figs/synth/hi2lo/hi2lo_pin_t_threshold} }}%
    \caption[]{Fraction area above the saturation point prediction comparison for the synthetic data set.}%
    \label{fig:frac_a}%
\end{figure}

Provided that crud growth exhibits a temperature thresholding behavior about the saturation point it is important to predict the fractional area of the rod surface which exists above this critical temperature. Figure \ref{fig:frac_a} shows a substantial difference in the fractional area predicted above the saturation point in each CTF face when utilizing the hi2lo model rather than the predictions generated from a CTF computation alone.  The more significant the thresholding behavior of crud growth, the more important it becomes to accurately compute areas of the rod surface in excess of the saturation point.

Figure \ref{fig:patchscatter} examines the surface frequency distributions of temperature, crud boron mass and TKE for the CTF patch denoted by the red box in figures \ref{fig:hi2lopincmass} and \ref{fig:hi2lopintke}.

\begin{figure}[H]
    \centering
    \includegraphics[width=0.99\linewidth]{figs/synth/patch_scatter_0_4_0_6}
    \caption[Single patch synthetic CFD data vs hi2lo sampled data.]{Single patch synthetic data vs hi2lo sampled data from patch centered on the rod at (3.14 $[rad]$, 2.85 $[m]$) at 300$[days]$.}
    \label{fig:patchscatter}
\end{figure}

After samples are independently drawn in each CTF face the temperature, TKE, and boundary heat flux samples were passed to a crud simulation packages as aboundary condtitions.  The crud simulation was stepped forward for 300 days with a resample step size of 50 days.  The resultant crud distribution is given in figure \ref{fig:hi2lopincmass}.

\begin{figure}[H]
    \centering
    \includegraphics[width=0.8\linewidth]{figs/synth/hi2lo/hi2lo_pin_cmass}
    \caption[Single pin crud mass distribution from synthetic TH data.]{Single pin crud mass distribution from synthetic TH data at 160\% nominal power conditions at 300 days simulation time.}
    \label{fig:hi2lopincmass}
\end{figure}
\begin{figure}[H]
    \centering
    \includegraphics[width=0.8\linewidth]{figs/synth/hi2lo/hi2lo_pin_t}
    \caption{Single pin temperature distribution from synthetic TH data at 160\% nominal power conditions at 300 days.}
    \label{fig:hi2lopint}
\end{figure}
\begin{figure}[H]
    \centering
    \includegraphics[width=0.8\linewidth]{figs/synth/hi2lo/hi2lo_pin_tke}
    \caption{Single pin surface TKE distribution from synthetic TH data at 160\% nominal power conditions at 300 days.}
    \label{fig:hi2lopintke}
\end{figure}

% \begin{figure}[H]%
%     \centering
%     \subfloat[Hi2lo patch cladding wall temperature spatial distrubtion]{{\includegraphics[width=0.45\linewidth]{figs/synth/hi2lo/hi2lo_patch_t} }}%
%     \qquad
%     \subfloat[Hi2lo patch TKE spatial distrubtion.]{{\includegraphics[width=0.45\linewidth]{figs/synth/hi2lo/hi2lo_patch_tke} }}%
%     \caption[]{Single hi2lo patch sample reordering demonstration.}%
%     \label{fig:reshuffle_a}%
% \end{figure}

\subsubsection{Crud Copula Parameter Sensitivity}
\label{sec:crud_copula_sensi}

Here the sensitivity of the crud result to the copula parameters is investigated.  Both the impact of the rank correlation coefficient, Kendall's $\tau$, and the Archimidean copula family are investigated.  The sensitivity results generated for the patch centered at $(\theta=3.14[rad], z=2.95[m])$ are shown in figure \ref{fig:patchcrudfit80}.  There is noise present in the crud predictions due to the Monte Carlo integration of equation \ref{eq:expected_crud} over the patch.  In this instance 2500 samples were used.  The crud is relatively insensitive to the choice of copula family, but the rank correlation coefficient is shown to have a significant influence on crud growth with an average boron deposition sensitivity of $\frac{\partial C_b}{\partial \rho_\tau} =$ -1.086e-7 $[g/cm^2/\tau]$ for this particular patch.  Accurately predicting Kendall's $\tau$ provided local core conditions is important.

\begin{figure}[H]
    \centering
    \includegraphics[width=0.7\linewidth]{figs/synth/patch_crud_fit_80}
    \caption{Single CTF face crud sensitivity to copula parameters.}
    \label{fig:patchcrudfit80}
\end{figure}

Next, two full single pin scenarios were considered. In the first scenario, shown if figure \ref{fig:crud_copula_fam_sensi}a, the best-fit copula on each patch as determined by the AIC metric is applied on each CTF face.  The second pin scenario enforces that a Gaussian copula model is used on every CTF face.  The crud results from these scenarios were then compared.  The data shows the choice of copula (between Gaussian, Frank, and Clayton) has a small overall impact on the total integrated rod boron mass.   The total integrated crud mass and crud boron mass for these scenarios at 300 days simulation time are given in table \ref{tab:crud_totals_copula}.

\begin{figure}[H]%
    \centering
    \subfloat[Best fit copula via AIC metric used in each CTF face.]{{\includegraphics[width=0.45\linewidth]{figs/synth/copula_compare/struct_pin_z_cmass_300_bestfit} }}%
    \qquad
    \subfloat[Gaussian copula used in each CTF face.]{{\includegraphics[width=0.45\linewidth]{figs/synth/copula_compare/struct_pin_z_cmass_300_gauss_only} }}%
    \caption[]{Influence of the choice parameters on the axial crud distribution.}%
    \label{fig:crud_copula_fam_sensi}%
\end{figure}


\begin{table}[h]
    \begin{center}
        \caption[Crud totals with different copula assumptions.]{Single pin crud totals at 300 days with different copula assumptions.}
        \begin{tabular}[h]{|l | l | l |}
            \hline
            Copula, $\Theta_c$ & Crud Boron Total: $C_B$ & Crud Mass Total: $C_m$ \\
            \hline  \hline
            Best Fit &  2.78953e-04 $[g]$ & 5.34015e-01 $[g]$ \\
            Gaussian &  2.78301e-04 $[g]$ & 5.32769e-01 $[g]$ \\
            \hline
            Rel Diff &  0.01724 $[\%]$ & 0.23396 $[\%]$ \\
            \hline
        \end{tabular}
        \label{tab:crud_totals_copula}
    \end{center}
\end{table}

The choice of the copula family, $\Theta_c$, has a negligible impact on the integrated crud results over a pin.  This result can reduce the complexity of the hi2lo model by removing the need to predict the correct copula family on each CTF patch in the core.  In section (), it is shown that CFD data exhibits a complex relationship between the best fitting copula family and the axial position along the rod.  This relationship is difficult to model using standard classification techniques, though further testing with a larger quantity of training data is warranted to ascertain if the copula family describing the dependence between the temperature and TKE fields on the rod surface can be accurately predicted given local core conditions.

\subsubsection{Crud Sample Size Study}

The number of samples, $N$, used to estimate the integral given in equation \ref{eq:expected_crud} is a parameter set at runtime of the hi2lo method.
As expected for a single rod the integrated crud variance is reduced by increasing the number of samples used per CTF face to estimate the integrated crud quantities of interest.   Section \ref{sec:Importance Sampling} demonstrates that improvements in sampling efficiency are possible by way of importance sampling.

Figure \ref{fig:cmprpintotalsviolinnsample} shows that increasing the number of samples used to in crud expected value Monte Carlo approximation yields an improvement in the variance of the predicted integrated crud results.  A single 300 day time step was conducted without re-sampling the underlaying density functions during this period.

\begin{figure}[H]
    \centering
    \includegraphics[width=0.7\linewidth]{figs/synth/nsample_study/cmpr_pin_totals_violin_nsample}
    \caption{Effect of sample size on the integrated crud results.}
    \label{fig:cmprpintotalsviolinnsample}
\end{figure}



\subsubsection{Importance Sampling}
\label{sec:Importance Sampling}

To obtain estimates for the efficiency gain offered by importance sampling to compute \ref{eq:expected_crud}, a singe patch was studied under synthetic TH data.

The choice of importance distribution is informed by the crud response.  To compute the integral \ref{eq:expected_crud} efficiently it is favorable to sample the TH distribution in regions which result in relatively large amounts of crud growth.  The response surface of the crud simulation code is presented in figures \ref{fig:crud_sensi1} to \ref{fig:crud_sensi3}.  Larger surface temperatures result in a higher crud growth rate.  Larger local TKE results in smaller crud growth rates due to the effects of erosion.  Additionally of note is the relatively small influence of the boundary heat flux.

\begin{figure}[H]%
    \centering
    \subfloat[Crud boron deposition sensitivity to temperature with TKE held fixed at $0.05 J/kg$]{{\includegraphics[width=0.45\linewidth]{../proposal/slides/seminar_slides/figs/dboron_dt_t} }}%
    \qquad
    \subfloat[Crud boron deposition sensitivity to TKE with temperature held fixed at $620 K$]{{\includegraphics[width=0.45\linewidth]{../proposal/slides/seminar_slides/figs/dboron_dt_tke} }}%
    \caption[]{Crud marginal response to varying temperature and TKE.}%
    \label{fig:crud_sensi1}%
\end{figure}

\begin{figure}[H]%
    \centering
    \subfloat[Crud boron deposition sensitivity with $q''=80 W/cm^2$.]{{\includegraphics[width=0.45\linewidth]{figs/crud/crud_t_tke_boron_response_80} }}%
    \qquad
    \subfloat[Crud boron deposition sensitivity $q''=120 W/cm^2$.]{{\includegraphics[width=0.45\linewidth]{figs/crud/crud_t_tke_boron_response_120} }}%
    \caption[]{Crud boron response to varying temperature and TKE.}%
    \label{fig:crud_sensi2}%
\end{figure}

\begin{figure}[H]%
    \centering
    \subfloat[Crud mass deposition sensitivity$q''=80 W/cm^2$.]{{\includegraphics[width=0.45\linewidth]{figs/crud/crud_t_tke_mass_response_bhf_80} }}%
    \qquad
    \subfloat[Crud mass deposition sensitivity $q''=120 W/cm^2$.]{{\includegraphics[width=0.45\linewidth]{figs/crud/crud_t_tke_mass_response_bhf_120} }}%
    \caption[]{Crud mass response to varying temperature and TKE.}%
    \label{fig:crud_sensi3}%
\end{figure}


The optimal importance distribution depends on the crud response surface and the TH probability density function.  Though an optimal importance distribution can be found [ref], the minimization problem is not solved in this work and is left as an avenue for future investigation.
Although the theoretically optimal importance distribution is forgone, a locally adaptive importance function was pursued in this work based on a distribution mixing approach.  With a known crud TH response surface existing in only $R^3$ it is feasible to design a near-optimal importance distribution by hand.

\begin{equation}
Q(\tilde{p}) = \sum_i^m a_{qi} Q_i(\tilde{p})
\end{equation}
Where $Q_i$ are the quantile functions as $a_{qi}$ are the weights.

Beta distributions with proscribed parameters are used in mixture with the original temperature and TKE density functions to produce a proposal density distribution for each patch.  The mixture weights can be adjusted.  This approach yields flexibility in the design of proposal density and by suitably tuning the parameters of the beta distributions.  Shown in figure \ref{fig:imp_sample2}, the sampling distribution can be skewed towards higher temperatures and lower TKE.

\begin{figure}[H]%
    \centering
    \subfloat[Temperature distributions.]{{\includegraphics[width=0.45\linewidth]{figs/imp_patch/temperature_importance_marginal_compare} }}%
    \qquad
    \subfloat[TKE distributions.]{{\includegraphics[width=0.45\linewidth]{figs/imp_patch/tke_importance_marginal_compare} }}%
    \caption[]{Proposal vs. original marginal distributions.}%
    \label{fig:imp_sample2}%
\end{figure}


\begin{figure}[H]%
    \centering
    \subfloat[Crud boron deposition rate.]{{\includegraphics[width=0.45\linewidth]{figs/imp_patch/bmass_sample_violin} }}%
    \qquad
    \subfloat[Crud mass deposition rate.]{{\includegraphics[width=0.45\linewidth]{figs/imp_patch/cmass_sample_violin} }}%
    \caption[]{Importance sampling trial results on a single CTF face.}%
    \label{fig:imp_sample1}%
\end{figure}

In figures \ref{fig:importancettkebmassscatter} the relative importance weight is denoted by the size of each point in the scatter plot.  Samples which have a small ratio $(h_i/\tilde h_i)$ appear as small points.  The sample weight is analogous to the rod surface area occupied by the sample.  In comparison, figure \ref{fig:originalttkebmassscatter} shows the same patch using a standard Monte Carlo sampling where each sample has the same weight.  The number of samples drawn in the upper tail of the temperature distribution is greater when importance sampling is applied, though these samples carry expectedly small sample weights.

\begin{figure}[H]
    \centering
    \includegraphics[width=0.99\linewidth]{figs/imp_patch/importance_t_tke_bmass_scatter}
    \caption[Importance sampled single patch crud result.]{Importance sampled single patch crud result.}
    \label{fig:importancettkebmassscatter}
\end{figure}

\begin{figure}[H]
    \centering
    \includegraphics[width=0.99\linewidth]{figs/imp_patch/original_t_tke_bmass_scatter}
    \caption[Standard Monte Carlo sampled patch crud result.]{Standard Monte Carlo sampled patch crud result.}
    \label{fig:originalttkebmassscatter}
\end{figure}

The importance sampling efficiency can be estimated by computing the variance ratio:  $\frac{\sigma^2_{MC}}{\sigma^2_{I}}$.  The variance of the patch-integrated crud result for the Monte Carlo and importance sampling schemes are provided figure \ref{fig:imp_sample1}.  The variance estimates were performed by running 1000 independent trials in which crud was grown on the patch for 300 days.  A total of 100 samples per patch per trial were used.  For the case studied, the application of importance sampling reduced the crud mass and boron mass sample variance by a factor of 2.02 [(4.979e-6)$^2$ / (3.503e-6)$^2$].  The mean crud predictions did not significantly deviate between two sampling schemes indicating that importance samples does not introduce any bias in the evaluation of the integral \ref{eq:expected_crud}.

The improvement in performance can be attributed to expending a larger proportion of the total available samples in the upper tail of the temperature distribution as this is a region which strongly contributes to crud growth.


\subsection{Single Pin with Time Stepping}

Stepping the crud simulation forward under the application of hi2lo supplied boundary conditions demands careful treatment of hot spot stationarity assumptions.  The time evolution of the crud simulation on the rod surface is strongly influenced by choices made both in the number of re-sampling steps taken as well as tunable constants which govern the sample remapping procedure and surface temperature mixing.

\subsubsection{Spatial Remapping with Time Stepping}

The influence of hotspot stationarity assumptions can be seen in figures \ref{fig:cmprpintotals0406} and \ref{fig:cmprpintotalsnoremap}.  When the surface temperature is allowed to randomly mix on each re-sampling event in each CTF face, the influence of the hot spots are smeared over the surface of the rod which leads to an overall under prediction in the total integrated crud mass.  Reordering the samples in each CTF face by their temperature improves the ability of the hi2lo model to preserve the impact of stationary hot and cold spots on the rod surface.  Good agreement with the original coupled CFD-Crud simulation data was achived by tuning the constants introduced in equation () to values of $\omega_T = 0.4$, and of $\omega_k = 0.6$.  This balanced seeks to preserve a heuristic thermal-hydraulic metric on the rod surface.  Since crud is sensitive to both the surface temperature and wall shear stress with a proxy of near wall turbulent kinetic energy.


\begin{figure}[H]
    \centering
    \includegraphics[width=0.7\linewidth]{figs/synth/cmpr_pin_totals_0_4_0_6}
    \caption[Total integrated crud born mass vs. time using approximately optimal remapping weights.]{Total integrated crud born mass vs. time using approximately optimal remapping weights ($\alpha_T=0.4, \alpha_{k}=0.6, \alpha_{q''}=0.0$).}
    \label{fig:cmprpintotals0406}
\end{figure}
\begin{figure}[H]
    \centering
    \includegraphics[width=0.7\linewidth]{figs/synth/cmpr_pin_totals_no_remap}
    \caption{Total integrated crud born mass vs. time without remapping samples.}
    \label{fig:cmprpintotalsnoremap}
\end{figure}

A spatial representation of the samples pre and post-remapping are shown in figure \ref{fig:remmap_comp}.  A visual representation of the remapping strategy presented is presented in figure \ref{fig:samplemapping}.

\begin{figure}[H]%
    \centering
    \subfloat[Remapped surface samples.]{{\includegraphics[width=0.45\linewidth]{figs/synth/patch_fields_0_4_0_6} }}%
    \qquad
    \subfloat[Non-remapped surface samples]{{\includegraphics[width=0.45\linewidth]{figs/synth/patch_fields_no_remap} }}%
    \caption[Re-mapped and non remaped temperature and TKE surface samples]{Re-mapped (a) and non remaped (b) temperature and TKE surface samples for a single CTF face.}%
    \label{fig:remmap_comp}%
\end{figure}

\subsection{Re-Sample Frequency}
\label{sec:resample_freq_study}

The frequency at which the distribution functions are sampled from on each CTF face influences the variance in the predicted integrated crud results.  To investigate this behavior, a parameter sweep was conducted in which the same pin was.  50 independent trials were conducted for each step size shown if table ().  Shown in figure \ref{fig:cmprpintotalsviolin} a smaller re-sampling steps size, $\Delta t_s$, results in a reduction in the variance of the rod integrated crud estimates at 300 days of simulation time.  It is also important to note that the variance of the rod integrated crud results increases as a function of time.

\begin{figure}[H]
    \centering
    \includegraphics[width=0.7\linewidth]{figs/synth/tstep_study/cmpr_pin_totals_violin}
    \caption{Influence of the re-sample frequency on the predicted integrated crud variance.}
    \label{fig:cmprpintotalsviolin}
\end{figure}

Performing a larger number of re-sampling events per VERA state results in reduced variance at little additional computational effort.
This is in part due to the minimal computational requirements of sampling the joint temperature and TKE disstribution on each patch.  Drawing samples from a bivariate copula density model is straight forward, as indicated in equation (), and can be done in parallel since each patch is treated as an independent sampling zone in this hi2lo approach.  The crud computation, by comparison, is more expensive.  Increasing the re-sampling frequency does not increase the total number of samples used per pin per time step, rather, this process only increases the nubmer of (re-sample) steps per VERA statepoint.  The reduction in variance stems from an improved sample density throughout time of the underlying random field.  In time, the underlaying random field is fixed throughout a VERA statepoint.  Repeatedly drawing samples from this field at small time steps rather than sampling the random field only once at the beginning of the VERA statepoint vastly increases the number of samples used to perform the time integration of the crud result on each CTF face.


\subsection{Section Takeaways}
\begin{itemize}
        \item Increasing the number of samples per patch decreases variance in total crud and total precipitated boron estimates.
        \item Increasing the number of re-sampling steps per VERA state point reduces variance in the final integrated crud results.
        \item After drawing samples from the predicted TH distribution a reordering of the samples on the rod surface is necessary to preserve hot spot stationarity.  This is required for an optimal time marching sampling strategy.
        \item Importance sampling was shown to reduce the variance in the integrated crud results.
        \item A synthetic data generation tool allows absolute control over the properties of joint distribution of TH boundary conditions which are fed into the crud simulation code.  Since the synthetic data has known properties, this data serves as an important data source for benchmarking and validation operations.
\end{itemize}

%! TEX root = ../outline.tex

\section{Single Pin Comparisons}

\begin{itemize}
    \item (\checkmark-) Show Kendall's tau vs Axial postion for a single pin.  The rank correlation correlation coefficient shows
    the influence of spacer grids on the joint distribution of temperature and turbulent kinetic energy.
    \item ($\cdot$) Show predicted Copula as a function of axial position.  Is there a clear trend here (i.e. is a guassian copula
        a good fit far away from spacer grids and the clayton copula a better fit near spacer grids?  (TODO: create this plot functionality - capability to compute best-fit copula already exists)
    \item (\checkmark-) Show axial crud comparisons for a single Hi2Lo, CFD and CTF pin.
    \item (\checkmark-) Show integrated crud comparisons for a single Hi2Lo, CFD and CTF pin.
\end{itemize}

\section{Multi Pin Comparisons}

\begin{itemize}
    \item (\checkmark-) Exercise hi2lo method with 5x5 CFD data set.
    \item (\checkmark-) Show pin-by-pin statistics to show if a geometric bias is present.  Are we always over predicting crud
    in pins near the edge of the assembly?
    \item (\checkmark-) Show axial CFD vs CTF vs Hi2lo results for each pin.
    \item (\checkmark-) Compare total integrated assembly crud mass and boron bass for CFD vs CTF vs Hi2lo.
\end{itemize}


%-----------------------------------------------------------------------------%
\chapter{Model Performance Under a CFD Data Source}
\label{chap:fw}
%! TEX root = ../dissertation_gurecky.tex
\label{sec:ml_cfd}

For deployment as an in-line statistically based downscaling tool which sits between a subchannel code and a crud simulation code in a core simulator such as VERA the model is required to perform the hi2lo mapping for all pins in the core at any operating condition.  In other words, the model must be evaluable at any local core conditions typical of an operating PWR.  Since the training data set cannot contain all possible pin geometries, loading configurations and operating conditions due to computational expense, the model must make a prediction for the copula and marginal distribution parameters between known states. 

One might envision a table-lookup approach where high fidelity CFD flow field maps are pre-computed and stored for a wide array of flow and power conditions.  A nearest neighbor interpolation scheme could then be applied to extract the best-matching CFD map provided some local core state.  This however, is not tractable since the number of CFD computations to build the data base would be prohibitively large.  

% It is appealing to transform the problem of hi2lo construction from performing spatial flow map prediction into one of summary statistics prediction where much less information is required adequately fill out the operating envelope.  

In this chapter the hi2lo methodology introduced in this work is exercised against a small CFD data set derived from a 5x5 fuel assembly operating at realistic PWR conditions.  A leave-one-out cross validation strategy is used to asses the predictive performance of the model.


\section{CFD Data Source}

For the generation of high fidelity CFD data sets the Westinghouse 5x5 test stand shown in figures \ref{fig:5x5topdown} and \ref{fig:5x5side} was used to prepare the CAD geometry.  The CFD mesh consisted of approximately 25 million cells and 1e5 surface elements per pin.  A matching CTF input deck for the 5x5 assembly was also constructed with 100 axial zones.  The CTF and CFD codes were then executed for a variety of flow conditions and power levels.

\begin{figure}[H]
    \centering
    \includegraphics[width=0.5\linewidth]{figs/5x5/5x5_top_down}
    \caption[Top down view of 5x5 pin Westinghouse facility.]{Top down view of 5x5 pin Westinghouse facility.  Assembly dimensions and pin powers redacted.}
    \label{fig:5x5topdown}
\end{figure}

\begin{figure}[H]
    \centering
    \includegraphics[width=0.5\linewidth]{figs/5x5/5x5_side}
    \caption[Side view of 5x5 pin Westinghouse facility.]{Side view of 5x5 pin Westinghouse facility.  Pin dimensions redacted.}
    \label{fig:5x5side}
\end{figure}


\subsection{Preprocessing}
\label{sec:preprocessing}

Pre-processing requires paired CFD and CTF results for a given pin generated with consistent boundary conditions between the codes.
First each of the CFD $\{T,k,q''\}$ fields are aggregated onto the CTF face centers.  In the aggregation procedure spatial information is discarded within each CTF patch.  The aggregated CFD data are associated with a unique CTF face and corresponding local core conditions which comprise the predictive variable set given in table \ref{tab:features}.  The aggregated CFD field distributions are subtracted from the mean CTF predictions in each CTF face and the resultant (CFD-CTF) residual distributions are sored in an HDF5 table along with predictive variables.  

Copula fitting by the maximum likelihood method with AIC model selection is carried out on each CTF face.  Additionally, the empirical Kendall's $\tau$ rank correlation coefficient is computed from the raw CFD data on each CTF face. Figure \ref{fig:copula_predicted} shows the copula parameters estimated from the raw CFD data on each pin as a function of axial position for the first 4 pins in the 5x5 CFD model.  There is a marked change in behavior of the copula between the pins.  This was an unexpected find since the flow patterns were speculated to be reasonably consistent from pin to pin.  Also, the influence of spacer grids on the correlation coefficient between the temperature and TKE fields is visible.  Across spacer grids the correlation coefficient first sharply falls indicating a tighter coupling between the TKE and temperature surface fields as the flow necks down when entering a grid.  This is followed by a sharp change in Kendall's $\tau$ towards unity indicating the temperature and TKE surface fields become less correlated immediately following the mixing vanes.  This change in correlation behavior is posited to be due to turbulent mixing effects.  The computed copula parameters are also stored alongside the raw temperature, TKE, and boundary heat flux aggregated residual distribution data in the HDF5 table.

\begin{figure}[H]%
    \centering
    \subfloat[Pin 1]{{\includegraphics[width=0.45\linewidth]{figs/preproc/copula_params_pin_1} }}%
    \qquad
    \subfloat[Pin 2]{{\includegraphics[width=0.45\linewidth]{figs/preproc/copula_params_pin_2} }}%
    \qquad
        \subfloat[Pin 3]{{\includegraphics[width=0.45\linewidth]{figs/preproc/copula_params_pin_3} }}%
    \qquad
        \subfloat[Pin 4]{{\includegraphics[width=0.45\linewidth]{figs/preproc/copula_params_pin_4} }}%
    \qquad
    \caption[Best fitting copula to CFD data.]{Best fitting copula determined by AIC model selection as a function of axial rod position.}%
    \label{fig:copula_predicted}%
\end{figure}

After pre-processing, the HDF5 table includes a list of predictive scalar values, $\mathbf p$, which are shown in table \ref{tab:features}, and a list of associated response variables; the copula parameters and residual sample distribution for $\{T,k,q''\}$ on each CTF face. 

\subsection{Feature Engineering}

The objective of feature engineering is to select a predictive variable set that can describe the behavior of the conditional quantiles and copula everywhere in the assembly.   
   
\begin{table}[h]
    \begin{center}
    \caption[Included exogenous training features.]{Features included in the gradient boosted models as exogenous variables.}
\begin{tabular}[h]{|l | l | l | l |}
    \hline
    Sym & Label & Feature & Unit \\
    \hline
    \hline
    $T$ & ctf\_twall\_avg & CTF Face surface temperature & $[K]$ \\
    $R_T$ & ctf\_twall\_range & Surface temperature range in 4 adjacent faces & $[K]$ \\ 
    $q''$ & ctf\_bhf\_avg & Local CTF face heat flux & $[W/m^2]$ \\
    $R_{q''}$ & ctf\_bhf\_range & Heat flux range in 4 adjacent faces & $[W/m^2]$ \\ 
    $u_z$ & w\_bulk & CTF subchannel bulk Z Velocity &  $[m/s]$ \\
    $k$ & ctf\_tke\_avg & Local CTF face near wall TKE &  $[J/kg]$ \\
    $R_k$ & ctf\_tke\_range & CTF TKE range in 4 adjacent faces & $[J/kg]$ \\ 
    $z$ & z & Global axial position & $[m]$ \\
    $\delta z_g$ & dz\_grid & Position relative to nearest spacer grid & $[m]$ \\
    $N_g$ & n\_upsteam\_grid  & Nearest upstream spacer grid ID & $[]$ \\
    $T_\infty$ & t\_bulk & Subchannel bulk temperature  &  $[K]$ \\
    \hline
\end{tabular}
\label{tab:features}
\end{center}
\end{table}

The predictive variables given in table \ref{tab:features} were selected based on two criteria:  availability and orthogonality.  A predictive variable must be made available by VERA or must be computable from CTF results alone.  This must be true since when evaluating the hi2lo model, the inputs to the hi2lo model must come from either previously stored information or information made available by CTF at runtime.  At this juncture, this criteria precludes using information such as geometric orientation of a given spacer grid since it is not possible to extract or infer this information from the CTF output.

It is not useful to include multiple variables which are co-linear. The bulk fluid density was not included in the predictive variable set as it strongly depends on the local temperature. Likewise the local static pressure was not used as a predictive variable since this would be one-to-one with the axial position.  The exclusion of this TH information is primarily done for computational saving when training the boosted models since, as opposed to other machine learning algorithms and statistical inference techniques, gradient boosting is robust to collinearity of features in the input space. 

In this case the inclusion of nuisance or collinear exogenous variables in the model will not necessarily reduce the model's ability to generalize to unseen data, only hamper computational efficiency.  The resulting feature importance plot shown in figure \ref{fig:ktauregfeatureimp} suggests that the relative axial position within a span does not provide predictive power since this information is redundant provided the absolute axial position and the nearest upstream spacer grid.  

\begin{figure}[H]
    \centering
    \includegraphics[width=0.6\linewidth]{figs/ktau_reg_feature_imp}
    \caption[Relative feature importance.]{Relative feature importances on kendall's $\tau$.}
    \label{fig:ktauregfeatureimp}
\end{figure}

TODO: Include example training table.  With predictive features and endogenous variable columns.



\subsection{Cross Validation}

An estimate of the interpolation error incurred when evaluating the trained models at unknown CFD states can be made by performing a leave one out (LOO) cross validation study.

\begin{figure}[h]
    \centering
    \includegraphics[width=0.3\linewidth]{figs/drawings/5x5_loo}
    \caption[Example pin layout for leave-one-out cross validation procedure.]{Example pin layout for leave-one-out cross validation procedure.  The gradient boosted models are trained on CFD and CTF data extracted from the blue pins.  Crud predictions are made on the missing pin.}
    \label{fig:5x5loo}
\end{figure}


As depicted in figure \ref{fig:5x5loo}, in this procedure a single CFD/CTF pin pair is removed from the database and then the model is retrained on remaining data.  Following this retraining, a crud prediction is made 
at the missing pin's TH conditions.  The crud results are compared against crud results generated using the full training data set.  This process is repeated by sequentially for each pin in the 5x5 set.  The differences are summarized and averaged to obtain a measure of model's predictive performance when applied to TH conditions that reside in the TH training envelope.

This cross validation technique only ascertains interpolation errors within the TH envelope enclosed by the original full training set.  The resulting interpolation error estimates cannot be extrapolated to core conditions that lay outside of the training set.  For a robust interpolation error analysis, a much larger training data set is required
that spans essentially all possible TH conditions encountered in an operational PWR.  This will require large scale CFD runs and is left as a avenue for future uncertainty quantification work.

\subsection{Quantile Regressors}

A principal goal of the machine learning model is to predict the shape of the conditional teperature and TKE distributions as a function of local core conditions.  

Quantile-quantile (Q-Q) plots of the  temperature and TKE residual distributions are used to elucidate bias introduced by the machine learning model in the conditional quantiles at a variety of axial positions and local core conditions.  Estimated quantiles are obtained for the left-out pin by evaluating the trained reduced LOO model and are compared to the expected CFD result.
A subset of the TKE residual quantile regression results are given in figure \ref{fig:tkepin1} to \ref{fig:tkepin3}.  A complete set of quantile regression results are provide in chapter \ref{chap:app_ml}, Appendix A. The Q-Q plots summarize the biases in the conditional quantile distributions when compared to the target golden standard CFD data.  The maximum and average Kolmogorov–Smirnov (KS) statistic is provided in the Q-Q figures for each pin which can be computed by equation \ref{eq:ks_stat}.
\begin{equation}
KS = \mathrm{sup}(\{\hat F(\tau_i) - F(\tau_i)\})
\label{eq:ks_stat}
\end{equation}
Where $\mathrm{sup}(\cdot)$ is the supremum of the set of distances between the predicted and empirical cumulative densities $\{\hat F(\tau_i) - F(\tau_i)\}$.
This statistic was computed at each axial level on the CTF grid.

\begin{figure}[H]%
    \centering
    \subfloat[TKE quantile regression results. CFD in solid line.  Predicted values as dashed.]{{\includegraphics[width=0.45\linewidth]{figs/ml_fit/q_tke_regression_1} }}%
    \qquad
    \subfloat[Q-Q plot of TKE quantile regression predictions from LOO cross validation study]{{\includegraphics[width=0.45\linewidth]{figs/ml_fit/qq_tke_pin_1} }}%
    \caption[Q-Q LOO TKE pin 1 results.]{Pin 1 TKE quantile regression predictions from LOO cross validation study.}%
    \label{fig:tkepin1}%
\end{figure}

\begin{figure}[H]%
    \centering
    \subfloat[TKE quantile regression results. CFD in solid line.  Predicted values as dashed.]{{\includegraphics[width=0.45\linewidth]{figs/ml_fit/q_tke_regression_2} }}%
    \qquad
    \subfloat[Q-Q plot of TKE quantile regression predictions from LOO cross validation study]{{\includegraphics[width=0.45\linewidth]{figs/ml_fit/qq_tke_pin_2} }}%
    \caption[Q-Q LOO TKE pin 2 results.]{Pin 1 TKE quantile regression predictions from LOO cross validation study.}%
    \label{fig:tkepin2}%
\end{figure}

\begin{figure}[H]%
    \centering
    \subfloat[TKE quantile regression results. CFD in solid line.  Predicted values as dashed.]{{\includegraphics[width=0.45\linewidth]{figs/ml_fit/q_tke_regression_3} }}%
    \qquad
    \subfloat[Q-Q plot of TKE quantile regression predictions from LOO cross validation study]{{\includegraphics[width=0.45\linewidth]{figs/ml_fit/qq_tke_pin_3} }}%
    \caption[Q-Q LOO TKE pin 3 results.]{Pin 1 TKE quantile regression predictions from LOO cross validation study.}%
    \label{fig:tkepin3}%
\end{figure}


A subset of the Temperature residual quantile regression results are given in figure \ref{fig:temppin1} to \ref{fig:temppin3}.  Similar to the TKE conditional quantiles, the conditional temperature distribution exhibits sharp changes in behavior across the spacer grids.  Discontinuities enforced the choice of a machine learning algorithm which is resilient to extremely sharp gradients in the response surface.  

\begin{figure}[H]%
    \centering
    \subfloat[Temperature quantile regression results. CFD in solid line.  Predicted values as dashed.]{{\includegraphics[width=0.45\linewidth]{figs/ml_fit/q_twall_regression_1} }}%
    \qquad
    \subfloat[Q-Q plot of Temperature quantile regression predictions from LOO cross validation study]{{\includegraphics[width=0.45\linewidth]{figs/ml_fit/qq_twall_pin_1} }}%
    \caption[Q-Q LOO Temperature pin 1 results.]{Pin 1 Temperature quantile regression predictions from LOO cross validation study.}%
    \label{fig:temppin1}%
\end{figure}

\begin{figure}[H]%
    \centering
    \subfloat[Temperature quantile regression results. CFD in solid line.  Predicted values as dashed.]{{\includegraphics[width=0.45\linewidth]{figs/ml_fit/q_twall_regression_2} }}%
    \qquad
    \subfloat[Q-Q plot of Temperature quantile regression predictions from LOO cross validation study]{{\includegraphics[width=0.45\linewidth]{figs/ml_fit/qq_twall_pin_2} }}%
    \caption[Q-Q LOO Temperature pin 2 results.]{Pin 2 Temperature quantile regression predictions from LOO cross validation study.}%
    \label{fig:temppin2}%
\end{figure}

\begin{figure}[H]%
    \centering
    \subfloat[Temperature quantile regression results. CFD in solid line.  Predicted values as dashed.]{{\includegraphics[width=0.45\linewidth]{figs/ml_fit/q_twall_regression_3} }}%
    \qquad
    \subfloat[Q-Q plot of Temperature quantile regression predictions from LOO cross validation study]{{\includegraphics[width=0.45\linewidth]{figs/ml_fit/qq_twall_pin_3} }}%
    \caption[Q-Q LOO Temperature pin 3 results.]{Pin 3 Temperature quantile regression predictions from LOO cross validation study.}%
    \label{fig:temppin3}%
\end{figure}

The ability of the gradient boosting method to predict conditional quantiles is  

\subsection{Kendall's $\tau$ Regression}

The rank correlation coefficient, Kendall's $\tau$ ($\rho_\tau$), is used to quantify the strength of dependence between the temperature and TKE on the rod surface in each CTF face.  A separate gradient boosted regression model was tasked with predicting this statistic as a function of local core conditions.   The growth rate of crud was shown to be sensitive to   $\rho_\tau$ in section \ref{sec:crud_copula_sensi}, figure \ref{fig:patchcrudfit80}.  It is therefore important to understand the error and uncertainty carried by the predicted $\hat \rho_\tau$ values in each CTF face.  

A subset of the 5x5 assembly's Kendall's $\tau$ regression results are given in figure \ref{fig:ktauregression} and the complete 5x5 $\rho_\tau$ LOO cross validation results are given in figure \ref{fig:ktauregressionmontage}.  There is a marked change in behavior of the rank correlation coefficient as a function of axial position in the core from pin to pin.  The influence of Kendall's $\tau$ on the CTF face-integrated crud results was discussed in section \ref{sec:crud_copula_sensi}, and it was shown to be an important parameter to accurately predict via the machine learning model.  Pins with large relative errors for Kendall's $\tau$ are expected to produce anomalously poor crud predictions.

The worst performing pin with respect to $\hat \rho_\tau$ prediction was pin 8, as indicated in table () and figure \ref{fig:ktauregressionmontage}.  Interestingly, this pin exhibited relatively good agreement between the predicted crud distribution and the expected CFD crud distribution as indicated in table \ref{tab:loo_crud_bmass} and figure \ref{fig:montageaxialbmasssm}.  This pin, however, was relatively cold in comparison to the others in the fuel bundle, growing only $5.9$e-2 $[g]$ of crud in 300 days when the hotest rods grew $1.4$e0 $[g]$ in the same time.  In the case of pin 8, since the majority of the rod surface exists below the saturation point the crud is not sensitive to the shape of the joint temperature and TKE distributions.  

\begin{figure}[H]%
    \centering
    \subfloat[Pin 1]{{\includegraphics[width=0.45\linewidth]{figs/ml_fit/ktau_regression_1} }}%
    \qquad
    \subfloat[Pin 2]{{\includegraphics[width=0.45\linewidth]{figs/ml_fit/ktau_regression_2} }}%
    \qquad
    \subfloat[Pin 3]{{\includegraphics[width=0.45\linewidth]{figs/ml_fit/ktau_regression_3} }}%
    \qquad
    \subfloat[Pin 4]{{\includegraphics[width=0.45\linewidth]{figs/ml_fit/ktau_regression_4} }}%
    \qquad
    \caption[Kendall's $\tau$ regression LOO results.]{Kendall's $\tau$ regression results from LOO cross validation study.}%
    \label{fig:ktauregression}%
\end{figure}
To improve the performance of the Kendall's $\tau$ regressors, a larger training set could be generated in future work.  For this limited 25 pin data set, it is hypothesized that each pin has a substantially unique flow field when compared to the other 24 pins.  Expelling a pin from the training data set for cross validation causes the predictive performance of the model to suffer since the remaining pins in the training set do not provide the requisite information about the local core conditions vs. Kendall's $\tau$ relationship for the missing pin.

\subsection{Copula Classifier}

In addition to the rank correlation coefficient, Kendall's $\tau$, the copula family is also required to recover the copula density function on each CTF face.  To this end a gradient boosted classifier was trained on the available CFD data.  Copula information extracted from the raw CFD results is shown in figure \ref{fig:copula_predicted}.

Figure \ref{fig:confusionmatrixavg} summarizes the LOO cross validation results of the copula classifier as a confusion matrix.  The diagonal entries of the confusion matrix represent the correctly labeled copula predictions made by the reduced LOO trained classifier for each copula family average over the entire 5x5 assembly.  It is shown that on average the classifier predicts an incorrect result more often than not.  

\begin{figure}[H]
    \centering
    \includegraphics[width=0.5\linewidth]{figs/confusion_matrix_avg}
    \caption[Copula classifier confusion matrix.]{Copula classifier confusion matrix.}
    \label{fig:confusionmatrixavg}
\end{figure}


The copula classifiers performance averaged over all pins is shown in figure \ref{fig:confusionmatrixavg}. The copula classifier struggles to predict the correct copula class given the local TH state and axial core position.  As indicated in figure \ref{fig:copula_predicted}, the behavior of the copula as a function of axial rod position are erratic and inconsistent from pin to pin.  Introducing other TH exogenous variables in addition to the axial position did not increase the classification score significantly.  We can conclude that the copula are not well described by local core condition and axial position.  It remains as future work to investigate if including additional geometric pin and grid attributes could improve the classification results.  Additional software infrastructure would be required to both write geometric pin and grid features from CTF and to utilize the geometric features in the current model.

Additional future work could include performing a transformation of the input space so that the copula labels are more easily separable in the transformed space.  A Potential candidate for building this transformation is the UMAP [ref] manifold learning algorithm.  

A complementary strategy to transforming the original data set as a preprocessing step is a ensemble machine learning technique known as stacking [ref].  Stacking combines the predictions of multiple classifiers using a meta-classifier.  Since model tuning and ML performance is not a focus of this work, the application of this technique to improve copula classification results is left as future work.  

Though improvements are possible, it should also be noted that section \ref{sec:crud_copula_sensi} and table \ref{tab:crud_totals_copula} show that the copula family does not significantly influence overall pin integrated crud results.  Gains in the copula classifier accuracy will not translate to a large improvement in crud prediction accuracy.

\section{Single Pin Comparisons}

\begin{itemize}
    \item (\checkmark-) Show Kendall's tau vs Axial position for a single pin.  The rank correlation coefficient shows
    the influence of spacer grids on the joint distribution of temperature and turbulent kinetic energy.
    \item ($\cdot$) Show predicted Copula as a function of axial position.  Is there a clear trend here (i.e. is a guassian copula
    a good fit far away from spacer grids and the clayton copula a better fit near spacer grids?  (TODO: create this plot functionality - capability to compute best-fit copula already exists)
    \item (\checkmark-) Compare Gaussian vs best-fit copula crud results.
    \item (\checkmark-) Show axial crud comparisons for a single hi2lo, CFD and CTF pin.
    \item (\checkmark-) Show integrated crud comparisons for a single hi2lo, CFD and CTF pin.
\end{itemize}

\begin{figure}[H]
    \centering
    \includegraphics[width=0.7\linewidth]{figs/5x5/imp/1_5_axial_bmass}
    \caption{Pin 1 CTF vs CFD vs Hi2lo axial crud boron mass distribution at 300 days.}
    \label{fig:15axialbmass}
\end{figure}
\begin{figure}[H]
    \centering
    \includegraphics[width=0.7\linewidth]{figs/5x5/imp/1_5_axial_cmass}
    \caption{Pin 1 CTF vs CFD vs Hi2lo axial crud mass distribution at 300 days.}
    \label{fig:15axialcmass}
\end{figure}
\begin{figure}[H]
    \centering
    \includegraphics[width=0.7\linewidth]{figs/5x5/imp/1_5_pin_bmass_time}
    \caption{Pin 1 CTF vs CFD vs Hi2lo integrated crud boron mass distribution as a function of time.}
    \label{fig:15pinbmasstime}
\end{figure}
\begin{figure}[H]
    \centering
    \includegraphics[width=0.7\linewidth]{figs/5x5/imp/1_5_pin_cmass_time}
    \caption{Pin 1 CTF vs CFD vs Hi2lo integrated crud mass distribution as a function of time.}
    \label{fig:15pincmasstime}
\end{figure}

\begin{figure}[H]
    \centering
    \includegraphics[width=0.7\linewidth]{figs/5x5/imp/tstep_5/pin_1/hi2lo_imp_pin_bmass}
    \caption{Pin 1 hi2lo 2D surface map of crud boron mass density.}
    \label{fig:hi2loimppinbmass}
\end{figure}
\begin{figure}[H]
    \centering
    \includegraphics[width=0.7\linewidth]{figs/5x5/imp/tstep_5/pin_1/hi2lo_imp_pin_cmass}
    \caption{Pin 1 hi2lo 2D surface map of crud mass density.}
    \label{fig:hi2loimppincmass}
\end{figure}


\begin{figure}[H]%
    \centering
    \subfloat[Hi2lo pin boron mass.]{{\includegraphics[width=0.46\linewidth]{figs/5x5/imp/tstep_5/pin_1/hi2lo_imp_pin_z_bmass} }}%
    \qquad
    \subfloat[CFD pin boron mass.]{{\includegraphics[width=0.46\linewidth]{figs/5x5/cfd/tstep_5/pin_1/CFD_pin_z_bmass} }}%
    \caption{Pin 1 crud boron mass results at 300 days.}%
    \label{fig:hi2loimppinzbmass}
\end{figure}


\begin{figure}[H]%
    \centering
    \subfloat[Hi2lo pin crud thickness.]{{\includegraphics[width=0.46\linewidth]{figs/5x5/imp/tstep_5/pin_1/hi2lo_imp_pin_z_cthick} }}%
    \qquad
    \subfloat[CFD pin crud thickness.]{{\includegraphics[width=0.46\linewidth]{figs/5x5/cfd/tstep_5/pin_1/CFD_pin_z_cthick} }}%
    \caption{Pin 1 crud thickness results at 300 days.}%
    \label{fig:hi2loimppinzcthick}
\end{figure}


% Redundant since we already have tempearture and TKE results
% \begin{figure}[H]
%    \centering
%    \includegraphics[width=0.7\linewidth]{figs/5x5/imp/tstep_5/pin_1/hi2lo_imp_pin_z_tke}
%    \caption{}
%    \label{fig:hi2loimppinztke}
% \end{figure}
% \begin{figure}[H]
%    \centering
%    \includegraphics[width=0.7\linewidth]{figs/5x5/imp/tstep_5/pin_1/hi2lo_imp_pin_z_twall}
%    \caption{}
%    \label{fig:hi2loimppinztwall}
% \end{figure}

\section{Multi Pin Comparisons}

\begin{itemize}
    \item (\checkmark-) Exercise hi2lo method with 5x5 CFD data set.
    \item (\checkmark-) Show pin-by-pin statistics to show if a geometric bias is present.  Are we always over predicting crud
    in pins near the edge of the assembly?
    \item (\checkmark-) Show axial CFD vs CTF vs hi2lo results for each pin.
    \item (\checkmark-) Compare total integrated assembly crud mass and boron bass for CFD vs CTF vs hi2lo.
\end{itemize}

% Hi2lo Bmass results
\begin{table}[h]
    \begin{center}
    \caption[Hi2lo crud boron mass results]{Crud boron mass hi2lo LOO result summary at 300 days.}
    \begin{tabular}[h]{|c|c|c|c|} 
        \hline
Pin & CTF Bmass Tot & CFD Bmass Tot & Hi2lo Bmass Tot  \\ 
\hline
1  & 1.2940e-07 & 7.5489e-08 & 6.3163e-08 \\ 
2  & 1.1458e-07 & 5.1953e-08 & 3.1487e-08 \\ 
3  & 1.0265e-07 & 3.2678e-08 & 3.4192e-08 \\ 
4  & 1.0111e-07 & 5.6847e-08 & 2.9848e-08 \\ 
5  & 9.5319e-08 & 1.8974e-08 & 2.2723e-08 \\ 
6  & 4.0505e-08 & 9.1686e-09 & 1.0065e-08 \\ 
7  & 8.0907e-09 & 4.0210e-09 & 3.7515e-09 \\ 
8  & 6.5705e-09 & 2.9861e-09 & 3.3475e-09 \\ 
9  & 6.7204e-09 & 3.3324e-09 & 3.3063e-09 \\ 
10  & 6.6850e-09 & 3.5892e-09 & 2.8171e-09 \\ 
11  & 8.7449e-09 & 4.0316e-09 & 3.7932e-09 \\ 
12  & 4.6693e-08 & 1.1722e-08 & 8.2669e-09 \\ 
13  & 1.0616e-07 & 2.2909e-08 & 2.6878e-08 \\ 
14  & 1.0617e-07 & 2.7471e-08 & 3.9548e-08 \\ 
15  & 1.0366e-07 & 4.5067e-08 & 3.3163e-08 \\ 
16  & 1.1594e-07 & 2.9707e-08 & 4.3385e-08 \\ 
17  & 9.1090e-08 & 2.6739e-08 & 2.6569e-08 \\ 
18  & 7.1616e-08 & 2.3468e-08 & 2.0092e-08 \\ 
19  & 6.1329e-08 & 1.1289e-08 & 1.4341e-08 \\ 
20  & 1.6370e-08 & 7.2277e-09 & 4.8991e-09 \\ 
21  & 1.2524e-08 & 4.5454e-09 & 4.4092e-09 \\ 
22  & 1.7007e-08 & 4.4576e-09 & 6.1729e-09 \\ 
23  & 6.2594e-08 & 1.5092e-08 & 1.1521e-08 \\ 
24  & 7.1144e-08 & 1.8479e-08 & 2.1561e-08 \\ 
25  & 4.1668e-08 & 1.3290e-08 & 8.6106e-09 \\ 
\hline
\end{tabular}
\label{tab:loo_crud_bmass}
\end{center}
\end{table}


% hi2lo cmass results
\begin{table}[h]
    \begin{center}
        \caption[Hi2lo crud mass results]{Crud mass hi2lo LOO result summary at 300 days.}
    \begin{tabular}[h]{|c|c|c|c|} 
        \hline
  Pin & CTF Cmass Tot & CFD Cmass Tot & Hi2lo Cmass Tot  \\ 
\hline
1  & 2.4316e-04 & 1.4232e-04 & 1.1899e-04 \\ 
2  & 2.1537e-04 & 9.8041e-05 & 5.9479e-05 \\ 
3  & 1.9302e-04 & 6.1962e-05 & 6.4702e-05 \\ 
4  & 1.9040e-04 & 1.0737e-04 & 5.6557e-05 \\ 
5  & 1.7982e-04 & 3.6318e-05 & 4.3346e-05 \\ 
6  & 7.6893e-05 & 1.7713e-05 & 1.9431e-05 \\ 
7  & 1.5980e-05 & 7.9597e-06 & 7.4350e-06 \\ 
8  & 1.3032e-05 & 5.9122e-06 & 6.6394e-06 \\ 
9  & 1.3329e-05 & 6.6094e-06 & 6.5574e-06 \\ 
10  & 1.3259e-05 & 7.1162e-06 & 5.5866e-06 \\ 
11  & 1.7213e-05 & 7.9487e-06 & 7.5059e-06 \\ 
12  & 8.8404e-05 & 2.2533e-05 & 1.5989e-05 \\ 
13  & 2.0002e-04 & 4.3636e-05 & 5.1048e-05 \\ 
14  & 1.9970e-04 & 5.2130e-05 & 7.4719e-05 \\ 
15  & 1.9474e-04 & 8.4999e-05 & 6.2756e-05 \\ 
16  & 2.1787e-04 & 5.6215e-05 & 8.1922e-05 \\ 
17  & 1.7124e-04 & 5.0727e-05 & 5.0361e-05 \\ 
18  & 1.3477e-04 & 4.4664e-05 & 3.8244e-05 \\ 
19  & 1.1573e-04 & 2.1674e-05 & 2.7412e-05 \\ 
20  & 3.1488e-05 & 1.4063e-05 & 9.6251e-06 \\ 
21  & 2.4285e-05 & 8.9271e-06 & 8.6669e-06 \\ 
22  & 3.2634e-05 & 8.7701e-06 & 1.2055e-05 \\ 
23  & 1.1797e-04 & 2.8810e-05 & 2.2075e-05 \\ 
24  & 1.3379e-04 & 3.5204e-05 & 4.0989e-05 \\ 
25  & 7.8681e-05 & 2.5506e-05 & 1.6677e-05 \\ 
\hline
\end{tabular}
\label{tab:loo_crud_cmass}
\end{center}
\end{table}

% RMS error table
\begin{table}[h]
    \begin{center}
        \caption[Hi2lo crud RMS summary.]{Hi2lo crud RMS summary.}
    \begin{tabular}[h]{|c|c|c|c|c|c|c|c|} 
        \hline
Pin  & RMS & RMS & RMS & RMS & RMS & RMS & RMS \\
\hline
1  & 6.2482e-04 & -2.3332e-05 & 3.3009e-07 & -1.2326e-08 & 7.5489e-08 & 1.4232e-04 & 0.0000e+00 \\ 
2  & 8.1392e-04 & -3.8562e-05 & 4.3183e-07 & -2.0466e-08 & 5.1953e-08 & 9.8041e-05 & 0.0000e+00 \\ 
3  & 2.0189e-04 & 2.7399e-06 & 1.0731e-07 & 1.5140e-09 & 3.2678e-08 & 6.1962e-05 & 0.0000e+00 \\ 
4  & 1.0737e-03 & -5.0812e-05 & 5.7008e-07 & -2.6998e-08 & 5.6847e-08 & 1.0737e-04 & 0.0000e+00 \\ 
5  & 2.2335e-04 & 7.0283e-06 & 1.1847e-07 & 3.7498e-09 & 1.8974e-08 & 3.6318e-05 & 0.0000e+00 \\ 
6  & 6.5623e-05 & 1.7178e-06 & 3.4608e-08 & 8.9595e-10 & 9.1686e-09 & 1.7713e-05 & 0.0000e+00 \\ 
7  & 2.1295e-05 & -5.2469e-07 & 1.0898e-08 & -2.6945e-10 & 4.0210e-09 & 7.9597e-06 & 0.0000e+00 \\ 
8  & 1.8572e-05 & 7.2717e-07 & 9.4019e-09 & 3.6144e-10 & 2.9861e-09 & 5.9122e-06 & 0.0000e+00 \\ 
9  & 1.0839e-05 & -5.1974e-08 & 5.4659e-09 & -2.6075e-11 & 3.3324e-09 & 6.6094e-06 & 0.0000e+00 \\ 
10  & 2.4348e-05 & -1.5296e-06 & 1.2298e-08 & -7.7213e-10 & 3.5892e-09 & 7.1162e-06 & 0.0000e+00 \\ 
11  & 2.9659e-05 & -4.4282e-07 & 1.5478e-08 & -2.3832e-10 & 4.0316e-09 & 7.9487e-06 & 0.0000e+00 \\ 
12  & 1.9073e-04 & -6.5447e-06 & 1.0111e-07 & -3.4547e-09 & 1.1722e-08 & 2.2533e-05 & 0.0000e+00 \\ 
13  & 3.9831e-04 & 7.4119e-06 & 2.1135e-07 & 3.9690e-09 & 2.2909e-08 & 4.3636e-05 & 0.0000e+00 \\ 
14  & 5.0327e-04 & 2.2589e-05 & 2.6950e-07 & 1.2076e-08 & 2.7471e-08 & 5.2130e-05 & 0.0000e+00 \\ 
15  & 5.5978e-04 & -2.2242e-05 & 2.9813e-07 & -1.1904e-08 & 4.5067e-08 & 8.4999e-05 & 0.0000e+00 \\ 
16  & 5.6026e-04 & 2.5707e-05 & 2.9830e-07 & 1.3678e-08 & 2.9707e-08 & 5.6215e-05 & 0.0000e+00 \\ 
17  & 1.2664e-04 & -3.6509e-07 & 6.6854e-08 & -1.7016e-10 & 2.6739e-08 & 5.0727e-05 & 0.0000e+00 \\ 
18  & 1.8205e-04 & -6.4199e-06 & 9.5976e-08 & -3.3756e-09 & 2.3468e-08 & 4.4664e-05 & 0.0000e+00 \\ 
19  & 1.2838e-04 & 5.7381e-06 & 6.8271e-08 & 3.0523e-09 & 1.1289e-08 & 2.1674e-05 & 0.0000e+00 \\ 
20  & 9.7915e-05 & -4.4377e-06 & 5.1674e-08 & -2.3286e-09 & 7.2277e-09 & 1.4063e-05 & 0.0000e+00 \\ 
21  & 1.7636e-05 & -2.6016e-07 & 9.2180e-09 & -1.3620e-10 & 4.5454e-09 & 8.9271e-06 & 0.0000e+00 \\ 
22  & 6.7042e-05 & 3.2845e-06 & 3.5199e-08 & 1.7153e-09 & 4.4576e-09 & 8.7701e-06 & 0.0000e+00 \\ 
23  & 1.7096e-04 & -6.7350e-06 & 9.0595e-08 & -3.5703e-09 & 1.5092e-08 & 2.8810e-05 & 0.0000e+00 \\ 
24  & 1.5885e-04 & 5.7841e-06 & 8.4133e-08 & 3.0827e-09 & 1.8479e-08 & 3.5204e-05 & 0.0000e+00 \\ 
25  & 1.9995e-04 & -8.8282e-06 & 1.0617e-07 & -4.6799e-09 & 1.3290e-08 & 2.5506e-05 & 0.0000e+00 \\ 
\hline
\end{tabular}
\label{tab:loo_rms}
\end{center}
\end{table}

\begin{figure}[H]
    \centering
    \includegraphics[width=0.7\linewidth]{figs/5x5/imp/asm_cmass_time}
    \caption{Assembly integrated CTF vs CFD vs Hi2lo crud mass as a function of time.}
    \label{fig:asmcmasstime}
\end{figure}

\begin{figure}[H]
    \centering
    \includegraphics[width=0.7\linewidth]{figs/5x5/imp/l2_boron_asm_errors_hmap}
    \caption{5x5 Axial RMS crud boron mass relative error distribution.}
    \label{fig:l2boronasmerrorshmap}
\end{figure}
\begin{figure}[H]
    \centering
    \includegraphics[width=0.7\linewidth]{figs/5x5/imp/l2_cmass_asm_errors_hmap}
    \caption{5x5 Axial RMS crud mass relative error distribution.}
    \label{fig:l2cmassasmerrorshmap}
\end{figure}


\begin{figure}[H]
    \centering
    \includegraphics[width=0.7\linewidth]{figs/5x5/imp/tot_bmass_rel_asm_errors_hmap}
    \caption{5x5 Integrated crud boron mass relative error distribution.}
    \label{fig:totbmassrelasmerrorshmap}
\end{figure}
\begin{figure}[H]
    \centering
    \includegraphics[width=0.7\linewidth]{figs/5x5/imp/tot_boron_asm_errors_hmap}
    \caption{5x5 Integrated crud boron mass absolute error distribution.}
    \label{fig:totboronasmerrorshmap}
\end{figure}
\begin{figure}[H]
    \centering
    \includegraphics[width=0.7\linewidth]{figs/5x5/imp/tot_cmass_asm_errors_hmap}
    \caption{5x5 Integrated crud mass absolute error distribution.}
    \label{fig:totcmassasmerrorshmap}
\end{figure}
\begin{figure}[H]
    \centering
    \includegraphics[width=0.7\linewidth]{figs/5x5/imp/tot_cmass_rel_asm_errors_hmap}
    \caption{5x5 Integrated crud mass relative error distribution.}
    \label{fig:totcmassrelasmerrorshmap}
\end{figure}

The correlation between errors committed by the multiple quantile and $\rho_\tau$ regressors on the errors seen in the crud results was investigated in an attempt to establish performance metrics.  If a statistically significant trend can be drawn between the ML model performance and predicted crud accuracy results then the trend could be used as an indicative tool for testing the a.  To do this, one could first compute the ML regression errors via a cross validation strategy.  Provided these error estimates and known sensitivities of the crud results to the ML errors one can estimate the expected accuracy of the crud predictions obtained via the hi2lo model.  This allows a user of the hi2lo tool to detect problems with the ML models before using the model to make crud predictions.

As shown in the upper-triangle of figure \ref{fig:asmerrorcorr} a Student-T test was conducted on the slope of each fitted linear trend model.  The null hypothesis was taken to be a slope of 0.  One can conclude that a  sample size greater than 25 pins should be used in future work to investigate the relative influence of machine learning errors on the predicted crud errors.  The standard deviation of the computed sensitivities is high when using a small sample size making it difficult to rigorously conclude that errors made by the machine learning models correspond to errors in the crud predictions.  

Statistically significant trends were found between the RMS error committed by the TKE quantile regressors and the crud boron and mass distribution errors, as measured by root-mean-squared error.  

\begin{figure}[H]
    \centering
    \includegraphics[width=0.95\linewidth]{figs/5x5/imp/asm_error_corr}
    \caption{Correlation of ML errors with crud prediction errors.}
    \label{fig:asmerrorcorr}
\end{figure}

\section{Section Takeaways}


\begin{itemize}
    \item Crud predictions made by the hi2lo model were compared against CFD/crud coupled results and stanalone CTF/crud results.  The Hi2lo model produced crud results closer to the CFD result than the CTF result, as anticipated.  The overall crud mass results for the 5x5 assembly favorably compared against the gold-standard CFD assembly integrated results. 
    \item A leave one out cross validation strategy was utilized to quantify the predictive performance of the model.  
    \item The prediction accuracy of the temperature and TKE quantile regression models was summarized though Q-Q plots for each pin in the LOO cross validation study.
    \item The prediction accuracy of the Kendall's $\tau$ regression model was assessed using the root-mean-square error for each pin in the LOO cross validation study.  \item Correlations between the errors committed by the machine learning models and the crud prediction errors were computed.  High uncertainty associated with these correlation measures did not permit a statistically significant link between poor Kendall's $\tau$ predictive performance and poor crud predictions to be drawn.
	\item The copula classifier performed poorly given the current set of considered explanatory variables and limited size of the training data set.  A Gaussian copula was assumed on each CTF face in place of the poorly predicted copula family from the classifier.  Recalling the results shown in \ref{sec:crud_copula_sensi}, this is not expected to reduce crud prediction accuracy.
\end{itemize}


%-----------------------------------------------------------------------------%
\chapter{Conclusion}
\label{chap:conc}
%-----------------------------------------------------------------------------%

This proposed technique does not try to provide a detailed spatial distribution
of the temperature, TKE, and CRUD on the rods' surface, rather the
statistical approach seeks to provide a detailed frequency distribution
of these fields.  The end result is model which correctly captures hot and cold
spot TH conditions that give rise to the largest and smallest boron
precipitation concentrations without precisely knowing where on the rods'
surface gave rise to the sampled TH conditions.

A large body of future work will focus on time steping and uncertainty propogation.
The changing TH conditoins throught a cycle must be accunted for in the simulation of CRUD build up over time.  Furthermore, the feedback between the CRUD layer and the rod surface temperature due to augmentation of the thermal resistance should be accounted for when stepping the simulation forward in time.
Uncertainties present in the gradient boosted models should be computed and
propogated into the CRUD results.   Uncertainties are expected to compound in time, thus there exists an incentive to minimize model induced uncertainties.

Additionally, an assesment of the regression kriging techniques should be made for this Hi2Low application.  Leveraging this technique for interpolating spatial fields is attractive because it naturally supplies uncertainty estimates for the value of the FOI at all interpolated locations.  However, the issue of sparsely available auxillery variables must be addressed before RK becomes applicable to this work.

A key measure of success for this Hi2Low work with respect to  CRUD predictions
in the vicinity of spacer grids is the computation time needed to build the
training data sets upon which a regression model is developed.
It has yet to be proven that the proposed Copula and GBM based framework outperforms
either a table-lookup approach or a spatial interpolation approach in which
spatial CFD information is explicitly preserved in the Hi2Low model.  This assessment
of computation requirements is a key avenue for future work.

%-----------------------------------------------------------------------------%
% %! TEX root = ../outline.tex

%-----------------------------------------------------------------------------%
\pagebreak
\section*{Project Schedule}

Legend:
\begin{itemize}
    \item {\color{blue} Blue text: New task (added post-proposal).}
    \item (\checkmark)  Complete
    \item (\checkmark-)  Underway but incomplete
    \item ($\cdot$)  Not started. Future Work.
    \item (\xmark)  \sout{Not started.  Will not do.}
\end{itemize}
\bigskip

\begin{enumerate}
\item \textbf{(\checkmark) Literature review.}
\item \textbf{(\checkmark) Develop CFD-CRUD tools for training data generation and extraction.}
          Platform used to generate datasets that are
          required to inform a scale-bridging model.
    \begin{enumerate}
        \item (\checkmark) Create CFD data extraction and post processing utilities.
        \item (\checkmark) Develop STAR-CCM+ plugin to extract cladding surface and volumetric TH quantities from a CFD computation.
        \item (\checkmark) Compute the required volume and surface integrals to
              distill finely resolved CFD datasets into subchannel-like results.
    \end{enumerate}
\item \textbf{(\checkmark) Demonstrate differences between CFD and Subchannel predictions.}
    \begin{enumerate}
        \item (\checkmark) Compute differences between volume/surface averaged CFD quantities and subchannel predictions.
        \item (\checkmark) Identify deficiencies in subchannel predictions wrt. CRUD growth and CILC.
    \end{enumerate}
\item \textbf{(\checkmark) Investigate CRUD model sensitives.}
    \begin{enumerate}
        \item (\checkmark) Identify CRUD (MAMBA) sensitivities to TH boundary conditions
        \item (\checkmark) Produce correlation coefficients and scatter plot depictions of the relationship(s) between input
              and output of the CRUD model.
          \item (\checkmark) Identify TH conditions under which CFD scale CRUD predictions diverge from subchannel-CRUD results.
    \end{enumerate}
\item \textbf{(\checkmark) Write proposal document}
%
\item \textbf{(\checkmark) Methods implementation and demonstration}
    \begin{enumerate}
        \item (\checkmark) De-trend pointwise CFD datasets \& compute residual distributions.
        \begin{enumerate}
            \item (\checkmark) Moving averaged approach (assumes CTF and CFD will agree on the mean)
            \item (\checkmark) CTF mean approach (requires CTF runs at identical CFD sample points)
        \end{enumerate}
        %
        \item (\checkmark) Create a tool to generate synthetic CFD-like training data sets to facilitate
            testing of the regression tools.
        \begin{enumerate}
            \item (\checkmark) Develop flexible dependence modeling package capable of capturing skewed covariance behavior (Copula).
            \item (\xmark) \sout{Develop Kriging model to capture spatial auto-correlation (may not be necessary?)}
        \end{enumerate}
        %
        \item (\checkmark) Construct a regression model $M(\cdot)$ (Gradient boosted tree model [GBM]).
        \begin{enumerate}
            \item (\checkmark) Regress copula and marginal model parameters on local-average CTF core conditions.
            \item (\checkmark) Evaluate machine learning regressions' ability to recover joint TH distributions on any CTF patch.
        \end{enumerate}
        %
        \item (\checkmark-) Error Estimation \& Propagation
        \begin{enumerate}
        	\item (\checkmark-) How many quantiles are required for temperature and TKE margin reconstruction
                                - How do the number of quantiles used in the reconstruction impact the uncertainty in CRUD predictions?
            \item ($\cdot$) {\color{blue} Account for uncertainty in CFD sample quantiles. The sample quantiles are known to be distributed according to a Gaussian distribution.  It is possible to propagate this uncertainty through the hi2lo model into the crud predictions.}
                \item ($\cdot$) {Estimate uncertainty introduced by the gradient boosted/Copula model}
            \item ($\cdot$) {Propagate uncertainty to ensemble CRUD calculations}
        \end{enumerate}
        %----------------- POST PROPOSAL ------------------%
        \item (\checkmark-) Multi-state point execution
        \begin{enumerate}
            \item (\checkmark) Implement time stepping scheme.  Support ability to change the power profile/flow conditions at each time step.
            \item (\checkmark) Implement ability to propagate uncertainty through time steps.
            \begin{itemize}
                \item (\checkmark) Account for sampling-induced uncertainty and demonstrate how these errors compound over time.
                \item (\checkmark) {\color{blue} Account for hot spot stationarity in time.}

            \end{itemize}
        \item ($\cdot$) {\color{blue} Detail how this hi2lo strategy fits into a multi-physics framework with TH/Neutronic/CRUD feedbacks.}
        %
        \end{enumerate}
        % -------------------------------------------------%
        \item (\xmark) \sout{Compute areas of the machine learning models' prediction surface with highest
                             sensitivity to changes core state conditions \& develop capability to super sample these regions.  This is a design of experiments problem: Where do we run our CFD computations?}
        %
        \item (\xmark) \sout{Demonstrate ability to perform dimensionality reduction of the input space via PCA.}

    \end{enumerate}

\item \textbf{(\checkmark-) Methods analysis and improvement}
    \begin{enumerate}
        \item (\checkmark) Perform CFD informed subchannel based CRUD prediction (hi2lo) for a \emph{single pin}.
        \item (\checkmark) Perform large (assembly) scale CFD runs to supply realistic data to hi2lo model.  (this was done by Bob Salko et. al for a 5x5 case)
        \item (\checkmark) Tune hi2lo algorithm's hyperparameters (re-sampling coefficients, machine learning hyper parameters)
        \item (\checkmark) {\color{blue} Implement and demonstrate variance reduction strategy (importance sampling). }
        \item ($\cdot$) Validate CFD informed subchannel CRUD results against plant data (IF AVAILABLE).
    \end{enumerate}
\item \textbf{(\checkmark) Final hi2lo model construction and demonstration}
    \begin{enumerate}
        \item (\checkmark) Extend to assembly scale cases.  Identify why this problem is difficult.  From a feature engineering standpoint: Each pin \& each CTF face must be uniquely identified in the core by some set of exogenous variables.
        \item (\checkmark) Demonstrate CFD informed subchannel model on a full-height, single assembly problem.
        \item (\checkmark) Assess Crud/Axial boron (CIPS) prediction improvements vs. base CTF/MAMBA model.
    \end{enumerate}
\item \textbf{(\checkmark) Write Dissertation}
\end{enumerate}

\begin{sidewaysfigure}
%---------------------------     YEAR ONE        ------------------------------%
\begin{ganttchart}[
        inline,
        x unit=1.5cm,
        y unit title =0.8cm,
        y unit chart=0.8cm,
        hgrid,
        vgrid,
        time slot format=isodate,
        time slot unit=month
    ]{2016-06-01}{2017-07-30}
    \gantttitlecalendar*[]{2016-06-01}{2017-07-30}{year, month=shortname} \\
    \gantttitlelist{1,...,14}{1} \\
%%%%%%% Phase 1
\ganttgroup{Proposal Era}{2016-06-01}{2017-06-30} \\  %elem0
\ganttbar{Lit. Review}{2016-06-01}{2016-11-01} \\  %elem1
\ganttbar{CFD Post Proc. Dev.}{2016-06-01}{2016-08-20} \\  %elem2
\ganttlinkedbar{CTF vs CFD}{2016-08-20}{2016-10-30} \ganttnewline  %elem3
\ganttbar{MAMBA Sensitivity}{2016-08-01}{2016-10-30} \ganttnewline  %elem4
\ganttbar{Copula Model Dev.}{2016-10-01}{2017-01-10}  \\  %elem5
\ganttlinkedbar{Synthetic CFD}{2016-12-01}{2017-01-30} \\  %elem6
\ganttlinkedbar{Gradient Boosting Dev.}{2016-12-01}{2017-03-30}  \\  %elem7
\ganttlinkedbar{Link Models w/ Crud Sim.}{2017-03-01}{2017-06-30}  \\  %elem8
\ganttbar{Write Proposal}{2017-04-01}{2017-06-30}  %elem9
\ganttmilestone{Proposal}{2017-06-30} \ganttnewline  %elem10
%%%%%%%% Extra Task Linkages
\ganttlink[link mid=0.2]{elem2}{elem4}
\end{ganttchart}
%-----------------------------------------------------------------------------%
\end{sidewaysfigure}

\begin{sidewaysfigure}
%---------------------------     YEAR TWO       ------------------------------%
\begin{ganttchart}[
        inline,
        x unit= 1.2cm,
        y unit title =0.8cm,
        y unit chart=0.8cm,
        hgrid,
        vgrid,
        time slot format=isodate,
        time slot unit=month
    ]{2017-06-01}{2018-12-01}
    \gantttitlecalendar*[]{2017-06-01}{2018-12-01}{year, month=shortname} \\
    \gantttitlelist{13,...,31}{1} \\
%%%%%%% Phase 2
\ganttbar{ }{2017-06-01}{2017-06-30}  %elem0
\ganttmilestone{Proposal}{2017-06-30} \ganttnewline  %elem1
\ganttgroup{Dissertation Era}{2017-07-1}{2018-10-31} \\  %elem2
\ganttbar{Methods Refinement (Hi2lo Refactor)}{2017-07-01}{2018-02-28} \ganttnewline  %elem3
\ganttbar{5x5 CFD Runs}{2018-01-01}{2018-03-30} \ganttnewline  %elem4
\ganttbar{Time Stepping}{2017-08-01}{2018-02-28} \ganttnewline  %elem5
\ganttbar{Uncert. Prop}{2017-12-01}{2018-02-28} \ganttnewline  %elem6
\ganttbar{Importance Sampling}{2018-03-01}{2018-07-31} \ganttnewline  %elem7
\ganttbar{Hi2lo Runs}{2018-03-28}{2018-07-01} \ganttnewline  %elem8
\ganttlinkedbar{Generate Figs}{2018-06-01}{2018-08-31} \ganttnewline  %elem9
\ganttbar{Write Dissertation}{2018-06-01}{2018-10-30}  %elem10
\ganttmilestone{Defense}{2018-10-30} \ganttnewline  %elem11
\ganttlinkedbar{Final Rev.}{2018-10-30}{2018-11-30}  \ganttnewline  %elem12
\ganttmilestone{End.}{2018-11-30} \ganttnewline  %elem13
%%%%%%%% Extra Task Linkages
\ganttlink[link mid=0.1]{elem4}{elem8}
\ganttlink[link mid=0.2]{elem5}{elem7}
\ganttlink[link mid=0.2]{elem5}{elem8}
\end{ganttchart}
%-----------------------------------------------------------------------------%
\end{sidewaysfigure}


\pagebreak


%-----------------------------------------------------------------------------%

%=================================APPX========================================%
%-----------------------------------------------------------------------------%
\chapter{Appendix A}
\label{chap:app_ml}


%! TEX root = ../dissertation_gurecky.tex

\section{5x5 Leave-One-Out Machine Learning Results}


Gradient boosted quantile regression model results are presented in figures \ref{fig:qtwallregressionmontagesm} and \ref{fig:qtkeregressionmontagesm}.  For each pin, the predictions are made for the left-out-pin and compared to the original CFD training data.  The gradient boosted results are shown as solid lines and the original CFD-CTF data is shown as broken lines. 

The presented quantile regression results are shown as a function of axial position along the rod for the residual surface temperature and TKE distributions, e.g $\hat q_{\tau}(z) = \mathbf b(z) + \varepsilon(z)$, where $\mathbf b(z) = \mu_{\mathrm{cfd}} - \mu_{\mathrm{ctf}}$.   The results were averaged over the 4 azimuthal CTF faces at each axial level in the CTF grid.   The root-mean-square (RMS) error of select quantiles prediction vs axial location are given in each figure.

Figures \ref{fig:qqtwallmontagesm} to \ref{fig:qqtkepinmontage} show quantile-quantile (Q-Q plots) for each pin in the 5x5 LOO results.  Each Q-Q plot summarizes the overall prediction quality afforded by the quantile regression averaged over the entire pin length.  At each CTF axial grid level, the Kolmogorov–Smirnov (KS) statistic was computed to quantify the goodness-of-fit of the quantile distribution reconstruction to the original, empirical CFD-CTF distribution. The average and maximum KS statistic encountered is recorded in each figure.

Additionally, the predicted rank correlation coefficient as a function of axial position made by the LOO-trained models are compared to the expected result in \ref{fig:ktauregressionmontage}.  The RMS error between the predicted $\hat \rho_\tau(z)$ and the CFD computed $\rho_\tau(z)$ is shown in each figure.

\newgeometry{left=1cm,right=1cm,top=1cm,bottom=1.5cm}
\begin{landscape}
\begin{figure}[H]
    \centering
    \includegraphics[width=0.96\linewidth]{figs/ml_fit/q_twall_regression_montage_sm}
    \caption{5x5 Axial surface temperature residual (CFD-CTF)  quantile predictions.}
    \label{fig:qtwallregressionmontagesm}
\end{figure}

\begin{figure}[H]
    \centering
    \includegraphics[width=0.96\linewidth]{figs/ml_fit/q_tke_regression_montage_sm}
    \caption{5x5 Axial TKE residual (CFD-CTF) quantile predictions.}
    \label{fig:qtkeregressionmontagesm}
\end{figure}

\begin{figure}[H]
    \centering
    \includegraphics[width=0.96\linewidth]{figs/ml_fit/qq_twall_montage_sm}
    \caption{5x5 surface temperature quantile predictions Q-Q goodness-of-fit summary.}
    \label{fig:qqtwallmontagesm}
\end{figure}

\begin{figure}[H]
    \centering
    \includegraphics[width=0.96\linewidth]{figs/ml_fit/qq_tke_pin_montage}
    \caption{5x5 TKE quantile predictions Q-Q goodness-of-fit summary.}
    \label{fig:qqtkepinmontage}
\end{figure}

\begin{figure}[H]
    \centering
    \includegraphics[width=0.96\linewidth]{figs/ml_fit/ktau_regression_montage}
    \caption{5x5 Kendall's $\tau$ vs axial position predictions.}
    \label{fig:ktauregressionmontage}
\end{figure}
\end{landscape}
\restoregeometry



\chapter{Appendix B}
\label{chap:app_b}
%! TEX root = ../dissertation_gurecky.tex

\section{5x5 Results}

All axial crud results for the 5x5 model are shown in figures \ref{fig:montageaxialbmasssm} and \ref{fig:montageaxialcmasssm}.  The axial distributions are shown at a simulated time of 300 days. Pin-integrated crud results plotted as a function of time are provided in figures \ref{fig:montagetimebmasssm} and \ref{fig:montagetimecmasssm}.

\newgeometry{left=1cm,right=1cm,top=1cm,bottom=1.5cm}
\begin{landscape}
\begin{figure}[H]
    \centering
    \includegraphics[width=.96\linewidth]{figs/5x5/imp/montage_axial_bmass_sm}
    \caption{5x5 axial crud boron mass results at 300 days.}
    \label{fig:montageaxialbmasssm}
\end{figure}
\begin{figure}[H]
    \centering
    \includegraphics[width=.96\linewidth]{figs/5x5/imp/montage_axial_cmass_sm}
    \caption{5x5 axial crud mass results at 300 days.}
    \label{fig:montageaxialcmasssm}
\end{figure}

\begin{figure}[H]
    \centering
    \includegraphics[width=0.96\linewidth]{figs/5x5/imp/montage_time_bmass_sm}
    \caption{5x5 rod integrated crud boron mass vs time.}
    \label{fig:montagetimebmasssm}
\end{figure}


\begin{figure}[H]
    \centering
    \includegraphics[width=0.96\linewidth]{figs/5x5/imp/montage_time_cmass_sm}
    \caption{5x5 rod integrated crud mass vs time.}
    \label{fig:montagetimecmasssm}
\end{figure}

\end{landscape}
\restoregeometry

\chapter{Appendix C}
\label{chap:app_c}
%! TEX root = ../dissertation_gurecky.tex

\section{Gradient Boosting Toolkit}
\index{Gradient Boosting!Software Implementation}

A gradient boosting library was developed in the python programming language to support the hi2lo work.  This package provides an easily extensible loss function class that a user can use to implement arbitrary loss functions in the gradient boosting framework.  As required by the hi2lo work, both quantile and least squares loss functions are included.  The package is applicable to both regression and classification problems.  CART tree construction controls are also provided allowing fine grained control over the weak learners.
The library interface was constructed to be similar to Scikit-learn's gradient boosting API so that the newly developed boosting algorithms can stand as drop in replacements for those available in Scikit-learn.

The gradient boosting package \emph{pCRTree} is available at \url{https://github.com/wgurecky/pCRTree.git}.

\section{Copula Toolkit}
\index{Copula!Software Implementation}

For copula simulation, the CDvine toolkit (GPLv3 licensed) is available for the R programming language. This packages does not implement all rotations of copula making it burdensome to handle negative dependence structures out-of-the-box.  Furthermore, the maximum likelihood fitting method included in CDVine does not allow the user to specify sample weights, a key feature for the CFD data under consideration since the CFD mesh cells vary in size.

To circumvent these deficiencies and potential license compatibility issues with VERA, a new copula toolkit was developed in python and is BSD3 licensed.
Careful attention was paid to develop a flexible abstract copula class which enables custom copula functions to be specified.  Importantly, all copula rotations are supported by default allowing one to model positive and negative dependence structures without duplication of code.
Canonical vine-copula construction and sampling algorithms are included in this package to handle the decomposition of arbitrary joint density functions of any dimension.
Copula parameters can be determined by a weighted maximum likelihood fit to empirically supplied data with included sample weights or by specifying a rank correlation coefficient in the case of Archimedean copula.  In the current hi2Low work, both capabilities are leveraged.

The \emph{StarVine} copula package and documentation is available at \url{https://github.com/wgurecky/StarVine.git}.

\section{Python Interfaces to CRUD Codes}
\index{MAMBA!Software Implementation}

As part of this work, python interfaces were developed for both the legacy CASL crud tool known as MAMBA1D and the state-of-the art CRUD package, Mamba.  The python wrappers to these Fortran codes facilitate rapid prototyping of hi2lo procedures which provide boundary conditions to the CRUD codes.  Additionally, the high level interface simplifies the process of orchestrating large CRUD sensitivity studies.

The python wrappers are available in the Virtual Environment for Reactor Analysis (VERA) developed by CASL \url{https://www.casl.gov}.

\section{Hi2lo Code}
\index{Hi2lo!Software Implementation}

A package that leverages all the aforementioned tools to produce estimates of crud growth rates was developed.  This high level package is the primary user facing result of the current work.  It should be noted this package is heavily dependent on CRUD simulation, copula construction, and gradient boosting technologies.
This package orchestrates the construction and evaluation of gradient boosted regression trees which provide the copula and marginal distribution parameters as a function of local core conditions.
Currently, multi pin, multi state point simulation is implemented with future work focused on parallelization, training data acquisition, and improvements to the machine learning model implementation.

The hi2lo crud growth package and documentation is available at \url{https://github.com/wgurecky/crudBoost.git}.


\section{Synthetic Training Data Generation}
\label{chap:synth}
\index{Synthetic CFD Data!Software Implementation}

A toolkit to overlay custom noise atop a CTF solution was developed to provide a secondary source of training data sets aside from running a CFD code.  The synthetic data generation tool provides training data sets with lower computational cost than CFD calculations.  Some properties of a true CFD solution field are preserved by the tool, namely that the shape of the marginal and copula distributions change as a function of position and local thermal hydraulic conditions in the core.  The synthetic data is not to be viewed as complete substitute for CFD data since it lacks the ability to capture spatial auto-correlation in the predicted spatial fields that arise naturally from the governing PDEs.  Neighboring points on the rod surface do not exchange any TH information in this tool.  Despite the unphysical nature of the synthetic data, the tool provides a means to verify that known relationships between the explanatory variables and the copula parameters are recovered by the gradient boosted regression model.  This is possible because the user specifies these relationships up-front as inputs to the surface field sampling routines. \\

An excerpt of an input to generate a synthetic single pin data set is given below:
\tiny
\begin{lstlisting}[language=XML]
{
    "pinID": 1,
    "chanID": 1,
    "averageHeatFlux": 1.2e6,
    "spans": {
              "0.0": {"model": "lower", "samples": 1000},
              "2.01": {"model": "upper", "samples": 4000},
              "2.53": {"model": "upper", "samples": 4000},
              "2.98": {"model": "upper", "samples": 4000}
    },
    "upper": {
            "0.0": {"copula":  {"family": "gauss", "params": [-0.5], "rot": 0},
                "tke": {"type": "gauss", "params": [0.001, 0.02]},
                "temp": {"type": "beta", "params": [5.0, 2.7], "loc": -9.2, "scale": 12.0},
                "bhf": {"type": "gauss", "params": [0.001, 2.6e4]}
                },
            "0.3": {"copula":  {"family": "gauss", "params": [-0.6], "rot": 0},
                "tke": {"type": "gauss", "params": [0.01, 0.008]},
                "temp": {"type": "beta", "params": [5.0, 1.7], "loc": -7.0, "scale": 8.0},
                "bhf": {"type": "gauss", "params": [0.01, 1.1e4]}
                },
            "1.0": {"copula":  {"family": "frank", "params": [4.0], "rot": 1},
                "tke": {"type": "gauss", "params": [0.01, 0.005]},
                "temp": {"type": "beta", "params": [5.0, 1.5], "loc": -4.0, "scale": 5.0},
                "bhf": {"type": "gauss", "params": [0.01, 0.9e4]}
                }
            },
    "lower": {
            "0.0": {"copula":  {"family": "gauss", "params": [-0.6]},
                "tke": {"type": "gauss", "params": [0.001, 0.0001]},
                "temp": {"type": "beta", "params": ["5.0*(t)/600.0", 5.0], "loc": -2.0, "scale": 4.0},
                "bhf": {"type": "gauss", "params": [0.01, 1.0e3]}
                },
            "1.0": {"copula":  {"family": "gauss", "params": [-0.6]},
                "tke": {"type": "gauss", "params": [0.001, 0.0002]},
                "temp": {"type": "beta", "params": [5.0, 5.0], "loc": -2.0, "scale": 4.0},
                "bhf": {"type": "gauss", "params": [0.01, 1.0e3]}
                }
            }
}
\end{lstlisting}
\normalsize
The synthetic data generation tool is available for download at \url{https://github.com/wgurecky/ctfpurt.git}


\chapter{Appendix D}
\label{chap:app_d}
%! TEX root = ../dissertation_gurecky.tex
% ------------------------- TH definitions ----------------------- %
\section{Subcooled Boiling and DNB}

The relationship between the surface temperature of a internally heated object and the heat flux from the surface into the surrounding fluid is shown in figure \ref{fig:boiling_curve}.

\begin{figure}[H]
    \centering
    \includegraphics[width=0.7\linewidth]{../proposal/images/boiling_curve}
    \caption{}
    \label{fig:boilingcurve}
\end{figure}


The curve can be approximated by equation \ref{eq:boil_h}.  Note that surface temperature $T_s$ is equivalent to the wall temperature $T_w$ in the equations which follow.
The critical heat flux (CHF) is the point at which film boiling begins to dominate and is accompanied by a precipitous drop in the heat transfer and a rise in the surface temperature.  This condition is known as departure from nucleate boiling (DNB) and must be avoided when operating a PWR.

\begin{equation}
q''(T_w) = 
\begin{cases}
      h(T_w-T_{\infty}), & \mbox{if } T_w < T_{sat} \\
      h(T_w-T_{\infty}) + q''_{nb} ,  & \mbox{if } T_{sat} \leq T_w < T_{CHF} 
\end{cases}
\label{eq:boil_h}
\end{equation}
Where $h$ is the single phase convective heat transfer coefficient which is in turn a function of the Nusselt number given in equations \ref{eq:htc} and \ref{eq:db}.  The contribution of nucleate boiling to the heat transfer can be approximated by the Rohsenow model given in \ref{eq:ros} \cite{rohsenow51}.

\begin{equation}
q''_{nb} = {{\mu }_{L}}{{h}_{fg}}{{\left[ \frac{g\left( {{\rho }_{L}}-{{\rho }_{v}} \right)}{\sigma } \right]}^{{}^{1}\!\!\diagup\!\!{}_{2}\;}}{{\left[ \frac{{{c}_{pL}}\left( {{T}_{w}}-{{T}_{sat}} \right)}{{{C}_{sf}}{{h}_{fg}}Pr_{L}^{n}} \right]}^{3}\;}
\label{eq:ros}
\end{equation}
Where $h_{fg}$ is the latent heat of vaporization, $\mu_L$ is the liquid viscosity, $\rho_v,\ \rho_L$ are the vapor and liquid phase densities, ${c}_{pL}$ is the specific heat of the liquid phase, and ${C}_{sf}$ is a tunable empirical constant.

\begin{equation}
h = \frac{k_l \mathrm{Nu}}{L} = \frac{q''}{T_w-T_{\infty}}
\label{eq:htc}
\end{equation}
Where $k_l$ is the thermal conductivity of the liquid, $L$ is the characteristic length scale, and $Nu$ is the Nusselt number.  For non-boiling flows over a flat vertical surface, the Nusselt number can be approximated by the
Dittus-Boelter equation:

\begin{equation}
\mathrm{Nu} = 0.023\, \mathrm{Re}^{4/5}\, \mathrm{Pr}^{n}
\label{eq:db}
\end{equation} 
Where $\mathrm{Re}$ is the Reynolds number and $\mathrm{Pr}$ is the Prandtl number.  $n$ is an empirically derived constant and is typically 0.4 for a heated flow.


%! TEX root = ../outline.tex

\bibliographystyle{unsrt}
\bibliography{sections/refs}



\end{document}
