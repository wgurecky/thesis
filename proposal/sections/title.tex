% Title.tex
% Author: William Gurecky
% Info:  Title page, table of contents
% Changlog:

%-----------------------------------------------------------------------------%
\begin{titlepage}
	\centering
	{\scshape\LARGE The University of Texas at Austin \par}
	\vspace{1cm}
	{\scshape\Large Nuclear \& Radiation Engineering \par}
	\vspace{1.5cm}
	{\huge\bfseries A CFD-Informed Gradient Boosted Model for Improving Subchannel Resolution CRUD Predictions\par}
	\vspace{2cm}
	{\Large William L. Gurecky \par}
	\vfill

	\begin{flushright}
	Dissertation Committee \par
	\bigskip
	Dr.~Derek \textsc{Haas}, Supervisor \par
	Dr. Sheldon Landsberger \par
	Dr. Benjamin Leibowicz \par
	Dr. Kevin Clarno \par
	Dr. Stuart Slattery \par
	\end{flushright}
	\vfill
	{\large \today\par}
\end{titlepage}
%-----------------------------------------------------------------------------%
\pagebreak
\tableofcontents
\pagebreak

%-----------------------------------------------------------------------------%
\clearpage
\vspace*{\fill}
\thispagestyle{empty} % suppress showing of page number
\begin{quotation}
\em % optional -- to switch to emphasis (italics) mode
In general, when building statistical models, we must not forget that the aim is to understand something about the real world. Or predict, choose an action, make a decision, summarize evidence, and so on, but always about the real world, not an abstract mathematical world: our models are not the reality—a point well made by George Box in his oft-cited remark that “all models are wrong, but some are useful”.
\medskip

--  David Hand
\end{quotation}
\vspace*{\fill}

\pagebreak
