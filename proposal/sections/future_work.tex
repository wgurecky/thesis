\section{Time Stepping}

TH conditions in the core change as the fuel is burned.  Therefore for any given CTF patch the TH boundary conditions to the CRUD simulator change as a function of time.  These time dependent changes can be accounted for in the buildup of CRUD by applying a time integration strategy.  Moreover, the CRUD layer both acts as a thermal resistor and augments the effective (boiling) heat transfer coefficient on the rod surface due to increasing the availability of vapor nucleation sites due to CRUD's porous structure.  This TH-CRUD feedback should be accounted for in an iterative time stepping scheme.
%A predictor-corrector time integration strategy, known as Heun's method, is proposed to step the CRUD solution forward in time and to resolve TH-CRUD feedbacks.  However, if the TH-CRUD feedback is weak, a simple forward Euler time integration strategy is sufficient.

The time stepping methodology must account for the fact that CRUD is grown in a Monte-Carlo fashion in each CTF patch, $j$.  Surface samples are weighted by their respective areas on the rod surface according to equation \ref{eq:weighted_crud}.  The process of growing CRUD on a single patch for a single time step is depicted in figure \ref{fig:crud_samples}.

\begin{figure}[hbtp]
\centering
\includegraphics[width=12cm]{images/crud_samples.png}
\caption{A single CRUD sample.}
\label{fig:crud_samples}
\end{figure}

\begin{align}
M_j &= \sum_{i=1}^N w_i m_{i} \\
B_j &= \sum_{i=1}^N w_i b_{i}
\label{eq:weighted_crud}
\end{align}
Where $N$ is the number of samples drawn per CTF patch.  The sample weights are given by $w_i = (A_j / N) [m^2]$ , where $A_j$ is the surface area of the CTF patch.  The CRUD sample mass and boron density are $m_j[kg/m^2]$ and $b_j [kg/m^2]$ respectively.  Note that the TH samples are drawn without regard to their spatial position inside the patch.  This presents a challenge when implementing a time stepping scheme.

A possible approach is to use the averaged CRUD result as an initial condition for the CRUD simulation at the next time index.
A primary concern with this approach is that the process of averaging the CRUD results between time steps does not preserve hot spot stationarity.  If the location of hot spots (though the magnitude the hot spot temperature may change) are stationary, then samples which resulted in a high temperature are likely to stay at a high temperature throughout time.  The averaging process disregards this potentially important information.  A future study of the averaging process's impact on the CRUD time evolution is planned. 

Equation \ref{eq:crud_state_single} is evaluated to generate an individual CRUD sample, $i$, at time $t+1$ on patch $j$.
\begin{equation}
\mathrm{\mathbf{C}}_{ij}^{t+1} = \mathbf{\Omega}(T_{ij}, TKE_{ij}, q_{ij}^{''}, \Delta t, \mathbf{\bar C}_j^t)
\label{eq:crud_state_single}
\end{equation}
The previous averaged CRUD state is included as an initial condition to the CRUD generator function.  The CRUD state passed to CTF is updated according to equation \ref{eq:crud_state}.
\begin{equation}
\mathrm{\mathbf{\bar C}}_j^{t+1} = \sum_{i=1}^N w_i \mathrm{\mathbf{C}}_{ij}^{t+1}
\label{eq:crud_state}
\end{equation}
Where $\mathbf{C}^t$ represents the CRUD state at time $t$.  $\mathbf{\Omega}$ represents the CRUD generator function which provides a CRUD result given TH boundary conditions and some initial CRUD state.   
The CRUD averaging process is depicted in figure \ref{fig:crud_samples_dt}.

\begin{figure}[hbtp]
\centering
\includegraphics[scale=.7]{images/crud_samples_dt.png}
\caption{CRUD averaging process.}
\label{fig:crud_samples_dt}
\end{figure}

The CRUD state is a non trivial object.  It is loosely defined to be a vector  comprised of many quantities such as porosity \& \ce{NiFe_2O_4} and \ce{Li} concentrations all tabulated as function of radial position in the CRUD layer.  The CRUD state definition is intentionally left ambiguous since its exact nature depends on the software implementation.   Future discussion with the developers of MAMBA will be required to expose the CRUD state vector as a user input.  In the current CRUD software, it is not possible to modify the internal CRUD state from the high level python interface.

\section{Uncertainty Propagation}

Future uncertainty quantification (UQ) tasks include estimating the impact of not knowing, precisely, the true shape of the margins or the copula.  These distributions are merely estimated from the available, noisy, CFD data.  Uncertainty in the shape, or more precisely the moments, of these distributions manifests as a broadening of the temperature and TKE margins which in turn propagates to an artificial broadening of the output CRUD distributions.

\section{Regression Kriging}

An alternative procedure based on regression kriging (RK) is under consideration for producing estimates and uncertainties of the desired FOI.
The available literature on the topic is substantial however much of the available resources focuses on kriging geostatistical data.  Regression kriging is typically introduced in the context of building
a surrogate model of a spatial field given a limited number of known sample points and a some set of predictive features that are known at a resolution greater than or equal to the endogenous variable sample resolution.
The initial steps taken to construct a RK model are similar to the currently proposed approach.  The primary focus of an RK model is to capture spatial auto-correlations in the field data.   If the ability to predict the full 2D distributions of the FOI are desired, an RK model could yeild the requisite fine scale surface distributions with uncertainty estimates. \\

First the target FOI is decomposed into a deterministic and stochastic component, as in equation \ref{eq:decomp}.
Let the spatial coordiantes, $\mathbf{s} = \{z, \theta\}$,
where $z$ is the axial position and $\theta$ is the azimuthal coordinate on the rod surface. Let the FOI be denoted as $T$.
\begin{equation}
T(\mathbf s) = \mu(\mathbf s) + \varepsilon(\mathbf s)
\label{eq:decomp}
\end{equation}

First, the mean field behavior must be estimated by either a least squares fit to the sampled CFD data, or by using the CTF result directly as the mean predictor, as is done in the proposed method. 
Additionally, the CTF results enter the kriging model as auxiliary predictive variables.  These variables serve as predictors of spatial auto-correlation.

The following example examines case of kriging the surface temperature field, $\hat T_\mathtt{RK}$.
Let $\mathbf{q}$ represent a predictive feature array comprised of the temperature,
TKE, and boundary heat flux from CTF.  Also included in this fictitious example, is a parameter $\Delta z$ which represents the axial distance to the nearest upstream spacer grid.

\begin{equation}
\mathbf{q} = \{TKE, q'', T, \Delta z \}_{CTF}
\end{equation}

In matrix form, the RK model evaluated at a point $\mathbf{s}_0 $ on the rod surface is given by equation \ref{eq:rk_eval} \cite{Hengl07}:

\begin{equation}
\hat T_\mathtt{RK}(\mathbf{s}_0 ) = \mathbf{q}_\mathbf{0}^\mathbf{T} \cdot \mathbf{\hat \beta}_\mathtt{OLS} + \mathbf{\lambda }_\mathbf{0}^\mathbf{T} \cdot (\mathbf T_{CFD}
- \mathbf{q} \cdot \mathbf{\hat \beta }_\mathtt{OLS} )
\label{eq:rk_eval}
\end{equation}
Where $\mathbf{q}_\mathbf{0}= \mathbf{q}(\mathbf s_0)$. The kriging weights at the sample location, $\mathbf{\lambda_0}$, are estimated by \ref{eq:simple_krige_weights} with the covariance function: $c(q_i, q_0)= \mathrm{Cov}(\mathbf e(q_i), \mathbf e(q_0))$ where $\mathbf e = (\mathbf T_{CFD} - \mathbf{q} \cdot \mathbf{\hat \beta }_\mathtt{OLS}$). 
\begin{equation}
\begin{pmatrix}\lambda_{0_1} \\ \vdots \\ \lambda_{0_n} \end{pmatrix}=
\begin{pmatrix}c(q_1,q_1) & \cdots & c(q_1,q_n) \\
\vdots & \ddots & \vdots  \\
c(q_n,q_1) & \cdots & c(q_n,q_n) 
\end{pmatrix}^{-1}
\begin{pmatrix}c(q_1,q_0) \\ \vdots \\ c(q_n,q_0) \end{pmatrix}
\label{eq:simple_krige_weights}
\end{equation}

The covariances must be estimated from the available CFD data.  

The residual vector, $\mathbf e$, naively includes all available CFD data points, but in practice a cut-off distance from the sample location, $\mathbf s_0$, can be specified to drastically reduce the amount of information required to construct and evaluate a covariance model.  This effectively reduces the length of the vectors $\mathbf{\lambda_0}$ and $\mathbf e$.

\begin{figure}[hbtp]
\centering
\includegraphics[scale=.3]{images/rk_example.png}
\caption{Regression kriging example \cite{Hengl07}.}
\label{fit:rk}
\end{figure}

RK is not straight forward to apply to the current Hi2Low problem because the available field estimates are very densely populated in space - but are sparse in the feature space of local core conditions. This is a non-typical situation for the application of RK. Therefore, building a RK model of the surface fields is left as an avenue for future investigation.
The application of RK is further complicated because the proposed set of auxiliary variables are not smoothly distributed in space.  Without the application of a smoothing pre-processor, these
fields jump in value when passing between CTF control volumes.

